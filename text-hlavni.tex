\documentclass{book}

\usepackage[utf8]{inputenc}
\usepackage[T1]{fontenc}
\usepackage{lmodern}
\usepackage[english,greek,russian,czech]{babel}
\let\ch\undefined
\usepackage{graphicx}
\graphicspath{ {./images_new/} }
\usepackage{xcolor}
\usepackage{geometry}
\geometry{a4paper}
\geometry{inner=2cm}
\geometry{outer=2cm}
\geometry{top=2cm}
\geometry{bottom=2cm}

\usepackage{chemmacros} %  https://ctan.org/pkg/chemmacros
% \chemsetup[chemformula]
% %\chemsetup{formula={chemformula}}
% \chemsetup{modules=all}
% \usepackage{mhchem}

\usepackage{rotating} %pro otočení jedné tabulky o~90 stupňů
\usepackage{ulem} %pro přeškrtnutý text příkazem \sout

% Přidáno při revizi AT
\usepackage{siunitx}
\DeclareSIUnit\torr{Torr}
\DeclareSIUnit\cal{cal}
% Ty jedna kundo, kvůli tomuhle jsem se docela zděsil (HH)
\sisetup{output-decimal-marker = {,}}

\usepackage{icomma}
\usepackage{upgreek}
\usepackage{hepnames}
\usepackage{caption}
\usepackage{subcaption}
\usepackage{tabto}
\usepackage[version=4]{mhchem}
\usepackage{enumitem}
\usepackage{svg}
\usepackage{chemfig}


%%% Načtení čínských znaků
\usepackage{CJKutf8}
\usepackage[overlap, CJK]{ruby}
\newenvironment{SChinese}{%
\CJKfamily{gbsn}%
\CJKtilde
\CJKnospace}{}
\newcommand{\cntxt}[1]{\begin{CJK}{UTF8}{}\begin{SChinese}#1\end{SChinese}\end{CJK}} 
%%% konec čínských znaků

%% záhlaví a zápatí
\usepackage{fancyhdr} % Load the package
\pagestyle{fancy} % Make pages fancy
\fancyhead[LE,RO]{Chemiklání}
\renewcommand{\chaptermark}[1]{\markboth{\MakeUppercase{\chaptername\ \thechapter: #1}}{}}
%\fancyhead[RE,LO]{}
%\fancyhead[RE,LO]{Section \thesection}
%\fancyfoot[CE,CO]{\leftmark}
%\fancyfoot[LE,RO]{\thepage}

% čtverečky na začátku úlohy
\usepackage{amssymb}
\newcommand{\nula}{$\square \, \square \, \square \, \square \; \; $}
\newcommand{\jeden}{$\blacksquare \, \square \, \square \, \square \; \; $}
\newcommand{\dva}{$\blacksquare \, \blacksquare \, \square \, \square \; \; $}
\newcommand{\tri}{$\blacksquare \, \blacksquare \, \blacksquare \, \square \; \; $}
\newcommand{\ctyri}{$\blacksquare \, \blacksquare \, \blacksquare \, \blacksquare \; \; $}

\usepackage[hidelinks]{hyperref}

\setlength{\parindent}{0pt}
\setlength{\parskip}{0.5em}
\usepackage{multirow}
\usepackage{multicol}
\title{\includegraphics[width=10cm]{logo_long} \\ \vspace{2cm} \\ Chemiklání 2016--2020: řešené úlohy}
\author{Jan Hrubeš, Adam Tywoniak, Stanislav Chvíla, Martin Balouch a kolektiv}
\date{}
\renewcommand{\thesubsection}{\alph{subsection}}
\usepackage[shortcuts]{extdash}

\renewenvironment{quotation}{\par}{\par} %zruší odsazování zadání

%%% volné stránky po \cleardoublepage nebo před kapitolama jsou úplně prázdné bez záhlaví
\makeatletter
\def\cleardoublepage{\clearpage\if@twoside \ifodd\c@page\else
  \hbox{}
  \thispagestyle{empty}
  \newpage
  \if@twocolumn\hbox{}\newpage\fi\fi\fi}
\makeatother

%%%%%%%%%%%%%%%%%%%%%%%%%%%%%%%%%%%%%%%%

\begin{document}

\maketitle


\cleardoublepage

\section*{Předmluva}
Milí studenti, vážení učitelé, přátelé chemie,

právě držíte v~ruce první vydání sbírky úloh, které se objevily v~soutěži Chemiklání v~letech 2016-2020. Všechny úlohy jsou řešené, řešení naleznete hned u~úlohy oddělené tečkovanou čárou. Úlohy jsou rozčleněné do kategorií podle jejich zaměření. Možná si říkáte, že Vám tu chybí úlohy na nějaké téma, nebo že v~některé kategorii je výrazně více úloh než v~jiné. Vězte, že i když jsme se v~rámci soutěžních úloh snažili o~rovnoměrné zastoupení všech oborů chemie, ne vše se nám povedlo obsáhnout. Proto zde kupříkladu není příliš mnoho úloh na sacharidy a lipidy, stejně jako zde můžete postrádat jednoduché vyčíslování rovnic. Naopak jsou zde obsaženy úlohy v~tematických oblastech, které se v~současnosti na většině středních škol (bohužel) nevyučují, a to hlavně instrumentální analytická chemie a NMR. Věříme, že i touto publikací můžeme přispět k~rozšíření povědomí o~těchto novinkách (mnohdy i více než padesát let starých).

Chemiklání se od většiny soutěží liší tím, že dovoluje soutěžícím nahlížet při řešení úloh do literatury. Sbírka proto může být skvělým podkladem pro tvorbu open-book testů, při kterých smějí žáci používat sešity a taháky. Věříme, že takovéto úlohy, které cílí na vyšší úrovně práce s~informacemi, než je pouhá pamětná znalost, mají mít v~našem vzdělávacím systému pevné místo.

Úlohy jsou též odstupňovány podle obtížnosti. Čím více černých čtverečků na začátku zadání najdete, tím je úloha těžší. Nejlehčí jsou úlohy označené \jeden, nejtěžší \ctyri. Řešení úloh s~více čtverečky mnohdy obnáší spoustu úmorného mechanického počítání. Nejtěžší z~úloh pak dají zabrat i vysokoškolskému studentovi.

Vydání sbírky by nebylo možné bez podpory Fakulty chemicko-technologické Univerzity Pardubice, jež v~uvedených letech hostila všechny ročníky soutěže a jejíž vydavatelství vzalo tuto publikaci pod svá křídla. Nesmíme též zapomenout na široký autorský tým, který za pět let sepsal více než 500 úloh, z~nichž bylo do soutěže nakonec vybráno 276. Z~těchto úloh Vám téměř všechny nabízíme na následujících stránkách.





Budeme rádi, pokud sbírka zpestří výuku chemie na středních školách, podnítí bystré žáky k~přemýšlení a třeba též k~účasti na dalším ročníku soutěže Chemiklání.\par
\vspace{18mm}
\noindent\-Za autorský kolektiv

Jan Hrubeš
\thispagestyle{empty}
\tableofcontents

\chapter{Obecná chemie, chemické výpočty}

\section{Výpočty hmotnostních a objemových zlomků, úlohy o~mísení}

% \subsection*{Ročník 4, úloha č. 0.16} 
\begin{quotation}
\jeden Běžnou rudou pro výrobu hliníku je bauxit, dihydrát oxidu hlinitého,
$\ce{Al2O3*2H2O}$. V~období studené války byl ale hliník jako důležitý
konstrukční materiál pro vojenské využití vyráběn v~bývalém SSSR v~množství okolo 3 megatun ročně také z~méně tradiční suroviny, nefelínu
($\ce{KAlSiO4}$). Spočítejte rozdíl v~množství hliníku (v kilogramech),
který lze získat z~jedné tuny každé z~rud.
\end{quotation} \dotfill \par 

Abychom zjistili, co se po nás chce, bude si vhodné spočítat, jakou část hmotnosti představuje hliník v~každé z~rud. Veličinou, která to dokáže vyjádřit, je hmotnostní zlomek.
\[
w = \frac{m_{\mathrm{hliník}}}{m_{\mathrm{ruda}}}
\]
Protože počítání s~hmotnostmi jednotlivých atomů by bylo nepraktické, budeme místo toho počítat s~molárními hmotnostmi atomů a molekul. 

Hmotnostní zlomek hliníku v~bauxitu je 

\[
w_{\ce{Al}/\mathrm{bauxit}}=\frac{2M_{\ce{Al}}}{2M_{\ce{Al}}+3M_{\ce{O}}+2M_{\ce{H2O}}}=0,39
\]

Hmotnostní zlomek hliníku v~nefelínu je

\[
w_{\ce{Al}/\ce{KAlSiO4}}=\frac{M_{\ce{Al}}}{M_{\ce{KAlSiO4}}}=0,17
\]

Z~jedné tuny nefelínu se tedy získá maximálně 0,17 tuny (170 kg) hliníku,
kdežto z~jedné tuny bauxitu získám 0,39 tuny (390 kg) hliníku. Prostým
odečtením těchto hodnot získám požadovaný výsledek:

\[
\Delta m_{\ce{Al}}=390-170=220\,\mathrm{kg}
\]


\hrulefill % \subsection*{Ročník 3, úloha č. 0.16}
\begin{quotation}
\jeden Železná ruda, kterou dovezli do provozu ostravských oceláren, obsahovala
46 \% hm. hematitu (též krevele, Fe$_{2}$O$_{3}$). Jiné sloučeniny
s obsahem železa nebyly v~rudě obsaženy. Jaké nejmenší množství této
rudy se spotřebuje na výrobu 15 tun surového železa, které obsahuje
5 \% nečistot?
\end{quotation} \dotfill \par  


V 15 tunách surového železa je 5 \% nečistot. Hmotnost čistého železa
tedy bude 
\[
m_{\mathrm{Fe}}=0,95\cdot15=14,25\,\mathrm{t}
\]

Hmotnostní zlomek železa v~hematitu je: 
\[
w_{\mathrm{Fe}}=\frac{2\cdot55,85}{2\cdot55,85+3\cdot16}=0,7=70\,\%
\]
Množství čistého hematitu dokážeme spočítat takto:

\[
m_{\mathrm{hem}}=\frac{m_{\mathrm{Fe}}}{w_{\mathrm{Fe}}}=\frac{14,25}{0,7}=20,36\,\mathrm{t}
\]

Obdobným postupem dopočítáme množství potřebné rudy:

\[
m_{\mathrm{ruda}}=\frac{m_{\mathrm{hem}}}{w_{\mathrm{hem}}}=\frac{20,36}{0,46}=44,3\,\mathrm{t}
\]


\hrulefill % \subsection*{Ročník 4, úloha č. 1.6}
\begin{quotation}
\dva Kauza lithium výrazně ovlivnila volby do Poslanecké sněmovny v~roce
2017. Podle Andreje Babiše\footnote{tehdejšího kandidáta jedné ze stran na premiéra ČR} šlo o~„kauzu za DVA TISÍCE MILIARD KORUN“.
Přehlédněte, prosím, Babišovu zřejmou neznalost číslovky bilion a
spočtěte, kolik tun cinvalditu, lithné rudy vyskytující se na Cínovci,
je potřeba na získání lithia za ony dva tisíce miliard korun! Cinvaldit
je minerál o~složení popsaném vzorcem $\ce{KLiFeAl(AlSi3)O10(OH)F}$. Cena lithia v~roce 2017 byla 200 000
Kč/tuna. Efektivita získávání lithia z~rudy je 80~\%. 
\end{quotation} \dotfill \par
Dva tisíce miliard korun je 
\[
2\cdot10^{3}\cdot10^{9}=2\cdot10^{12}\ \mathrm{K\check{c}}
\]
Při ceně lithia $2\cdot10^{5}\ \mathrm{K\check{c}\cdot t^{-1}}$ je
pro získání této částky zapotřebí 
\[
\frac{2\cdot10^{12}}{2\cdot10^{5}}=10^{7}\ \mathrm{t\, lithia}
\]
Při molární hmotnosti cinvalditu $M_{\mathrm{ cin}}=436,10\ \mathrm{g\cdot mol^{-1}}$
a molární hmotnosti lithia $M_{\mathrm{Li}}=6,94\ \mathrm{g\cdot mol^{-1}}$
by za předpokladu stoprocentní efektivity získávání lithia z~cinvalditu
bylo potřeba následující množství cinvalditu:
\[
\frac{436,10}{6,94}\cdot10^{7}=6,28\cdot10^{8}\ \mathrm{t}
\]
Protože je efektivita tohoto procesu jen 80 \%, je ve
skutečnosti nutno vytěžit cinvalditu o~to více\footnote{Na Cínovci se nalézá asi $6\cdot10^{8}\ \mathrm{t}$ této rudy, a
tak je zřejmé, že výpočty bývalého ministra financí v~kauze lithium
jsou i při zanedbání ceny těžby a rafinace rudy asi o~30 \% chybné.
Při započtení těžebních nákladů, na jejichž pokrytí je třeba asi 80
\% zisku by se Babiš dopustil chyby asi 600 \%. Jelikož vy nejste
politici, ale chemici, byly vám uznány výsledky, které se neliší od
autorského řešení o~více než 5 \%, tj. $7,45\cdot10^{8} - 8,25\cdot10^{8}\ \mathrm{t}$
cinvalditu.}
:
\[
\frac{100}{80}\cdot6,28\cdot10^{8}=7,85\cdot10^{8}\ \mathrm{t}
\]

\hrulefill % \subsection*{Ročník 4, úloha č. 0.15}
\begin{quotation}
\jeden Kyselina chlorovodíková se běžně dodává o~koncentraci 35 \%. Nepozorný
student biochemie Jardík (se kterým se v~této sbírce ještě párkrát setkáte) chtěl namíchat pro svůj experiment 15\% HCl. Bohužel
se spletl a omylem připravil kyselinu 10\%. V~jakém poměru má nyní
smíchat 35\% a 10\% kyselinu chlorovodíkovou, aby výsledkem byl 15\%
roztok? Všechna procenta v~úloze jsou hmotnostní. 
\end{quotation} \dotfill \par 
Pro řešení úlohy lze například použít směšovací rovnici: 
\[
\left(m_{1}+m_{2}\right)w_{\mathrm{celk}}=m_{1}w_{1}+m_{2}w_{2}
\]
 Jelikož nás zajímá poměr hmotností, vydělíme rovnici hmotností 10\%
HCl. 

\[
\frac{m_{1}}{m_{2}}\cdot w_{\mathrm{celk}} + w_{\mathrm{celk}}=\frac{m_{1}}{m_{2}}w_{1}+w_{2}
\]

\[
\left(\frac{m_{1}}{m_{2}}+1\right)w_{\mathrm{celk}}=\frac{m_{1}}{m_{2}}w_{1}+w_{2}
\]

Po převedení všech členů s~poměrem hmotností na jednu stranu rovnice a vytknutí vyjde
\[
\frac{m_{1}}{m_{2}}\left(w_{1}-w_{\mathrm{celk}}\right)=w_{\mathrm{celk}}-w_{2}
\]

\[
\frac{m_{1}}{m_{2}}=\frac{w_{\mathrm{celk}}-w_{2}}{w_{1}-w_{\mathrm{celk}}}
\]

\[
\frac{m_{1}}{m_{2}}=\frac{0,15-0,1}{0,35-0,15}=\frac{0,05}{0,2} =\frac{1}{4}
\]

Roztoky je tedy třeba smíchat v~poměru 1:4 (35\%:10\%).

Tento příklad lze též řešit i jinými postupy, například pomocí tzv. křížového pravidla.

\hrulefill % \subsection*{Ročník 5, úloha č. 0.13}
\begin{quotation}
\jeden V~pozdním prosincovém odpoledni si autor této úlohy připravoval fosfátový
pufr dle návodu. Vzal 1~tabletu pro přípravu pufru a rozpustil ji
ve 200,0 ml vody. Pomocí pH metru zjistil, že pH vzniklého roztoku
je 7,76. Jelikož však cílem experimentu bylo simulovat prostředí krve
při pH 7,4, rozhodl se autor této úlohy roztok okyselit pomocí 5\%
kyseliny fosforečné. Vzal láhev s~kyselinou fosforečnou (85 objemových \%)
a nalil 3,5~ml do odměrného válce. Kolika mililitry vody musel roztok zředit,
aby získal 5\% (objemová procenta) kyselinu fosforečnou? Zanedbejte objemovou kontrakci.
\end{quotation} \dotfill \par 

Pokud se prokoušete množstvím dat pro vyřešení úlohy nepotřebných, zjistíte, že výpočet lze jednoduše provést pomocí směšovací rovnice, kdy mísíme 3,5 ml 85\% kyseliny
(roztok 1) s~čistou vodou (roztok 2).

\[
V_{1}\phi_{1}+V_{2}\phi_{2}=(V_{1}+V_{2})\cdot\phi
\]
\[
3,5\cdot0,85+V_{2}\cdot0=(3.5+V_{2})\cdot0,05
\]
\[
V_{2}=56\ \mathrm{ml}
\]


\hrulefill % \subsection*{Ročník 5, úloha č. 0.26}
\begin{quotation}
\jeden \textit{Je libo šálek kávy?} Existují dva základní druhy kávy: robusta a arabica. Oba se liší
chemickým složením některých sekundárních metabolitů, a tedy i chutí,
vůní, a hlavně obsahem kofeinu. Robusta obsahuje 2,5 hm.~\% kofeinu,
kdežto arabica jen 1,5 hm. \%. Vysokoškolští studenti mají průměrnou
spotřebu 2~šálky kávy denně, ve zkouškovém období se ale jejich spotřeba kávy zdvojnásobí a vypijí tak 4 šálky za den. Předpokládejte, že jeden šálek obsahuje 100 mg kofeinu.
V jednom nejmenovaném obchodním řetězci prodávají kilogramovou směs
arabiky a robusty v~poměru 70:30. Kolik celých vysokoškoláků si může
během jednoho dne zkouškového z~tohoto pytle udělat kávu?
\end{quotation} \dotfill \par 
Pokud mám 1 kg směsi složené z~arabiky a robusty v~poměru 70:30, 
znamená to, že mám 700 g arabiky a 300 g robusty. Když vynásobím tyto hmotnosti příslušným hmotnostním zlomkem a sečtu je, dostanu hmotnost  kofeinu v~celém balení. 
\[
700\cdot0,015+300\cdot0,025=18\,\mathrm{g}
\]
V 1 kg směsi je 18 g, tedy 18 000 mg kofeinu. Jeden student běžně
potřebuje během zkouškového období celkem 4~šálky kávy -- oproti běžné denní spotřebě dvou šálků vypije ještě další dva denně navíc. A protože v~jednom šálku
je 100 mg kofeinu, je celková denní studentova spotřeba kofeinu ve zkouškovém 400 mg. 

To, kolik studentů si za jeden den udělá kávu, spočítám tak, že celkové
množství kofeinu podělím denní spotřebou jednoho studenta. Kávu si
tedy udělá 45 studentů.

\newpage %%

% \subsection*{Ročník 2, úloha č. 2.3 }
\begin{quotation}
\dva Smrtelná dávka kofeinu pro člověka je 120 mg/kg živé váhy. Uvažujme,
že průměrná hmotnost studenta před maturitou je 75 kg. Jedno 100g
balení instantní kávy obsahuje 4 g kofeinu, běžný energetický nápoj
pak 32 mg kofeinu ve 100 ml. Vypočítejte, kolik bezprostředně po sobě
vypitých hrnečků s~instantní kávou zalitou energetickým nápojem studenta
zabije. Nápoj se připravuje zalitím čtyř gramů instantní kávy 200
ml energetického nápoje. Neuvažujte jiný důvod smrti než předávkování
kofeinem.
\end{quotation} \dotfill \par 
Smrtelná dávka kofeinu pro 75kg člověka je:
\[
\mathrm{LD}=120\cdot75=9000\,\mathrm{mg}
\]
Jeden ,,energy hrneček`` obsahuje $2\cdot32 = 64$ mg kofeinu
z energy drinku a $4\cdot\frac{4}{100}\cdot 1000 = 160$ mg kofeinu z~kávy, což celkem
dělá 224~mg na jeden hrnek. Počet hrnků pak zjistíme jednoduchým podělením:
\[
N=\frac{9000}{224}=40,18
\]
Počet hrnků však musíme zaokrouhlit nahoru,
protože ve 40 hrncích ještě není obsaženo dost kofeinu na usmrcení studenta. Smrtelné je tím pádem požití 41 energyhrnků.

\hrulefill % \subsection*{Ročník 5, úloha č. 2.3}
\begin{quotation}
\dva Biochemikovi Jardíkovi se v~knihovně dostala do ruky učebnice buněčné
biologie a samozřejmě si chtěl hned něco vyzkoušet. Nechal si proto
vysvětlit ovládání průtokového cytometru a z~mrazáku vzal připravenou
kulturu kmenových buněk. Po roztátí a výměně média odpipetoval vzorek,
nastavil cytometr a dozvěděl se, že v~1 ml jeho suspenze je přítomno průměrně
1,6 milionu buněk o~středním průměru 12 mikrometrů. Jaký je objemový
zlomek cytoplasmy v~suspenzi? Buňky uvažujte jako stejně velké koule
bez organel, s~membránou zanedbatelné tloušťky.
\end{quotation} \dotfill \par 
Objemový zlomek spočítáme jako podíl objemu veškeré přítomné cytoplasmy na celkovém objemu suspenze. Víme, že cytoplasma je tekutina uvnitř buněk, její objem v~suspenzi tedy bude určen objemem všech buněk. Ten dokážeme jednoduše určit ze změřené početní hustoty a průměru pomocí vztahu pro objem koule:
\[
V_{\mathrm{buňka}} = \frac{4}{3} \pi \left( \frac{d}{2} \right) ^3 = \frac{4}{3} \pi \left( \frac{12}{2} \right) ^3 \doteq \SI{900}{\cubic\micro\metre}
\]
 900 mikrometrů krychlových činí 900 femtolitrů, což je $9\cdot10^{-10}$ ml. Potom má 1,6 milionu buněk (a cytoplasma v~nich) objem 1,44 mikrolitru:
\[
V_{\mathrm{cyt}} = 9\cdot10^{-10} \cdot 1\, 600\, 000 = 0,00144 \, \mathrm{ml} ,
\]

což vzhledem k~objemu vzorku 1 ml činí 1,44 objemových promile\footnote{Pro účely soutěže jsme uznávali hodnoty mezi 1,4 a 1,5 ‰.}.

\hrulefill % \subsection*{Ročník 5, úloha č. 5.6}
\begin{quotation}
\tri Oddělování pevných částic od kapaliny je rutinní praxí v~široké škále
procesů chemického průmyslu; s~filtrací se jistě již každý seznámil
v laboratoři nebo i v~domácím prostředí při přípravě kávy nebo při
koupání v~bazénu. Industriální filtr, používaný pro velké objemy
suspenzí, je kalolis. Do filtračního prostoru kalolisu se přivádí
suspenze materiálu. Tento prostor je uzavřen plachetkami, které umožňují
prostup vody, ovšem prostupu pevných částic zabraňují. Těchto filtračních
prostorů je za sebou mnoho a díky velké filtrační ploše je možné zpracovat
velké množství kalu.

Vaším úkolem bude spočítat, kolik tun kaolinu o~vlhkosti 28 \% (hmotnostních)
získáme za jeden filtrační cyklus, pokud budeme zpracovávat kaolinový
kal o~hustotě 1,187 $\mathrm{g\cdot cm^{-3}}$ s~hmotnostní koncentrací
kaolinu v~kalu 300 $\mathrm{g\cdot dm^{-3}}$. Průtok suspenze je 3,2 $\mathrm{dm^{3}\cdot s^{-1}}$,
filtrační cyklus trvá 50~minut.
\end{quotation} \dotfill \par 
Nejdříve vypočítáme celkový objem kalu zpracovaného za 50 minut jako
násobek průtoku suspenze a celkového času filtrace $t$. Nesmíme zapomenout
na přepočet jednotek. 
\[
V=t\cdot\dot{V}=(50\cdot60)\cdot3,2=9600\mathrm{~dm^{3}}
\]

\newpage %text%

Nyní vypočítáme množství sušiny kaolinu v~kalu jako násobek hmotnostní
koncentrace kaolinu a celkového objemu suspenze. Pro lepší přehlednost
můžeme převést hmotnost z~gramů na tuny

\[
m_{\text{sušina}}=\rho_{\text{kaolin}}\cdot V=300\cdot9600=2,88\cdot10^{6}\mathrm{~g}=2,88\mathrm{~t}
\]

Pokud známe vlhkost materiálu, známe i hmotnostní zlomek suchého kaolinu
ve vlhkém kaolinu.\[
w_{\text{sušina}}=1-w_{\text{voda}}
\]
Na základě vypočteného množství sušiny a se znalostí
vlhkosti materiálu je nyní možné vypočítat hmotnost vlhkého materiálu
v útrobách kalolisu. 
\[
m_{\text{vlhký kaolin}}=\frac{m_{\text{sušina}}}{1-w_{\text{voda}}}=\frac{2,88}{0,72}=4\mathrm{~t}
\]
Za jeden filtrační cyklus se získají čtyři tuny vlhkého materiálu.

\hrulefill % \subsection*{Ročník 3, úloha č. 2.5}
\begin{quotation}
\jeden Vězte, že organizátoři Chemiklání, ač tomu formulace a obtížnost úloh mnohdy
nenasvědčují, jsou také lidé a při jedné z~porad se rozhodli mírně
se rozptýlit degustací \cntxt{泸州老窖}\footnote{\textit{Luzhou Laojiao} (\cntxt{泸州老窖}) je čínský destilát typu \textit{baijiu}, získávaný fermentací čiroku dvoubarevného.}. Ačkoliv se nikomu
z přítomných nezdál nápoj dostatečně kvalitní (mírně řečeno), zaznamenal
jeden z~autorů nápis na štítku:\\,,100~ml, 56~perc.~vol.~ethanol``.
Pokuste se nyní spočítat objemový a hmotnostní zlomek vody v~\cntxt{泸州老窖}.
Předpokládejte, že přítomnými složkami jsou: voda ($\rho_{\mathrm{voda}}=0,9982\,\mathrm{g\cdot cm^{-3}}$)
a ethanol ($\rho_{\mathrm{ethanol}}=0,7893\,\mathrm{g\cdot cm^{-3}}$), jejich molární hmotnosti zaokrouhlete na celá čísla (např. 42) a výsledky
na tři platné číslice (např. 6,66). 
\end{quotation} \dotfill \par 
Objemový zlomek vody získáme odečtením objemového zlomku ethanolu
od jedničky:

\[
\phi_{\ch{H2O}}=1-0,560=0,440=44,0\,\%
\]

Hmotnostní zlomek je pak definován jako podíl hmotnosti vody ku hmotnosti
celého roztoku. To lze poměrně jednoduše vyjádřit pomocí objemů a
hustot. Objemy se pak dají zaměnit za příslušné objemové zlomky dle $V_A = \phi \cdot V$ (objem roztoku se vykrátí)

\[
w_{\ch{H2O}}=\frac{V_{\ch{H2O}}\cdot\rho_{\ch{H2O}}}{V_{\ch{H2O}}\cdot\rho_{\ch{H2O}}+V_{\mathrm{EtOH}}\cdot\rho_{\mathrm{EtOH}}}=\frac{\phi_{\ch{H2O}}\cdot\rho_{\ch{H2O}}}{\phi_{\ch{H2O}}\cdot\rho_{\ch{H2O}}+\phi_{\mathrm{EtOH}}\cdot\rho_{\mathrm{EtOH}}}
\]

Číselně pak
\[
w_{\ch{H2O}}=0,498=49,8\,\%
\]


\hrulefill % \subsection*{Ročník 4, úloha č. 4.3  }
\begin{quotation}
\dva Sušení je prastarou konzervační metodou. V~dnešní době se často provádí
mechanicky, zejména pokud jde o~ovoce či zeleninu. Čerstvé okurky
obsahují přibližně 97 hmotnostních procent vody. O~kolik procent klesne
jejich hmotnost po prvním cyklu sušení, pokud klesne obsah vody na
95 \%?
\end{quotation} \dotfill \par 
Celou úlohu jde vyřešit i obecně, my zde pro jednodušší představu ukážeme, jak bude vypadat sušení 100 g okurek\footnote{Mohli bychom samozřejmě vzít i jakoukoliv jinou hmotnost, se sto gramy je výpočet nejvíce názorný.}. Před sušením ve 100 gramech okurek najdeme 3 g sušiny a 97 g vody.
Při sušení se zbavujeme vody, to znamená, že se hmotnost sušiny nemění. Po prvním cyklu tvoří voda 95 procent a 3 g sušiny zbylých 5 procent hmotnosti. Z~hmotnosti a hmotnostních procent po prvním cyklu sušení pak lze jednoduše vypočítat hmotnost celku po prvním sušení. 
\[
m_1 =\frac{3}{0,05}=\SI{60}{\gram}
\]
Hmotnost 60 g po prvním cyklu sušení odpovídá poklesu hmotnosti o~40 procent.

\newpage %%

 % \subsection*{Ročník 4, úloha č. 3.1   }
\begin{quotation}
\jeden Oscillococcinum\textsuperscript{\textregistered} je homeopatický přípravek užívaný při
příznacích chřipky a nachlazení. Přípravek se připravuje homeopatickým
ředěním suspenze získané z~kachního srdce a jater až ke zředění 200C
(1 díl kachních jater na $10^{400}$ dílů cukru). Určete, z~jakého
nejmenšího počtu kachen musela být játra odebrána pro produkci tohoto
homeopatika od roku 1950, víte-li, že jedno balení obsahuje 10 g přípravku
a ročně se takových balení po celém světě prodá průměrně jedna miliarda.
Uvažujte neomezenou trvanlivost získávané suspenze, hmotnost jater
průměrné kachny odhadněte jako 100 g; kachna není Prometheus, játra
jí lze tedy odebrat pouze jednou.
\end{quotation} \dotfill \par 
Ředění, ve kterém se homeopatikum prodává, dalece přesahuje všechny
fyzikálně představitelné meze\footnote{Podle různých zdrojů je ve vesmíru kolem $10^{100}$ atomů, tedy abychom
spotřebovali "jeden atom z~kachních jater" (ať už si pod tím představíme cokoliv), museli bychom mít asi $10^{300}$
vesmírů homeopatika Oscilococcinum.}. Proto nezávisle na zadaných hmotnostech, prodejích či velikosti kachen
můžeme směle tvrdit, že na veškerou produkci Oscillococcina\textsuperscript{\textregistered} od počátku věků stačí játra z~jedné jediné kachny.

\hrulefill % \subsection*{Ročník 4, úloha č. 2.3 }
\begin{quotation}

\textit{,,Biochemie je jako vynášení odpadu. Je dobře, že to někdo
dělá, a je dobře, že to nemusím být já!``}\footnote{Toto rozhodně neřekl biochemik Jardík.} 

\dva Podívejme se nyní na vynášení odpadu v~biochemickém významu. Živočichy
lze podle způsobu, jakým se zbavují přebytečného dusíku například
z odbouraných aminokyselin, rozdělit na amonotelní, urikotelní a ureotelní.
Prvně jmenovaní vylučují dusík ve formě amonných iontů, druzí jako
kyselinu močovou a třetí jako močovinu. Vzorce posledních dvou jmenovaných
látek najdete níže, v~uvedeném pořadí zleva. Nejspíš se vám někdy
stalo, že jste spolu s~odpadky vyhodili i nějaké ještě použitelné
předměty. Pokuste se zjistit, která z~uvedených skupin živočichů takto
plýtvá nejvíce: vyberte z~navržených tří látek tu s~nejnižším hmotnostním
obsahem dusíku v~molekule a ten uveďte v~procentech. Pro výpočet použijte
elektroneutrální formy jednotlivých látek.
\begin{center}

\includegraphics{/4/2-3-1.eps}\includegraphics{/4/2-3-2.eps}

\par\end{center}

\end{quotation} \dotfill \par 
Obsah dusíku v~hmotnostních procentech se spočítá jako podíl molární
hmotnosti všech atomů dusíku ku molární hmotnosti celé molekuly ($\nu_{N}$
je počet dusíků v~molekule).

\[
w_{\mathrm{N}}=\frac{\nu_{\mathrm{N}}\cdot A_{\mathrm{N}}}{M_{\mathrm{molekula}}}
\]
Výsledky pro jednotlivé sloučeniny sumarizuje následující tabulka:
\noindent \begin{center}
\begin{tabular}{c|c|c}

Látka & Sumární vzorec & Obsah dusíku (hm. \%)\tabularnewline
\hline 
\hline 
amoniak & $\ce{NH3}$ & 82,4\tabularnewline
\hline 
kyselina močová & $\ce{C5H4N4O3}$ & 33,3\tabularnewline
\hline 
močovina & $\ce{CH4N2O}$ & 46,7\tabularnewline

\end{tabular}
\par\end{center}

Je zřejmé, že nejvíce plýtvají organismy vylučující kyselinu močovou.

\newpage %%

\section{Látkové množství}

 % \subsection*{Ročník 4, úloha č. 0.13 }
\begin{quotation}
\jeden Na úlevu od soutěžního stresu je při řešení úvodní série trochu brzy.
Mnozí jako pomoc ve stresových situacích vyhledávají klid, tmu, příjemnější
prostředí mimo budovu univerzity, nebo také sahají po krabičce cigaret.
V nich obsažený nikotin je rostlinný pyridinový alkaloid obsažený
v~tabáku, který je velmi jedovatý. Při kouření přestupuje do organismu
pouze malá část nikotinu, při perorální\footnote{Nějaký lék nebo jed můžeme do těla dostat mnoha způsoby. Ten nejjednodušší, \textit{perorální}, znamená, že látku prostě sníme. Mezi další způsoby podání patří například \textit{inhalační} (vdechování), \textit{transdermální} (masti vstupující skrz kůži) nebo \textit{intravenózní} (přímo do žíly). Způsob podání samozřejmě ovlivňuje, kolik látky se ve skutečnosti vstřebá do těla -- z~tohoto pohledu je nejpřímější intravenózní podání.} konzumaci cigaret je ale vstřebávání
nikotinu téměř stoprocentní. Kolik bezprostředně po sobě požitých
cigaret usmrtí dospělého člověka, je-li průměrná smrtelná dávka pro
průměrného dospělého člověka 0,3\,mmol, 1 cigareta obsahuje 10~mg
nikotinu a molární hmotnost nikotinu je 162~$\mathrm{g\cdot mol^{-1}}$? Uvažujte pro odhad
stoprocentní absorpci nikotinu při konzumaci.
\end{quotation} \dotfill \par 
Vypočteme množství nikotinu potřebné k~usmrcení dospělého člověka:
\[
m=n\cdot M
\]
po dosazení: 

\[
m=0,3\cdot10^{-3}\cdot162=0,0486\,\mathrm{g=48,6\,\mathrm{mg}}
\]

Pomocí přímé úměry (lze též trojčlenkou) dopočítáme počet cigaret: 
\[
N_{\mathrm{cig}}=\frac{48,6}{10}=4,86\,\mathrm{cigaret}
\]
Dospělého člověka usmrtí 5 bezprostředně za sebou požitých cigaret.

\hrulefill % \subsection*{Ročník 5, úloha č. 0.15 }
\begin{quotation}
\jeden \textit{Haloquadratum walsbyi} je druh organismu z~domény staruší
(\textit{Archaea}) objevený v~80. letech 20. století v~nádržích
se solankou na Sinajském poloostrově. Vyznačuje se plochými buňkami
čtvercového tvaru (odtud jeho název) a patří mezi extremofilní organismy:
nejlépe se mu žije v~roztocích chloridu sodného, ve kterých se odpařováním
vody sráží pevná sůl. Protože nás zajímalo, kolik soli naše breberka
snese, odlili jsme si 250 ml roztoku ze solného jezírka, nechali veškerou
vodu odpařit a zbylo nám 90,5 g soli. V~jezírku byla teplota 35 °C.
Určete molární koncentraci chloridu sodného ve vzorku.
\end{quotation} \dotfill \par 
Molární koncentrace je definována jako podíl látkového množství a
objemu. Látkové množství určíme vydělením hmotnosti chloridu sodného
jeho molární hmotností, která je $M_{\ch{NaCl}}=58,44\ \mathrm{g\cdot mol^{-1}}$.
Vyjde:
\[
 n=\frac{m}{M_{\ch{NaCl}}}=\frac{90,5}{58,44}=1,55\ \mathrm{mol}
\]
Molární koncentrace je tedy 
\[
c=\frac{n}{V}=\frac{1,549}{0,250}=6,19\ \mathrm{mol\cdot dm^{-3}}
\]
Údaj o~teplotě k~řešení nepotřebujeme.

\hrulefill % \subsection*{Ročník 1, úloha č. 1.2 }
\begin{quotation}
\jeden Makromolekulární chemie je obor, který se rozvíjí již desítky let. Od svého počátku v~hlubokém 20. století přinesla nespočet polymerů rozličných vlastností s~využitím v~mnoha oblastech lidské činnosti.
Podívejme se nyní na molekulu v~dnešní době docela obyčejnou, takový polystyren.
Kolik váží jeden mol polystyrenu, jehož relativní molekulovou hmotnost aproximujeme jako 1 000 000? 
\end{quotation} \dotfill \par 
Relativní molekulová hmotnost je číselně shodná s~molární hmotností
v jednotkách $\mathrm{g\cdot mol^{-1}}$, tudíž hmotnost jednoho molu makromolekul polystyrenu
bude $1\, 000\, 000\, \mathrm{g}=1000\, \mathrm{kg}=1\, \mathrm{t}.$

\newpage %%

 % \subsection*{Ročník 5, úloha č. 1.4}
\begin{quotation}
\dva Fruktóza je hlavním energetickým zdrojem pro spermie, proto se také
vyskytuje v~seminální tekutině ve vysokých koncentracích. Spočítejte,
kolik miligramů fruktózy ze sebe vydá muž za jeden měsíc, pokud koncentrace
fruktózy ve spermatu dosahuje průměrně 7,5 mmol~$\cdot$~dm$^{-3}$,
průměrný objem ejakulátu činí 5 ml a dodržuje pouze lékaři doporučované
množství ejakulací měsíčně, což je 12. Napovíme, že fruktóza je isomer
glukózy.
\end{quotation} \dotfill \par 
Koncentrace fruktózy je 7,5~mmol~$\cdot$~dm$^{-3}$, tedy $7,5\cdot10{}^{-3}\,\mathrm{mol\cdot dm^{-3}}$.
Objem jednoho ejakulátu je 5 ml, tedy $5\cdot10^{-3}\, \mathrm{dm^{-3}}$. Látkové
množství fruktózy v~jedné dávce lze vypočítat jako 
\[
n_{\text{dávka}} = c_{\text{fruktóza}}\cdot V=7,5\cdot 10^{-3} \cdot 5\cdot 10^{-3} =3,75\cdot10^{-5}\mathrm{\ mol}
\]
Celkově se za měsíc, ve kterém proběhne 12 ejakulací, jedná o~takovéto látkové množství fruktózy:
\[
n_{\text{fruktóza}} = 12\cdot3,75\cdot10^{-5}=4,5\cdot10^{-4}\mathrm{\ mol}
\]
Molární hmotnost fruktózy určíme na základě sumárního
vzorce, který je $\ch{C6H12O6}$ (stejný jako u~glukózy), tedy $M_{\text{fruktóza}}=180\mathrm{\ g\cdot mol^{-1}}$.
Celková hmotnost fruktózy je potom
\[
m_{\text{fruktóza}}=n_{\text{fruktóza}}\cdot M_{\text{fruktóza}}=4,5\cdot10^{-4}\cdot180=0,081\mathrm{\ g}=81\ \mathrm{mg}
\]

\hrulefill % \subsection*{Ročník 2, úloha č. 1.5 }
\begin{quotation}
\dva Snem mnoha lidí je z~různých příčin navázání komunikace s~vyspělou
mimozemskou civilizací. Pokud by se nám takový kontakt zdařil, je
možné, že by přišlo na výměnu znalostí. Problém by nastal při pokusu
o předání našich fyzikálních zákonů, protože jedna ze sedmi základních
fyzikálních jednotek, kilogram, je definována podle hmotnosti mezinárodního
prototypu kilogramu.\footnote{Tuto úlohu jsme předložili k~řešení v~roce 2017, tj. před redefinicí základních jednotek SI, která proběhla o~rok později. Od té doby se definice kilogramu opírá o~Planckovu konstantu.
} Ten je, jak známo, uložen v~Mezinárodním úřadu
pro míry a váhy v~Sèvres u~Paříže. Mimozemšťané by tedy pro pochopení
našeho systému fyzikálních jednotek museli vážit cestu do Francie
a vyvstává obava, že se jim nebude chtít. Důsledkem této definice
kilogramu je fakt, že Avogadrova konstanta není konstantou, protože
se může měnit s~tím, jak se mění hmotnost prototypu. Současná hodnota
Avogadrovy konstanty se udává jako $N_{\mathrm{A}}=(6,022140857\pm0,000000074)\cdot10^{23}\, \mathrm{mol^{-1}}$.
Pokud bychom změřili Avogadrovu konstantu zcela přesně, mohli bychom
vytvořit novou definici kilogramu a mimozemšťané by tak nemuseli do
Francie. Pro vás máme úkol o~poznání jednodušší. Kolik atomů manganu
je v~55 ng dokonale čistého $\mathrm{Mg(OH)_{2}}$? 
$A_\mathrm{r} (\mathrm{Mg}) = 24,31$, $A_\mathrm{r} (\ch{O}) = 16,00$, $A_\mathrm{r} (\ch{H}) = 1,01$.
\end{quotation} \dotfill \par 
V hydroxidu hořečnatém se mangan nenachází. Správnou odpovědí je tedy
0 atomů.

\hrulefill % \subsection*{Ročník 5, úloha č. 1.6 }
\begin{quotation}
\jeden V~mnoha zemích je zákonem zakázán prodej destilátů obsahujících více
než 60 obj. \% alkoholu. Tam, kde to předpisy dovolují, najdeme v
obchodech i směsi blížící se azeotropickému limitu pro binární směs
ethanol-voda, který je 96 hm. \%. Takovýto alkohol, označovaný v~Polsku
jako „Spirytus rektyfikowany 95 \%`` a vyráběný z~brambor, kdežto
v~USA z~obilí či kukuřice jako ,,190-proof``, se používá pro výrobu
bylinných tinktur, ochucených likérů či jako rozpouštědlo. Určete
molární koncentraci zcela čistého (tedy stoprocentního) ethanolu,
pokud by se nám podařilo obejít termodynamická a praktická destilační
omezení. Jeho hustota je 789 kg$\cdot$m$^{-3}$.
\end{quotation} \dotfill \par 
Molární koncentrace je definována jako podíl látkového množství a
objemu 
\[
c=\frac{n}{V}
\]
Látkové množství je definováno jako podíl hmotnosti látky s~její
molární hmotností 
\[
n=\frac{m}{M}
\]
Objem potom získáme podělením hmotnosti látky hustotou 
\[
V=\frac{m}{\rho}
\]

Po dosazení dostáváme 
\[
c=\frac{\rho}{M}
\]
Před dosazením je třeba zkontrolovat kompatibilitu jednotek, nejlépe
volbou těch základních. Molární hmotnost ethanolu ($M=46\,\mathrm{g\cdot mol^{-1}}$)
proto převedeme na 0,046 kg$\cdot$mol$^{-1}$ a můžeme dosadit:

\[
c=\frac{789}{0,046}=17152\,\mathrm{mol\cdot m^{-3}}=17,2\mathrm{\,mol\cdot dm^{-3}}
\]

Místo dosazení do definičního vztahu můžeme uvažovat libovolné množství
ethanolu, například 1 litr, jehož hmotnost je při zadané hustotě 789
g. Podělením této hmotnosti molární hmotností $46\,\mathrm{g\ mol^{-1}}$
získáme tentýž výsledek:

\[
c=\frac{m}{V\cdot M} = \frac{789}{1\cdot 46} = 17,2\mathrm{\,mol\cdot dm^{-3}}
\]

\hrulefill % \subsection*{Ročník 1, úloha č. 2.5}
\begin{quotation}
\jeden Jakkoli je voda látkou na Zemi prakticky všudypřítomnou a podrobně zkoumanou, je s~podivem, kolik pokusů o~hlubší porozumění jejím fyzikálně-chemickým vlastnostem končí jen částečným úspěchem. Mnohé výpočetní postupy, užitečné pro popis molekulárních struktur na první pohled komplikovanějších, u~prosté \ce{H2O} narážejí na své limity.
Místo toho pro vás máme výpočetní úkol s~lépe dosažitelným výsledkem: zkuste spočítat molární koncentraci molekuly $\ch{H2O}$ v~1 litru destilované vody.
\end{quotation} \dotfill \par 
Naším úkolem je zjistit molární koncentraci ,,vody ve vodě“. Jinými slovy -- zjistit, kolik molů vody je v~jednom litru. Podobně jako u~předchozí úlohy lze vše odvodit ze základních vzorců pro koncentraci a látkové množství (hustotu vody uvažujeme $1000\, \mathrm{kg\cdot m^{-3}}$). Při dosazování do vzorce je třeba dát pozor na jednotky\footnote{Správně bychom měli dosadit hustotu v~gramech na litr, jelikož molární hmotnost zde máme v~gramech na mol a výsledná koncentrace by měla být v~molech na litr. Hustota v~gramech na litr je ovšem číselně shodná s~hustotou v~kilogramech na metr krychlový.}.
\[
c=\frac{n}{V}=\frac{\frac{m}{M}}{\frac{m}{\rho}}=\frac{\rho}{M}=\frac{1000}{18}=55,56\, \mathrm{mol\cdot dm^{-3}}
\]

\hrulefill % \subsection*{Ročník 4, úloha č. 3.3}
\begin{quotation}
\tri Ještě v~roce 1997 unikaly v~ČR do ovzduší 2 miliony tun oxidu siřičitého
ročně. Jedním z~jeho hlavních zdrojů bylo spalování hnědého uhlí v
severočeských tepelných elektrárnách. Významná část tohoto oxidu se
v atmosféře přemění na kyselinu sírovou. Budeme pro účely této úlohy předpokládat, že tato
reakce probíhá kvantitativně a že se tedy na kyselinu sírovou přemění veškerý oxid siřičitý.

Vypočítejte, jakou molární koncentraci
měla kyselina sírová v~kyselých deštích dopadajících na Krušné hory,
pokud plochu těchto hor aproximujete\footnote{Aproximací se myslí zjednodušení problému na nějaký jiný, který dokážeme vyřešit. V~tomto případě by bylo obtížné počítat plochu Krušných hor na základě jejich přesných hranic, vhodně navržený lichoběžník nám proto může celkem zjednodušit práci. Možná se ptáte, proč jsme pro aproximaci nezvolili obdélník -- vždyť ten by dokázal plochu Krušných hor taky nějak obstojně reprezentovat a počítalo by se s~ním lépe. Zde je třeba říci, že jsme jako autoři byli poněkud škodolibí.} jako rovnoramenný lichoběžník,
jehož základny měří 100 km a 120 km, jeho ramena mají 30 km. Počítejte,
že na oblast Krušných hor dopadlo 40 \% roční produkce $\ce{SO2}$
a že ročně v~Krušných horách naprší 1100 mm srážek.
\end{quotation} \dotfill \par 
Abychom mohli spočítat molární koncentraci kyseliny sírové v~kyselých
deštích, potřebujeme znát látkové množství kyseliny a objem, ve kterém
je rozpuštěná. S~látkovým množstvím je to jednoduché, to je 0,4násobek
látkového množství oxidu siřičitého. 

\[
c=\frac{n_{\ce{H2SO4}}}{V}
\]

\[
n_{\ce{H2SO4}}=0,4\cdot n_{\ce{SO2}}=0,4\cdot\frac{m_{\ce{SO2}}}{M_{\ce{SO2}}}=1,25\cdot10^{10}\,\mathrm{mol}
\]

\newpage %text% 

Objem se pak dopočítá jako obsah
podstavy lichoběžníku násobený ročním úhrnem. Pozor na to, že je třeba
dopočítat ještě výšku lichoběžníku\footnote{Pokud tedy nepočítáte obsah pomocí Brahmaguptova vzorce: $S=\sqrt{(s-a)(s-b)(s-c)(s-d)}$,
kde $s$ je polovina obvodu.}!
\begin{center}
\includegraphics[scale=0.5]{/4/3-3}
\par\end{center}

\[
V=\frac{(a+c)v}{2}\cdot1,1=\frac{(a+c)\sqrt{b^{2}-e^{2}}}{2}\cdot1,1=\frac{(a+c)\sqrt{b^{2}-\left(\frac{a-c}{2}\right)^{2}}}{2}\cdot1,1
\]
\\
Objem je také třeba dosadit v~litrech, aby koncentrace vyšla v~klasických jednotkách $\mathrm{mol\cdot dm^{-3}}$.
\[
V=3,4\cdot10^{9}\,\mathrm{m^{3}}=3,4\cdot10^{12}\,\mathrm{dm^{3}}
\]
\[
c=\frac{n_{\ce{H2SO4}}}{V}=\frac{1,25\cdot10^{10}}{3,4\cdot10^{12}}=0,0037\,\mathrm{mol\cdot dm^{-3}} =3,7\,\mathrm{mmol\cdot dm^{-3}}
\]


\hrulefill % \subsection*{Ročník 2, úloha č. 5.1 }
\begin{quotation}
\tri Podle legislativy platné v~EU od roku 2010 nesmí denní průměr koncentrace
ozonu za osm hodin překročit hodnotu \SI[inter-unit-product = \ensuremath{{}\cdot{}}]{120}{\micro\gram\per\cubic\metre}.
V USA je podle federálních standardů vydávaných EPA (Environmental
Protection Agency) tento limit 70 ppb. Určete, který z~těchto limitů
je přísnější. Svou odpověď doložte přepočtem jednoho limitu na jednotky,
ve kterých je uveden druhý limit.
\end{quotation} \dotfill \par 
Přepočtěme například americký limit na jednotky hmotnostní koncentrace:
Zadaný údaj 70 ppb představuje 70~částic z~jedné miliardy (anglicky
billion), tedy molární zlomek $7\cdot10^{-8}$. V~jednom metru krychlovém
vzduchu je za normálních podmínek přítomno 
\[
n=\frac{1}{V_{\mathrm{m}}}=\frac{1}{0,02271}=44,03\,\mathrm{mol}
\]
ideálního plynu\footnote{Zde je velká debata, jestli používat molární objem plynu 22,41 l či
22,71 l. Tento nesoulad vychází z~ne zcela jednotné definice standardních
a normálních podmínek, resp. standardního a normálního tlaku. Molární
objem 22,41 l odpovídá 1 mol plynu za teploty 0 °C a tlaku 101~325 Pa,
molární objem 22,71 l pak novější definici podmínek dle IUPAC, kde
tlak činí 100~000 Pa. V~soutěži jsme uznávali obojí, stejně jako různé
varianty výpočtu ze stavové rovnice ideálního plynu při zadávání
teplot v~rozmezí 0--25 °C.}. Vynásobením tohoto množství zadaným molárním zlomkem získáme látkové
množství ozonu v~1 m$^{3}$:
\[
n_{ozon} = 44,03 \cdot 7\cdot 10^{-8} =3,082\cdot10^{-6}\,\mathrm{mol}
\]

Při molární hmotnosti ozonu $48\,\mathrm{g\cdot mol^{-1}}$ má toto
množství hmotnost \SI{148}{\micro\gram}. Evropský limit je tedy přísnější. 

V\ případě dosazení normálních podmínek, tj. teploty 293,15 K a tlaku 101~325~Pa, vyjde hmotnostní koncentrace \SI[inter-unit-product = \ensuremath{{}\cdot{}}]{140}{\micro\gram\per\cubic\metre}.

Zvolíme-li opačný směr výpočtu, dojdeme k~výsledku, že evropskému
limitu odpovídá koncentrace mezi 57 a 60 ppb (v závislosti na zvolených
podmínkách). 

\newpage %%

\section{Přepočet hmotnostního a molárního zlomku}

% \subsection*{Ročník 4, úloha č. 0.21   }
\begin{quotation}
\jeden Téměř ve všech solidních periodických tabulkách můžete najít také
relativní atomové hmotnosti. Většina prvků je tvořena směsí několika
izotopů, brány v~potaz jsou také radioaktivní nuklidy. Relativní hmotnost
prvku uváděná v~tabulce pak představuje hmotnost jednoho molu atomů,
v~němž jsou jednotlivé izotopy zastoupeny stejně jako v~přírodě. Relativní
atomová hmotnost chloru o~přírodním složení je 35,45. Vypočítejte
molární zlomek chloru 35 v~přírodním zastoupení, pokud předpokládáme,
že v~přírodě vyskytující chlor se skládá pouze z~izotopů 35 a 37.
Relativní atomové hmotnosti izotopů považujte za rovny jejich nukleonovým
číslům.
\end{quotation} \dotfill \par 
Molární hmotnost prvku uvedená v~tabulce je molární hmotností směsi
všech jeho izotopů tak, jak se vyskytují v~přírodě. Tuto molární hmotnost
můžeme vypočítat jako: 
\[
M_{\mathrm{tabulka}}=x_{1}M_{1}+x_{2}M_{2}
\]
Nás zajímá molární zlomek $x_{1}$ ve směsi. Zároveň pro naši dvousložkovou
směs platí: 
\[
x_{1}+x_{2}=1
\]
Dosazením do první rovnice získáme výraz pro výpočet molárního zlomku
chloru 35 ve směsi. 
\[
M_{\mathrm{tabulka}}=x_{1}M_{1}+\left(1-x_{1}\right)M_{2}
\]
\[
x_{1}\left(M_{1}-M_{2}\right)=M_{\mathrm{tabulka}}-M_{2}
\]
\[
x_{1}=\frac{M_{\mathrm{tabulka}}-M_{2}}{M_{1}-M_{2}}
\]
\[
x_{1}=\frac{35,45-37}{35-37}=0,775
\]
V přírodním zastoupení je 77,5 mol.~\% chloru 35. 

\hrulefill % \subsection*{Ročník 4, úloha č. 1.5   }
\begin{quotation}
\jeden Vypočítejte molární hmotnost plynné směsi složené z~1,0 gramu molekul lehkého
vodíku a 1,0 gramu molekul deuteria. Relativní atomové hmotnosti vodíku
a deuteria považujte za rovné jejich nukleonovému číslu.
\end{quotation} \dotfill \par 
Molární hmotnost směsi vypočteme z~následujícího definičního vztahu: 
\[
M=\frac{m_{\mathrm{celk}}}{n_{\mathrm{celk}}}=\frac{m_{H}+m_{D}}{n_{H}+n_{D}}
\]
\[
M=\frac{m_{H}+m_{D}}{\frac{m_{H}}{M_{H}}+\frac{m_{D}}{M_{D}}}=\frac{1+1}{\frac{1}{2}+\frac{1}{4}}=2,67\ \mathrm{g\cdot mol^{-1}}
\]
Molární hmotnost molekuly lehkého vodíku je $M_H = 2\, \mathrm{g\cdot mol^{-1}}$, molární hmotnost molekuly těžkého vodíku(deuteria) je $M_D = 4\, \mathrm{g\cdot mol^{-1}}$. Molární hmotnost plynné směsi je $2,67\ \mathrm{g\cdot mol^{-1}}$.

\newpage %%

\hrulefill % \subsection*{Ročník 1, úloha č. 8.3}
\begin{quotation}
\ctyri V~přírodě se olovo vyskytuje zcela běžně v~horninách, v~lidském organismu jeho přítomnost ovšem vyvolává nervové poruchy, je toxické rovněž pro krevní
oběh a kumuluje se v~měkkých tkáních a v~mozku. Aby vaše šedé buňky nechřadly, následuje tematická
mozková rozcvička. Olovo se totiž v~přírodě vyskytuje ve 4 nuklidech
s~různým zastoupením. Jedná se o~$^{204}\mathrm{Pb}$, $^{206}\mathrm{Pb}$, $^{207}\mathrm{Pb}$,
$^{208}\mathrm{Pb}$. 

Vypočtěte molární zastoupení $^{206}\mathrm{Pb}$
v přírodě, víte-li, že molární poměr zastoupení $^{208}\mathrm{Pb}:{}^{207}\mathrm{Pb}$
je 2,371 : 1. Dále víte, že molární zlomek $^{204}\mathrm{Pb}$ je 1,4\% a
molární poměr $^{206}\mathrm{Pb}:(^{204}\mathrm{Pb}+{}^{207}\mathrm{Pb})$ je 1,026 : 1. Předpokládejte
přitom, že relativní atomová hmotnost izotopu je rovna jeho nukleonovému
číslu.

Relativní atomová hmotnost olova je $A_{\mathrm{r}}(\mathrm{Pb})=207,2$.

\end{quotation} \dotfill \par 
Zadání maskuje skutečnost, že se jedná o~čtveřici rovnic o~čtyřech
neznámých -- tři jsou výslovně určeny v~textu, čtvrtou maskuje skutečnost,
že $A_{r}(\mathrm{Pb})=207,2$. Numericky vyjádřeno: 

\[
x_{^{204}\mathrm{Pb}}=0,014
\]

\[
\frac{x_{^{206}\mathrm{Pb}}}{x_{^{204}\mathrm{Pb}}+x_{^{207}\mathrm{Pb}}}=1,026
\]

\[
\frac{x_{^{208}\mathrm{Pb}}}{x_{^{207}\mathrm{Pb}}}=2,371
\]

\[
204\cdot
x_{^{204}\mathrm{Pb}}+206\cdot
x_{^{206}\mathrm{Pb}}+207\cdot
x_{^{207}\mathrm{Pb}}+208\cdot
x_{^{208}\mathrm{Pb}}=207,2
\]

Nyní je soustava čtyř lineárních rovnic o~čtyřech neznámých definována
jednoznačně, výsledky je možné určit za pomocí manuálního či kalkulačkového
dopočtu. Matematické řešení této soustavy tu podrobně nerozebíráme, uvedeme pouze výsledné hodnoty molárních zlomků s~přesností na 3 desetinná místa:

$x_{^{204}\mathrm{Pb}}=0,014$; $x_{^{206}\mathrm{Pb}}=0,241$; $x_{^{207}\mathrm{Pb}}=0,221$;
$x_{^{208}\mathrm{Pb}}=0,524$. 

\hrulefill  % \subsection*{Ročník 3, úloha č. 0.22 }
\begin{quotation}
\jeden Vajíčka jsou nedílnou součástí české kuchyně. Všichni jistě víte,
že jejich skořápky jsou tvořeny uhličitany, nejčastěji CaCO$_{3}$,
ale v~menším množství i MgCO$_{3}$. Vzorek skořápky o~hmotnosti 0,9845
g obsahuje neznámé množství uhličitanů. Při reakci s~10 ml koncentrované
HCl se uvolnilo 50 ml plynu. Uvažujte molární objem plynu $V_{\mathrm{ m}}=22,41\,\mathrm{dm^{3}\cdot mol^{-1}}$.
Jaký je hmotnostní zlomek \textbf{uhličitanového aniontu} ve vzorku?
\end{quotation} \dotfill \par 
Hmotnostní zlomek uhličitanového aniontu je definován jako podíl hmotnosti
všech molekul aniontu ku hmotnosti vzorku. Vzorcem:

\[
w=\frac{m_{\ch{CO3^{2-}}}}{m_{\mathrm{vz}}}
=\frac{n_{\ch{CO3^{2-}}}\cdot M_{\ch{CO3^{2-}}}}{m_{\mathrm{vz}}}
\]

Počet molekul uhličitanu bude rovný látkovému množství oxidu uhličitého,
která z~uhličitanů vzniká dle rovnice:
\shorthandoff{-}
\ch{CO3^{2-} + 2 H+ -> CO2 + H2O}
\shorthandon{-}
Toto látkové množství spočítáme pomocí zadaného molárního objemu:

\[
n_{\ch{CO3^{2-}}}=n_{\ch{CO2}}=\frac{V_{\ch{CO2}}}{V_{\mathrm{m}}}=\frac{50}{22,41}= 2,231\,\mathrm{mol}
\]

Po dosazení příslušných hodnot ($M_{\ch{CO3^{2-}}}=60\,\mathrm{g\cdot mol^{-1}}$)
vyjde hodnota $w=0,136=13,6\,\%$.

\newpage %%
% \subsection*{Ročník 4, úloha č. 0.23 }
\begin{quotation}
\jeden O~samolibosti autorského kolektivu Chemiklání se dozvídáte nejen osobně,
ale také prostřednictvím námi předkládaných úloh. Tato samolibost
již překročila úroveň vymýšlení nejrůznějších postů v~organizační
struktuře a krčení nosu -- představte si, že nejeden člen autorského
kolektivu by byl radši, kdyby mu proudil v~žilách místo obyčejného
železa nějaký ušlechtilejší kov, kupříkladu palladium. Vypočítejte,
kolik by přibral biochemik Jardík, kdybychom nahradili
veškeré jeho hemové železo palladiem. Předpokládejte, že lidské tělo
obsahuje 3,581 g hemového železa.
\end{quotation} \dotfill \par 
Celková hmotnost hemového železa je 3,581 g. Hmotnost palladia zjistíme z~látkového množství, které bude stejné jako celkové látkové množství železa v~těle. Po nahrazení veškerého
hemového železa palladiem se hmotnost Honzy změní pouze o~rozdíl hmotnosti
palladia a železa.
\[
n_{\mathrm{Pd}}=\frac{m_{\mathrm{Fe}}}{M_{\mathrm{Fe}}}
\]
\[
\Delta m=\frac{m_{\mathrm{Fe}}}{M_{\mathrm{Fe}}}\cdot M_{\mathrm{Pd}}-m_{\mathrm{Fe}}=\frac{3,581}{55,845}\cdot106,421-3,581=3,243\,\mathrm{g}
\]


\hrulefill % \subsection*{Ročník 5, úloha č. 2.4 }
\begin{quotation}
\dva Italská kapela \textit{Lacuna Coil }vydala v~roce 2012 album \textit{Dark
Adrenaline}. Molekula z~definice věci temná být nemůže\footnote{Dovolíme si opomenout pojem \textit{temný stav} ve~fotofyzice, popisující atomy a molekuly, které nemohou pohlcovat ani vyzařovat fotony.}, ale udělejme
z ní heavy adrenaline. O~kolik procent se změní hmotnost molekuly
adrenalinu, pokud všechny atomy vodíku v~ní vyměníme za deuterium? 
\begin{center}
\includegraphics{images_new/5/2-4.eps}
\par\end{center}

\end{quotation} \dotfill \par 
Pokud chceme určit, jak se změní molekula hmotnost po nahrazení všech
lehkých atomů vodíku za deuterium, musíme určit strukturu adrenalinu
a jeho původní molekulovou hmotnost. Ze struktury lze poměrně
přímočaře stanovit molekulový vzorec $\ch{C9H13NO3}$, z~něhož pak
jednoduše vypočítáme molární hmotnost jako 
\[
9\cdot12+13\cdot1+1\cdot14+3\cdot16=183\mathrm{\ g\cdot mol^{-1}}
\]
Po ztěžknutí adrenalinu se změní příspěvek každého atomu vodíku na
\SI[inter-unit-product = \ensuremath{{}\cdot{}}]{2}{\gram\per\mole}, což vede k~molární hmotnosti
$9\cdot12+13\cdot\textbf{2}+1\cdot14+3\cdot16=196\mathrm{\ g\cdot mol^{-1}}$.
Výpočtem poměrů či trojčlenkou pak dospějeme k~poměru
hmotností $\frac{196}{183}\cdot100\,\%$, tedy, že
nárůst hmotnosti činí 7,1 \%.


\hrulefill % \subsection*{Ročník 1, úloha č. 3.5}
\begin{quotation}
\jeden Příroda je plná rozmanitých izotopů nejrůznějších atomů. S~některými izotopy se spíše než v~přírodě, kde je jejich zastoupení velmi malé, setkáte spíše v~laboratoři, kde slouží vědcům k~provádění různých pokusů.
Jedním z~takových izotopů je deuterium -- atom vodíku, který má v~jádře kromě protonu ještě neutron. Tahle drobná změna může mít vliv na různé fyzikální, ale i chemické vlastnosti. Co třeba plně deuterovaný led? Na rozměrech mřížky se výměna vodíku $\mathrm{^1 H}$ za deuterium neprojeví, led však bude jinak hustý (bude mít jinou hustotu). Na vás je tuto hustotu spočítat. Hustota lehkého ledu (kde všechny atomy vodíku tvoří izotop $\mathrm{^1 H}$) činí \SI[inter-unit-product = \ensuremath{{}\cdot{}}]{917}{\kg\per\cubic\metre}. 
\end{quotation} \dotfill \par 
Jelikož obsah jiného izotopu nemění mřížkové parametry, dá se hustota
těžkého ledu vypočítat například i následující trojčlenkou, ve které vystupují
molární hmotnosti a hustoty. Poměr hustot je stejný jako poměr molárních
hmotností.

18,02  $\mathrm{g\cdot mol^{-1}}$ …………. 917 $\mathrm{kg\cdot m^{-3}}$ 

20,02 $\mathrm{g\cdot mol^{-1}}$ …………… x  $\mathrm{kg\cdot m^{-3}}$ 

\smallskip{}
Hustota těžkého ledu je tedy \SI[inter-unit-product =\ensuremath{{}\cdot{}}]{1019}{\kg\per\cubic\metre}.
\newpage %%

% \subsection*{Ročník 3, úloha č. 0.25 }
\begin{quotation}
\jeden Běžně jsme zvyklí udávat krevní hladinu glukosy v~$\mathrm{mmol\cdot l^{-1}}$. V~anglofonních
zemích preferují jiné, mnohdy mnohem komplikovanější soustavy jednotek,
např. v~USA se pro udávání „cukru v~krvi“ používá jednotka $\mathrm{mg\cdot dl^{-1}}$.
Za normální hodnotu se v~českém prostředí považuje koncentrace glukosy
3,9 až 5,6~$\mathrm{mmol\cdot l^{-1}}$. Vyjádřete tento údaj v~obvyklých amerických jednotkách. 
\end{quotation} \dotfill \par 
Určíme molární hmotnost glukosy: $M_{\mathrm{C_{6}H_{12}O_{6}}}=180\,\mathrm{g\cdot mol^{-1}}$
a převedeme dolní hranici na jednotky $\mathrm{mmol\cdot l^{-1}}$. $c_{1}=3,9\cdot10^{-3}\,\mathrm{mol\cdot dm^{-3}}$.
Tuto molární koncentraci následně převedeme na hmotnostní:
\[
m=c\cdot M
\]
tedy
\[
m_{1}=3,9\cdot10^{-3}\cdot180=0,702\,\mathrm{g\cdot l^{-1}}
\]
Poté přepočteme údaj, aby odpovídal americkým jednotkám: 
\[
m_{1}=70,2\,\mathrm{mg\cdot dl^{-1}}
\]
Pro výpočet druhé hraniční hodnoty lze využít analogický postup jako
pro dolní hraniční hodnotu nebo lze použít trojčlenku. Výsledek je poté $m_{2}=100,8\,\mathrm{mg\cdot dl^{-1}}$.

\section{Stavba atomu}

% \subsection*{Ročník 5, úloha č. 0.11  }
\begin{quotation}
\jeden Mnoho zábavy si nejen ve škole užijete už s~nejlehčím prvkem, vodíkem.
Je nejen mimořádně lehký, hořlavý či explozivní, ale v~celém vesmíru
naprosto všudypřítomný. Proto jej potkáte v~různých podobách v~celém
Chemiklání, třeba jako součást celulosy, na kterou píšete své odpovědi,
nebo ve vodě, kterou budete později při řešení potit. Vodík je ve
skutečnosti směsí tří nuklidů, lehkého vodíku, deuteria a tritia.
Rozhodněte, která tvrzení popisují který nuklid, a označte je jejich
nuklidem (P za lehký vodík neboli protium, D za deuterium a T za tritium).
Správně je vždy pouze jedna možnost. 
\begin{enumerate}
\item Tento nuklid má ve svém jádře právě jeden neutron. 
\item Tento nuklid je nejtěžší. 
\item Tento nuklid je nejběžnější. 
\item Tento nuklid je radioaktivní. 
\item Tento nuklid je nejlehčí. 
\end{enumerate}
\end{quotation} \dotfill \par 
Protium má jádro tvořeno protonem, deuterium protonem a neutronem
a tritium protonem a dvěma neutrony. Proto je zjevně správné přiřadit
1D, 2T a 5P. Nejběžnější je lehký vodík -- protium,
takže 3P. Radioaktivní je nejtěžší z~nuklidů, tritium, tedy 4T.

\hrulefill % \subsection*{Ročník 3, úloha č. 0.14 }
\begin{quotation}
\jeden Chemie je věda o~valenčních elektronech, tedy těch, které se mohou
účastnit chemické vazby. Celkem často se stává, že jsou elektrony
v rámci jedné vazby sdíleny mezi dvěma partnery nerovnoměrně: jeden
si jich užívá víc než druhý. Označte v~následujících molekulách atom
s největším částečným kladným nábojem.
\begin{center}

\includegraphics{/3/0-14}

\par\end{center}

\end{quotation} \dotfill \par 
V prvním případě se jedná o~atom vodíku, vazba O--H je jediná polární
vazba v~molekule. V~druhém případě se jedná o~hořčík, který má zdaleka
nejnižší elektronegativitu, navíc je jako jediný atom vázán k~silně
elektronegativnímu bromu. Ve třetím případě je to uhlík, který, podobně
jako v~druhém případě, je z~celé molekuly nejméně elektronegativní.

\newpage%%

 % \subsection*{Ročník 3, úloha č. 0.15   }
\begin{quotation}
\jeden Na scéně alternativní medicíny se v~posledních letech objevují různé
způsoby, které by měly obyčejnou vodu nějak „harmonizovat“, „tachyonizovat“,
nebo například „naspinovat“. Všechny tyto metody jsou samozřejmě založené
na lžích, které mají za úkol hlavně vyždímat peněženky důvěřivých
zákazníků. Kdyby však autoři poslední zmíněné metody měli na mysli
elektronový spin, bylo by vhodné vědět, kolik elektronů molekula vody
vlastně obsahuje. Určete celkový počet elektronů v~molekule vody. 
\end{quotation} \dotfill \par 
Molekula vody obsahuje dva atomy vodíku a jeden atom kyslíku. Vodíky
mají každý po jednom elektronu, kyslík má celkem 8 elektronů (z toho
šest v~nejvyšší, valenční vrstvě). Celkem je tedy v~molekule vody
elektronů 10.

\hrulefill % \subsection*{Ročník 2, úloha č. 1.4}
\begin{quotation}
\jeden Již ve třetím tisíciletí před naším letopočtem se Chetité naučili
vyrábět primitivní železné nástroje, čímž získali při boji s~dalšími
národy velkou výhodu: železné zbraně byly totiž oproti bronzovým výrazně
odolnější. Nejsme ale vojáci, nýbrž chemici, tak se na železo podíváme
z chemického pohledu. Zapište elektronovou konfiguraci železitého
kationtu. Můžete využít zkráceného zápisu pomocí elektronové konfigurace
vzácného plynu. 
\end{quotation} \dotfill \par 
Elektronová konfigurace železitého kationtu je následující:

\noindent \begin{center}
\[
\mathrm{Fe^{3+}:\;[Ar]\,4s^{0}\,3d^{5}}
\]
\par\end{center}

Elektrony se všechny přesunou do orbitalu $d$, ten pak bude právě z\ půlky
zaplněný a bude tak stabilní.

\hrulefill % \subsection*{Ročník 1, úloha č. 5.5}
\begin{quotation}
\tri Není bez zajímavosti, že svět atomů má rád sudá čísla více než lichá.
Přesto však existují výjimky. Počet protonů a neutronů v~jádře je
toho příkladem. V~drtivé většině případů je stabilní izotop ten, co
má sudý počet alespoň jednoho druhu částic. A teď k~těm výjimkám:
Napište všechny čtyři prvky, jejichž stabilní izotopy mají lichý počet
protonů i lichý počet neutronů.
\end{quotation} \dotfill \par 
Licho-lichá jádra: \ce{^{2}_{1}H}, \ce{^{6}_{3}Li}, \ce{^{10}_{5}B}, \ce{^{14}_{7}N}.


\section{Jaderné reakce, radioaktivní rozpady}

% \subsection*{Ročník 5, úloha č. 0.24   }
\begin{quotation}
\jeden Možná si ještě vzpomenete, že pár stránek nazpět se v~této sbírce objevila úloha o~izotopech vodíku. Nejtěžší
z těchto izotopů se samovolně rozkládá na radioaktivní helium dle následující
rovnice s~poločasem rozpadu 12,3 roku:
\shorthandoff{-}
\begin{equation*}
\ch{{^{3}_{1}H} -> {^{3}_{2}He} + X}
\end{equation*}
\shorthandon{-}
Určete částici X.
\end{quotation} \dotfill \par 
Aby platila rovnost nukleonového i protonového čísla na obou stranách
rovnice, musí být částice $_{-1}^{\phantom{-}0}$X, tzn. neobsahuje
proton ani neutron a má náboj --1. Správnou odpovědí je elektron.
Na stejnou odpověď přijdeme i uvědoměním si, že se jedná o~$\upbeta\textsuperscript{{-}}$ rozpad\footnote{Při této přeměně vzniká i elektronové antineutrino. Pro jednoduchost byla tato skutečnost v~úloze zatajena.}.

\hrulefill % \subsection*{Ročník 3, úloha č. 3.1   }
\begin{quotation}
\jeden Při výbuchu uranové atomové bomby dochází například k~následující
jaderné reakci: neutron se srazí s~jádrem $^{235}$U, přičemž vzniknou
3 neutrony, atom $^{93}$Kr a ještě jedna částice. Určete tuto částici
včetně nukleonového čísla.
\end{quotation} \dotfill \par 
Pro řešení této úlohy je třeba si uvědomit, že při jaderných reakcích
platí zákon zachování protonového i nukleonového čísla. Zapsaná rovnice
popsané přeměny vypadá následovně:
\shorthandoff{-}
\begin{equation*}
\ch{{^{1}_{0}n} + {^{235}_{92}U} -> 3 {^{1}_{0}n} + {^{93}_{36}Kr} + {^{140}_{56}Ba}}
\end{equation*}
\shorthandon{-}
Hledanou částicí je tedy atom barya \ch{^{140}_{56}Ba}.
\newpage %%

% \subsection*{Ročník 5, úloha č. 3.2}
\begin{quotation}
\dva Největší test vodíkové bomby provedený armádou Spojených států byl
uskutečněn roku 1954 na atolu Bikini na Marshallových ostrovech. Bomba
byla založena na principu tzv. Jetterova cyklu: 
\begin{center}
\includegraphics{/5/3-2} 
\par\end{center}
Jako palivo byl použit deuterid lithný s~obsahem 35~\% \ch{^{6}Li}
(zbytek tvořil izotop \ch{^{7}Li}). Výsledná exploze byla přibližně třikrát silnější, než bylo plánováno, což způsobilo značné nepříjemnosti jednotkám pověřeným tímto experimentem. \ch{^{7}Li} se totiž rovněž účastnil jaderné reakce podle rovnice níže, což velmi zesílilo
explozi. 
\shorthandoff{-}
\begin{equation*}
\ch{{^{7}Li} + {^{1}n} -> X + Y + {{^{1}n}}}
\end{equation*}
\shorthandon{-}
Napište, co se v~cyklu a rovnici skrývá pod písmeny \textbf{X} a \textbf{Y}.
\end{quotation} \dotfill \par 

Pro určení identity částic \textbf{X} a \textbf{Y} je třeba na základě diagramu vypsat další děje, kterých se tyto částice účastní:

\shorthandoff{-}
\begin{align*}
\ch{{^{2}D} + Y &-> {{^{5}He}}}\\
\ch{{{^{5}He}} &-> X + {{^{1}n}}}
\end{align*} 
\shorthandon{-} 

V těchto rovnicích jaderných reakcí platí zachování protonového a nukleonového čísla. Součet protonového a nukleonového čísla u~reaktantů je tak roven jejich součtu u~produktů. Na základě toho jsme schopní identifikovat neznámé částice: \textbf{X} = \ch{{^{4}He}}; \textbf{Y} = $^3$T = \ch{{^{3}H}}

Můžeme ověřit, že rovnost je splněna pro původní děj ze zadání:
\shorthandoff{-}
\[
\ch{{^{7}_{3}Li} + {^{1}_{0}n} -> {^{4}_{2}He} + {^{3}_{1}H} + {^{1}_{0}n}}
\]
\shorthandon{-} 

\hrulefill % \subsection*{Ročník 2, úloha č. 3.4}
\begin{quotation}
\dva Radiokarbonová metoda datace je založena na měření obsahu izotopu uhlíku ve vzorcích biologického původu. Radioaktivní izotop zvaný radiouhlík vzniká v~atmosféře srážkami neutronů z~kosmického záření s~atomy dusíku: 
\shorthandoff{-}
\begin{equation*}
\ch{{^{14}_{7}N} + {^{1}_{0}n} -> {^{14}_{6}C} + {^{1}_{1}p}}
\end{equation*}
\shorthandon{-} 

\noindent Uhlík \ch{^{14}_{6}C} je ve formě oxidu „radiouhličitého“ vstřebáván
rostlinami a stává se tak součástí biosféry. Jelikož živý organismus
radiouhlík průběžně přijímá z~okolního prostředí, lze jeho obsah v~živé hmotě považovat za konstantní. Od okamžiku, kdy organismus přestane
vyměňovat látky s~okolním prostředím, klesá relativní obsah radiouhlíku
v mrtvých tkáních v~důsledku radioaktivní přeměny, konkrétně $\upbeta-$rozpadu
spojeného s~uvolněním elektronového antineutrina:
\shorthandoff{-}
\begin{center}
\ch{{^{14}_{6}C} -> {^{14}_{7}N} + {^{0}_{-1}e^{-}} +  $\bar{\nu}_{\mathrm{e}}$}
\end{center}
\shorthandon{-}
Rozpad jader je typickým dějem prvního řádu, rychlost poklesu obsahu izotopu je tedy přímo úměrná jeho okamžitému množství. Závislost počtu jader na čase popisuje exponenciální vztah
\[
N(t)=N_{0}\cdot e^{-\frac{t\cdot\ln2}{\tau_{\nicefrac{1}{2}}}}
\]

kde $N_{0}$ je počet radioaktivnách jader na počátku a $\tau_{1/2}$
je poločas rozpadu daného izotopu (pro \ch{^{14}_{6}C} 5730
let). Vaším úkolem je rozhodnout a doložit výpočtem, zda kousek kosti
nalezený ve slovanském hradišti v~Mikulčicích mohl patřit svatému
Metodějovi, víte-li, že relativní obsah radiouhlíku byl stanoven jako
89,435 \% hodnoty v~živé hmotě. Jako výsledek uveďte také počet let,
který uplynul od smrti zkoumaného organismu.
\end{quotation} \dotfill \par 

\newpage %text%
Ze zadání plyne, že okamžitá hodnota $N(t)=0,89435\cdot N_{0}$. Dosazením
do exponenciálního vztahu dostaneme rovnici s~jednou neznámou $t$:
\[
0,89435=e^{-\frac{t\cdot\ln2}{\tau_{\nicefrac{1}{2}}}}
\]

Postupnými úpravami

\[
\ln(0,89435)=-\frac{t\cdot\ln2}{\tau_{\nicefrac{1}{2}}}
\]

dojdeme k~výsledku: 

\[
t=-\ln(0,89435)\cdot\frac{\tau_{\nicefrac{1}{2}}}{\ln2}=+0,11166\cdot\frac{5730}{0,69315}=923\,\mathrm{let}
\]

Od okamžiku, kdy zkoumaný organismus přestal vyměňovat látky s\ okolím,
tedy uplynulo asi 923 let. Vzhledem k~tomu, že se jako datum úmrtí
Metoděje ze Soluně nejčastěji uvádí 6. duben 885, můžeme s~jistotou
říci, že zkoumané ostatky nepatří tomuto významnému moravskému světci,
ale spíše nějakému současníkovi přemyslovského knížete Břetislava
II. 

\hrulefill % \subsection*{Ročník 5, úloha č. 3.4 }
\begin{quotation}
\dva Stáří vzorků vody či vodných roztoků se dá zjistit stanovením obsahu
radioaktivního tritia, které přirozeně vzniká v~přírodě působením
kosmického záření. Vypočítejte stáří vzorku vody, která má oproti
běžné čisté vodě 10\texttimes{} menší radioaktivitu. Předpokládejte,
že tam, kde byl vzorek uložen, nepronikalo žádné ionizující kosmické
záření a že na počátku měla voda běžnou koncentraci tritia. Poločas
rozpadu tritia je 12,32 roků.
\end{quotation} \dotfill \par 
K výpočtu se nejlépe hodí známý rozpadový zákon: 
\[
N=N_{0}\cdot\mathrm{e}^{-\lambda t}
\]
Aktivita je přímo úměrná počtu jader ($A=\lambda\cdot N$), čímž dosazením
dostáváme vztah 

\[
A=A_{0}\cdot\mathrm{e}^{-\lambda t} 
\]

Je zadáno, že aktivita našeho vzorku je desetinová, což lze vyjádřit
takto 
\[
A=0,1\cdot A_{0}=A_{0}\cdot\mathrm{e}^{-\lambda t}
\]
Po pokrácení $A_{0}$ vyjádříme čas jako 
\[
t=-\frac{\ln0,1}{\lambda}
\]
Vztah mezi konstantou $\lambda$ a poločasem rozpadu $t_{1/2}$ lze
najít v~tabulkách či odvodit podobně dosazením\\ 
$N=0,5N_{0}$ do rozpadového zákona.
\[
\lambda=\frac{\ln2}{t_{1/2}}
\]
Kombinací dvou předchozích vztahů dostaneme
\[
t= - \frac{\ln 0,1}{\ln2}\cdot t_{1/2}=-\frac{\ln 0,1}{\ln 2} \cdot 12,32=40,9 \, \mathrm{let.}
\]

\hrulefill % \subsection*{Ročník 4, úloha č. 5.3 }
\begin{quotation}
\tri Nepochybně je vám, řešitelům, známý koncept tzv. nejdůležitější odpovědi,
42. V~letošním ročníku vám onu klíčovou otázku ještě neodhalíme, můžeme
však poukázat na některé zajímavé souvislosti tohoto čísla. Počáteční
aktivita vesmírného radioaktivního vzorku je 42\,TBq, ten obsahuje
7,7 mmol částic. Jedovatý plyn se do krabice se živou kočkou vypustí,
až se rozpadne 60 \% původních částic. Spočítejte nejpravděpodobnější
datum smrti kočky, pokud tento plyn kočku zabije okamžitě. 

Pokus začal 12. 8. 1887\footnote{V tento den se narodil Erwin Schrödinger.}.
\end{quotation} \dotfill \par 

\newpage %text%
Aktivita vzorku je definovaná jako počet proběhnuvších rozpadů ve
vzorku za jednu sekundu. Z~ní můžeme vypočítat snadno rozpadovou konstantu
$\lambda$, postačí aktivitu vydělit množstvím částic ve vzorku, množství pak určíme vynásobením látkového množství částic Avogadrovou konstantou.

\[
\lambda = \frac{A}{n\cdot N_\mathrm{A}}
\]

\[
\lambda=\frac{42\cdot10^{12}}{7,7\cdot10^{-3}\cdot6,022\cdot10^{23}}=9,0577\cdot10^{-9}\,\mathrm{s^{-1}}
\]


Tuto konstantu poté dosadíme do rovnice rozpadového zákona:

\[
N=N_{0}\cdot e^{-\lambda t}
\]

\[
t=\frac{-\ln\frac{N}{N_{0}}}{\lambda}=\frac{-\ln0,4}{9,0577\cdot10^{-9}}=1,0116\cdot10^{8}\,\mathrm{s}
\]
Jeden den se skládá z~$24\cdot3600=86400$~s. Vydělením spočítaného
času touto hodnotou zjistíme počet dní, které uplynuly od začátku
pokusu:

\[
1,0116\cdot10^{8}\,\mathrm{s}=1170,85\doteq1171\,\mathrm{dn\mathring{u}}
\]

Teď už jen zbývá dopočítat, jaké to bude datum. Nezapomeňme, že rok
1888 je přestupný\footnote{O den odlišné odpovědi byly při soutěži tolerovány.}!

Poté nám vyjde, že pokus trval tři roky a 75 dní: datum nejpravděpodobnější
smrti je tedy 26. říjen 1890.

\hrulefill
% \subsection*{Ročník 3, úloha č. 4.2  }
\begin{quotation}
\dva Jedna sekunda je dle soustavy SI definována jako „doba, za niž dojde
k 9 192 631 770 přechodům mezi dvěma hyperjemnými hladinami základního
stavu atomu $^{133}$Cs“. Tyto chemické stopky měří naši současnost,
avšak zkoumání chemického složení nám dovoluje poodhalit i minulost.
Existuje několik známých i méně známých datovacích metod, kterými
lze s~chronometrem nahlédnout i do dob dávno minulých, z~nichž pětici
vám představujeme níže. Přiřaďte k~sobě správné dvojice metody a popisu. 
\begin{center}
\begin{tabular}{ m{6cm} | m{9cm} }
\textbf{1)} Radiouhlíková metoda & \textbf{A)} Metoda používaná k~určování stáří vody nebo vín. Její přesnost
v minulém století narušily pokusy při vývoji vodíkové bomby.  \\ \hline
\textbf{2)} Samarium-neodymová metoda & \textbf{B)} Metoda založená na jedné z~rozpadových řad, v~minulosti výrazně posunula tehdy známou hodnotu stáří Země. \\ \hline
\textbf{3)} Tritiová metoda & \textbf{C)} Metoda využívaná k~určování stáří neporézních hornin či minerálů. \\ \hline
\textbf{4)} Draslíko-argonová metoda & \textbf{D)} Archeologicky nejvyužívanější z~radiodatovacích metod. \\ \hline
\textbf{5)} Metoda uran-olovo & \textbf{E)} Z~výše uvedených metod má nejhlubší dosah do minulosti, její použití je ale omezené kvůli nízkému přirozenému výskytu zúčastněných species. 
\end{tabular}
\end{center}
\end{quotation} \dotfill \par 
Jsou známé čtyři rozpadové řady, jejichž konečným členem je vždy některý
z izotopů olova. Proto můžeme přiřadit \textbf{5B}. Ve vodě i víně je poměrně
hojně obsažen vodík, v~menším množství i deuterium a tritium. To lze
spojit se zmínkou o~vodíkové bombě, dalším přiřazením tedy bude \textbf{3A}.
Ze zbylých tří metod je zdaleka nejznámější metoda radiouhlíková,
která se používá pro určení stáří kosterních pozůstatků, dřeva a jiných
organických materiálů -- zjevně \textbf{1D}. Co zmínka o~neporézních horninách?
Ta souvisí s~metodou draslíko-argonovou, protože rozpadem izotopu
$^{40}$K vzniká ve 12\% případů argon, který je uvězněn ve vzniklé
dutince v~minerálu. Jeho únik by dataci znemožnil. Proto \textbf{4C}. Zbývá
přiřazení \textbf{2E}, také proto, že samarium a neodym jsou prvky vskutku
vzácnými. 

\newpage %nadpis%
\section{Kyselost, pH}

 % \subsection*{Ročník 5, úloha č. 0.22}
\begin{quotation}
\dva Homeopatie je pseudovědecká metoda založená na ředění látky až do
tak malých koncentrací, že v~,,účinném`` přípravku není přítomna
ani jedna molekula účinné látky. Biochemik Jardík chtěl připravit
homeopatickou kyselinu chlorovodíkovou. Připravil si 0,01M roztok
HCl a změřil pH = 2,0. Tento roztok 100\texttimes{} naředil a změřil
pH = 4,0. Opět 100\texttimes{} zředil a změřil pH = 6,0. Jaké pH Jardík
změřil po dalším stonásobném zředění?
\end{quotation} \dotfill \par 
Výpočtem za použití běžně užívaných vzorců pro silné kyseliny dojdeme ke zdánlivému výsledku, že pH takto
zředěného roztoku HCl by mělo být 8,0. To ale není v~souladu s~principem,
že kyseliny mají při standardní teplotě pH v~rozmezí 0--7. Náš roztok je tedy až tak zředěný,
že středoškolský vzorec 
\[
\pH=-\log[\ch{H+}]\approx-\log\bigl(c_{\ch{HCl}}\bigr)
\]
neplatí, takže jej můžeme považovat\footnote{Pro silně zředěné roztoky nesmíme zanedbat iontový součin vody. Silná
kyselina zcela disociuje a roztok musí být elektroneutrální. Sestavíme
si rovnice: 
\[
c _{\mathrm{HCl}}  ={[}\mathrm{Cl^{-}}{]} 
\]
\[ {[}\mathrm{H^+ }{]}{[}\mathrm{OH^{-}}{]} =10^{-14}
\]
\[
{[}\mathrm{H^+ }{]} ={[}\mathrm{OH^{-}}{]}+{[}\ch{Cl^{-}}{]}
\]
Dosazením první a druhé rovnice do rovnice třetí po algebraických úpravách
získáme kvadratickou rovnici: 
\[
 {[}\mathrm{H^+ }{]}^{2}-c_{\mathrm{HCl}}\cdot {[}\mathrm{H^+ }{]}-10^{-14}=0
\]

 Řešením a zlogaritmováním kladného výsledku získáme pH = 6,98. Pro
potřeby soutěže byly uznávány hodnoty mezi 6,9 -- 7,0.} za destilovanou vodu s~pH = 7,0.

\hrulefill
% \subsection*{Ročník 1, úloha č. 6.2}
\begin{quotation}
\dva Kombinace neúměrného pití alkoholu a sportu může mít devastující účinky.
Představte si (samozřejmě čistě hypoteticky), že jeden z~autorů jednoho
večera přebral a byl domluven, že následujícího dne má jít ráno plavat.
Bohužel tento svůj slib nedodržel. Představme si, že by se autor přemohl
a šel si zaplavat, avšak tato pohybová aktivita by mu neudělala dobře.
Jak se změní pH bazénu, pokud do něj vyzvracíme 200 ml roztoku HCl
o pH = 2? Uvažte, že zvracený roztok je pouze směs HCl a vody. Bazén
přitom uvažujte jako nádobu o~rozměrech $50\times25~\mathrm{m}$ naplněnou do výše
2,5~m destilovanou vodou. pH uveďte v~přesnosti na jedno desetinné
místo. 
\end{quotation} \dotfill \par 
Bazén s~obsahem zvratků představuje mohutně zředěný roztok
kyseliny chlorovodíkové. Při výpočtu za použití aproximace vycházející
z předpokladu, že veškeré protony v~roztoku pocházejí z~disociace
kyseliny, dostaneme pH~>~7, tj. v~alkalické oblasti, což je zjevný nesmysl.
V tomto případě totiž nelze zanedbat autoprotolýzu vody, tedy koncentraci
protonů vzniklých disociací \shorthandoff{-}
\ch{2 H2O -> H3O^{+} + OH^{-}}
\shorthandon{-}.
Výsledné pH tak bez složitějšího výpočtu můžeme pokládat za rovné
7,0 (přesným výpočtem dostaneme hodnotu 6,9988). 

\hrulefill % \subsection*{Ročník 4, úloha č. 0.24 }
\begin{quotation}
\jeden Kyselina mléčná vzniká z~pyruvátu\footnote{Pyruvát je tříuhlíkatý anion, meziprodukt spalování cukrů v~našem těle.} ve svalech při vysoké námaze (třeba při sprintu). Ačkoliv jde o~kyselinu spíše slabou, v~těle se vyskytuje
téměř výhradně ve formě aniontu. Přesvědčte nás o~tom výpočtem 
poměru látkových množství deprotonované a protonované formy kyseliny mléčné při fyziologickém
pH 7,4. Hendersonova-Hasselbalchova rovnice je zadaná takto:

\[
\mathrm{pH=p}K_{\mathrm{A}}-\log\frac{c_{\mathrm{HA}}}{c_{\mathrm{A^{-}}}}
\]
$\mathrm{p}K_{\mathrm{A}}(\mathrm{kyselina\,ml\acute{e}\check{c}n\acute{a}})=3,86$
\end{quotation} \dotfill \par 
Po dosazení do Hendersonovy-Hasselbalchovy rovnice dostaneme vztah:
\[
\mathrm{pH}=\mathrm{p}K_{\mathrm{A}}-\log\frac{c_{\mathrm{HA}}}{c_{\mathrm{A^{-}}}}
\]
 
\[
7,4=3,86-\log\frac{c_{\mathrm{HA}}}{c_{\mathrm{A^{-}}}}
\]
Po úpravě vztahu dostaneme: 
\[
\frac{c_{\mathrm{HA}}}{c_{\mathrm{A^{-}}}}=10^{-3,54}=2,884\cdot10^{-4}
\]
 Po převrácení zlomku dostáváme: 
\[
\frac{c_{A^{-}}}{c_{HA}}=\frac{1}{2,884\cdot10^{-4}}=3467
\]
Na jednu molekulu protonované formy tedy připadá 3467 molekul deprotonované
formy.

\hrulefill % \subsection*{Ročník 3, úloha č. 2.2   }
\begin{quotation}
\jeden Baryum je prvek, jehož rozpustné sloučeniny jsou toxické. Jednou z
mála nerozpustných sloučenin je síran barnatý, který se používá jako
kontrastní látka při rentgenovém vyšetření trávicího traktu: baryum
jakožto atom s~velkým poloměrem dobře pohlcuje rentgenové záření.
Mezi ty rozpustné (a tedy jedovaté) sloučeniny patří například hydroxid
barnatý. Vypočítejte, kolik Ba(OH)$_{2}$ musíme navážit pro přípravu
250 ml roztoku o~pH 11. Uvažujte úplnou disociaci hydroxidu barnatého.
\end{quotation} \dotfill \par 
Hydroxid barnatý je silný hydroxid, ve vodě zcela disociuje podle
rovnice

\shorthandoff{-}
\begin{center}
\ch{Ba(OH)2 -> Ba^{2+} + 2 OH^{-}}\\
\end{center}
\shorthandon{-}
A protože z~rovnice iontového součinu vody
při 25 °C vyplývá $\mathrm{pH+pOH=14}$, dá se potřebná koncentrace hydroxylových aniontů vypočítat jako

\[
\mathrm{[OH^{-}]=10^{14-pH}}=0,001\,\mathrm{mol\cdot dm^{-3}}
\]
Jelikož se z~každé molekuly hydroxidu barnatého uvolní dva hydroxidové anionty, bude koncentrace hydroxidu dvakrát menší (poloviční).

\[
c_{\ch{Ba(OH)2}}=\frac{1}{2}[ \,\ch{OH^{-}}] \,=5\cdot10^{-4}\,\mathrm{mol\cdot dm^{-3}}
\]

Následně vynásobíme koncentraci objemem roztoku a molární hmotností hydroxidu, čímž dojdeme k~výsledku.

\[
m=c\cdot V\cdot M=0,0005\cdot0,25\cdot171,34=0,0214\,\mathrm{g=21,4\,mg}
\]


\hrulefill % \subsection*{Ročník 2, úloha č. 4.2}
\begin{quotation}
\tri Soli slabých kyselin a silných zásad vykazují mírně zásaditou reakci.
Výpočet přesného pH není ale zas tak triviální záležitostí, jak by
se na první pohled mohlo zdát. Jaká musí být koncentrace octanu vápenatého
v roztoku, aby jeho pH bylo stejné jako pH roztoku citronanu trisodného
o koncentraci 0,01 mol/l? Potřebné údaje:
$\pKa~\mathrm{(k.~octov\acute{a})=4,76}$,
$\pKa[3]\mathrm{(k.~citronov\acute{a})=6,40}$. 
\end{quotation} \dotfill \par 
pOH slabých zásad se s~uspokojivou přesností spočítá následujícím
vztahem:

\[
\pOH=\frac{1}{2}\left(\pKb-\log c_{\mathrm{z\acute{a}sada}}\right)
\]

Pokud vezmeme v~úvahu, že $\ch{\pOH}=14-\pH$ a že $\ch{\pKb}=14-\mathrm{p}K_{a}$
($\mathrm{p}K_{a}$ konjugované kyseliny), lze psát

\[
14-\pH=\frac{1}{2}\left(14-\pKa-\log c_{\mathrm{z\acute{a}sada}}\right)
\]
Po matematických úpravách dostaneme vztah
\[
\pH=7+\frac{1}{2}\left(\pKa+\log c_{\mathrm{z\acute{a}sada}}\right)
\]

Tento vztah je univerzální a lze pomocí něho spočítat celý příklad.
Nejprve si pomocí něho spočítáme pH roztoku citronanu: 

\[
\pH=7+\frac{1}{2}\left(6,4+\log0,01\right)=9,2
\]

\newpage %text% 
Tuto hodnotu pH dosadíme do vztahu pro pH octanu: 

\[
\mathrm{9,2}=7+\frac{1}{2}\left(4,76+\log c_{\mathrm{octan}}\right)
\]

Po vyřešení této exponenciální rovnice vyjde $c_{\mathrm{ octan}}=0,437\,\mathrm{mol\cdot dm^{-3}}$.
Jedna molekula octanu vápenatého ovšem uvolňuje dva octanové anionty,
výslednou koncentraci je proto třeba ještě podělit dvěma. Pak vyjde
\[
c_{\mathrm{Ca(OAc)_{2}}}=0,219\,\mathrm{mol\cdot dm^{-3}}
\]



\section{Rozpustnost}

 % \subsection*{Ročník 3, úloha č. 1.2 }
\begin{quotation}
\dva Thallium tvoří zajímavé komplexy, třeba Tl$_{3}${[}TlCl$_{6}${]}.
Tento komplex krystaluje v~hexagonální soustavě, tvoří žluté třpytivé
krystaly, které jsou poněkud málo rozpustné ve vodě: při 25 °C se
v litru vody rozpustí pouze 0,0033 molu tohoto komplexu\footnote{REMY, Heinrich. Anorganická chemie, I. díl, s. 394, \textit{vlevo
dole}. Vydalo SNTL v~Praze roku 1961. Údaje byly zjištěny během hledání
článku Ferdinanda Peroutky ,,Hitler je gentleman,`` autor může potvrdit,
že ten se v~Remyho učebnici opravdu nenachází.}. Určete, kolik gramů komplexu budu potřebovat k~přípravě čtyř tekutých
uncí (fl oz) roztoku nasyceného při 25 °C. Uvažujte 1 ml = 1/30~fl~oz.

$A_{r}(\mathrm{Tl})=204,38$; $A_{r}(\mathrm{Cl})=35,45$.
\end{quotation} \dotfill \par 
Relativní molekulovou hmotnost daného komplexu spočítáme jako součet
zadaných hmotností jednotlivých atomů v~komplexu.

\[
M_{\mathrm{r}}=4\cdot A_\mathrm{r}(\mathrm{Tl})+6\cdot A_{r}(\mathrm{Cl})=1030,22
\]

Dále je třeba spočítat objem potřebného roztoku. Dle zadání $1\,\mathrm{fl\,oz=30\,ml}$.
Požadované 4 tekuté unce tedy odpovídají 120 mililitrům. Potřebné
množství komplexu na roztok zjistíme takto ($c$ je zadaná koncentrace
nasyceného roztoku):

\[
n=c\cdot V=0,0033\cdot0,12=3,96\cdot10^{-4}\,\mathrm{mol}
\]

Relativní molekulová hmotnost je číselně shodná s~molární hmotností
v jednotkách $\mathrm{g\cdot mol^{-1}}$. Následný výpočet hmotnosti
je už triviální:

\[
m=n\cdot M=3,96\cdot10^{-4}\cdot1030,22=0,408\,\mathrm{g}
\]


\section{Výpočty z~chemických reakcí}

 % \subsection*{Ročník 5, úloha č. 3.6 }
\begin{quotation}
\dva Triatlon je jedním z~nejnáročnějších sportů, který člověk vůbec může
provozovat. Jeho královská varianta je závod Ironman. Při první fázi
musí sportovec uplavat 3,8~km v~rozbouřených vodách, následně musí
ujet 180 km na kole a po této projížďce ho čeká ještě maratonský běh
dlouhý 42,2 km. Tento závod je velmi náročný fyzicky i psychicky.
Jak moc je však tento závod náročný energeticky?

Enzymy zajišťující pohyb svalů energii získávají hydrolýzou ATP, univerzálního
energetického platidla lidského těla. Jedním z~účastníků loňského
mistrovství světa v~Ironmanu byl i český profesor chemického inženýrství.
Kolik kilogramů ATP spotřebovalo tělo prof. Františka Štěpánka, aby
pokrylo energetické požadavky na zvládnutí triatlonu Ironman na Havaji,
pokud chytré hodinky po závodu ukázaly kalorickou spotřebu 12~500 kcal?
Molární hmotnost molekuly ATP je 507,18 g$\cdot$mol$^{-1}$. (1 cal
= 4,18 J)

Předpokládejte, že energie uvolněná hydrolýzou ATP je daná následující
rovnicí a že tělo využije energii beze zbytku. 

\shorthandoff{-}

\begin{tabular}{ m{6cm} m{6cm} }
 \ch{ATP + H2O -> ADP + P_{i}} & \Delta G^{\circ}=-57~\mathrm{kJ\cdot mol}^{-1}
\end{tabular}

\shorthandon{-}
\end{quotation} 
\dotfill \par 
Počet molů ATP potřebného na pokrytí energetického požadavku vypočítáme
jako energetickou spotřebu v~kilojoulech dělenou volnou energií reakce
štěpení ATP (v kilojoulech na mol): 
\[
n=\frac{12500\cdot4,18}{57}=916,7\mathrm{\ mol}
\]
Celkovou hmotnost ATP vypočteme jako násobek látkového množství ATP
potřebného na syntézu a molární hmotností jednoho molu ATP 
\[
m=916,7\cdot507,18=464\,932\mathrm{\ g}=464,9\mathrm{\ kg}
\]
Tato hodnota několikanásobně převyšuje hmotnost celého sportovce,
což se může zdát na první pohled nemožné, nicméně molekuly ATP se
po štěpení na ADP a fosfát zpětně spojují na ATP a celý kruh se mnohonásobně
opakuje. Druhým důvodem, proč je hmotnost syntetizovaného ATP tak
vysoká, je i relativně vysoká molární hmotnost molekuly ATP.

\hrulefill
% \subsection*{Ročník 4, úloha č. 5.5 }
\begin{quotation}
\tri Dimethylether (DME) se vyrábí kondenzací methanolu na kovovém katalyzátoru.
Methanol přiváděný do reaktoru je produktem katalytické redukce oxidu
uhelnatého a obsahuje 0,004 obj. \% pevných částic katalyzátoru z
předchozího stupně výroby, které se usazují na dně. Určete, jak vysoká
vrstva tohoto nánosu se vytvoří na dně reaktoru pro výrobu DME, který
je nepřetržitě provozován po dobu 5~let s~denní produkci 15 tun
DME. Hustota methanolu je $792\,\mathrm{kg\cdot m^{-3}}$, reaktor
má tvar svisle stojícího válce o~vnitřním průměru 5 metrů a výšce
7 metrů. Uvažujte, že na DME se přemění 75 \% přivedeného methanolu;
vrstva nánosu není upěchovaná a je ze 40 \% tvořena mezerami a s~produktem
neodcházejí žádné pevné částice.
\end{quotation} \dotfill \par 

Základem výpočtu je celková produkce DME: při zadané době provozu
5 let (tj. 1826 dní) a denní produkci $\dot{m}=15\ \mathrm{t\cdot den^{-1}}$
se vyrobí celkem $m=27394\,\mathrm{t}\cong2,74\cdot10^{7}\mathrm{\,kg}$
DME, což je $5,95\cdot10^{8}\,\mathrm{mol}$.

Podle stechiometrie reakce
\[
\ce{2CH3OH} \rightarrow \ce{CH3OCH3 + H2O}
\]
je třeba přivést dvojnásobné látkové množství methanolu:
\[
n_{\mathrm{MeOH}}^{\mathrm{teor}}=2\cdot n_{\mathrm{DME}}=1,19\cdot10^{9}\,\mathrm{mol}
\]
Protože se na DME přemění jen 75 \% přivedeného methanolu, je skutečné
množství větší: 
\[
n_{\mathrm{MeOH}}^{\mathrm{vstup}}=\frac{n_{\mathrm{MeOH}}^{\mathrm{teor}}}{0,75}=1,59\cdot10^{9}\:\mathrm{mol}
\]
což při dané molární hmotnosti představuje $5,08\cdot10^{7}\,\mathrm{kg}$.

Objem methanolu, který je třeba do reaktoru přivést, určíme podělením
hmotnosti hustotou: 
\[
V_{\mathrm{MeOH}}=\frac{m_{\mathrm{MeOH}}}{\rho_{\mathrm{MeOH}}}=\frac{5,08\cdot10^{7}}{792}=6,42\cdot10^{4}\,\mathrm{m}^{3}
\]
 Podle zadání obsahuje přiváděný methanol 0,004 obj. \% pevných částic,
jejich objem je tedy 
\[
V_{\mathrm{\check{c}\acute{a}stice}}=\frac{V_{\mathrm{MeOH}}}{1-0,00004}\cdot0,00004=2,57\, \mathrm{m^{3}}
\]
Samotné částice ale představují jen část vzniklé vrstvy (40 \% tvoří
vzduch), celkový objem vrstvy je větší: 
\[
V_{\mathrm{vrstva}}=\frac{V_{\mathrm{\check{c}\acute{a}stice}}}{1-0,40}=\frac{V_{\mathrm{\check{c}\acute{a}stice}}}{0,60}=4,28\,\mathrm{m^{3}}
\]
Pro určení výšky nánosu je třeba vypočítat obsah dna reaktoru za
zadaného vnitřního průměru: 
\[
S=\pi\frac{d^{2}}{4}=\pi\frac{5^{2}}{4}=19,6\,\mathrm{m^{2}}
\]

\newpage %text%

 Vzniklá vrstva pak bude mít výšku 
\[
h_{\mathrm{vrstva}}=\frac{V_{\mathrm{vrstva}}}{S}=\frac{4,28}{19,6}=0,218\,\mathrm{m}=21,8\,\mathrm{cm}
\]
Pro účely soutěže jsme vzhledem k~možnému řetězení zaokrouhlovacích
chyb uznávali výsledky v~intervalu 20~až~25~cm.

\hrulefill % \subsection*{Ročník 1, úloha č. 6.1}
\begin{quotation}
\ctyri Iontový hydrid o~hmotnosti 4,2 g byl vhozen do vody, se kterou velmi
bouřlivě reagoval. Při reakci se uvolňoval plyn, který měl za laboratorních
podmínek $t = 77\,^{\circ}\mathrm{F}$ a $p = 1~\mathrm{bar}$ objem $V = 1,19~\mathrm{dm}^{3}$. Roztok vykazoval
zjevně zásaditou reakci. Napište vzorec hydridu, uvažujte ideální
chování plynu.
\end{quotation} \dotfill \par 
Reakce iontových hydridů s~vodou vede ke vzniku elementárního
vodíku (synproporcionací hydridového aniontu a atomu vodíku z~molekuly
vody), vedlejším produktem je hydroxid příslušného kovu. Reakce probíhá
v~případě hydridu alkalického kovu podle rovnice
\shorthandoff{-}
\[
\ch{MH + H2O -> H2 + MOH}
\]
\shorthandon{-}
reakce hydridu kovu z~druhé skupiny v~ox. čísle +II je popsána rovnicí
\shorthandoff{-}
\[
\ch{MH2 + 2 H2O -> 2 H2 + M(OH)2}
\]
\shorthandon{-} 
Pro určení, o~jaký hydrid se jedná, potřebujeme znát jeho molární
hmotnost. Protože známe hmotnost hydridu, který byl vhozen do vody,
vyjdeme ze vztahu
\[
M=\frac{m}{n}
\]
Látkové množství hydridu $n$ určíme na základě
úvahy o~stechiometrii reakce, tj. že v~případě hydridu MH se látkové
množství hydridu rovná látkovému množství uvolněného vodíku, v~případě
hydridu \ch{MH2} je rovno polovině množství vodíku. 

Látkové množství vodíku vypočteme ze stavové rovnice ideálního plynu: zadanou teplotu přepočteme
na kelviny\footnote{$T=\frac{5}{9} (t_{\text{°F}} - 32) + 273,15$}, tlak na pascaly, objem na metry krychlové a z~rovnice
vyjádříme látkové množství vodíku
\[
n_{H_2}=\frac{pV}{RT}=\frac{100\,000\cdot 0,00119}{8,314\cdot 298,15}
\]
kde $R$ je molární plynová
konstanta. Látkové množství vodíku vyjde 0,048~mol, čemuž podle výše zmíněného vztahu
\[
M=\frac{m}{n}
\]
odpovídá molární hmotnost 87,49 $\mathrm{g\cdot mol^{-1}}$ za předpokladu,
že se jedná o~hydrid se vzorcem MH, u~kterého je látkové množství
uvolněného vodíku rovno množství hydridu. Pokud by se jednalo o~hydrid MH$_2$ se stechiometrií 2:1, bylo by látkové množství hydridu poloviční (24 mmol) a molární hmotnost 174,98 $\mathrm{g\cdot mol^{-1}}$. 

Nahlédnutím do periodické tabulky prvků snadno zjistíme, že žádný hydrid kovu
z druhé skupiny nemůže mít relativní molekulovou hmotnost větší než
150, tedy se jedná o~hydrid kovu z~první skupiny. Hodnotě relativní
molekulové hmotnosti 87,49 jednoznačně nejlépe odpovídá rubidium (relativní
atomová hmotnost 85,47). Jak ale vysvětlit, že rozdíl relativních
hmotností hydridu a samotného rubidia je dvě jednotky, což by odpovídalo
dvěma atomům vodíku, když má jít o~hydrid se vzorcem MH? Jednoduše:
relativní atomová hmotnost 1,008 odpovídá vodíku v~přírodním izotopovém
složení, kdežto náš hydrid je tvořen těžkým vodíkem (deuteriem) s~relativní hmotností 2,014. Správný\footnote{Jeden tým vypekl během soutěže organizační tým tím, že jako výsledné složení směsi uvedl hydrid se zvýšeným obsahem izotopu rubidia $^{87}\mathrm{Rb}$. Tento výsledek, který je též správný, byl samozřejmě uznán.} vzorec neznámého hydridu je tedy
RbD, \ch{Rb^{2}_{}D} nebo \ch{Rb^{2}_{}H}.

\hrulefill % \subsection*{Ročník 1, úloha č. 6.3 }
\begin{quotation}
\dva Chemie halogenů je chemií pestrou, a to z~mnoha důvodů -- fluor má
díky své elektronegativitě v~tabulce, v~laboratoři i v~nejedné teoretické
příručce vskutku výsadní postavení, zbylé tři halogeny se ale v~reaktivitě
některých svých sloučenin rozhodně mohou s~fluorem měřit. Jeho soused
z nižšího patra, chlor, je většině středoškolských studentů znám především
jako zelenožlutý, potenciálně nebezpečný plyn s~desinfekčními účinky.
Přestože se v~kyslíkatých kyselinách vyskytuje výhradně v~lichých
oxidačních stavech, chemie jeho oxidů je na oxidační stavy bohatší
a možná právě o~to třaskavější. Kupříkladu oxid chloričitý, za běžných
podmínek jantarově oranžový plyn, se používá jako bělidlo či jako
desinfekce pitné vody, protože ve vodě zanechává méně vedlejších produktů
než chlor. V~poslední době se však též začalo šířit jeho používání
jako panacey, tedy všeléku výhradně podávaného z~rukou alternativních
léčitelů pod názvem MMS či CDS. Oxid chloričitý se v~jejich podání
připravuje z~22,5 \% (hm.) roztoku $\mathrm{NaClO_{2}}$ a vhodné
kyseliny, kupříkladu HCl či $\mathrm{NaHSO_{4}}$. Rovnice reakce
NaClO s~$\mathrm{NaHSO_{4}}$ zapsaná v~iontovém tvaru: 
\[
\mathrm{4\,ClO_{2}^{-}+2\,H^{+}\rightarrow2\,ClO_{2}+ClO_{3}^{-}+Cl^{-}+H_{2}O}
\]
Spočítejte, kolik gramů 22,5\% roztoku $\mathrm{NaClO_{2}}$ budeme
potřebovat na přípravu jednoho litru roztoku $\mathrm{ClO_{2}}$ o
koncentraci 250 ppm (molární zlomek; uvažujte hustotu roztoku 1 $\mathrm{g\cdot cm^{-3}}$).
\end{quotation} \dotfill \par 
Vzhledem k~tomu, že koncentrace $\ch{ClO2}$ je zadaná jako molární zlomek, bude potřeba vypočítat látkové množství oxidu chloričitého
v jednom litru panacey. Označení ppm \textit{(parts per million)} pak vyjadřuje, že se jedná o~miliontiny (podobně jako procenta vyjadřují setiny).
\[
n_{\ch{ClO2}}=x_{\ch{ClO2}}\cdot n_{\ch{H2O}}=x_{\mathrm{ppm}}\cdot\frac{m_{\ch{H2O}}}{M_{\ch{H2O}}}=250.10^{-6}\cdot\frac{1000}{18}=0,0139\, \mathrm{mol}
\]

Potřebné množství 22,5\% roztoku chloritanu sodného pak vyplývá ze stechiometrie reakce -- na vznik dvou molekul $\ch{ClO2}$ je třeba čtyř aniontů $\ch{ClO2^{-}}$, celkové množství je proto třeba vynásobit dvěma.
\[
m_{\ch{NaClO2}}=n_{\ch{NaClO2}}\cdot M_{\ch{NaClO2}}=2\cdot n_{\ch{ClO2}}\cdot M_{\ch{NaClO2}}=2\cdot 0,0139\cdot 90,44=2,514\, \mathrm{g}
\]
\[
m_{roztok}=\frac{m_{\ch{NaClO2}}}{w_{\ch{NaClO2}}}=\frac{2,514}{0,225}=11,17\, \mathrm{g}
\]


\hrulefill % \subsection*{Ročník 5, úloha č. 7.2 }
\begin{quotation}
\dva Častou přísadou do těsta je jedlá soda (hydrogenuhličitan sodný),
která těsto ,,nadýchává``. Z~chemického hlediska lze tento proces
relativně jednoduše objasnit. Jedlá soda se teplem rozkládá na uhličitan
sodný, vodu a oxid uhličitý. Plynné produkty vznikající při této reakci
pak těsto probublávají a tím ho nadýchají. Přestože pro většinu pečiva
jedlá soda stačí, v~cukrářství se též používá uhličitan amonný (též
cukrářské droždí), který se rozkládá na amoniak, vodu a oxid uhličitý,
díky čemuž nadýchá těsto více.

Vaším úkolem je spočítat, kolikrát větší bude objem všech plynných
produktů vzniklých rozkladem 42 gramů uhličitanu amonného oproti objemu
plynných produktů vzniklých rozkladem hydrogenuhličitanu
sodného o~stejné hmotnosti. Předpokládejte, že vznikající voda je v~plynném skupenství
(pečeme v~troubě při teplotě 175 °C) a teplota a tlak jsou v~obou
případech stejné.
\end{quotation} \dotfill \par 

Vzhledem k~tomu, že víme, že objem jednoho molu plynu je za daných podmínek pořád stejný nezávisle na tom, jaké molekuly plyn tvoří, můžeme usoudit, že 
poměr objemu vzniklých plynů bude dán pouze poměrem látkových množství
vzniklých plynů.

K výpočtu se budou hodit vyčíslené rovnice zadaných reakcí: 
\shorthandoff{-}
\begin{center}
\ch{2 NaHCO3 -> H2O (g) + CO2 (g) + Na2CO3 (s)}\\
\ch{(NH4)2CO3 -> H2O (g) + CO2(g) + 2 NH3(g)}\\
\end{center}
\shorthandon{-}
Vznikající plyny jsou v~rovnicích označené\footnote{Jedná se o~standardní způsob značení skupenství v~chemických rovnicích -- do závorky se píše buď s~jako pevná látka (z angl. \textit{solid}), l jako kapalina (\textit{liquid}) g jako plyn (\textit{gas}), případně též aq pro vodný roztok (to je odvozené z~latinského \textit{aqua}).} (g). Z~vyčíslených rovnic plyne, že ze dvou molů jedlé sody obdržíme dva moly plynu, zatímco z~jednoho molu cukrářského droždí obdržíme
4 moly plynu. Žádaný poměr pak získáme podělením látkových množství plynů, které by vznikly z~obou droždí stejné hmotnosti:
\[
r=\frac{4\cdot \frac{m_{\mathrm{droždí}}}{M_{\ch{NaHCO3}}}}{\frac{2}{2} \cdot \frac{m_{\mathrm{droždí}}}{M_{\ch{(NH4)2CO3}}}} = \frac{4\cdot M_{\ch{NaHCO3}}}{1\cdot M_{\ch{(NH4)2CO3}}}=\frac{4\cdot84}{1\cdot96}=3,5
\]

\newpage %%
 % \subsection*{Ročník 5, úloha č. 6.6 }
\begin{quotation}
\tri Ve čtyřicátých a padesátých letech minulého století se jako raketové
palivo používal mimo jiné anilin ve směsi s~dýmavou kyselinou dusičnou
(počítejte 100\%) jako okysličovadlem. Tyto dvě látky jsou ve směsi
hypergolické, což znamená, že nepotřebují další vnější zážeh. Kolik
ml kyseliny dusičné potřebujeme na úplné spálení jednoho litru anilinu
za předpokladu, že produkty hoření jsou jen dusík, voda a oxid uhličitý?

Molární hmotnosti: $M_{\mathrm{anilin}}=93,13\ \mathrm{g\cdot mol^{-1}}$,
$M_{\ch{HNO3}}=63,01\ \mathrm{g\cdot mol^{-1}}$

Hustoty: $\rho_{\mathrm{anilin}}=1,03\ \mathrm{g\cdot ml^{-1}}$,
$\rho_{\ch{HNO3}}=1,51\ \mathrm{g\cdot ml^{-1}}$
\end{quotation} \dotfill \par 
Pro výpočet musíme nejprve sestavit vyčíslenou rovnici spalování.
Uhlíky anilinu s~C--H vazbou mají oxidační číslo \textminus I, uhlík
vedle amino skupiny +I. Oxidační číslo dusíku v~aminoskupině je \textminus III
a oxidační čísla všech vodíků jsou +I. Na oxidaci jedné molekuly anilinu
tedy potřebujeme 25 elektronů na uhlíky s~C--H vazbou, 3 elektrony
na uhlík vedle aminoskupiny a 3 elektrony na dusík aminoskupiny. Celkem
tedy k~oxidaci jedné molekuly anilinu potřebujeme 31 elektronů. Kyselina
dusičná se redukuje o~pět elektronů. Po aplikaci křížového pravidla
rovnici vyčíslíme. 
\begin{center}
\includegraphics{/5/6-6} 
\par\end{center}

Pro poměr látkových množství reaktantů tedy platí: 
\[
\frac{n_{\ch{HNO3}}}{n_{\mathrm{anilin}}}=\frac{31}{5}
\]
Látkové množství anilinu převedeme na pravou stranou a za obě látková
množství dosadíme následující výraz: 
\[
n=\frac{m}{M}=\frac{\rho\cdot V}{M}
\]
Získáme tedy 
\[
\frac{\rho_{\ch{HNO3}}\cdot V_{\ch{HNO3}}}{M_{\ch{HNO3}}}=\frac{31}{5}\cdot\frac{\rho_{\mathrm{anilin}}\cdot V_{\mathrm{anilin}}}{M_{\mathrm{anilin}}}
\]

Nyní už nám stačí vyjádřit objem dýmavé kyseliny dusičné a po dosazení
získáme výsledek: 
\[
V_{\ch{HNO3}}=\frac{31}{5}\cdot\frac{\rho_{\mathrm{anilin}}\cdot M_{\ch{HNO3}}}{\rho_{\ch{HNO3}}\cdot M_{\mathrm{anilin}}}\cdot V_{\mathrm{anilin}}=2861\mathrm{~ml}
\]


\section{Vyčíslování rovnic}

 % \subsection*{Ročník 4, úloha č. 1.3 }
\begin{quotation}
\jeden Přílohou k~úloze byl text \textit{Balady o~detekční trubičce} od Ivo Jahelky\footnote{
Tento text zde nezveřejňujeme jak z~důvodu,
že by se sem nevešel, tak z~důvodu ochrany autorského práva. Jak pan
Jahelka ve svých písních dokazuje, právu poměrně dobře rozumí a my
bychom se neradi stali aktéry jedné z~jeho dalších písní.}. Text této písně se odvolává na rozdílnou reakci dichromanu v~detekční trubičce pro určování alkoholu v~krvi v~závislosti
na počasí. Jistě je Vám však jasné, že počasí s~reakcí rozhodně nesouvisí. Uvnitř skleněné detekční trubičky je na silikagelu zvlhčeném kyselinou
sírovou nanesen dichroman draselný. Účinkem vydechovaného alkoholu
dochází k~barevné reakci, kterou lze při silniční kontrole zjistit,
zda řidič alkohol požil, či nikoliv. Během reakce vzniká mimo jiné
kyselina octová a chromitá sůl. Napište a vyčíslete tuto reakci.
\end{quotation} \dotfill \par 
\[
\ce{2K2Cr2O7 + 8H2SO4 + 3C2H5OH \rightarrow} \, 3\, \ce{CH3COOH + 2Cr2(SO4)3 + 2K2SO4 + 11H2O}
\]
Oranžový $\ce{K2Cr2O7}$ se při této reakci redukuje na zelený $\ce{Cr2(SO4)3}$
a oxiduje tak ethanol v~dechu na kyselinu octovou.

\newpage %%
% \subsection*{Ročník 5, úloha č. 2.1  }
\begin{quotation}
\dva Celosvětová produkce arsenu se opírá především o~tři minerály: arsenopyrit,
auripigment a realgar. Poslední jmenovaný o~stechiometrickém vzorci AsS vytváří
hezké rudé krystaly, pro něž bývá někdy nazýván „rudá síra“. Jeho
barva však mizí na denním světle po kontaktu se vzdušným kyslíkem,
protože se minerál mění na jedovatý arsenik (oxid arsenitý) a sytě
žlutý auripigment, stechiometricky vzato sulfid arsenitý. Vyčíslenou
chemickou rovnicí popište tuto přeměnu.
\end{quotation} \dotfill \par 

Arsen se oxiduje o~jeden elektron, kyslík se o~dva redukuje. Problém nám ovšem při vyčíslování bude dělat počet atomů v~jednotlivých produktech. Nejjednoduššeji budeme balancovat kyslík - na pravé straně je potřeba mít alespoň dvě molekuly $\ch{As2O3}$, abychom mohli mít u~kyslíku na levé straně celočíselný stechiometrický koeficient. Redukce tří molekul $\ch{O2}$ pak odpovídá 12 elektronům, tedy oxidaci 12 atomů arsenu.

Vyčíslená rovnice je níže:
\[
\ch{12 AsS + 3 O2}\rightarrow4\,\ch{As2S3 + 2 As2O3}
\]
Uznávána byla i následující rovnice, která lépe vystihuje vzájemné vazebné vztahy arsenu a síry v~realgaru:
\[
\ch{3 As4S4 + 3 O2}\rightarrow4\,\ch{As2S3 + 2 As2O3}
\]


\hrulefill
% \subsection*{Ročník 5, úloha č. 5.4 }
\begin{verse}
\textit{Kde domov můj, kde domov můj? }\\
 \textit{V kraji znáš-li Bohu milém }\\
 \textit{duše útlé v~těle čilém, }\\
 \textit{mysl jasnou vznik a zdar }\\
 \textit{a tu sílu vzdoru zmar? }\\
 \textit{To je Čechů slavné plémě,}\\
 \textit{mezi Čechy domov můj, mezi Čechy domov můj!} 
\end{verse}
\begin{quotation}
\ctyri Málokdo zná druhou sloku této písně z~divadelní hry Josefa Kajetána
Tyla, která, narozdíl od sloky první, nikdy nebyla státním symbolem.
Podobně jsou ve stínu slávy jejího skladatele Františka Škroupa opomíjeny
úspěchy, kterých dosáhl jeho synovec Zdenko Hans. Jako profesor chemie
na univerzitách ve Štýrském Hradci (rektor 1903/04) a Vídni vychoval
několik generací významných přírodovědců a přispěl k~rozvoji chemie
přírodních látek: jeho příprava chinolinu z~anilinu, glycerolu a
vhodného oxidačního činidla (např. nitrobenzenu) v~přítomnosti kyseliny
sírové patří mezi zavedené jmenné reakce (\textit{Skraup synthesis}).
Vaším úkolem je vyčíslit rovnici analogické syntézy 1,10-fenanthrolinu
z~$o$-fenylendiaminu, 2-nitroanilinu a glycerolu. 
\begin{center}
\includegraphics{/5/5-4} 
\par\end{center}

\end{quotation} \dotfill \par 
Pokud bychom se rovnici snažili vyčíslit pomocí počítání elektronů v~redoxních dějích, dostali bychom se do nepříjemností. Oxidačním činidlem je sice jen nitroanilin, kde se redukuje dusík v~nitroskupině o~šest elektronů, ovšem oxidaci podléhají atomy uhlíku ve všech třech reaktantech, což znamená změnu tří průměrných oxidačních čísel. Bez základního vhledu do mechanismu reakce není vůbec snadné rovnici tímto způsobem vyčíslit.


\newpage %text%
K vyčíslení ovšem znalost mechanismu nepotřebujeme, existuje i cesta, jak se bez ní obejít. Stačí zapsat sumární vzorce a vyčíslit rovnici pomocí bilančních rovnic:
\begin{center}
\shorthandoff{-}
\ch{\textit{a}~{C6H8N2} + \textit{b}~C6H6N2O2 + \textit{c}~C3H8O3 -> \textit{d}~C12H8N2 + \textit{e}~H2O}
\shorthandon{-}
\end{center}

Sestavíme bilanční rovnice pro jednotlivé prvky. 
\begin{align*}
\ch{N}:~ & 2a+2b=2d\\
\ch{O}:~ & 2b+3c=e\\
\ch{C}:~ & 6a+6b+3c=12d\\
\ch{H}:~ & 8a+6b+8c=8d+2e
\end{align*}
Výsledkem jsou čtyři rovnice pro pět neznámých, které můžeme vyřešit
s jednou neznámou jako parametrem. Tím získáme 
\[
b=2a;\ c=6a;\ d=3a;\ e=22a
\]

To znamená, že rovnice je splněna pro stechiometrické koeficienty
1:2:6:3:22 a jejich násobky.

Doplňme, že glycerol slouží jako zdroj tříuhlíkatých řetězců, ze kterých
vzniknou boční aromatická jádra hlavního produktu. Mechanismus je
poměrně dlouhý a počíná dvojnásobnou dehydratací glycerolu za vzniku
akroleinu, který kondenzuje s~aminoskupinou, následuje cyklizace\footnote{Eine Synthese des Chinolins. Skraup, Z.H. Monatshefte
für Chemie (1880) 1: 316. DOI: \href{https://doi.org/10.1007/BF01517073}{\underline{10.1007/BF01517073}}}.

\hrulefill
% \subsection*{Ročník 2, úloha č. 7.1 }
\begin{quotation}
\textit{»A tak jsem jela s~těma sedmi pětimarkovkama na trh, kterej bejval
ráno u~menzy technický univerzity, obstarala si fet a už při pořádným
absťáku si na záchodě konečně píchla. Matka mi teď každej den kontrolovala
ruce, jestli tam nemam čerstvý vpichy. Píchala jsem se do ruky, pořád
do stejnýho místa. Měla jsem tam už strup. Matce jsem nakecala, že
je to poranění, který se mi špatně hojí. Jednou si přece jen všimla,
že tam mám čerstvej vpich. Řekla jsem: „No jo, jasně. Jenom dneska.
Dělám to jenom někdy, to vůbec neškodí...“« }\footnote{Wir Kinder vom Bahnhof Zoo. Christiane F., podle magnetofonových záznamů sepsali Kai Hermann a Horst Rieck, 1. vydání, Hamburg, Gruner und Jahr, 1978, (Ein Stern-Buch), ISBN 3-570-02391-5}

\tri Heroin je droga hojně zneužívaná narkomany, která se před aplikací
rozehřeje na lžičce a pak vstříkne („vpálí“) do žíly. Nebylo tomu
tak vždy: Od roku 1896, kdy byl vyvinut proces jeho syntézy v~průmyslovém
měřítku, byl heroin široce užívaným prostředkem proti kašli a bolestem.
O osm let později bylo ale potvrzeno, že užívání heroinu vede ke vzniku
tolerance a závislosti dokonce rychleji, než je tomu u~morfia. V~této
úloze si ale vyzkoušíte proces opačný k~syntéze heroinu, totiž jeho
totální oxidaci. Zapište vyčíslenou rovnici reakce heroinu s~manganistanem
v kyselém prostředí. Heroin se oxiduje na dusík, oxid uhličitý a vodu.
Strukturní vzorec heroinu máte uveden na následujícím obrázku.

\begin{center}
\includegraphics{/2/7-1}
\par \end{center}
\end{quotation} \dotfill \par 
Sumární vzorec heroinu je $\mathrm{C_{21}H_{23}NO_{5}}$. Pokud budeme chtít zjistit, kolik elektronů je třeba odebrat heroinu pro oxidaci, je třeba určit oxidační čísla jednotlivých skupin atomů. Oxidační číslo kyslíků a vodíků zůstává stejné v~heroinu i v~produktech, mění se tedy pouze oxidační číslo dusíku a uhlíku. Dusík má v~heroinu oxidační
číslo --III, průměrné oxidační číslo uhlíku pak je $-\frac{23-10-3}{21}=-\frac{10}{21}$.

Rovnice oxidací jsou níže:
\[
\mathrm{N^{-III}\rightarrow N^{0}+3\,e^{-}}
\]

\[
\mathrm{C_{21}^{-\nicefrac{10}{21}}\rightarrow21\,C^{IV}+94\,e^{-}}
\]

Celkem se oxidací uvolní 97 elektronů. Ty navážeme do manganistanu,
který se v~kyselém prostředí redukuje na manganatou sůl:

\[
\mathrm{MnO_{4}^{-}+5\,e^{-}+8\,H^{+}\rightarrow Mn^{2+}+4\,H_{2}O}
\]

Na oxidaci pěti molekul heroinu je potřeba 97 molekul manganistanu.
Vyčíslená rovnice je potom:
\[
\mathrm{5\,C_{21}H_{23}NO_{5}+97\,MnO_{4}^{-}+291\,H^{+}\rightarrow105\,CO_{2}+\frac{5}{2}\,N_{2}+97\,Mn^{2+}+203\,H_{2}O}
\]
Pokud se nespokojíte s~necelým koeficientem u~dusíku, lze vynásobit všechny
koeficienty dvěma:
\[
\mathrm{10\,C_{21}H_{23}NO_{5}+194\,MnO_{4}^{-}+582\,H^{+}\rightarrow210\,CO_{2}+5\,N_{2}+194\,Mn^{2+}+406\,H_{2}O}
\]

\hrulefill
% \subsection*{Ročník 3, úloha č. 8.6 }
\begin{quotation}
\ctyri Vyčíslování chemických rovnic patří k~základním vědomostem a schopnostem
každého dobrého chemika. Některé rovnice jsou jednoduché a jdou řešit
uhádnutím od boku. Nad jinými, komplexnějšími reakcemi, je třeba se trochu víc rozmyslet. Obzvláště to platí, pokud reakce zahrnuje složitější komplexní sloučeniny s~různými typy ligandů.

Spočítejte podíl součinu veškerých nejmenších možných stechiometrických
koeficientů následující reakce a hodnoty Avogadrova čísla. $N_{\mathrm{A}}=6,022140857\cdot10^{23}$

\begin{align*}
\mathrm{[Cr\{(N_{2}H_{4})CO\}_{6}]_{4}[Cr(CN)_{6}]_{3}+KMnO_{4}+H_{2}SO_{4}} \rightarrow\mathrm{K_{2}Cr_{2}O_{7}+MnSO_{4}+CO_{2}+KNO_{3}+K_{2}SO_{4}+H_{2}O}
\end{align*}
\textit{Nápověda: Oxidační číslo dusíku v~hydrazinu je --II, CO skupina
je neutrální ligand.}
\end{quotation} \dotfill \par 

Oxidaci podléhají následující atomy:

 4 Cr$\mathrm{^{+III}}$(z kompexního kationtu, 12 elektronů)

 3 Cr$\mathrm{^{+II}}$ (z komplexního aniontu, 12 elektronů)

48 N$\mathrm{^{-II}}$ (z hydrazinu, 336 elektronů)

24 C$\mathrm{^{+II}}$ (z ligandu CO, 48, elektronů)

18 C$^{\mathrm{+II}}$ (z ligandu CN, 36 elektronů)

18 N$^{\mathrm{-III}}$ (z ligandu CN, 144 elektronů)

Celkem na oxidaci potřebujeme 588 elektronů. Protože ale vzniká dichroman
a ve výchozí molekule je lichý počet atomů chromu, musíme počítat
s dvojnásobkem elektronů, abychom nedostali u~dichromanu neceločíselný
koeficient, tedy 1176 elektronů.

Oxidačním činidlem bude manganistan draselný, který podléhá pětielektronové
redukci na manganatou sůl. Celou rovnici jsme násobili kvůli vzniku
dichromanu dvěma, dvě molekuly KMnO$_{4}$ tedy poskytnou 10 elektronů.
Po dopočítání koeficientů u~ostatních složek vyjde rovnice:

\noindent 
\begin{align*}
\mathrm{10\,[Cr\{(N_{2}H_{4})CO\}_{6}]_{4}[Cr(CN)_{6}]_{3}+1176\,KMnO_{4}+1399\,H_{2}SO_{4} & \rightarrow}\\
\rightarrow\mathrm{35\,K_{2}Cr_{2}O_{7}+1176\,MnSO_{4}+420\,CO_{2}+660\,KNO_{3}+223\,K_{2}SO_{4}+1879\,H_{2}O}
\end{align*}
Podíl součinu všech koeficientů a Avogardova čísla je 0,1306 (7,656,
pokud dělíme obráceně).

\chapter{Fyzikální chemie}

\section{Stavová rovnice ideálního plynu}

% \subsection*{Ročník 3, úloha č. 0.26   }
\begin{quotation}
\jeden V~mnohých z~předchozích úloh jste uvažovali molární objem ideálního plynu 22,41 litru.
Tato poučka ovšem neplatí vždy, platí pouze při konkrétním tlaku a
teplotě. Jaká musí být teplota ideálního plynu (ve °C) při tlaku 1
atmosféry (101~325 Pa), aby tento ideální plyn zaujímal molární objem
22,41 litru?
\end{quotation} \dotfill \par 
Teplotu lze spočítat ze stavové rovnice ideálního plynu $pV=nRT$.

Po vyjádření a dosazení:

\[
T=\frac{pV}{nR}=\frac{101325\cdot0,02241}{1\cdot8,314}\doteq273\,\mathrm{K}.
\]

Po převodu na stupně Celsia je teplota 0\,°C.

\hrulefill % \subsection*{Ročník 1, úloha č. 1.6}
\begin{quotation}
\jeden 11. května 1976 se v~Houstonu (Texas, USA) na mimoúrovňové křižovatce
převrhla cisterna s~amoniakem. Devatenáct tun amoniaku se vylilo na
zem, ihned po vylití se 21~\% veškerého amoniaku odpařilo. Z~tohoto
množství vznikl válcovitý mrak, který zasáhl oblast v~kruhu o~průměru
54 metrů. Pokud aproximujete mrak jako válec vysoký 2~metry plný pouze
amoniaku, který se chová jako ideální plyn, spočítejte, jaká byla
po uvolnění v~mraku teplota (ve stupních Celsia). V~daný den bylo
hezky, atmosférický tlak byl 1,05~bar. 
\end{quotation} \dotfill \par 
Využijeme stavové rovnice ideálního plynu a po úpravě dostaneme 
\[
T=\frac{pV}{nR}
\]

kde $p$ je tlak, $R$ je molární plynová konstanta, $n$ je počet molů vzniklého
plynu a $V$ je objem mraku. Pro $n$ platí: 
\[
n=\frac{m}{M}
\]
kde $m$ je hmotnost vypařeného plynu a $M$ jeho molární hmotnost. Objem válce
se spočítá jako $V=\pi r^{2}v$. Dosazením těchto vztahů do stavové rovnice dostáváme
\[
T=\frac{p\pi r^{2}v}{\frac{m}{M}R} = \frac{105000 \cdot \pi\cdot 27^{2}\cdot 2}{\frac{19\cdot 10^6 \cdot 0,21}{17}\cdot 8,314} =246 \, \mathrm{K}
\]
Při dosazování je třeba dbát na to, aby dosazované veličiny byly ve správných jednotkách. Též je třeba nezapomenout na to, že se odpařilo pouze 21 \% z~vylitého amoniaku.
Převod stupňů Celsia na kelviny je pak poměrně triviální -- od výsledné teploty v~kelvinech odečteme 273. Vyjde nám, že mrak měl teplotu --27\,°C.

\newpage %%
% \subsection*{Ročník 3, úloha č. 3.5 }
\begin{quotation}
\dva Airbagy v~autech zachraňují spoustu lidských životů při závažnějších
dopravních nehodách. Jedním ze způsobů, jak rychle nafouknout airbag,
je rozkladná reakce azidu sodného (NaN$_{3}$). Tato reakce produkuje
velké množství molekulárního dusíku; jiné látky obsahující dusík reakcí
nevznikají. Kolik gramů azidu musí vybouchnout, aby vznikající plyn
naplnil polštář airbagu tak, že má vůči okolí přetlak 0,5 bar? Polštář
aproximujme jako kvádr o~rozměrech 0,40 \texttimes{} 0,40 \texttimes{}
0,15\,m. Uvažujte tlak okolního vzduchu 1 bar, teplota plynů po explozi
je 300 K. Uvažujme ideální chování plynu.
\end{quotation} \dotfill \par 
Azid se rozkládá na prvky podle rovnice\footnote{Vznikající sodík je oxidován nějakým vhodným oxidovadlem, tento děj
je však pro úlohu irelevantní, a tak byl pro jednoduchost vynechán.}:

\[
\mathrm{2\,NaN_{3}\rightarrow2\,Na+3\,N_{2}}
\]

Vznikající dusík musí naplnit objem celého kvádru tak, aby měl tlak
1,5 bar (1 bar atmosférický + 0,5 bar přetlak) při teplotě 300 K.
Látkové množství k~tomu potřebného dusíku nejlépe spočítáme ze stavové
rovnice ideálního plynu:

\[
n_{\ch{N2}}=\frac{pV}{RT}=\frac{150000\cdot0,4\cdot0,4\cdot0,15}{8,314\cdot300}=1,4433\,\mathrm{mol}
\]

Látkové množství potřebného azidu bude podle stechiometrie reakce
dvoutřetinové, takže jeho výslednou hmotnost určíme jako:

\[
m_{\mathrm{azid}}=M_{\mathrm{azid}}\cdot n_{\mathrm{azid}}=\nicefrac{2}{3}\,M_{\mathrm{azid}}\cdot n_{\ch{N2}}=62,54\,\mathrm{g}
\]

\section{Termodynamika}

% \subsection*{Ročník 4, úloha č. 3.6 }
\begin{quotation}
\dva V~listopadu roku 2017 proběhla v~médiích zpráva, že jistý elektrický
nákladní tahač vyráběný v~Kalifornii potřebuje nabíječku s~příkonem
jako několik tisíc domácností, tj. v~řádu megawattů. Zkuste pro srovnání
vypočítat výkon odpovídající chemické energii přenesené při tankování
vozidla se spalovacím motorem u~stojanu čerpací stanice, který naplní
padesátilitrovou nádrž benzínem za 1 minutu a 40 sekund. Benzín považujte
za oktan o~hustotě $750\,\mathrm{kg\cdot m^{-3}}$ se standardní spalnou
enthalpií $\Delta_{\mathrm{spal}}H = -5400\,\mathrm{kJ\cdot mol^{-1}}$. 
\end{quotation} \dotfill \par 
Načerpá se $V=50\,\mathrm{dm^{3}}$ za $t=100$ sekund, což představuje
objemový tok

\[
\dot{V}=\frac{V}{t}=0,5\ \mathrm{dm^{3}\cdot s^{-1}}
\]
Vynásobením hustotou přepočteme objemový tok na hmotnostní: 
\[
\dot{m}=\rho\cdot\dot{V}=375\ \mathrm{g\cdot s^{-1}}
\]
Podělením molární hmotností oktanu $M_{\ce{C8H18}}=114,23\,\mathrm{g\cdot mol^{-1}}$
určíme molární tok 
\[
\dot{n}=\frac{\dot{m}}{M}=\frac{375}{114,23}=3,28\ \mathrm{mol\cdot s^{-1}}
\]
Kolik chemické energie proudí do nádrže, určíme vynásobením molárního
toku spalnou enthalpií: 
\[
\dot{E}=\dot{n}\cdot\Delta_{\mathrm{spal}}H=3,28\cdot5400=17700\ \mathrm{kJ\cdot s^{-1}}
\]
Energie za jednotku času již má rozměr výkonu\footnote{Pokud se Vám koncept objemového, hmotnostního a molárního toku nelíbí, lze časem tankování samozřejmě podělit až celkovou přenesenou energii. V~takovém případě budou všechny mezivýsledky výše (celková přenesená hmotnost, látkové množství a energie) stokrát větší.}, tedy 17,7 MW (podle $P=E/t$).

\newpage %%
% \subsection*{Ročník 5, úloha č. 6.3 }
\begin{quotation}
\tri V~kruzích organizátorů Chemiklání oblíbený drink drátěnka vzniká slitím
jednoho panáka (40~ml) octa (8~\% hm. kyseliny octové, hustota 1,01 g$\cdot$ml$^{-1}$)
a jednoho panáka rumu (40\% obj. ethanolu, hustota 0,947 g$\cdot$ml$^{-1}$). V~této směsi by ale mohla probíhat esterifikace -- kyselina octová a ethanol by mohly zreagovat na ethyl-acetát. Tato reakce je však rovnovážná -- na ethyl-acetát zreaguje jen malá část dostupných molekul. Rovnice této rovnováhy je zde:

\begin{center}
\includegraphics{/5/6-3}    
\end{center}


Určete, kolik obj. procent ethanolu obsahuje výsledný nápoj po ustavení
rovnováhy.

Molární hmotnosti: \\
 $M_{\text{voda}}=18,02$ g$\cdot$mol$^{-1}$\\
 $M_{\text{EtOH}}=46,07$ g$\cdot$mol$^{-1}$\\
 $M_{\text{AcOH}}=60,05$ g$\cdot$mol$^{-1}$\\
 \\
 Hustota absolutního ethanolu: 0,7951 g$\cdot$ml$^{-1}$\\
 Rovnovážná konstanta esterifikace $K=5,3$ (do konstanty dosazujeme
koncentrace všech složek včetně vody).

Předpokládejte, že objem směsí je dán součtem objemů složek a že změna
objemu drátěnky v~průběhu reakce je zanedbatelná.
\end{quotation} \dotfill \par 
Nejprve spočteme objem drátěnky podle zadaného předpokladu: 
\[
V_{\text{final}}=V_{\text{ocet}}+V_{\text{rum}}=80\mathrm{~ml}
\]
Pro další výpočet budeme potřebovat znát látková množství jednotlivých
složek drátěnky. Látkové množství ethanolu je následující: 
\[
n_{\mathrm{EtOH}}=\frac{m_{\mathrm{EtOH}}}{M_{\mathrm{EtOH}}}=\frac{V_{\text{rum}}\cdot\phi_{\mathrm{EtOH}}\cdot\rho_{\mathrm{EtOH}}}{M_{\mathrm{EtOH}}}=0,2761\mathrm{~mol}
\]
Obdobně spočteme látkové množství kyseliny octové: 
\[
n_{\mathrm{AcOH}}=\frac{m_{\mathrm{AcOH}}}{M_{\mathrm{AcOH}}}=\frac{V_{\text{ocet}}\cdot\rho_{\mathrm{ocet}}\cdot w_{\mathrm{AcOH}}}{M_{\mathrm{AcOH}}}=0,0533\mathrm{~mol}
\]

Látkové množství vody spočteme z~její hmotnosti, která spolu s~hmotností
ethanolu a kyseliny octové\footnote{Hmotnosti ethanolu a kyseliny octové jsou vyjádřené v~rovnicích výše.} dává celkovou hmotnost drátěnky: 
\[
n_{\ch{H2O}}=\frac{m_{\mathrm{final}}-m_{\mathrm{AcOH}}-m_{\mathrm{EtOH}}}{M_{\ch{H2O}}}=3,440\mathrm{~mol}
\]
Nyní už můžeme počítat s~rovnováhou reakce. Ve drátěnce může probíhat
následující reakce: 
\[
\ch{EtOH + AcOH \ \rightleftharpoons\ \mathrm{H_{2}O} + EtOAc}
\]

Rovnovážnou konstantu této reakce lze potom při jednotkových aktivitních
koeficientech zapsat pomocí rovnovážné koncentrace. Rovnovážné koncentrace
můžeme vyjádřit pomocí látkových množství spočtených výše. Je ale
nutné od látkových množství reaktantů odečíst látkové množství $x$,
které se reakcí přemění na produkty. Toto látkové množství pak přičteme
k látkovým množstvím produktů. Pro přehlednost uvádíme bilanční tabulku:

\begin{center}
\begin{tabular}{l|l|c|r}
       & začátek & změna & rovnováha  \\\hline \hline
EtOH   & $n_{\mathrm{EtOH}}$  & $-x$  & $n_{\mathrm{EtOH}}-x$ \\\hline
AcOH   & $n_{\mathrm{AcOH}}$  & $-x$  & $n_{\mathrm{AcOH}}-x$ \\\hline
H$_2$O & $n_{\mathrm{voda}}$   & $+x$  & $n_{\mathrm{voda}}+x$  \\\hline
EtOAc  & 0       & $+x$  & $x$       
\end{tabular}
\end{center}

\newpage %text%
Dosazením získáme: 
\[
K=\frac{\left(\frac{n_{\mathrm{voda}}+x}{V_{\mathrm{fin}}}\right)\left(\frac{x}{V_{\mathrm{fin}}}\right)}{\left(\frac{n_{\mathrm{EtOH}}-x}{V_{\mathrm{fin}}}\right)\left(\frac{n_{\mathrm{AcOH}}-x}{V_{\mathrm{fin}}}\right)}=\frac{(n_{\text{voda}}+x)\cdot x}{(n_{\mathrm{EtOH}}-x)(n_{\mathrm{AcOH}}-x)}
\]
Výsledné objemy drátěnky se pokrátí a získáme rovnici pro neznámou
$x$. Jejím řešením získáme $x=0,01464$ mol. Nyní už nám jen stačí
zbylé látkové množství ethanolu přepočítat na jeho objemový zlomek\footnote{Pro porovnání: pokud by reakce neproběhla, byl by objemový zlomek ethanolu roven
20 \%.}: 
\[
\phi_{\mathrm{EtOH}}=\frac{V_{\mathrm{EtOH}}}{V_{\mathrm{fin}}}=\frac{(n_{\mathrm{EtOH}}-x)\cdot M_{\mathrm{EtOH}}}{V_{\mathrm{final}}\cdot\rho_{\mathrm{EtOH}}}=0,1894.
\]

\hrulefill % \subsection*{Ročník 3, úloha č. 5.3  }
\begin{quotation}
\dva Úklid není oblíbenou činností snad žádného studenta. Mnozí se k~němu
uchylují především ve chvílích, kdy se blíží zkouškové období, mnohem
užitečnější je ale vykonávat jej pravidelně a vynakládat tak úhrnem
mnohem méně energie než při jednom velkém úklidu. Třídění je ale také
neodmyslitelnou součástí chemického průmyslu -- například v~chemických
reaktorech, kde nemusejí konverze výchozích látek probíhat se stoprocentní
účinností. Třeba při Haberově-Boschově syntéze amoniaku se výtěžek
pohybuje v~závislosti na použité metodě v~rozsahu 20 až 30 procent. Vznikající amoniak je pak třeba oddělit od nezreagovaného dusíku a vodíku.

Minimální práce potřebná pro rozdělení takové směsi se dá vypočítat
ze vztahu 
\[
W=-nRT\cdot\sum (x_{i}\cdot \ln x_{i})
\]
kde $n$
je celkový počet molů směsi, $R$ univerzální plynová konstanta a
$x_{i}$ molární zlomek $i$-té složky směsi. 

Spočítejte, kolik práce musí minimálně vynaložit separátor plynů oddělující
pouze veškerý amoniak ze směsi vycházející z~autoklávu o~objemu 600
m$^{3}$ a teplotě 850 K pod tlakem 18 MPa, obsahující 15 mol. \% amoniaku,
20 mol. \% dusíku a 65 mol. \% vodíku. Při výpočtu předpokládejte
platnost stavové rovnice ideálního plynu. 
\end{quotation} \dotfill \par 
Zadání velmi svádí k~výpočtu látkového množství tříděné směsi jakožto
jediné neznámé ve stavové rovnici. Tato hodnota je však nepotřebná,
protože můžeme člen $nRT$ jednoduše vyměnit za levou stranu stavové
rovnice, na které obě veličiny známe, a přímo zjistit, že na úplné
rozseparování na jednotlivé složky (tj. à la Popelka) vynaložíme

\begin{align*}
W & =-pV\cdot\sum(x_{i}\ln x_{i})=-18\cdot10^{6}\cdot600\cdot(0,15\ln0,15+0,2\ln0,2+0,65\ln0,65)\\
W & =9,574\,\mathrm{GJ}
\end{align*}

To však nestačí: pokud chceme oddělit pouze amoniak, musíme zbylý
vodík a dusík zamíchat à la macecha, a tedy připočítat práci „získanou“
zamícháním. Ta bude mít pochopitelně opačné znaménko. Nesmíme však zapomenout, že
se po odebrání amoniaku změní molární zlomky vodíku a dusíku, a že
také dojde k~poklesu tlaku, který koresponduje poklesu látkového množství!

\[
p_{\mathrm{out}}=p(1-x_{\mathrm{NH_{3}}})=\SI[output-decimal-marker = {,}]{15,3}{\mega\pascal}
\]

\[
x_{\mathrm{N_{2}\,out}}=\frac{x_{\mathrm{N_{2}}}}{x_{\mathrm{N_{2}}}+x_{\mathrm{H_{2|}}}}=\frac{0,2}{0,2+0,65}=0,235
\]

\[
x_{\mathrm{H_{2}\,out}}=1-x_{\mathrm{N_{2}\,out}}=0,765
\]

\begin{align*}
W_{\mathrm{mix}} & =p_{\mathrm{out}}\cdot\-V\cdot\sum(x_{i,\mathrm{\,out}}\cdot\-\ln x_{i,\mathrm{\,out}})=15,3\cdot10^{6}\cdot600\cdot(0,235\ln0,235+0,765\ln0,765)\\
W_{\mathrm{mix}} & =-5,005\,\mathrm{GJ}
\end{align*}
Pro získání kýžené námahy, se kterou oddělíme amoniak, stačí sečíst
získané hodnoty prací, je tedy třeba vynaložit minimálně 4,569 GJ.

\section{Slučovací a spalná enthalpie, Hessův zákon}

% \subsection*{Ročník 4, úloha č. 2.6   }
\begin{quotation}
\jeden Standardní spalná enthalpie kapalného benzenu při 298 K je --3300 $\mathrm{kJ\cdot mol^{-1}}$. Napište
vyčíslenou chemickou rovnici reakce, které může být tato spalná enthalpie
přiřazena jako reakční. Nezapomeňte u~všech reaktantů a produktů uvést
jejich skupenství: (g) pro plyn, (l) pro kapalinu, (s) pro pevnou
látku nebo (aq) pro vodný roztok.
\end{quotation} \dotfill \par 
Spalná enthalpie udává uvolněné teplo\footnote{Správně bychom měli dodat, že se jedná o~uvolněné teplo při isobarickém průběhu reakce -- tedy za konstatntního tlaku.} při reakci 1 molu látky s~kyslíkem na nejstabilnější produkty oxidace za standardní teploty. Vzniká tedy plynný oxid uhličitý a kapalná voda.

\[
\ce{C6H6}\,(\mathrm{l})+\frac{15}{2}\ce{O2}\,(\mathrm{g})\rightarrow\ce{6CO2}\,(\mathrm{g})+\ce{3H2O}\,(\mathrm{l})
\]


\hrulefill % \subsection*{Ročník 1, úloha č. 2.3 }
\begin{quotation}
\jeden Z~dob, kdy organičtí chemici určovali struktury molekul destruktivními
metodami, si už odnášíme spíše teoretické poznámky, které však na
mnoha přednáškách citují profesoři rádi a se slzou v~oku. Obdobná
data však mohou být skvěle použitelná, třeba stabilitu různě substituovaných alkenů můžeme vyjádřit pomocí rozdílných hydrogenačních enthalpií.

Zkuste si obdobné výpočty na jednodušším příkladě, zjistěte standardní
slučovací enthalpii ethylenu za pomoci následujících standardních
spalných enthalpií (hodnoty jsou uvedeny v~$\mathrm{kJ\cdot mol^{-1}}$): 

\begin{center}
\begin{tabular}{l|l}
látka & $\Delta _{spal} H$   \\\hline
\hline
$\mathrm{C_{2}H_{4}}$ (g)  & $-1411$                          \\\hline
    $\mathrm{H_{2}\, (g)}$  &     $-286$                                 \\\hline
  C (s)    &          $-393,5$                           
\end{tabular}
\end{center}
\end{quotation} \dotfill \par 
Vyjádřeno chemickou rovnicí, chceme enthalpii následující reakce: 

\[
\mathrm{2\, H_{2}\, (g)+2\, C\, (s)\text{→}\,C_{2}H_{4}\,(g)}
\]

takže enthalpii této reakce vyjádříme jako součet spalných enthalpií
reaktantů, od kterých odečteme spalné enthalpie produktů. Všechny
enthalpie před sčítáním vynásobíme stechiometrickými koeficienty.
Vyjde tedy:
\[
\Delta _\mathrm{r} H=2\cdot (-286) + 2\cdot(- 393,5) - (-1411) = +52\, \mathrm{kJ\cdot mol^{-1}}
\]

\hrulefill % \subsection*{Ročník 1, úloha č. 5.2 }
\begin{quotation}
\dva Při studiu chemie se mnohdy studenti musí spokojit s~jídlem, které
zklidní jejich kňourající prázdný žaludek a přitom nezatíží jejich peněženku
(nebo jí spíše neodlehčí). Třeba při pokusech o~přiblížení se italské
kuchyni mnohdy hladový chemik sáhne po surovinách, které blízké obchodní
řetězce nabízejí za přístupnější ceny. To se samozřejmě také projeví
na termodynamických aspektech pečení a trávení. Vypočítejte změnu
enthalpie spojenou s~upečením šunkové pizzy, pokud je spalná enthalpie
jedné pizzy --975,1 kcal. Ingredience na jednu pizzu:
\begin{itemize}
    \item 5 rajčat 
    \item 150 g výrobního salámu
    \item 100 g sýra Eidam 30\% 
    \item kruhové těsto o~průměru 40 cm 
\end{itemize}

\newpage %text%

Spalná enthalpie rajčat je --33,22 kcal pro jedno průměrné rajče. 

Spalná enthalpie 100 g výrobního salámu je --134,8 kcal. 

Spalná enthalpie 1 $\mathrm{m^{2}}$ pizza těsta je --1,369 Mcal. 

Spalná enthalpie sýra Eidam 30\% (na výrobce neberme ohled) je --2629 $\mathrm{kcal\cdot kg^{-1}}$.

\textit{Nápověda: Pizza se dělá tak, že se vezmou všechny ingredience a nechají
se zreagovat\footnote{Otázkou by byla entropická bilance upečení pizzy, přeci jen jde o~poměrně uspořádaný systém. Na druhou stranu lze spatřit velký nárůst entropie v~porcování jednotlivých surovin. Vzhledem k~vysoké diskutabilitě této záležitosti se do debat o~entropii raději nebudeme pouštět.}}.
\end{quotation} \dotfill \par 
Nejprve vypočítáme spalné enthalpie výchozích látek (ingrediencí)
odpovídající jejich množství uvedenému v~receptu. Pro výpočet energie pizza těsta bude třeba si spočítat jeho plochu z~vzorce pro obsah kruhu:
\[
S=\pi \cdot \frac{d^2}{4}=400\pi \doteq 1257 \mathrm{cm^2}=0,1257 \mathrm{m^2}
\]
\begin{center}
\begin{tabular}{l|l}
surovina & $\Delta _{spal} H$ [kcal]   \\
\hline\hline
5 rajčat  & --166,1                      \\\hline
150 g výrobního salámu &   --202,2                                   \\\hline
100 g eidamu & --262,9 \\\hline
0,1257 $\mathrm{m^2}$ těsta & --172,1 (převedeno z~Mcal)
\end{tabular}
\end{center}

Od součtu těchto spalných enthalpií výchozích látek odečteme spalnou enthalpii jedné
pizzy. Výsledná změna enthalpie je 
\[
\Delta H = -166,1-202,2-262,9-172,1-(-975,1)=+171,8\, \mathrm{kcal}=+718,1\, \mathrm{kJ}.
\]

\hrulefill % \subsection*{Ročník 2, úloha č. 8.2 }
\begin{quotation}
\tri Směs práškového hliníku a oxidu železitého, běžně nazývaná termit,
je dobře známá pro svou schopnost uvolňovat při hoření velké množství
tepla. Britský youtuber \textit{Colin Furze} se rozhodl tuto reakci
využít pro konstrukci rychlovarné konvice. Vaším úkolem je vypočítat,
kolik gramů stechiometrické směsi oxidu železitého a hliníku je třeba
na ohřátí vody pro přípravy jedné konvice čaje (1~pinty, tj. 568~ml).
Na přípravu čaje je nejvhodnější voda o~teplotě 90\,°C. Voda z~vodovodu,
konvice i reakční směs mají teplotu 20\,°C. Standardní slučovací enthalpie
oxidu hlinitého a oxidu železitého jsou:

$\Delta_{\mathrm{sluč}}H_{\ch{Al2O3}}^{\ominus}=-1676\,\mathrm{kJ\cdot mol^{-1}}$

$\Delta_{\mathrm{sluč}}H_{\ch{Fe2O3}}^{\ominus}=-826\,\mathrm{kJ\cdot mol^{-1}}.$

Molární tepelné kapacity jsou 

$c_{\ch{Al2O3}}=111,06\,\mathrm{J\cdot K^{-1}\cdot mol^{-1}}$, 

$c_{\ch{Fe}}=33,87\,\mathrm{J\cdot K^{-1}\cdot mol^{-1}}$, 

$c_{\ch{H2O}}=76,36\,\mathrm{J\cdot K^{-1}\cdot mol^{-1}}$. 

Tepelná kapacita konvice je $C_{\mathrm{konv}}=150\,\mathrm{J\cdot K^{-1}}$. 

Všechny tepelné kapacity považujte za nezávislé na teplotě, počítejte
s hustotou vody 1~$\mathrm{g\cdot ml^{-1}}$. 
\end{quotation} \dotfill \par 
Při hoření probíhá následující reakce:

\[
\mathrm{2\,Al+Fe_{2}O_{3}\rightarrow Al_{2}O_{3}+2\,Fe}
\]

Její enthalpii vyjádříme pomocí slučovacích enthalpií jednotlivých
produktů a reaktantů (pomocí Hessova zákona):

\[
\Delta_{\mathrm{r}}H^{\ominus}=\Delta_{\mathrm{sluč}}H_{\ch{Al2O3}}^{\ominus}-\Delta_{\mathrm{sluč}}H_{\ch{Fe2O3}}^{\ominus}
\]
\[
\Delta_{\mathrm{r}}H^{\ominus}=-1676+826=-850\,\mathrm{kJ\cdot mol^{-1}}
\]

Látkové množství ohřívané vody spočteme jednoduše. Protože máme v~zadání, že uvažujeme hustotu vody $1\,\mathrm{g\cdot cm^{-3}}$ v~celém teplotním intervalu úlohy, objem vody v~mililitrech se číselně
rovná její hmotnosti v~gramech. Lze tedy psát

\[
n_{\ch{H2O}}=\frac{568}{18}=31,56\,\mathrm{mol}.
\]

Celková tepelná kapacita soustavy voda-konvice pak činí

\[
C_{\mathrm{vše}}=C_{\mathrm{konv}}+c_{\ch{H2O}}\cdot n_{\ch{H2O}}=31,56\cdot 76,36+150=12,56\,\mathrm{kJ\cdot K^{-1}}
\]

Kromě ohřevu konvice a vody jsou ohřáty také produkty spalovací
reakce ($\mathrm{Al_{2}O_{3}+Fe}$), které přijmou část vyprodukovaného tepla.
\[
q_{\mathrm{zahř}}=70\cdot (c_{\ch{Al2O3}}+2c_{\mathrm{Fe}})
\]

\[
q_{\mathrm{zahř}}=0,07\cdot (111,06+2\cdot33,87)=12,52\,\mathrm{kJ\cdot mol^{-1}}
\]

Konečná teplotní bilance tedy vypadá následovně:
\[
(\Delta_{\mathrm{r}}H^{\ominus}+q_{\mathrm{zahř}})\cdot n_{\mathrm{reakce}}+C_{\mathrm{vše}}\cdot \Delta T=0
\]

Vyjádříme z~této rovnice látkové množství:
\[
n_{\mathrm{reakce}}=\frac{-C_{\mathrm{vše}}\cdot \Delta T}{\Delta _{\mathrm{r}}H^{\ominus}+q_{\mathrm{zahř}}}=\frac{-179195}{-850+12,52}=0,214\, \mathrm{mol}
\]

Výpočet hmotností oxidu železitého a hliníku je pak triviální:
\begin{align*}
m_{\ch{Fe2O3}}=159,69\cdot0,2140=34,17\,\mathrm{g}\\
m_{\ch{Al}}=2\cdot27\cdot0,2140=11,55\,\mathrm{g}
\end{align*}

Celková hmotnost termitu je pak součtem těchto dvou hmotností:
\[
m_{\mathrm{termit}}=11,55+34,17=45,72\,\mathrm{g}
\]
\hrulefill % \subsection*{Ročník 5, úloha č. 8.2 }
\begin{quotation}
\tri Dikyanoacetylen (DCA) je za pokojové teploty a atmosférického tlaku
bezbarvá kapalina, která je pro chemika zajímavá hned ze dvou důvodů.
Tato látka je jednak schopna reagovat s~durenem (1,2,4,5-tetramethylbenzenem)
ve smyslu Dielsovy--Alderovy reakce, ale také při spalování v~proudu
kyslíku poskytuje nejteplejší kyslíkový plamen. Na druhou zajímavost
se podíváme podrobněji. Spočtěte adiabatickou teplotu plamene dikyanoacetylenu
a stechiometrického množství kyslíku ve stupních Celsia. Vstupní teplota
reaktantů je 25 °C.\\
(Adiabatická teplota plamene je model, který předpokládá
úplné využití reakčního tepla na ohřev plynných produktů. Kapalina
shoří až po předchozím vypaření.) Počítejte s~těmito vstupními daty:
\vspace{0,5cm}

\begin{tabular}{l|l}
Slučovací enthalpie oxidu uhličitého $\Delta_{\mathrm{f}}H^{\circ}$($\ch{CO2}$(g)): & $-$393,5 kJ $\cdot$ mol$^{-1}$\tabularnewline
\hline
Slučovací enthalpie dikyanoacetylenu $\Delta_{\mathrm{f}}H^{\circ}$($\ch{C4N2}$(l)): & 500,4 kJ $\cdot$ mol$^{-1}$\tabularnewline \hline
Výparná enthalpie $\Delta_{\mathrm{vap}}H$($\ch{C4N2}$): & 28,9 kJ $\cdot$ mol$^{-1}$\tabularnewline \hline
Tepelná kapacita $\ch{CO2}$(g) za konst. tlaku & 92,09 J $\cdot$ K$^{-1}$ $\cdot$ mol$^{-1}$ \tabularnewline \hline
Tepelná kapacita $\ch{N2}$(g) za konst. tlaku & 55,26 J $\cdot$ K$^{-1}$ $\cdot$ mol$^{-1}$\tabularnewline 
\end{tabular}
\vspace{0,5cm}

\noindent\-Tepelné kapacity považujte za nezávislé na teplotě.
\end{quotation} \dotfill \par 
Z~principu modelu adiabatického plamene spočteme nejprve standardní
reakční enthalpii. Rovnice stechiometrického spalování je následující:
\[
\ch{C4N2 (g) + 4 O2 (g)\ \rightarrow}\ 4\ \ch{CO2 (g) + N2 (g)}
\]
Standardní reakční enthalpii spočteme jako rozdíl standardních slučovacích
enthalpií produktů a reaktantů. Standardní slučovací enthalpie prvků
jsou nulové. Standardní slučovací enthalpie dikyanoacetylenu
v plynné fázi je součtem té ve fázi kapalné a jeho výparné enthalpie:
\[
\Delta_{\mathrm{r}}H=4\cdot\Delta_{\mathrm{f}}H(\ch{CO2})-\Delta_{\mathrm{f}}H(\ch{C4N2})-\Delta_{\mathrm{vap}}H(\ch{C4N2})
\]
\[
\Delta_{\mathrm{r}}H=4\cdot (-393,5) - 500,4 - 28,9 =-2103,3\;\mathrm{kJ\cdot mol^{-1}}
\]

Celá reakční enthalpie je v~našem modelu využita na ohřev plynných
produktů. Plamen se tedy ohřeje o~následující rozdíl teplot: 
\[
\Delta T=\frac{-\Delta_{\mathrm{r}}H}{4\cdot\-c_{\mathrm{p}}(\ch{CO2})+c_{\mathrm{p}}(\ch{N2})}=\frac{-(-2103,3 \cdot 10^3 )}{4\cdot 92,09 + 55,26} =4965\mathrm{~K}
\]
Protože jsme produkty spalování ohřívali z~teploty 25\,°C, je výsledná
adiabatická teplota plamene 4990\,°C.

\hrulefill % \subsection*{Ročník 5, úloha č. 8.6  }
\begin{quotation}
\tri Svůj odraz nalezne chemie také ve světě šachu. V~košaté teorii šachových
zahájení své místo najdou například ,,sodium attack`` (v anglické
notaci 1. Na3), „barium attack“ (1. b3 s~následným 2. Ba3) nebo „borane
attack“ (1. g3 s~následným 2. Bh3). V~roce 2018 
proběhlý zápas o~mistrovství světa mezi Magnusem Carlsenem a Fabiano
Caruanou skončil šňůrou dvanácti remíz „v základním čase“, tedy v~partiích
s klasickou časovou kontrolou FIDE. Na kvalitě zápasu se podepsala
skutečnost, že v~partiích, kdy vedl bílé figury americký velmistr,
oba hráči prozkoumávali stále stejnou variantu sicilské hry; když
hrál černými, snažil se vyhýbat všem rizikům, získat jen malou poziční
výhodu či udržet remízu. Zabývejme se nyní myšlenkou, že by Fabiano
Caruana opustil šestkrát zahraný tah 1. e4 a oživil by svůj repertoár
jedním „sodným útokem“ -- že by pokoutně přineseného jezdce, který
by byl vyroben z~čistého sodíku, vhodil do sklenice vody přinesené
jeho norským soupeřem. Spočítejte, na jakou teplotu by se ohřál vzniklý
roztok hydroxidu sodného ve sklenici.

Potřebná data:

Magnus Carlsen si přinesl 0,5 litru vody o~teplotě 20 °C a hustotě
0,9982 g$\cdot$cm$^{-3}$. Tepelná kapacita vzniklého roztoku NaOH
$c_{p}$ {[}$\mathrm{J\cdot g^{-1}\cdot K^{-1}}${]} má ve sledované
oblasti závislost 
\[
c_{p}=4,152-0,182b+0,023b^{2}
\]
kde $b$ je molalita hydroxidového nápoje v~mol NaOH na kg $\ch{H2O}$.

Slučovací enthalpie dotyčných figur: \\
 $\ch{Na+}$: $-240,1\,\mathrm{kJ\cdot mol^{-1}}$, $\ch{OH^{-}}$:
$-230\,\mathrm{kJ\cdot mol^{-1}}$, $\ch{H2O}$: $-285,8\,\mathrm{kJ\cdot mol^{-1}}$.

Hmotnost jezdce: $m$ = 10 g. Zanedbejte tepelnou kapacitu sklenice a
únik tepla do okolí. Počítejte též s~tím, že nedojde k~explozi.
\end{quotation} \dotfill \par 
Výpočet lze rozdělit na dva kroky: zjištění uvolněného tepla a následný
ohřev vody. Ve sklenici proběhne známá reakce 
\[
\ch{Na + H2O} \rightarrow \ch{Na+ + OH^{-} + 1/2 H2}
\]
Jelikož standardní slučovací enthalpie vodíku je nulová, pro reakční
enthalpii získáme hodnotu 
\begin{align*}
\Delta_{\mathrm{r}}H_{\mathrm{m}} & =\Delta_{\mathrm{f}}H_{\ch{Na^{+}}}+\Delta_{\mathrm{f}}H_{\ch{OH^{-}}}-\Delta_{\mathrm{f}}H_{\ch{H2O}}\\
 & =-240,1-230-(-285,8)=-184,3\mathrm{~kJ\cdot mol^{-1}}
\end{align*}
\[
\Delta_{\mathrm{r}}H_{\mathrm{skut}}=n_{\ch{Na}}\cdot\Delta_{\mathrm{r}}H_{\mathrm{m}}=\frac{m_{\ch{Na}}}{M_{\ch{Na}}}\cdot\Delta_{\mathrm{r}}H_{\mathrm{m}}=\frac{10}{23}\cdot(-184,3)=-80,166\mathrm{~kJ}
\]
Nyní k~samotnému ohřevu -- nejdříve spočteme molalitu vzniklého
roztoku. Látkové množství vzniklého NaOH se rovná látkovému množství
vhozeného Na, 
\[
n_{\ch{NaOH}}=n_{\ch{Na}}=0,435\mathrm{~mol}
\]
Počáteční množství vody (pozor na jednotky -- objem je třeba dosazovat
v mililitrech) je 
\[
n_{\ch{H2O}}=\frac{\rho\cdot V}{M}=\frac{0,9982\cdot500}{18}=27,728\mathrm{~mol}
\]
Po reakci zbyde 27,728 -- 0,435 = 27,293 mol vody. Molalita je pak
\[
b=\frac{n_{\ch{NaOH}}}{n_{\ch{H2O}}\cdot M_{\ch{H2O}}}=\frac{0,435}{27,293\cdot0,018}=0,885\mathrm{~mol\cdot kg^{-1}}
\]
čemuž po dosazení odpovídá tepelná kapacita $c_{p}=4,0089\mathrm{~J\cdot K^{-1}\cdot g^{-1}}$.
Posledním krokem je dopočet konečné teploty. Pro zcela rigorózní přesnost
našeho šachového propočtu nesmíme opomenout, že do sklenice přibylo
10~g sodíku a vyšumělo malé množství vodíku: 
\begin{align*}
m_{\mathrm{roztok}} & =m_{\ch{H2O}}+m_{\ch{Na}}-m_{\ch{H2}}=0,9982\cdot500+10-\frac{n_{\ch{Na}}}{2}\cdot M_{\ch{H2}}\\
 & =0,9982\cdot500+10-0,438=508,7\mathrm{~g}
\end{align*}
Nyní již matový závěr -- teplota roztoku hydroxidu ve sklenici: 
\begin{align*}
T & =T_{0}+\frac{|\Delta_{\mathrm{r}}H_{\mathrm{skut}}|}{m\cdot c_{\mathrm{p}}}\\
T & =20+\frac{80169}{508,7\cdot4,0089}\\
T & =59,3\,{^{\circ}}\mathrm{C}.
\end{align*}

\hrulefill % \subsection*{Ročník 3, úloha č. 8.2 }
\begin{quotation}
\ctyri S~využitím Bornova-Haberova cyklu a Kapustinského rovnice, která 
umožňuje přibližně vypočítat teplo uvolněné při vzniku krystalu iontové látky
z~iontů $U_{\mathrm{L}}$, vypočtěte teplo (enthalpii) vzniku jednoho molu Xe{[}PtF$_{6}${]}
z~plynného xenonu a pevného fluoridu platinového.

Poloměr Xe$^{+}$: $r^{+}=190\,\mathrm{pm}$, 

Poloměr PtF$_{6}^{-}$: $r^{-}=200\,\mathrm{pm}$,

Ionizační energie xenonu: $IE=1170\,\mathrm{kJ\cdot mol^{-1}}$, 

Elektronová afinita PtF$_{6}$: $EA=786\,\mathrm{kJ\cdot mol^{-1}}$. 

\noindent\-PtF$_{6}$~je ochotně sublimující pevná látka, a proto můžeme teplo potřebné
na její sublimaci zanedbat. Kapustinského rovnice: 

\[
U_{\mathrm{L}}=K\cdot\frac{\nu\cdot|z^{+}|\cdot|z^{-}|}{r^{+}+r^{-}}\cdot\left(1-\frac{d}{r^{+}+r^{-}}\right)
\]

kde $K=1,2025\cdot10^{-4}\,\mathrm{J\cdot m\cdot mol^{-1}}$, $\nu$
je počet iontů v~empirickém vzorci, $d=3.45\cdot10^{-11}\,\mathrm{m}$. 

\textit{Nápověda: Bornův-Haberův cylkus pro \ch{CaBr2}:}
\begin{center}
\includegraphics[]{Born_Haber_1.pdf}
\end{center}
\end{quotation} \dotfill \par 

\newpage %text%
Nejprve je nutné sestavit Bornův-Haberův cyklus. Jelikož je xenon
jednoatomový plyn, tak ho není na rozdíl od vápníku nutno atomizovat,
jinak se cykly obou sloučenin v~principu neliší.

\begin{center}
\includegraphics[]{Born_Haber_2.pdf}
\end{center}
Z~Hessova zákona víme, že enthalpická bilance uzavřeného cyklu je nulová,
tedy: $-\Delta H_{\mathrm{f}}+IE+S+EA+U_{\mathrm{L}}=0$ (cyklus procházíme v~jednom
směru -- zde proti směru hodinových ručiček -- a pokud jdeme proti
reakční šipce, má příslušný energetický příspěvek opačné znaménko).
O sublimačním teplu ze zadání víme, že je zanedbatelné a položíme
tedy $S=0$. Dále je potřeba z~Kapustinského rovnice dosazením spočítat
$U_{\mathrm{L}}$:

\begin{align*}
U_{\mathrm{L}} & =1,2025\cdot10^{-4}\cdot\frac{2\cdot1\cdot1}{(190+200)\cdot10^{-12}}\cdot\left(1-\frac{3,45\cdot10^{-11}}{(190+200)\cdot10^{-12}}\right)\\
U_{\mathrm{L}} & =562,1\,\mathrm{kJ\cdot mol^{-1}}
\end{align*}

Posledním úskalím před dosazením do rovnice získané z~Hessova zákona
je znaménková konvence. Je třeba si uvědomit, že vznikem krystalu
z iontů se energie vždy uvolní, a $U_{\mathrm{L}}$ je tedy třeba dosazovat se záporným znaménkem. Podobně je tomu u~elektronové afinity, která je definovaná
jako energie uvolněná při přijetí elektronu a opět ji tedy musíme
dosazovat s~opačným znaménkem. Když jsme vyřešili těchto pár drobných
nepříjemností, nic nám už nebrání v~dosazení a vypočtení uvolněného
reakčního tepla: 

\[
-\Delta H_{\mathrm{f}}+1170000+0-786000-562115=0
\]
\[
\Delta H_{\mathrm{f}}=-178\,\mathrm{kJ\cdot mol^{-1}}
\]


\section{Chemická kinetika}

% \subsection*{Ročník 4, úloha č. 2.4   }
\begin{verse}
\textit{Krabička cigaret }\\
\textit{a do kafe rum, rum, rum }\\
\textit{dvě vodky a fernet }\\
\textit{a teď doktore čum čum čum}\footnote{NOHAVICA, Jarek. Pochod marodů {[}píseň{]}. V~tom roce pitomém. Panton,
1990.}
\end{verse}

\begin{quotation}
\dva Jedna věc je chvástat se v~písni, ale realita bývá jiná. Spoustě z~vás
je jasné, že jít zcela opilý za lékařem (či třeba prezidentem Ruské
federace) není nejlepší nápad. Spočítejte Jarkovi o~hmotnosti 85~kg
a objemu krve 5,5~litru, za jak dlouho může jít k~lékaři po pozření
tří rumů, dvou vodek a jednoho fernetu. Uvažujte, že lékař na Jarkovi
nepozná opilost, pokud má Jarek v~krvi 0,2~promile (obj.) alkoholu.
Předpokládejte, že Jarek pil panáky velké 5~cl, všechny vypité druhy
alkoholu obsahují 40 objemových procent ethanolu. Dále předpokládejte,
že v~krvi se projeví pouze 6~\% vypitého alkoholu (zbytek je v~dalších
tkáních, kde se též odbourává). Vylučování ethanolu z~organismu se řídí kinetikou 0.~řádu
tak, že Jarek odbourá z~krve 1~ml čistého ethanolu za 70~minut.
\end{quotation} \dotfill \par 
\newpage %text%
Nejdříve vypočítejme objem alkoholu, který bude mít Jarek v~sobě ihned
po vypití a také po „vystřízlivění“. 
\[
V_{\mathrm{start}}=\eta\cdot\varphi_{EtOH}\cdot V_{\mathrm{vypito}}=0,06\cdot0,40\cdot50\cdot6=7,2\ \mathrm{ml}
\]
\[
V_{\mathrm{end}}=\varphi_{\mathrm{krev}}\cdot V_{\mathrm{krev}}=0,0002\cdot5,5=0,0011\ \mathrm{l}=1,1\ \mathrm{ml}
\]
Z~výsledků vidíme, že je potřeba odbourat 6,1~ml čistého alkoholu.
A jelikož se odbourá 1~ml za 70~minut, tak Jarek vystřízliví na požadovanou
hodnotu za 427~minut, tzn. přibližně 7~hodin. 

\hrulefill % \subsection*{Ročník 5, úloha č. 7.3 }
\begin{quotation}
\ctyri Chemická kinetika je oblast chemie, která se zabývá studiem rychlostí
chemických reakcí.\\Mějme hypotetickou reakci 
\[
\ch{X + Y + Z ~\rightarrow \mathrm{produkty}},
\]
kde X, Y, Z~značí nějaké náhodné molekuly. Pro rychlost $r$ této
reakce platí: 
\begin{equation}
r=k_{\mathrm{r}}\ch{[X]}^{x}\ch{[Y]}^{y}\ch{[Z]}^{z}\tag{1} 
\end{equation}

kde $k_{\mathrm{r}}$ je rychlostní konstanta reakce, {[}X{]}, {[}Y{]}
a {[}Z{]} jsou koncentrace látek X, Y a Z~a exponenty $x$, $y$,
$z$ nazýváme řády reakce vůči X, Y a Z. Vaším úkolem je určit celkový
řád reakce $N=x+y+z$, přičemž je zadána $k_{\mathrm{r}}=43~285~\mathrm{s^{-1}\cdot mol^{-4}\cdot dm^{12}}$
a byla naměřena následující data: 
\begin{center}
\begin{tabular}{c|c|c|c}
$r$ / mol $\cdot$ dm$^{-3}$ $\cdot$ s$^{-1}$ & {[}X{]} / mol $\cdot$ dm$^{-3}$ & {[}Y{]} / mol $\cdot$ dm$^{-3}$ & {[}Z{]} / mol $\cdot$ dm$^{-3}$\tabularnewline\hline\hline
106,05 & 0,1 & 0,5 & 0,7\tabularnewline\hline
530,24 & 0,7 & 0,1 & 0,5\tabularnewline\hline
75,759 & 0,5 & 0,7 & 0,1\tabularnewline\hline
\end{tabular}
\par\end{center}

\end{quotation} \dotfill \par 
Nejjednodušeji lze k~danému problému přistoupit přes rozměrovou analýzu -- podíváme se na to, jakou má zadaná konstanta jednotku.
Z~rovnice (1) po dosazení jednotek místo veličin plyne
\begin{align*}
\mathrm{mol\cdot dm^{-3}\cdot s^{-1}} & =\mathrm{s^{-1}\cdot mol^{-4}\cdot dm^{12}}\cdot(\mathrm{mol\cdot dm^{-3}})^{x+y+z}\\
\mathrm{s^{-1}}(\mathrm{mol\cdot dm^{-3}})^{1} & =\mathrm{s^{-1}}(\mathrm{mol\cdot dm^{-3}})^{-4}(\mathrm{mol\cdot dm^{-3}})^{N}
\end{align*}
Logaritmováním dostaneme celkem jednoduchý vztah pro $N$: $N-4=1$,
tím pádem $N=5$.\\
Alternativně lze sestavit soustavu tří rovnic o~třech neznámých,
kdy do rovnice (1) postupně dosadíme data z~jednotlivých měření.

\[
\begin{array}{ccc}
106,05 & = & 43\,285\cdot 0,1^{x}\cdot0,5^{y}\cdot0,7^{z}\\
530,24 & = & 43\,285\cdot 0,7^{x}\cdot0,1^{y}\cdot0,5^{z}\\
75,759 & = & 43\,285\cdot 0,5^{x}\cdot0,7^{y}\cdot0,1^{z}
\end{array}
\]

Řešením soustavy\footnote{Po zlogaritmování všech rovnic dostaneme jednoduchou soustavu tří lineárních rovnic o~třech neznámých.} dostaneme $x=2$, $y=1$, $z=2$; což v~součtu poskytuje
též $N=5$.

\hrulefill % \subsection*{Ročník 1, úloha č. 8.2 }
\begin{quotation}
\dva Vypočtěte aktivační energii chemické reakce, pokud víte, že se při
zvýšení reakční teploty z~25\,°C na 45\,°C zvýší rychlost reakce na
trojnásobek. Předpokládejte, že koncentrace jednotlivých reaktantů
i produktů jsou v~obou případech stejné. 
\end{quotation} \dotfill \par 
Podle Arrheniovy rovnice souvisí reakční rychlost (respektive rychlostní konstanta reakce, která je reakční rychlosti přímo úměrná) s~aktivační energií
vztahem
\[
k=e^{\frac{-E_{A}}{R\cdot T}}
\]
kde $E_{\mathrm{A}}$ je aktivační
energie, $R$ molární plynová konstanta a $T$ termodynamická teplota
(v kelvinech). Vlivem koncentrací reaktantů a produktů a dalšími faktory
se pro účely této úvahy nezabýváme. Víme-li, že s~určitou změnou teploty
vzrostla reakční rychlost na trojnásobek, můžeme psát:
\[
k_{2}=3\cdot k_{1}
\]
Po dosazení číselných hodnot dostaneme exponenciální
rovnici 
\[
e^{\frac{-E_{\mathrm{A}}}{8,314\cdot318,15}}=3\cdot e^{\frac{-E_{\mathrm{A}}}{8,314\cdot298,15}}
\]
kterou vyřešíme logaritmováním a dostaneme $E_{\mathrm{A}}=43320\, \mathrm{J\cdot mol^{-1}}=43,32\, \mathrm{kJ\cdot mol^{-1}}$. 

\hrulefill % \subsection*{Ročník 4, úloha č. 8.6}
\begin{quotation}
\ctyri Katalytická oxidace ethylenu na ethylenoxid probíhá na povrchu palladiového
katalyzátoru následujícím mechanismem, kde M značí volné katalytické
místo na povrchu katalyzátoru, M-X značí katalytické místo obsazené
molekulou X: 

\[
\ce{Et}+\ce{M} \rightleftharpoons \ce{M-Et} \tag{1} \label{eq:Pd1}
\]
\[
\ce{O2}+\ce{2M} \rightleftharpoons 2\,\ce{M-O} \tag{2} \label{eq:Pd2}
\]
\[
\ce{M-Et}+\ce{M-O\rightarrow M}+\ce{EtO} \tag{3} \label{eq:Pd3}
\]
\[
\ce{M-EtO} \rightleftharpoons \ce{M}+\ce{EtO} \tag{4} \label{eq:Pd4}
\]

kde rovnicím \eqref{eq:Pd1} a \eqref{eq:Pd2} náleží rovnovážné konstanty adsorpce\footnote{Český ekvivalent slova \textit{adsorpce} by zněl nejspíše \textit{přilnutí}.}. Reakce \eqref{eq:Pd3} probíhající na povrchu katalyzátoru není rovnovážná a přísluší jí rychlostní konstanta. Rovnici \eqref{eq:Pd4} náleží rovnovážná konstanta desorpce.\\
Předpokládejte, že desorpce probíhá okamžitě a volný ethylenoxid okamžitě
opouští okolí katalyzátoru. Napište vztah pro reakční rychlost (rychlost
tvorby ethylenoxidu) v~závislosti na koncentracích volných reaktantů,
rychlostní konstantě povrchové reakce a rovnovážných konstantách.
Poznámka: Obdobou rovnovážných koncentrací jsou pro povrchově navázané
molekuly jejich relativní zastoupení mezi všemi katalytickými místy.
\end{quotation} \dotfill \par 
Nejdříve napišme vztah pro reakční rychlost. Jelikož rychlost desorpce
je okamžitá, rychlost vzniku ethylenoxidu bude stejná jako rychlost
vzniku povrchově vázaného ethylenoxidu, 
\[
r=k_{3}\theta_{\mathrm{Et}}\theta_{\mathrm{O}}
\]
 kde $\theta_{i}$ značí relativní zastoupení míst obsazených látkou
$i$. Dále vyjádříme obě konstanty adsorpce. 
\[
K_{1}=\frac{\theta_{\mathrm{Et}}}{\theta_{\mathrm{nic}}[\ce{Et}]}
\]
\[
K_{2}=\frac{\theta_{\mathrm{O}}^{2}}{\theta_{\mathrm{nic}}^{2}[\ce{O2}]}
\]
 Nyní dosaďme do vztahu pro rychlost za $\theta{}_{\ce{Et}}$ a $\theta_{\mathrm{O}}$
\[
r=k_{3}\theta_{\mathrm{nic}}K_{1}[\ce{Et}]\theta_{\mathrm{nic}}\sqrt{[\ce{O2}]K_{2}}=k_{3}K_{1}[\ce{Et}]\theta_{\mathrm{nic}}^{2}\sqrt{[\ce{O2}]K_{2}}
\]
 Dále platí, že součet relativních zastoupení všech obsazených i prázdných
míst je roven 1. Místa obsazená ethylenoxidem neuvažujeme, neboť je
zadáno, že desorpce je okamžitá. 
\[
\theta_{\mathrm{nic}}+\theta_{\mathrm{Et}}+\theta_{\mathrm{O}}=1
\]
Dosadíme opět za $\theta_{\ce{Et}}$ a $\theta_{\mathrm{O}}$: 
\[
\theta_{\mathrm{nic}}+\theta_{\mathrm{nic}}K_{1}[\ce{Et}]+\theta_{\mathrm{nic}}\sqrt{[\ce{O2}]K_{2}}=1
\]
\[
\theta_{\mathrm{nic}}\left(1+K_{1}[\ce{Et}]+\sqrt{[\ce{O2}]K_{2}}\right)=1
\]
\[
\theta_{\mathrm{nic}}=\frac{1}{1+K_{1}[\ce{Et}]+\sqrt{[\ce{O2}]K_{2}}}
\]
Nyní můžeme dosadit do rovnice pro rychlost reakce. 
\[
r=k_{3}\theta_{\mathrm{nic}}K_{1}[\ce{Et}]\theta_{\mathrm{nic}}\sqrt{[\ce{O2}]K_{2}}=\frac{k_{3}K_{1}[\ce{Et}]\sqrt{[\ce{O2}]K_{2}}}{\left(1+K_{1}[\ce{Et}]+\sqrt{[\ce{O2}]K_{2}}\right)^{2}}
\]
Rychlost reakce se dá vyjádřit v~závislostech na koncentracích\footnote{Správnější než koncentrace by bylo uvažovat parciální tlaky reagujících
látek, protože ethylen i kyslík jsou při této reakci v~plynném skupenství.} reaktantů, rychlosti povrchové reakce a konstantách adsorpce jako
\[
r=\frac{k_{3}K_{1}[\ce{Et}]\sqrt{[\ce{O2}]K_{2}}}{\left(1+K_{1}[\ce{Et}]+\sqrt{[\ce{O2}]K_{2}}\right)^{2}}
\]


\section{Koligativní vlastnosti}

% \subsection*{Ročník 3, úloha č. 0.21}
\begin{quotation}
\jeden Chlorid vápenatý je známý pro své výborné kryoskopické vlastnosti,
což znamená, že roztok CaCl$_{2}$ ve vodě bude tát (či tuhnout) při teplotě nižší
než 0\,°C. Teplotní posun je samozřejmě závislý na koncentraci roztoku,
koncentrovanější roztoky obvykle snesou nižší teploty zůstávajíce
v kapalném stavu. Tento jev samozřejmě velmi dobře znáte například
ze zimního posypu silnic. Po vás v~této jednoduché úloze budeme chtít
vypočítat, při jaké teplotě taje 0,59~ g čistého CaCl$_{2}$, pokud
víte, že 1,703~g CaCl$_{2}$ taje při teplotě 775\,°C. V~obou případech
měříme za atmosférického tlaku. 

$M_{\mathrm{r,\ch{CaCl2}}}=109,9$ 

Rozpustnost ve vodě je 0,745~g/ml při 20\,°C. 

Hustota pevného CaCl$_{2}$ je 2,15 g/ml při 68\,°F, hustota 1M roztoku
je 1,086~g/ml. 

Molární entropie kapalného \ch{CaCl2} je \SI[inter-unit-product = \ensuremath{{}\cdot{}},output-decimal-marker = {,}]{123,87}{\joule\per\kelvin\per\mole},

molární entropie pevného \ch{CaCl2} je \SI[inter-unit-product = \ensuremath{{}\cdot{}},output-decimal-marker = {,}]{104,62}{\joule\per\kelvin\per\mole}.
\end{quotation} \dotfill \par 
Teplota tání nezávisí na tom, kolik čisté látky máme (není to extenzivní
veličina). Teplota tání 0,592\,g CaCl$_{2}$ bude tedy také 775\,°C\footnote{Všimněte si, kolik údajů jste k~vyřešení této úlohy nepotřebovali.}.

\hrulefill % \subsection*{Ročník 5, úloha č. 6.2 }
\begin{quotation}
\tri Koligativní vlastnosti roztoků jsou takové vlastnosti, které nezávisí
na rozpouštěné látce, ale pouze na její látkové koncentraci. Jednou
z těchto vlastností je i snížení bodu tuhnutí. Pokud v~nějakém rozpouštědle,
řekněme ve vodě, rozpustíme jakoukoliv látku, například sůl, tak dojde
ke snížení teploty tuhnutí této látky. Toho se například využívá v
zimě při sypání chodníků a silnic solí. Kvantitativně snížení teploty
tuhnutí popisuje následující rovnice: 
\[
\Delta T=\frac{R\cdot c\cdot T^{2}}{\Delta H}
\]
kde $R$ je molární plynová konstanta, $c$ je součet koncentrací
všech rozpuštěných částic, $T$ je teplota tuhnutí čistého rozpouštědla
a $\Delta H$ je enthalpie tuhnutí rozpouštědla.

Chcete-li snížit teplotu tuhnutí vody (o jeden stupeň Celsia) v~bazénu
o rozměru 2\texttimes 5\texttimes 20 m$^{3}$ tím, že v~ní rozpustíte
dostatečné množství jedné ze sloučenin v~tabulce, za kterou sloučeninu
z~nabídky byste zaplatili nejméně? Uvažujte, že všechny soli zcela disociují
na ionty\footnote{Tedy 1 mol síranu sodného by disocioval na 3 moly částic.}.

\begin{tabular}{ll}
$T=\SI[inter-unit-product = \ensuremath{{}\cdot{}},output-decimal-marker = {,}]{273}{\kelvin}$~ & $\Delta H=\SI[inter-unit-product = \ensuremath{{}\cdot{}},output-decimal-marker = {,}]{79,72}{\cal\per\gram}$

\tabularnewline
$R=\SI[inter-unit-product = \ensuremath{{}\cdot{}},output-decimal-marker = {,}]{62,64}{\cubic\dm\per\torr\per\mole}$ & $\rho_{\text{voda}}=\SI[inter-unit-product = \ensuremath{{}\cdot{}},output-decimal-marker = {,}]{996}{\kg\per\cubic\metre}$\tabularnewline
\end{tabular}
\begin{center}
\begin{tabular}{r|r|r}
sloučenina & molární hmotnost (g $\cdot$ mol$^{-1}$) & cena (CZK $\cdot$ kg$^{-1}$)\tabularnewline \hline\hline
bromid zlatitý & 436,7 & 3 150 000,00\tabularnewline  \hline
chlorid draselný & 74,6 & 3,80\tabularnewline  \hline
octan draselný & 98,2 & 200,00\tabularnewline  \hline
chlorid sodný & 58,4 & 3,96\tabularnewline \hline
kyanid sodný & 49,0 & 49,00\tabularnewline \hline
síran amonný & 132,1 & 225,00\tabularnewline \hline
chlorid zlatitý & 303,3 & 3 468 000,00\tabularnewline \hline
chlorid vápenatý & 111,0 & 3,40\tabularnewline \hline
močovina & 60,6 & 21,00\tabularnewline
\end{tabular}
\par\end{center}

\end{quotation} \dotfill \par 
V zadaném vzorečku jsou molární plynová konstanta $R$, teplota tuhnutí
vody $T$ i enthalpie tuhnutí rozpouštědla $\Delta H$ konstantní.
Proto je jasné, že jediné, čím má smysl se zabývat, je koncentrace
všech rozpuštěných částic. Čím více částic v~roztoku, tím víc se
sníží teplota tuhnutí.

Počet částic, na které jedna molekula zkoumané látky disociuje,
označme $z$. Čím větší molární hmotnost látky, tím větší hmotnost
potřebujeme k~tomu, abychom jí rozpustili 1 mol (a tím víc za ni
musíme zaplatit). Z~toho plyne, že sloučenina, za kterou byste zaplatili
nejméně, musí mít nejvyšší následující poměr $r$: 
\[
r=\frac{z}{\text{cena}\cdot M}
\]
Pohledem na tabulku můžeme vyloučit vše, co stojí více než 5 korun.
Zbývají chlorid vápenatý, chlorid sodný a chlorid draselný. Spočítáme-li
hodnoty $r$ pro každou z~těchto látek, dostaneme: 
\begin{align*}
r(\ch{NaCl}) & =8,65\cdot10^{-3}\mathrm{~kg^{2}~mol^{-1}~CZK^{-1}}\\
r(\ch{KCl}) & =7,06\cdot10^{-3}\mathrm{~kg^{2}~mol^{-1}~CZK^{-1}}\\
r(\ch{CaCl2}) & =7,95\cdot10^{-3}\mathrm{~kg^{2}~mol^{-1}~CZK^{-1}}
\end{align*}

Jak je vidět, nejvyšší poměr $r$ je u~chloridu sodného, tedy pro
snížení teploty tuhnutí vody, zajímá-li nás pouze cena, je nejlepší
chlorid sodný\footnote{Gratulujeme všem řešitelům, kteří se zamysleli, čím se v~praxi solí silnice, a tento výpočet neprováděli.}.

\section{Kvantová chemie, fotochemie}

 % \subsection*{Ročník 2, úloha č. 6.2}
\begin{quotation}
\dva Běh na 100 metrů je jednou z~nejsledovanějších atletických disciplín.
Však se také držitel světového rekordu na této trati může pyšnit titulem
„nejrychlejší člověk planety“. Od roku 1912, kdy IAAF (Mezinárodní
atletická federace) začala světové rekordy evidovat, se na pozici
světového rekordmana vystřídalo více než 50~sportovců a nejlepší čas
na této trati se snížil o~více než sekundu. V~současnosti je držitelem
světového rekordu fenomenální jamajský sprinter Usain Bolt. Jaká byla
průměrná de Broglieho vlnová délka Usaina Bolta během závodu, v~němž
ustanovil světový rekord v~běhu na 100~metrů? Usain Bolt proběhl 16.
8. 2009 v~Berlíně tuto trať za 9,58~s. Uvažujte, že atlet při závodu
vážil 94~kg. Planckova konstanta je $6,626\cdot10^{-34}\,\mathrm{J\cdot s}$.
\end{quotation} \dotfill \par 
De Broglieho vlnová délka se počítá ze známého vztahu $\lambda=\frac{h}{p}$,
kde $h$ je zadaná Planckova konstanta a $p$ je hybnost. Z~mechaniky
víme, že hybnost je rovna součinu hmotnosti a rychlosti, a rychlost
spočítáme z\ dráhy a času. Vznikne vztah, ve kterém známe všechny
veličiny, takže můžeme dosadit. 
\[
\lambda=\frac{h\cdot t}{m\cdot s}=\frac{6,626\cdot10^{-34}\cdot9,58}{94\cdot100}=6,75\cdot10^{-37}\mathrm{\,m}
\]

Nutno podotknout, že takto spočítaná délka je o~dva řády menší než Planckova délka. Hodnota tak nemá vůbec žádný fyzikální smysl. Byla proto uznávána i odpověď, že de Broglieho vlnová délka Usaina Bolta je nulová.

\hrulefill % \subsection*{Ročník 2, úloha č. 2.6}
\begin{quotation}
\jeden Složitějším sledem reakcí se můžeme
dostat od sytě modré látky indigo -- systematicky 2,2’-bis(2,3-dihydro-3-oxoindolyliden) -- až k~molekule s~touto strukturou: 

\begin{center}
\includegraphics{images_new/2/2-6_8-6-eps-converted-to.pdf}
\par\end{center}

Tato látka má silný absorpční pás v~elektromagnetickém spektru kolem
vlnové délky 450 -- 460~nm. Jaké barvy budou krystalky této látky?
\end{quotation} \dotfill \par 
Uvedená vlnová délka odpovídá modré barvě. Krystalky budou mít barvu
doplňkovou k~této barvě, a to žlutooranžovou. Uznáváme i pouze žlutou.

\hrulefill % \subsection*{Ročník 2, úloha č. 3.6}
\begin{quotation}
\dva Látka popsaná v~předešlé úloze je zajímavá nejen svojí barvou, ale také
tím, že velice ráda fluoreskuje. Fluorescence
je jev, při kterém molekula absorbuje světlo či ultrafialové záření, které vybudí
elektron z~nejvyššího obsazeného orbitalu do vyššího (ne nutně nejbližšího)
neobsazeného orbitalu. Vzhledem k~tomu, že tento stav je energeticky
nevýhodný, chová se elektron jako zralé jablko nad Isaacem Newtonem.
Tedy párkrát se „zhoupne“ a tím mírně sníží svou energii tzv. nezářivými
přechody, pak velice rychle (v řádu nanosekund) padá na pomyslnou
Newtonovu hlavu -- do základního stavu. Vyberte z~možností 1 --
10 správné názvy jevů, znázorněných šipkami a, b, c v~energetickém,
tzv. Jablonského\footnote{správně Jabłońskiego, čti Jabuoňskjego} diagramu.

\begin{figure}[ht]
\begin{subfigure}{.6\textwidth}
\centering
\includegraphics[scale=0.5]{/2/3-6}
\label{fig:subim1}
\end{subfigure}
\begin{subfigure}{.4\textwidth}
\begin{tabular}{r|l} 
 
 1 & Excitace\\ 
 \hline
 2 & Usínání\\
 \hline
 3 & Padání\\
 \hline
 4 & Snoozing\\
 \hline
 5 & Relaxace\\
 \hline
 6 & Neosvětlený přechod\\
 \hline
 7 & Nezářivý přechod\\
 \hline
 8 & Newtonovská fluorescence\\
 \hline
 9 & Singletové štěpení\\
 \hline
 10 & Tripletová anihilace\\
\end{tabular}
\label{fig:subim2}
\end{subfigure}
% \label{fig:image2}
\end{figure}
\end{quotation} \dotfill \par 
Šipka \textbf{a} ukazuje excitaci elektronu (\textbf{1}), též buzení, do energeticky vyššího orbitalu. Šipka \textbf{b} pak nespecifikovaný nezářivý přechod (\textbf{7}) a nakonec \textbf{c} relaxaci (\textbf{5}) (též deexcitaci) elektronu.

\hrulefill % \subsection*{Ročník 2, úloha č. 8.6}
\begin{quotation}
\ctyri Pár úloh nazpátek jsme se vás ptali na barvu krystalků molekuly Cibalakrotu
(na obrázku), přičemž jste dostali vlnovou délku absorpčního maxima.
Jak by se ale tato vlnová délka dala spočítat či spíše odhadnout? 

\begin{center}
\includegraphics{/2/2-6_8-6}
\end{center}

Uvažujme následující model: 

\begin{enumerate}
    

\item  Uvažujme planární 2D molekulu. 

\item  Delokalizovaný systém $\pi$ elektronů má pro elektrony v~něm obsažené
nulový potenciál. Potenciál v~ostatním prostoru je nekonečně velký. (Jedná se o~nekonečně hlubokou potenciálovou jámu) 

\item  Modelu se účastní pouze $\pi$ elektrony nacházející se v~delokalizovaném
systému. Dva fenyly nenáleží delokalizovanému systému a jejich $\pi$
elektrony se modelu neúčastní. 

\item  Pro energii elektronu na hladině \textit{x}, \textit{y} platí: 
\[
E_{x,y}=\frac{h^{2}}{8m_{\mathrm{e}}}\cdot\left(\frac{n_{x}^{2}}{l_{x}^{2}}+\frac{n_{y}^{2}}{l_{y}^{2}}\right)
\]
kde $h$ je Planckova konstanta, $m_{\mathrm{e}}$ hmotnost elektronu, $n_{x},\;n_{y}$
jsou kvantová čísla s~hodnotami nabývajícími přirozených čísel, $l_{x},\;l_{y}$
jsou rozměry potenciálové jámy. 

\item  Absorpční maximum náleží přechodu mezi HOMO (Highest Occupied Molecular
Orbital) a LUMO (Lowest Unoccupied Molecular Orbital) elektronovými
hladinami. 

\item  V~molekulových orbitalech platí Pauliho vylučovací princip a Hundovo
pravidlo. 

\item  Delokalizovaný systém aproximujeme jako obdélník s~délkami stran
400 pm a 800 pm. 
\end{enumerate}
Vypočítejte, jaká vlnová délka náleží maximu absorpce elektronu podle
uvedeného modelu. 
\end{quotation} \dotfill \par 
V zadaném modelu se nachází celkem 11 elektronových párů. Je třeba
znát minimálně 12 nejnižších hladin tohoto 2D systému. Dosazením do
zadané rovnice pro energii hladin získáme následující pořadí hladin:

\noindent \begin{center}
\begin{tabular}{c|c|c|c}

č. & $n_{x}$ & $n_{y}$ & $E$/J$\cdot10^{19}$\tabularnewline
\hline 
\hline 
1 & 1 & 1 & 4,71\tabularnewline
\hline 
2 & 1 & 2 & 7,53\tabularnewline
\hline 
3 & 1 & 3 & 12,24\tabularnewline
\hline 
4 & 2 & 1 & 16,00\tabularnewline
\hline 
5 & 1 & 4 & 18,83\tabularnewline
\hline 
6 & 2 & 2 & 18,83\tabularnewline
\hline 
7 & 2 & 3 & 23,53\tabularnewline
\hline 
8 & 1 & 5 & 27,30\tabularnewline
\hline 
9 & 2 & 4 & 30,12\tabularnewline
\hline 
10 & 3 & 1 & 34,83\tabularnewline
\hline 
11 & 1 & 6 & 37,65\tabularnewline
\hline 
12 & 3 & 2 & 37,65\tabularnewline
\hline 
13 & 2 & 5 & 38,60\tabularnewline

\end{tabular}
\par\end{center}

HOMO (nejvyšší obsazené) jsou 2 degenerované orbitaly s~energií $37,65\cdot 10^{-19}\, \mathrm{J}$, LUMO (nejnižší neobsazený)
je orbital s~energií $38,60\cdot 10^{-19}\, \mathrm{J}$. Foton absorbovaný přechodem
mezi HOMO a LUMO má energii $9,5\cdot 10^{-20}\, \mathrm{J}$. To odpovídá vlnové délce
přibližně \SI[]{2}{\micro\metre}:
\[
\lambda = \frac{h\cdot c}{\Delta E} = \frac{6,64\cdot 10^{-34}\cdot 3\cdot 10^8 }{9,5 \cdot 10^{-20}} = 2,097 \cdot 10^{-6}\, \si{\metre}
\]

Z~rozdílu mezi experimentální hodnotou a vypočtenou
hodnotou lze vidět, že zvolený model byl naprosto nevhodný. 

\section{Fázové rovnováhy, další úlohy}

 % \subsection*{Ročník 5, úloha č. 4.2 }
\begin{quotation}
\tri Raw strava spočívá v~konzumaci tepelně neupravených potravin. Hlavním
argumentem jejích zastánců je, že nedochází k~tepelnému zničení živin
a enzymů. Za prahovou teplotu, kterou při tvorbě raw pokrmu nesmíme
překročit, se považuje 42~°C. Za jakého tlaku si můžeme v~pravém
slova smyslu uvařit vodu pro přípravu raw polévky? Použijte Clausiovu--Clapeyronovu
rovnici v~integrálním tvaru, jejíž tvar je následující: 
\[
\ln\left(\frac{p_{2}}{p_{1}}\right)=-\frac{\Delta_{\mathrm{výp}}H}{R}\left(\frac{1}{T_{2}}-\frac{1}{T_{1}}\right)
\]

kde $p_{i}$ je tlak nasycených par látky při teplotě $T_{i}$, $\Delta_{\mathrm{výp}}H$
je výparná enthalpie látky a $R$ je molární plynová konstanta. Výparnou
enthalpii vody uvažujte konstantní, a to $\Delta_{\mathrm{výp}}H(\ch{H2O})=40\,660\mathrm{~J\cdot mol^{-1}}$.
\end{quotation} \dotfill \par 
Clausiova--Clapeyronova rovnice popisuje závislost tlaku nasycených
par látky nad její vroucí kapalnou fází v~závislosti na teplotě.
V našem případě známe teplotu normálního bodu varu $T_{2}=373,15\,\mathrm{K}$,
která se vztahuje k~normálnímu tlaku $p_{2}=101\,325$~Pa. Chceme
zjistit tlak $p_{1}$ znajíce jemu příslušející teplotu $T_{1}=315,15$
K. Z~rovnice pak několika úpravami vyjádříme požadovaný tlak a dosadíme:
\[
\ln p_{2}-\ln p_{1}=-\frac{\Delta_{\mathrm{výp}}H}{R}\left(\frac{1}{T_{2}}-\frac{1}{T_{1}}\right)
\]
\[
p_{1}=\exp\left(\frac{\Delta_{\mathrm{výp}}H}{R}\left(\frac{1}{T_{2}}-\frac{1}{T_{1}}\right)+\ln p_{2}\right) =9082\mathrm{~Pa}
\]

\newpage %%
 % \subsection*{Ročník 2, úloha č. 6.5}
\begin{quotation}
\dva Kolikrát minimálně bude nutné provést destilaci směsi voda-ethanol
o molárním zlomku ethanolu $x = 0,1$, abychom získali alespoň nepatrné
množství směsi o~obsahu ethanolu minimálně 84 hmotnostních procent?
Rovnováha mezi plynnou a kapalnou fází\footnote{tj. jaké je složení plynné fáze nad kapalnou fází určitého složení} je zadána grafem.
\begin{center}
\includegraphics[scale=0.45]{images_new/2/6-5a.pdf}
\end{center}
\end{quotation} \dotfill \par 
Nejprve přepočteme požadovaný obsah ethanolu z~hmotnostních procent
na molární zlomek.

Ve 100 gramech směsi bude 84 g ethanolu a 16 g vody. Molární množství
ethanolu ve směsi je 
\[
n_\mathrm{{EtOH}}=\frac{m_\mathrm{{EtOH}}}{M_\mathrm{{EtOH}}}=\frac{84}{46}=1,826\,\mathrm{mol}
\]

Molární množství vody pak spočítáme analogicky:
\[
n_{\ch{H2O}}=\frac{16}{18}=0,889\,\mathrm{mol}
\]

Molární zlomek ethanolu pak lze spočítat následujícím způsobem:
\[
x_\mathrm{{EtOH}}=\frac{n_\mathrm{{EtOH}}}{n_\mathrm{{EtOH}}+n_{\ch{H2O}}}=\frac{1,826}{1,826+0,889}=0,673
\]

Při složení kapalné fáze, kde $x_{\mathrm{EtOH,poč}}=0,1$ je obsah
ethanolu v~parách cca 45 mol. \% (odečteno z~grafu). Pokud tato fáze
později zkondenzuje, mohu s~ní provést další destilaci. Molární zlomek
ethanolu v\ parách při složení kapalné fáze $x_{\mathrm{EtOH,p1}}=0,45$ je
zhruba 0,62-0,63. To je stále málo, budeme tedy muset provést ještě
jednu destilaci, kterou se v~ideálním případě dostaneme nad 70 mol.
\% ethanolu ve směsi.\\
Správná odpověď je tedy taková, že je třeba provést destilaci \textbf{třikrát.}\\
Úlohu lze řešit i graficky, způsobem vyobrazeným níže.

\begin{center}
\includegraphics[scale=0.45]{images_new/2/6-5b.pdf}
\end{center}


\hrulefill % \subsection*{Ročník 4, úloha č. 7.2   }
\begin{quotation}
\ctyri Známým pravidlem, kterým se popisuje chování lidí při hledání přátelských
vztahů, je otřepané „vrána k~vráně sedá“. Toto platí přeneseně i v
chemickém světě -- podobné substance se dobře vzájemně rozpouštějí,
vykazují analogickou reaktivitu a ve směsi se nechovají příliš odlišně
od čistých stavů. Například směs hexanu a heptanu se chová z~hlediska
fyzikálně-chemického téměř ideálně. Vypočítejte složení parní fáze
nad směsí 50 g kapalného hexanu a 50 g kapalného heptanu při teplotě
50 °C po dosažení rovnováhy. Tlak nasycených par čistých složek lze
spočítat z~Antoineovy rovnice: 
\[
\ln p_{\mathrm{s}}=A-\frac{B}{T+C}
\]
kde $T$ je teplota v~K, $p_{s}$ tlak sytých par v~kPa a \textit{A}, \textit{B} a \textit{C} jsou experimentální koeficienty, které máte zadány níže: 

Hexan: \textit{A} = 13,8216, \textit{B} = 2697,55, \textit{C} = --48,78 

Heptan: \textit{A} = 13,8744, \textit{B} = 2895,51, \textit{C} = --53,97
\end{quotation} \dotfill \par 

\textit{Pro celé řešení platí, že veličiny s~dolním indexem 6 náleží hexanu, veličiny pro heptan mají pak dolní index 7.}

Nejdříve je třeba vypočítat tlak sytých par obou látek při 50\,°C. 

\[
p_{\mathrm{s6}}=\exp\left({A-\frac{B}{T+C}}\right)=\exp\left({13,8216-\frac{2697,55}{323,15-48,78}}\right)=54,04\ \mathrm{kPa}
\]

\[
p_{\mathrm{s7}}=\exp\left({A-\frac{B}{T+C}}\right)=\exp\left({13,8744-\frac{2895,51}{323,15-53,97}}\right)=22,59\ \mathrm{kPa}
\]

Za předpokladu ideálního chování směsi platí Raoultův zákon pro každou
ze složek: 
\[
py_{6}=p_{\mathrm{s6}}x_{6}
\]
\[
py_{7}=p_{\mathrm{s7}}x_{7}
\]
kde $x_{\mathrm{i}}$ značí molární zlomek složky v~kapalné fázi a $y_{\mathrm{i}}$ značí molární zlomek složky v~plynné fázi. Víme, že složení kapalné
fáze je 50~g hexanu a 50~g heptanu, je třeba spočítat jeho molární
složení: 
\[
x_{6}=\frac{n_{6}}{n_{6}+n_{7}}=\frac{\frac{m_{6}}{M_{6}}}{\frac{m_{6}}{M_{6}}+\frac{m_{7}}{M_{7}}}
\]
\[
x_{6}=\frac{\frac{50}{86,18}}{\frac{50}{86,18}+\frac{50}{100,21}}=0,538
\]
\[
x_{7}=1-x_{6}=0,462
\]
Dále můžeme nahradit jeden ze zlomků v~plynné fázi druhým ze součtu
$y_{6}+y_{7}=1$. Dostaneme dvě rovnice pro dvě neznámé: $y_{6}$
a $p$.

\[
py_{6}=p_{\mathrm{s6}}x_{6}\tag{1} \label{eq:721}
\]
\[
p\left(1-y_{6}\right)=p_{\mathrm{s7}}x_{7}\tag{2} \label{eq:722}
\]
Dosazením $p$ z~rovnice \eqref{eq:721} do \eqref{eq:722} můžeme vyjádřit $y_{6}$: 
\[
\frac{p_{\mathrm{s6}}x_{6}}{y_{6}}\left(1-y_{6}\right)=p_{\mathrm{s7}}x_{7}
\]
\[
p_{\mathrm{s6}}x_{6}\left(\frac{1}{y_{6}}-1\right)=p_{\mathrm{s7}}x_{7}
\]
\[
\frac{1}{y_{6}}-1=\frac{p_{\mathrm{s7}}x_{7}}{p_{\mathrm{s6}}x_{6}}
\]
 
\[
\frac{1}{y_{6}}=\frac{p_{\mathrm{s7}}x_{7}}{p_{\mathrm{s6}}x_{6}}+1
\]
\[
y_{6}=\frac{1}{\frac{p_{\mathrm{s7}}x_{7}}{p_{\mathrm{s6}}x_{6}}+1}=\frac{1}{\frac{22,59\cdot0,462}{54,04\cdot0,538}+1}=0,736
\]
 
\[
y_{7}=1-y_{6}=1-0,736=0,264
\]
Složení parní fáze nad směsí je 73,6~\% hexanu a 26,4~\% heptanu. 

\hrulefill % \subsection*{Ročník 5, úloha č. 2.2   }
\begin{quotation}
\dva Práškový PVC se pro průmyslové potřeby neznačí střední molární
hmotností (jak by člověk očekával), ale takzvanou $K$ hodnotou. $K$
hodnota PVC je jeho experimentální charakteristika zjišťovaná v~technické
praxi následujícím postupem (citováno): 
\begin{enumerate}
\item Změříme viskozitu $\eta_{0}$ čistého cyklohexanonu. 
\item Změříme viskozitu $\eta$ cyklohexanonu s~rozpuštěným PVC. 
\item Dopočítáme tisícinásobek hodnoty $K$ z~rovnice 
\[
\log\left(\frac{\eta}{\eta_{0}}\right)=\frac{75K^{2}}{1+1,5K}+K
\]
\end{enumerate}
Vypočtěte, jaká viskozita ($\eta$) byla naměřena u~PVC ($1000K=65$),
jestliže viskozita čistého cyklohexanonu vyšla $\eta_{0}=2,5\,\mathrm{mPa\cdot\-s}$.
\end{quotation} \dotfill \par 
Text úlohy byl opsán z~návodu, který se používá při průmyslové praxi
v nejmenované firmě. Z~krkolomného textu je třeba dovodit, že jestliže
$1000K=65$, pak $K=0,065$, což můžeme dosadit do zadané rovnice a následně ji vyřešit.
\[
\log\left(\frac{\eta}{2,5}\right)=\frac{75\cdot0,065^{2}}{1+1,5\cdot0,065}+0,065
\]
\[
\log\left(\frac{\eta}{2,5}\right)=0,354
\]
\[
\left(\frac{\eta}{2,5}\right)=10^{0,354}=2,258
\]
\[
\eta=5,65\mathrm{\ mPa\cdot\-s}
\]


\chapter{Anorganická chemie}

\section{Nepřechodné kovy a polokovy}

 % \subsection*{Ročník 3, úloha č. 0.24 }
\begin{quotation}
\jeden Alkalické kovy jsou obecně velmi reaktivními substancemi, reagují
skoro se vším. Když hodíme sodík do vody, vzniká během bouřlivé reakce
hydroxid sodný a vodík. Když ale vhodíme sodík do ethanolu, reakce
probíhá o~poznání méně bouřlivě. 

Napište rovnici reakce kovového sodíku s~ethanolem. 
\end{quotation} \dotfill \par 
Sodík reaguje s~ethanolem v~acidobazické reakci za vzniku ethanolátu
sodného a vodíku.

\[
\mathrm{C_{2}H_{5}OH+Na\rightarrow C_{2}H_{5}ONa+\frac{1}{2}\,H_{2}}
\]


\hrulefill % \subsection*{Ročník 2, úloha č. 1.1}
\begin{quotation}
\jeden Reakce alkalických kovů s~vodou je evergreenem školních chemických
pokusů. Obvykle se do vody s~přidaným acidobazickým indikátorem fenolftaleinem
hodí kus sodíku, který pak začne rejdit po hladině, zanechávaje za
sebou fialovou stopu. Jiný alkalický kov, lithium, se dnes díky své
pozici na samém levém konci Beketovovy řady kovů používá v~lithium-iontových
akumulátorech (Li-Ion). Tyto baterie jsou ale poněkud citlivé na neopatrné
vnější zásahy: pokud porušíte jejich obal nebo je vystavíte vysoké
teplotě, riskujete požár, případně explozi. Vraťme se ale na začátek.
Co se stane, když hodíme kovové lithium do vody? Napište vyčíslenou
chemickou reakci.
\end{quotation} \dotfill \par 
Lithium reaguje s~vodou za vzniku hydroxidu lithného a vodíku. Vyčíslená chemická rovnice reakce je zde:

\[
\mathrm{Li+H_{2}O\rightarrow LiOH+\frac{1}{2}\,H_{2}}
\]

\hrulefill % \subsection*{Ročník 2, úloha č. 3.2}
\begin{quotation}
\jeden Alkalické  kovy se v~přírodě vyskytují pouze ve sloučeninách, jelikož jsou tak reaktivní,
že by v~elementárním stavu (na rozdíl třeba od zlata) nevydržely.
Nelze je získávat ani redukcí uhlíkem v~pecích tak jako železo, protože
na to jsou v\ Beketovově řadě kovů příliš nalevo. Musíme si tedy
pomoct jinak: vyrábíme je elektrolýzou. Nemůžeme však elektrolyzovat
vodné roztoky, protože alkalické kovy s~vodou reagují; proto elektrolyzujeme
přímo taveniny jejich sloučenin. Napište poloreakce, které probíhají na
katodě a anodě při elektrolýze taveniny chloridu rubidného. Uveďte
také, která poloreakce přísluší katodě a která anodě.
\end{quotation} \dotfill \par 
\newpage %text%
Jak je v~textu zadání napovězeno, v~tavenině nemůže vznikat nic jiného
než prvky, ze kterých se elektrolyzovaná sůl skládá. Dále je třeba
mít na paměti, že na anodě probíhá vždy oxidace. 

Poloreakce na anodě: 

\[
\mathrm{2\,Cl^{-}\rightarrow\ Cl_{2}+2\,e^{-}}
\]

Poloreakce na katodě:
\[
\mathrm{2\,Rb^{+}+2\,e^{-}\rightarrow\ 2\,Rb}
\]


\hrulefill % \subsection*{Ročník 4, úloha č. 3.4 }
\begin{quotation}
\jeden Při výrobě hliníku se v~uhlíkové elektrodové vaně (samotná vana je
elektrodou) elektrolyzuje tavenina surové rudy hliníku (pro zjednodušení
uvažujme, že rudou je pouze oxid hlinitý), do níž jsou shora ponořeny
uhlíkové elektrody. Tyto elektrody při elektrolýze ubývají a je potřeba
je v~průběhu času vyměňovat. Napište jednotlivé poloreakce na obou
elektrodách a napište, zda je elektrodová vana katodou, či anodou.
\end{quotation} \dotfill \par 
Na katodě, kterou je elektrodová vana, probíhá redukce hliníku. Ten
pak můžeme jednoduše z~prostoru odpouštět, protože je těžší než samotná
tavenina.

\[
\ce{2Al^{3+}}+6\,\mathrm{e^{-}}\rightarrow\ce{2Al}
\]

Na anodě probíhá elektrolytické spalování uhlíku:

\[
\ce{3O^{2-}}+\ce{3C}\rightarrow\ce{3CO}+6\,\mathrm{e^{-}}
\]
Bylo též uznáno, pokud jste napsali, že na anodě vzniká oxid uhličitý.
Pouhý vznik kyslíku uznáván nebyl vzhledem k~tomu, že v~zadání bylo
definováno, že uhlíkové elektrody během elektrolýzy ubývají.

\hrulefill % \subsection*{Ročník 4, úloha č. 6.1 }
\begin{quotation}
\tri Asi 5 g práškovitého kysličníku \textbf{X} prvku, který je druhým
nejhojněji zastoupeným prvkem v~zemské kůře, zahřejeme ve zkumavce
s~8~g práškového kovu \textbf{A}, jenž se často používá při reakcích
dle Grignarda. Zkumavku přitom nedržme rukou, nýbrž zavěsme ji do
nějakého držáku a to ještě ústím od obličeje, ježto směs zahříváním
náhle se rozžhavuje a stříká někdy až ven. Při rozžhavení směsi vznikne
vedle kysličníku \textbf{B} i látka \textbf{C}, která jest dvouprvkovou
sloučeninou. Potom zkumavku rozbijeme a obdrženou hmotu vpravíme na
dno vyvíjecí láhve, volný konec plynovodné trubice od této láhve vnoříme
do misky s~vodou, načež celou láhev i s~plynovodnou trubicí naplníme
skrze nálevku zúplna vodou, abychom tak vypudili z~přístroje všechen
vzduch. Pak ke hmotě \textbf{C} na dně láhve přilijeme nálevkou koncentrované
kyseliny solné. Tím začne se vyvinovati plyn, jenž vycházeje bublinami
z vody v~misce ven, ve styku se vzduchem se ihned spaluje, při čemž
se po každé bublině tvoří bělavý prsténec spálením vzniklého kysličníku
\textbf{X}\footnote{Podle R. Kout: Navedení k~chemickým pokusům. Knihovna přírody a školy, Mor. Ostrava 1906}.

Pojmenujte látky \textbf{A, B, C, X} a napište jejich racionální vzorce.
Nápověda: Kysličník \textbf{X} je v~jedné ze svých alotropických modifikací
též znám jako drahokam křišťál.
\end{quotation} \dotfill \par 
Látkou \textbf{X} je oxid křemičitý $\ce{SiO2}$. Ten reaguje s~látkou
\textbf{A} -- práškovým hořčíkem -- za vzniku oxidu hořečnatého MgO
(látka \textbf{B}) a silicidu hořečnatého $\ce{Mg2Si}$ (látka \textbf{C}). Silicid
hořečnatý po reakci s~HCl uvolňuje silan $\ce{SiH4}$, který ihned
reaguje s~kyslíkem ve vzduchu za vzniku $\ce{SiO2}$ a $\ce{H2O}$.

\newpage %nadpis%
\section{Přechodné a vnitřně přechodné prvky}

 % \subsection*{Ročník 4, úloha č. 0.12   }
\begin{quotation}
\jeden Trendem poslední doby je klást velký důraz na stravování a složení
konzumovaných potravin -- „vím, co jím“. V~chemické laboratoři se
toto pravidlo samozřejmě zapovídá a nahrazuje se univerzálnějším „vím,
co dělám“. To je v~tomto okamžiku důležitější, jelikož následující
sloučeninu byste rozhodně příliš ochutnávat nechtěli. Co o~ní víme?
\begin{itemize}
\item Jde o~oranžovou pevnou látku.
\item Ve vodném roztoku přidáním hydroxidu mění barvu do žluté.
\item V~anorganické, organické i analytické chemii se používá jako oxidační
činidlo.
\item Látka je klasifikována jako vysoce toxická a karcinogenní.
\item Kov, který je přítomen v~kationtu, má nejstabilnější izotop s~nukleonovým
číslem 39.
\end{itemize}
Zapište vzorec či název této sloučeniny.
\end{quotation} \dotfill \par 
Jedná se o~dichroman draselný, $\ce{K2Cr2O7}$.

\hrulefill % \subsection*{Ročník 5, úloha č. 0.23 }
\begin{quotation}
\jeden V~dnešní době analytická chemie spoléhá především na data z~přístrojů.
Historicky však chemici používali především metody senzorické (t.j.
své smysly). Ochutnávání je však dnes už zakázáno, ani čichová analýza
není příliš doporučená. Zatím se však můžeme na chemické látky dívat
a je vskutku několik chemikálií, které můžeme poznat na první pohled.
Ve školním skladu chemikálií jste na polici nalezli čtyři lahvičky
s pevnými látkami následujících barev:

žlutá/zlatá, fialová, oranžová, modrá.

Popisky lahviček už jsou dávno stržené, ale podle seznamu chemikálií
víte, že na dané polici by se měly nacházet následující sloučeniny:

dihydrát síranu vápenatého, pentahydrát síranu měďnatého, manganistan
draselný, dusičnan sodný, dichroman draselný, jodid olovnatý, heptahydrát
síranu železnatého, uhličitan vápenatý.

Určete, která chemikálie se nachází v~které lahvičce.
\end{quotation} \dotfill \par 
Správné dvojice jsou:

žlutá/zlatá -- jodid olovnatý; fialová -- manganistan draselný;
oranžová -- dichroman draselný; modrá -- pentahydrát síranu měďnatého

\hrulefill % \subsection*{Ročník 2, úloha č. 6.3}
\begin{quotation}
\ctyri Pozoruhodná komplexní sloučenina kobaltu má následující elementární
složení: 3,84~\%~H, 71,20~\%~N, 24,96~\%~Co. Její molární hmotnost
je 472,17 $\mathrm{g\cdot mol^{-1}}$. Zapište její racionální vzorec, víte-li, že obsahuje komplexní kation s~nábojem 3+ a komplexní anion s~nábojem opačným.
\end{quotation} \dotfill \par 
Hmotnostní procenta jednotlivých prvků podělíme jejich molárními hmotnostmi,
abychom dostali poměr, ve kterém jsou jednotlivé atomy vůči sobě:

\[
n_{\mathrm{H}}:n_{\mathrm{N}}:n_{\mathrm{Co}}=3,8095:5,0832:0,4236
\]
Poměr upravíme vydělením všech čísel číslem 0,4236, abychom dostali
celočíselné koeficienty:
\[
n_{\mathrm{H}}:n_{\mathrm{N}}:n_{\mathrm{Co}}=9:12:1
\]
Empirický vzorec zkoumaného komplexu je tedy $\ch{H9N12Co}$.
Molekulová hmotnost tomu odpovídající je 236,085. Pokud podělíme touto
hmotností molekulovou hmotnost komplexu, zjistíme, kolikrát musíme
vynásobit koeficienty empirického vzorce, abychom dostali vzorec sumární.
$\frac{472,17}{236,08}\doteq2$, sumární vzorec neznámého komplexu
je tedy $\mathrm{Co_{2}N_{24}H_{18}}$. Protože kation i anion jsou
komplexní, hledáme komplex s~racionálním vzorcem $\mathrm{[Co(X)_{n}][Co(Y)_{m}]}$. 

Nejstabilnějším oxidačním stavem kobaltu je +III. Předpokládejme tedy,
že se kobalt komplexu účastní ve formě $\mathrm{Co^{3+}}$ kationtů.

Kobalt tvoří stabilní oktaedrické komplexy. V~těchto komplexech má
kobalt koordinační číslo 6, většinou je na něj tedy navázáno šest
ligandů. Můžeme tedy předpokládat $n=m=6$.

Jedním z~ligandů bude pravděpodobně neutrální amin $\mathrm{NH_{3}}$.
Ten se může šestkrát navázat na kobaltitý kation, výsledný náboj komplexního
kationtu $\mathrm{[Co(NH_{3})_{6}]}$ bude 3+. Po odečtení šesti dusíků
a osmnácti uhlíků nám v~sumárním vzorci jako materiál pro ligandy
aniontu zbude pouze 18 atomů dusíku. Komplexní kobalt bude mít opět
nejpravděpodobněji ox. č. 3+, tudíž každý z~atomů dusíku by měl mít
náboj $-\nicefrac{1}{3}$, nebo také skupina $\mathrm{N_{3}}$ musí
mít náboj $-1$. Ligandem je tedy azidový anion  $\mathrm{N_{3}^{-}}$.
Naším komplexním aniontem je potom tedy $\mathrm{[Co(N_{3})_{6}]^{3-}}$.  Hledaným komplexem je $\mathrm{[Co(NH_{3})_{6}][Co(N_{3})_{6}]}$.

\hrulefill % \subsection*{Ročník 1, úloha č. 8.4}
\begin{quotation}
\dva Nedávno byla vydána nová směrnice IUPAC, která nezávazně nabádá členské
státy, aby přistoupily k~úpravě názvoslovných směrnic pro anorganickou
chemii. Čeští (i další) chemici jsou však tvorové tvrdohlaví, a proto
na směrnici nereagují a mnohem raději používají názvosloví, které
vyjadřuje oxidační stavy za pomocí koncovek. To však není předmětem
úlohy. Ještě raději totiž čeští chemici používají názvosloví triviální.
Prokážete, že jste stejně dobří jako průměrný český chemik… nebo ještě
lepší? Přiřaďte triviální názvy ke vzorcům. 

\noindent \begin{center}
\begin{tabular}{r|c||r|c}

\multicolumn{2}{c||}{Vzorec}  &\multicolumn{2}{c}{Název}\\
\hline 
\hline 
1 &$\mathrm{ZnSO_{4}\cdot7\,H_{2}O}$ & a & žlutá krevní sůl\tabularnewline
\hline 
2&$\mathrm{FeSO_{4}\cdot7\,H_{2}O}$ & b & kryolit\tabularnewline
\hline 
3& $\mathrm{Na_{2}SO_{4}\cdot10\,H_{2}O}$ & c & Glauberova sůl\tabularnewline
\hline 
4&$\mathrm{Fe_{4}[Fe(CN)_{6}]_{3}}$ & d & sublimát\tabularnewline
\hline 
5&$\mathrm{K_{3}[Fe(CN)_{6}]}$ & e & minium\tabularnewline
\hline 
6&$\mathrm{K_{4}[Fe(CN)_{6}]}$ & f & bílá skalice\tabularnewline
\hline 
7&$\mathrm{Hg_{2}Cl_{2}}$ & g & kalomel\tabularnewline
\hline 
8&$\mathrm{HgCl_{2}}$ & h & berlínská modř\tabularnewline
\hline 
9& $\mathrm{Na_{3}[AlF_{6}]}$ & i & červená krevní sůl\tabularnewline
\hline 
10&$\mathrm{Pb_{3}O_{4}}$ & j & zelená skalice\tabularnewline
\end{tabular}
\end{center}
\end{quotation} \dotfill \par 
Správně přiřazené dvojice včetně doplněného systematického názvu: 

\begin{center}
\begin{tabular}{c|c|c}
1 & f & heptahydrát síranu zinečnatého \\ \hline
2 & j & heptahydrát síranu železnatého \\ \hline
3 & c & dekahydrát síranu sodného \\ \hline
4 & h & hexakyanoželeznatan železitý \\ \hline
5 & i & hexakyanoželezitan draselný \\ \hline
6 & a & hexakyanoželeznatan draselný \\ \hline
7 & g & chlorid rtuťný \\ \hline
8 & d & chlorid rtuťnatý \\ \hline
9 & b & hexafluoridohlinitan sodný \\ \hline
10 & e & oxid olovnato-olovičitý (stechiometricky 2:1) \\ 
\end{tabular}
\end{center}

\hrulefill % \subsection*{Ročník 3, úloha č. 8.4   }
\begin{quotation}
\ctyri Oxidace iontů Ag$^{+}$ pomocí silných oxidačních činidel v~prostředí
správných ligandů může vést k~tvorbě komplexů stříbra s~vyšším oxidačním
stavem. Pro syntézu a analýzu komplexu \textbf{Z} byly provedeny následující
procedury: Vodný roztok obsahující 0,500~g AgNO$_{3}$ a 2~ml pyridinu
($\rho=0,982\,\mathrm{g\cdot ml^{-1}}$\SI[inter-unit-product = \ensuremath{{}\cdot{}},output-decimal-marker = {,}]{0,982}{\gram\per\ml}) byl přidán do míchaného ledově
vychlazeného roztoku obsahujícího 5,000~g K$_{2}$S$_{2}$O$_{8}$.
Reakční směs po čase zežloutla, poté z~roztoku vypadla oranžová sraženina
\textbf{Z} o~hmotnosti 1,719~g. Elementární analýza látky \textbf{Z}
poskytla hmotnostní procenta C, H, N: 38,96 \%, 3,28 \%, 9,09 \%.

0,6164~g látky \textbf{Z} bylo přidáno do vodného roztoku čpavku.
Suspenze byla přivedena k~varu a byla vařena do úplného vymizení sraženiny.
Roztok byl okyselen přebytkem vodného roztoku HCl a vytvořená sraženina
byla zfiltrována, vysušena a zvážena. Hmotnost sraženiny byla 0,1433~g.
Do filtrátu byl přidán nadbytek BaCl$_{2}$ za vzniku 0,4668~g bílé
sraženiny. Měřením magnetického momentu látky \textbf{Z} byl zjištěn
magnetický moment odpovídající jednomu nepárovému elektronu v~centrálním
atomu. Napište racionální vzorec (např. {[}Cu(NH$_{3}$){]}$_{4}$SO$_{4}$)
komplexní sloučeniny \textbf{Z}, pokud víte, že obsahuje pouze jeden
typ stříbra a jeden typ ligandu.
\end{quotation} \dotfill \par 
Je vhodné nejprve určit empirický vzorec sloučeniny \textbf{Z}. Hmotnostní
procenta uhlíku, vodíku a dusíku máme zadána, potřebujeme k~tomu přidat
údaje pro další prvky. Zaměříme se tedy na pokus s~0,6164~g látky
\textbf{Z}. V~prvním případě se vyloučil chlorid stříbrný, z~jeho
množství můžeme zjistit látkové množství stříbra v~odebrané části
sraženiny. Obdobně budeme postupovat v~případě druhé sraženiny, kdy
se jedná o~BaSO$_{4}$.

\[
n_{\mathrm{Ag}}=\frac{m_{\mathrm{AgCl}}}{M_{\mathrm{AgCl}}}=\frac{0,1433}{143,1}=0,001\mathrm{\,mol}
\]

\[
n_{\ch{SO4}}=\frac{m_{\ch{SO4}}}{M_{\ch{SO4}}}=\frac{0,4668}{233,4}=0,002\mathrm{\,mol}
\]

Ze znalosti hmotnosti odebrané ze sraženiny, látkových množství a
molárních hmotností stříbra a síranového aniontu můžeme určit hmotnostní
zlomky těchto species v~látce \textbf{Z}. 

\[
w_{\mathrm{Ag}}=\frac{n_{\mathrm{Ag}}\cdot M_{\mathrm{Ag}}}{m_{Z}}=\frac{0,001\cdot107,87}{0,6164}=0,175
\]

\[
w_{SO_{4}}=\frac{n_{SO_{4}}\cdot M_{SO_{4}}}{m_{Z}}=\frac{0,002\cdot96,06}{0,6164}=0,3117
\]

Nyní můžeme určit poměr stříbra, síranových aniontů, uhlíku, vodíku
a dusíku v~molekule \textbf{Z}, což nám poskytne empirický vzorec
látky \textbf{Z}:

\[
\mathrm{Ag:SO_{4}:C:H:N}=\frac{0,175}{107,87}:\frac{0,3117}{96,06}:\frac{38,96}{12,01}:\frac{3,28}{1,01}:\frac{9,09}{14,01}=1:2:20:20:4
\]

Empirický vzorec molekuly Z~tedy je: Ag(SO$_{4}$)$_{2}$C$_{20}$H$_{20}$N$_{4}$.

Nyní musíme zapojit chemickou intuici. Ze zadání víme, že centrální
atom stříbra má jeden nepárový elektron a jelikož celá úloha pojednává
o oxidaci stříbra, tak bude stříbro v~oxidačním stavu +II. Z~empirického
vzorce dále vidíme, že látka Z~obsahuje 2 ekvivalenty síranových aniontů.
Jelikož se ale látka připravovala z~peroxodisíranu, bude aniontem v~molekule bude  peroxodisíranový anion: $\ch{S2O8^{2-}}$.
Nyní zbývá dořešit, čemu odpovídá skupina atomů C$_{20}$H$_{20}$N$_{4}$.
Víme, že při přípravě se přidával do reakce pyridin. Sumární vzorec
pyridinu je C$_{5}$H$_{5}$N, veškeré uhlíky, vodíky a dusíky tak
odpovídají čtyřem pyridinům koordinovaným na centrální atom stříbra.

Výsledný racionální vzorec molekuly tedy je: {[}Ag(py)$_{4}${]}S$_{2}$O$_{8}$

\hrulefill % \subsection*{Ročník 4, úloha č. 8.5 }
\begin{quotation}
\ctyri Redukcí hexachloridotechnecičitanu $\ce{(Et4N)2[TcCl6]}$ plynným
vodíkem v~koncentrované kyselině bromovodíkové lze připravit sůl
$\ce{(Et4N)2[\{Tc6(\mu-Br)6Br6\}Br2]}$, která obsahuje elektroneutrální
cluster $\ce{[Tc6(\mu-Br)6Br6]}$. Rentgenostrukturní analýzou byla
zjištěna trigo\-nálně-prizmatická struktura s~výrazně odlišnými délkami
vazeb Tc--Tc. 
\begin{center}

\includegraphics[scale=0.35]{Tc.pdf}

\par\end{center}
Navrhněte na základě počtu elektronů, které mají atomy Tc k~dispozici,
celočíselné řády vazeb Tc--Tc v~této molekule. Předpokládejte, že
vazby Tc--Tc mají charakter dvouelektronových bicentrických vazeb,
elektrony do vazeb Tc--Br pocházejí výhradně od ligandů a na atomech
Tc nejsou volné elektronové páry.
\end{quotation} \dotfill \par 
\newpage %text%
Technecium je ve stejné skupině jako mangan, cluster $\mathrm{Tc}_{6}^{12+}$
proto obsahuje $6\cdot7-12=30$ valenčních elektronů. Každý fragment Tc tedy vytváří pět vazeb
k sousedním Tc-fragmentům. Jelikož jsou délky všech vazeb Tc-Tc v~podstavách hranolu (trojúhelníkových plochách) identické (2,66 Å),
uvažujeme pro tyto vazby stejný řád. Vazby mezi podstavami jsou výrazně
kratší a očekáváme u~nich vyšší řád vazby. Výsledkem této úvahy je
přiřazení trojných vazeb třem vazbám mezi rovinami a jednoduchých
šesti vazbám v~podstavách. Z~každého fragmentu Tc tak vychází pět
vazeb typu kov-kov\footnote{Úloha převzata z~Gade, L. H., Koordinationschemie. Wiley-VCH: Weinheim,
1998. ISBN 978-3-527-29503-6 DOI: \href{https://onlinelibrary.wiley.com/doi/book/10.1002/9783527663927}{\underline{10.1002/9783527663927}}.

Primární literatura: Spitzin, V. I., Kryutchkov, S. V., Grigoriev,
M. S. and Kuzina, A. F. (1988), Polynuclear clusters of technetium.
I. Synthesis, crystal and molecular structure of bromide octanuclear
prismatic and hexanuclear octahedral clusters of technetium. \textit{Z. anorg.
allg. Chem.}, 563: 136-152. DOI: \href{https://doi.org/10.1002/zaac.19885630118}{\underline{10.1002/zaac.19885630118}}

Symmetric vs. asymmetric linear M-X-M linkages in molecules, polymers,
and extended networks Ralph A. Wheeler, Myung Hwan. Whangbo, Timothy.
Hughbanks, Roald. Hoffmann, Jeremy K. Burdett, and Thomas A. Albright
\textit{J. Am. Chem. Soc.}, 1986, 108 (9), 2222-2236 DOI: \href{https://doi.org/10.1021/ja00269a018}{\underline{10.1021/ja00269a018}}
}
\noindent \begin{center}

\includegraphics[scale=0.75]{/4/8-5-1}

\par\end{center}

\hrulefill % \subsection*{Ročník 3, úloha č. 7.5  }
\begin{quotation}
\dva Manganometrie je mezi analytickými metodami tak trochu evergreenem.
V současnosti jsou používány metody mnohem přesnější, titrace intenzivně
barvícím fialovým roztokem si ale vyzkoušel snad každý student chemie
nejpozději v~laboratořích na vysoké škole. My se dnes ale nebudeme
zabývat běžnými reakcemi v~kyselém prostředí, kdy se manganistan redukuje
na manganaté kationty, a představíme si některá méně známá manganometrická
stanovení a jejich aspekty. Kupříkladu lze manganometricky stanovit
kyselinu mravenčí či její soli -- formiáty, a to v~bazickém prostředí
(\textbf{1}). Oxidovaný produkt této reakce zůstává za velmi bazických podmínek
v roztoku jako uhličitanový anion, manganistanový anion podléhá pouze
jednoelektronové redukci. Vzniklý anion mangananový však nemusí v
roztoku zůstat -- dochází k~jeho pozvolnému rozkladu (\textbf{2}), kdy se
roztok začíná zvolna hnědě zakalovat a zároveň zabarvovat.\\Vyčíslenými
rovnicemi popište průběh stanovení (\textbf{1}) a děje (\textbf{2}). 
\end{quotation} \dotfill \par 
V textu k~úloze je rozeseto velké množství nápověd, které by měly
řešitele poměrně rychle dovést ke správnému sestavení rovnic. Pro
první děj známe většinu reaktantů, některé produkty a prostředí, ve
kterém probíhá, a není tak těžké sestavit rovnici

\[
\mathrm{2\,KMnO_{4}+HCOOK+3\,KOH\rightarrow2\,K_{2}MnO_{4}+K_{2}CO_{3}+2\,H_{2}O}
\]

Druhý děj je také bystřejším mladým chemikům známý -- manganany totiž
nejsou v~roztocích příliš stálé a dochází k~jejich postupné disproporcionaci
na burel a manganistan.

\[
\mathrm{3\,K_{2}MnO_{4}+2\,H_{2}O\rightarrow2\,KMnO_{4}+MnO_{2}+4\,KOH}
\]

Uznáván je i správný iontový zápis obou rovnic. 

\newpage %nadpis%
\section{Dusík a jeho sloučeniny}

% \subsection*{Ročník 3, úloha č. 3.3 }
\begin{quotation}
\dva Od Cimrmanova úspěšného pokusu o~velice krátkou chemickou anekdotu
„H$_{2}$SO$_{5}$“ uplynulo již mnoho let. Autoři Chemiklání se však
nyní pokusí vyrovnat Jeho géniu a obdobně jako myslitelé divadla Járy
Cimrmana se vynasnaží vytvořit anekdotu s~chemickou tematikou a stejně
vtipnou. A ta tedy zní: „HNO$_{4}$!“ Všimněte si! Jde o~ještě kratší
anekdotu. V~zájmu pravdy je nutno říci, že tato sloučenina rovněž
existuje. Pojmenujte tuto sloučeninu a nakreslete její strukturní
elektronový vzorec. Nezapomeňte na volné elektronové páry. 
\end{quotation} \dotfill \par 
Jedná se o~kyselinu peroxodusičnou. Její vzorec a rezonanční struktury
přikládáme níže:
\noindent \begin{center}

\includegraphics{/3/3-3}

\par\end{center}

\hrulefill % \subsection*{Ročník 4, úloha č. 2.5 }
\begin{quotation}
\jeden Největší potíže při kreslení elektronových vzorců činí středoškolákům
nepochybně dusík. Narozdíl od ostatních prvků 15. (dříve V.A) skupiny
nemůže být pětivazný, protože má k~dispozici pouze 4 valenční orbitaly\footnote{Jako protiargument je možné uvést struktury diazomethanu či azoxidu bez formálních nábojů. Pro hlubší vhled odkážeme na diskusi o~teorii valenční vazby, 
\textit{Chem. Soc. Rev.}, 1997, \textbf{26}, 87-100, DOI: 
\href{https://doi.org/10.1039/CS9972600087}{\underline{10.1039/CS9972600087}}}.
Sloučeniny dusíku s~oxidačním číslem +V však běžně existují. Nakreslete strukturní elektronový (Lewisovský) vzorec jedné z~nejznámějších sloučenin dusíku: oxidu dusičného. 
\end{quotation} \dotfill \par 
Oxid dusičný má racionální vzorec $\ce{N2O5}$. Jeho strukturní elektronový
vzorec je zde:
\noindent \begin{center}

\includegraphics{/4/2-5}

\par\end{center}

\hrulefill % \subsection*{Ročník 3, úloha č. 0.13 }
\begin{quotation}
\jeden Anorganická sůl obsahující pouze dusík, vodík a kyslík v~hmotnostních
poměrech 10,8 : 1,54 : 18,5 byla rozpuštěna ve vodě za vzniku bezbarvého
roztoku. Po přilití koncentrovaného louhu se z~roztoku začal uvolňovat
páchnoucí plyn, který barvil navlhčený univerzální indikátorový papírek
modře. Napište vzorec tohoto plynu.
\end{quotation} \dotfill \par 
Zmíněný plyn je amoniak, česky čpavek. Vzorec plynu je $\mathrm{NH_{3}}.$

\hrulefill % \subsection*{Ročník 1, úloha č. 4.1 }
\begin{quotation}
\dva Oxid dusný (jinak též azoxid) je znám pro své anestetické účinky.
Ještě známější (a to i pro nechemiky) je jeho schopnost vyvolávat
nekontrolovaný smích, z~čehož má svůj triviální název rajský plyn.
Pojďme se na něj podívat trochu více zblízka: nakreslete rezonanční
struktury této molekuly a určete její tvar.
\end{quotation} \dotfill \par 
\begin{center}
\includegraphics{/1/4-1}
\end{center}

Molekula $\mathrm{N_{2}O}$ je lineární a některé její atomy vždy
nesou formální náboj\footnote{Trváme-li na dodržení oktetového pravidla, vizte předchozí poznámku.}. K~výsledné struktuře molekuly více přispívají
ty struktury, v~nichž je součet velikostí nábojů menší, tj. vlevo
a uprostřed. Vpravo je zobrazena další možná, ale energeticky méně
výhodná struktura. 

\newpage %nadpis%
\section{Halogeny a jejich sloučeniny}

% \subsection*{Ročník 3, úloha č. 1.5 }
\begin{quotation}
\jeden Brněnští kriminalisté potřebují pomoc s~řešením případu mrtvé uklízečky
nalezené ve sprchách nejmenovaných kolejí. Na místě činu byly nalezeny
dvě převržené otevřené lahve. V~jedné byl známý čisticí prostředek
obsahující chlornan sodný, zatímco ve druhé byl čistič proti vodnímu
kameni obsahující kyselinu chlorovodíkovou. Zjistilo se, že se uklízečka
udusila jedovatým plynem. Napište chemickou rovnici vzniku tohoto
jedovatého plynu. 
\end{quotation} \dotfill \par 
Vzniká chlor, rovnice jeho vzniku z~výše popsaných chemikálií:

\[
\mathrm{NaClO+2\,HCl\rightarrow Cl_{2}+NaCl+H_{2}O}
\]


\hrulefill % \subsection*{Ročník 3, úloha č. 2.4 }
\begin{quotation}
\jeden Napište a vyčíslete rovnici chemické reakce, jejímž jediným produktem
je NaCl, tedy kuchyňská sůl. Reakci nezapisujte iontovými rovnicemi.
Reaktanty musí existovat jako samostatné entity.
\end{quotation} \dotfill \par 
Jediná reakce, která splňuje výše uvedené požadavky, je syntéza NaCl
z prvků, tedy:

\[
\mathrm{2\,Na+Cl_{2}\rightarrow2\,NaCl}
\]


\hrulefill % \subsection*{Ročník 1, úloha č. 2.1}
\begin{quotation}
\jeden Jste v~laboratoři a máte na stole tři kádinky, o~nichž víte, že obsahují
vodné roztoky amoniaku (25\%), kyseliny chlorovodíkové a chlornanu
sodného. Určete, který roztok je ve které kádince, víte-li, že po
přilití obsahu kádinky č. 3 do kádinky č. 1 se uvolnil žlutozelený
nebezpečný plyn, a že v~místě styku par z~kádinek č. 2 a č. 3 vznikl zdánlivě ze vzduchu bílý prášek.
\end{quotation} \dotfill \par 
Žlutozelený nebezpečný plyn je chlor, který vzniká synproporcionací
chloridu a chlornanu. Bílým práškem vznikajícím reakcí par je chlorid
amonný (triviálně salmiak). Kyselina chlorovodíková je tedy obsažena
v~roztoku v~kádince č. 3, v~kádince č. 1 je chlornan sodný (oxidační
činidlo) a roztok amoniaku je v~kádince č. 2. 

\hrulefill % \subsection*{Ročník 5, úloha č. 4.4 }
\begin{quotation}
\dva Soli kyseliny chlorné jsou oxidační činidla, široce používaná pro
dezinfekci vody k~přímé spotřebě a v~plaveckých bazénech, kde téměř
úplně nahradila přímé ošetření vody plynným chlorem. V~záznamech
o průmyslových haváriích ale přesto najdeme několik událostí, kdy
došlo k~vývinu oblaku chloru v~důsledku neúmyslného smísení chlornanu
s nějakou kyselinou. Dvě takové nehody se staly v~rozmezí přibližně
jednoho roku: 8. dubna 1983 v~Knoxville ve státě Tennessee pracovníci
čistírny odpadních vod omylem přidali do nádrže s~chlornanem 600
galonů vodného chloridu železitého, událost se obešla bez zranění.
V~listopadu dalšího roku dovezla automobilová cisterna z~okolí Manchesteru
roztok chloridu železitého k~odběrateli v~hrabství Yorkshire. Řidič
obdržel nesprávnou dokumentaci a v~cíli náklad přečerpal do nádrže
s~chlorovým bělidlem. Celkem 29 osob muselo být ošetřeno.

Zapište chemickou rovnici reakce, ke které došlo v~provozech; železo
při ní nepodléhá redoxním změnám. Místo chlornanu uvažujte vodný roztok
kyseliny chlorné.
\end{quotation} \dotfill \par 

Dojde k~synproporcionaci za vzniku plynného chloru a oxidu železitého.
\[
\ch{FeCl3 (aq) + 3 HClO(aq)} \rightarrow \ch{3 Cl2 (g) + Fe(OH)3 (s)}
\]


\hrulefill % \subsection*{Ročník 3, úloha č. 6.2  }
\begin{quotation}
\tri Ačkoli je periodická tabulka cenným nástrojem pro vyhledávání trendů,
které mnohdy mohou studenta navést ke správnému vyřešení testové úlohy,
není radno spoléhat na podobné analogie vždy a bez výjimky. Krásným
případem jsou trendy ve skupině halogenů -- ať už zavádíme chlor
do studené vody nebo alkalického roztoku, bude vždy vzorně disproporcionovat
na kyselinu chlorovodíkovou a kyselinu chlornou, popřípadě jejich
soli. Oproti tomu zavádění fluoru do studené vody sice vede ke vzniku
jedné z~kyselin obsahující fluor \textbf{A}, druhým produktem je ale
plynný prvek \textbf{X} mírně těžší než vzduch. Reakce analogická
disproporcionaci chloru probíhá až za velmi nízkých teplot, kdy se
opatrně pokoušíme fluorovat led při teplotě $-40\,{^\circ}\mathrm{C}$,
kde jsou produkty dvě sloučeniny fluoru \textbf{A} a \textbf{B}. Fluor
probublávaný roztokem hydroxidu sodného poskytuje kromě vody také
sůl kyseliny \textbf{A} -- látku \textbf{D}, ale také velmi toxickou
binární sloučeninu \textbf{C} se stechiometrií LM$_{2}$, kterou lze
formálně považovat za anhydrid kyseliny \textbf{B}. Identifikujte
látky \textbf{A, B, C, D} a prvek \textbf{X}, které ve výše popsaných
dějích vznikají.
\end{quotation} \dotfill \par 
Látka \textbf{A} je kyselina fluorovodíková, HF.

Látka \textbf{B }je sloučenina HOF. Formálně je však problém nazývat
tuto látku kyselinou fluornou, fluor zde má totiž oxidační číslo -I,
kyslík zde má oxidační číslo rovné nule.

Látka \textbf{C }je fluorid kyselnatý (difluorid kyslíku), OF$_{2}$.

Látka \textbf{D} je fluorid sodný, NaF.

Prvkem \textbf{X} je kyslík.

Pro snadnější orientaci přikládáme ještě chemické rovnice dějů v~zadání:

\[
\mathrm{2\,F_{2}+2\,H_{2}O\,(l)\rightarrow4\,HF\,(\mathrm{\boldsymbol{\mathrm{A}}})+O_{2}\,(\boldsymbol{\mathrm{X}})}
\]
\[
\mathrm{F_{2}+H_{2}O\,(s)\rightarrow HF\,(\mathrm{\boldsymbol{\mathrm{A}}})+HOF\,(\mathrm{\boldsymbol{\mathrm{B}}})}
\]
\[
\mathrm{2\,F_{2}+2\,NaOH\rightarrow2\,NaF\,(\mathrm{\boldsymbol{\mathrm{D}}})+OF_{2}\,(\mathrm{\boldsymbol{\mathrm{C}}})+H_{2}O}
\]


\hrulefill % \subsection*{Ročník 4, úloha č. 5.4 }
\begin{quotation}
\ctyri Karl Otto Christe (narozen 1936) je anorganickým chemikem, který se
proslavil výsledky práce s~extrémně reaktivními látkami a vysloužil
si titul 'The Fluorine God'. Syntézou kationtu pentazenia $\mathrm{N}^{5+}$ 
přispěl k~rozvoji chemie vysokoenergetických
materiálů. Při 100.~výročí objevu fluoru Henri Moissanem publikoval
první chemickou syntézu elementárního fluoru, tedy reakci, jejímž
produktem je $\mathrm{F}_{2}$. Napište vyčíslenou rovnici popsané
syntézy, pokud znáte reaktanty a víte, že mangan podléhá jednoelektronové
redukci.

\[
\ce{K2MnF6 +SbF5} \rightarrow
\]

\end{quotation} \dotfill \par 
Produktem reakce je kromě fluoru též hexafluoroantimoničnan draselný a fluorid manganitý\footnote{Chemical synthesis of elemental fluorine. Karl O. Christe. \textit{Inorg. Chem.},
\textbf{1986}, 25(21), pp 3721--3722 DOI: \href{https://doi.org/10.1021/ic00241a001}{\underline{10.1021/ic00241a001}}.
}
\[
\ce{2 K2MnF6 + 4 SbF5 \rightarrow} 2\, \ce{MnF3 + 4 KSbF6 +F2}
\]

Pokud byste se spokojili se zlomkovým stechiometrickýmo koeficientem u~plynného flouru, lze samozřejmě všechny koeficienty vydělit dvěma:

\[
\ce{K2MnF6 + 2 SbF5 \rightarrow MnF3 + 2 KSbF6 + \frac12 F2}
\]

\section{Uhlík a jeho anorganické sloučeniny}

% \subsection*{Ročník 4, úloha č. 0.14 }
\begin{quotation}
\jeden Před několika lety byl velmi intenzivně inzerován jistý přípravek,
který odstraňoval vodní kámen z~topného tělesa myčky. V~současnosti
se s~těmito problémy potýkají především domácnosti v~podhůří Bílých
Karpat, které se spoléhají na vlastní studnu. Pokud se doma potýkáte
s vodním kamenem, jednou z~možností, jak se ho zbavit, je použití
zahřátého octa. Napište vyčíslenou chemickou rovnici reakce octa s
vodním kamenem \textbf{v iontovém tvaru}.
\end{quotation} \dotfill \par 
Rovnice reakce octa s~obecným vodním kamenem, který je tvořen uhličitany, vypadá takto:

\[
\ce{2CH3COOH + \ch{CO3^{2-}} \rightarrow} \ 2\, \ce{CH3COO- + CO2 + H2O}
\]

\newpage %%
 % \subsection*{Ročník 4, úloha č. 0.26 }
\begin{quotation}
\dva Podle školních pravidel pro názvy kyselin lze odvodit kyselinu uhelnatou
$\ce{H2CO2}$, která by měla mít následující strukturní vzorec.
\begin{center}

\includegraphics{/4/0-26-1}

\par\end{center}
Povšimněte si volného elektronového páru na atomu uhlíku. Jako organická
sloučenina má název dihydroxymethyliden, zapisuje se $\ce{C(OH)2}$ a byla detekována pouze v~plynném skupenství. Její
odvozenou bází je uhelnatanový anion \(\ch{CO2^{2-}}\) a za nízkých
teplot (15 K) byla prokázána existence solí, například $\ce{Li2CO2}$.
Protonací těchto solí ovšem nevzniká volná kyselina uhelnatá, ale
její isomer, který zpravidla řadíme mezi organické sloučeniny. Napište
název a vzorec této kyseliny, ze kterého bude jasné propojení atomů (racionální či strukturní, ne sumární).
\end{quotation} \dotfill \par 

Vzniká kyselina mravenčí, nejjednodušší karboxylová kyselina (samozřejmě
také $\ce{H2CO2}$)\footnote{Schreiner, P. and Reisenauer, H., Spectroscopic Identification
of Dihydroxycarbene. \textit{Angew. Chem. Int. Ed.} \textbf{2008}, 47:
7071-7074. DOI:
\href{https://doi.org/10.1002/anie.200802105}{\underline{10.1002/anie.200802105}}}.
\noindent \begin{center}

\includegraphics{/4/0-26-2}

\par\end{center}

\hrulefill % \subsection*{Ročník 5, úloha č. 2.6 }
\begin{quotation}
\jeden Oxid uhelnatý představuje důležitou součást zpracování ropy, vzniká
především při zplyňování koksu a při parním reformingu zemního plynu.
V ropném průmyslu je to meziprodukt při výrobě látky důležité v~širokém
spektru chemického průmyslu. Zapište rovnici reformace \textbf{oxidu uhelnatého}
vodní parou (včetně skupenství všech látek) a označte hlavní produkt,
kvůli kterému se reforming provádí. Napovíme, že onen žádoucí produkt
je lehčí než vzduch.
\end{quotation} \dotfill \par 
\[
\ch{CO (g) + H2O (g)\  \rightarrow \mathrm{CO_2 } (g) + H2 (g)}
\]

Tato reakce probíhá při vysokých teplotách v~průmyslových chemických
reaktorech, hlavním produktem, na který jsme se ptali, je vodík. Oxid
uhličitý je sice také užitečným produktem, na rozdíl od vodíku ale není možné například jeho energetické využití.

\hrulefill % \subsection*{Ročník 5, úloha č. 5.1}
\begin{quotation}
\dva Elementární chlor v~alkalických roztocích disproporcionuje za vzniku
iontů chlornanových a chloridových. Dikyan, vzorcem $\ch{(CN)2}$,
vykazuje obdobné chování. Zapište vyčíslenou reakci dikyanu v~alkalickém
vodném roztoku.
\end{quotation} \dotfill \par 
Stejně jako disproporcionuje chlor, disproporcionuje dikyan na kyanid
a kyanatan. 
\[
\ch{(CN)2 + 2 OH^{-}\rightarrow \mathrm{CN^{-}} + OCN^{-} + H2O}
\]


\hrulefill % \subsection*{Ročník 1, úloha č. 7.1 }
\begin{quotation}
\dva V~knize „Bylo nás pět“ Karel Poláček píše:\textit{ „Když jsme byli v~polích,
tak Bejval pravil, jestli víme, že když se ta látka (sloučenina \textbf{A})
hasí vodou (reakce \textbf{1}), tak se z~ní vyvine plyn (sloučenina \textbf{B}), který
byv zapálen, vydává jasné světlo (reakce \textbf{2}). Odvětili jsme, že to
víme, a já jsem pravil, že jsem jednou viděl takovou lampu, jak si
s ní o~pouti svítil medák na turecký med. Bejval pravil, dobře že
to víme, a pak nám vysvětlil, v~čem spočívá ten vynález, z~něhož bude
hrozná legrace. Na poli leželo sněhu velice moc a my jsme to všechno
zahrabali do sněhu, a když to bylo zahrabaný, tak jsme škrtli zápalkou
a chvilku drželi u~toho. Za chvilánku to začalo prudce syčet a pak
to chytlo a začalo hořet jasným plamenem. Kdo by nevěděl, co v~tom
je, tak by myslil, že hoří sníh, což by mu bylo divné. A my jsme šli
od toho dál, a když jsme byli dál, tak šlehaly plameny velice vysoko
a bylo velké světlo a my jsme se z~toho radovali.“} Napište, co jsou
zmiňované látky \textbf{A} a \textbf{B} a reakce \textbf{1} a \textbf{2} zapište chemickými rovnicemi.
\end{quotation} \dotfill \par 
Sloučenina \textbf{A} je karbid/acetylid/dikarbid vápenatý, $\ch{CaC2}$. Sloučenina \textbf{B} je acetylen/ethyn, $\ch{C2H2}$.

Reakce \textbf{1} -- $\mathrm{CaC_{2}+2\,H_{2}O\rightarrow Ca(OH)_{2}+C_{2}H_{2}}$ 

Reakce \textbf{2} -- $\mathrm{2\,C_{2}H_{2}+5\,O_{2}\, \rightarrow\, 4\,CO_{2}+2\,H_2 O}$

\hrulefill % \subsection*{Ročník 4, úloha č. 6.3 }
\begin{quotation}
\dva Kyanidový anion patří mezi nejčastější ligandy v~koordinační chemii
a je známo mnoho jeho komplexních sloučenin s~přechodnými kovy. Příprava
komplexu obsahujícího terminálně vázaný ligand \linebreak{}
$\mathrm{{-\,C\equiv\hspace{0pt}P}}$, v~angličtině nazývaný cyaphide, se poprvé podařila
až v\,roce 2006. Níže najdete citaci z~článku, vaším úkolem je přiřadit
k~vyobrazeným strukturám čísla \textbf{1-5}.

\begin{center}
\includegraphics[scale=0.7]{Cyaphide sbírka.pdf}
\end{center}

Conversion of \textbf{1} into the Grignard reagent followed by treatment
with $\ce{PCl3}$ produced the silylsubstituted alkyl phosphonous
dichloride (\textbf{2}), in over 87 \% yield. Employing 2.2 equiv
of DABCO (1,8-diazabicyclo{[}2.2.2{]}octane) effected the dehydrohalogenation
reaction, and \textbf{2} was fully transformed into the triphenylsilyl-substituted
phos\-phaalkyne (\textbf{3}). The reaction occurred at ambient temperature
and in multiple solvents in less than one hour. Addition of a solution
of \textbf{3} in toluene to 0.5 equiv of $\ce{[RuH(dppe)2]OTf}$ in
$\ce{CH2Cl2}$ resulted in a color change from dark red to yellow-orange.
After workup, the colorless $\eta^{1}$-coordinated complex \textbf{4}
was isolated in over 80 \% yield as the only product. However, when
\textbf{4} was treated with a slight excess of sodium phenoxide (NaOPh)
in THF, the immediate formation of an intermediate was observed which
was then converted into a new product (\textbf{5}) after 14 h. The
results (\ldots) demonstrate the first structurally characterized metal
cyaphide complex\footnote{Cordaro, J. G., Stein, D., Rüegger, H. and Grützmacher, H. (2006),
Making the True “CP” Ligand. \textit{Angew. Chem. Int. Ed.}, \textit{45}, 6159-6162. DOI: 
\href{https://onlinelibrary.wiley.com/doi/full/10.1002/anie.200602499}{\underline{10.1002/anie.200602499}}}.
\end{quotation} \dotfill \par 
Struktury přiřazené k~číslům v~reakční sekvenci jsou na obrázku: 
\noindent \begin{center}

\includegraphics{/4/6-3-5}

\par\end{center}

Všimněte si, že úloha byla ve své podstatě spíše přiřazovací úlohou na anglické
názvosloví, než úlohou na chemii $\ce{-C#P}$ ligandu.

\hrulefill % \subsection*{Ročník 5, úloha č. 7.6 }
\begin{quotation}
\dva Izomerie je neodmyslitelně spojena s~organickou chemií. Poprvé však
byla izomerie pozorována u~dvou anorganických solí stříbra.

Látka \textbf{A} může být připravena reakcí dusičnanu stříbrného s
močovinou. Jako jediný vedlejší produkt vzniká dusičnan amonný. Látka
\textbf{B} vzniká reakcí kovového stříbra s~kyselinou dusičnou a
ethanolem. Tato reakce je značně složitá a vzniká v~ní velké množství
vedlejších produktů. Látka \textbf{A} je stabilní a nepříliš zajímavý
šedivý prášek. Látka \textbf{B} je velmi citlivou třaskavinou, která
se používá například jako náplň bouchacích kuliček. Elementární analýzou
bylo zjištěno, že obě látky obsahují pouze stříbro, dusík, uhlík a
kyslík. Nakreslete strukturní elektronové vzorce látek \textbf{A}
a \textbf{B}.
\end{quotation} \dotfill \par 

Látka \textbf{A} je kyanatan stříbrný. Látka \textbf{B} je pak její isomer, isokyanatan stříbrný. Strukturní vzorce obou látek jsou níže:
\begin{center}
\includegraphics{/5/7-6}
\par\end{center}

\hrulefill % \subsection*{Ročník 4, úloha č. 8.4 }
\begin{quotation}
\ctyri Výchozí látkou pro přípravu žlutého pigmentu je substance \textbf{X}, která je známá pro svou reakci s~vodou, při níž se uvolňuje acetylen.
Tato látka reaguje s~molekulárním dusíkem za vzniku uhlíku a látky
\textbf{A}. Tato reakce hrála klíčovou roli v~procesu získávání amoniaku
před objevem Haberovy syntézy amoniaku. Látka \textbf{A} je dále částečně
hydrolyzována na dva produkty: hydroxid vápenatý a látku \textbf{B},
která reakcí s~dusičnanem olovnatým poskytuje mimo jiné požadovaný
žlutý pigment. Napište vyčíslenou rovnici popisující poslední krok
přípravy.
\end{quotation} \dotfill \par 
Reakce látky \textbf{X} s~molekulárním dusíkem je reakce karbidu vápenatého,
vzniká kyanamid vápenatý, též známý jako dusíkaté vápno,  $\ce{CaCN2}$:
\[
\ch{CaC2 + N2 \rightarrow \mathrm{CaCN_2 } + C }
\]
Částečná hydrolýza $\ce{CaCN2}$ poskytuje
kromě hydroxidu vápenatého ještě látku \textbf{B} -- hydrogenkyanamid vápenatý $\ce{Ca(HCN2)2}$.
\[
\ch{CaCN2 + H2O \rightarrow \mathrm{Ca(OH)2} + Ca(HCN)2}
\]

Požadovanou reakcí látky \textbf{B} -- $\ce{Ca(HCN2)2}$ s~dusičnanem
olovnatým pak vznikne žlutý kyanamid olovnatý.
\[
\mathrm{Ca(HCN_{2})_{2}+Pb(NO_{3})_{2}\rightarrow PbCN_{2}+Ca(NO_{3})_{2}+H_{2}CN_{2}}
\]

\newpage %nadpis%
\section{Ostatní prvky a jejich sloučeniny}

 % \subsection*{Ročník 4, úloha č. 1.2  }
\begin{quotation}
\jeden Prakticky každému laboratornímu praktiku předchází krátká přednáška
o bezpečnosti práce v~laboratoři. Studenti jejímu průběhu bohužel většinou
nevěnují mnoho pozornosti, ale základní bezpečnostní povědomí se hodí
každému. Jedním z~možných bezpečnostních problémů může být únik plynů
z tlakových lahví. Které plyny byste byli schopni detekovat na základě
jejich charakteristického zápachu? 
\begin{itemize}
\item kyanovodík 
\item chlor 
\item methan 
\item oxid uhelnatý 
\item amoniak 
\item oxid siřičitý
\end{itemize}
\end{quotation} \dotfill \par 
Po čichu bezpečně poznáte kyanovodík páchnoucí po hořkých mandlích,
chlor podle zápachu velmi podobného dezinfekci, amoniak podle zcela
nezaměnitelného štiplavého odéru a po pekle páchnoucí oxid siřičitý.

\hrulefill % \subsection*{Ročník 2, úloha č. 3.3}
\begin{quotation}
\jeden „Bledník, známka jest Bl., mocnina 11 neb 135 až 136 dobývá se z~bledny,
starým již známé salaje, proti blednici uživané, a i k~pájení na tvrdo.
Bledník s~kyslíkem dává kys bledenec, který v~lihu rozpuštěn bledě
zeleně hoří. Tento bledenec se žíravou sodou dává blednu. Bledenec
náleží mezi kysy v~ohni stálé a jest tedy dosti podoben ku křemenu.
Bledník co prvek jest hnědý prach.“\footnote{Amerling, Karel. Karla Amerlinga Orbis Pictus: čili svět v
obrazích. Stupeň druhý, Co pokračování prvního stupně, jejž sepsal
Amos Komenský. V~Praze: B. F. Mohrmann, 1852.}

Napište dnešní název pro bledník.
\end{quotation} \dotfill \par 
Bledník, prvek o~atomové hmotnosti 11, jehož sloučeniny hoří v~lihu zeleně,
je bor.

\hrulefill % \subsection*{Ročník 5, úloha č. 0.14   }
\begin{quotation}
\jeden Čistý kyslík se v~průmyslové praxi vyrábí například destilací zkapalněného vzduchu.
Do laboratoří, ale též například do nemocnic se pak dodává v~tlakových lahvích.
Nejjednodušší cestou, jak kyslík připravit laboratorně, je termický
rozklad vhodné anorganické látky. Napište vyčíslenou chemickou rovnici
termického rozkladu chlorečnanu draselného.
\end{quotation} \dotfill \par 
Jak bylo napovězeno doprovodným textem, termický rozklad chlorečnanu
draselného vede k~molekulovému kyslíku jako produktu. Termické rozklady
velmi často vedou k~velmi stabilním produktům a jelikož kyslík z~$\ch{KClO3}$
může všechen skončit jako $\ch{O2}$, nejjednodušší a stabilní produkt
ze zbytku molekuly bude chlorid draselný. Vyčíslená rovnice:

\[
\ch{2 KClO3}\rightarrow 2\,\ch{KCl + 3 O2}
\]


\chapter{Organická chemie}

\section{Úlohy na názvosloví}

 % \subsection*{Ročník 5, úloha č. 1.5 }
\begin{quotation}
\jeden Kouř z~cigarety obsahuje až 4 000 různých chemických sloučenin, z
nichž bylo přes 50 prokazatelně identifikováno jako karcinogenní pro
člověka. Přiřaďte názvy některých těchto jedovatých látek vyskytujících
se v~cigaretovém kouři k~jejich chemickým vzorcům.

Furan, nitromethan, oxiran, akrylonitril, dimethylhydrazin. 
\begin{center}
\includegraphics{/5/1-5}
\par\end{center}

\end{quotation} \dotfill \par 
Zleva doprava jsou postupně vyobrazeny akrylonitril, nitromethan,
oxiran, furan a dimethylhydrazin.

\hrulefill % \subsection*{Ročník 1, úloha č. 5.6}
\begin{quotation}
\dva Triviální názvy mohou být poměrně zajímavé. Zde kupříkladu všechny začínají na F. Přiřaďte následující triviální názvy k~jejich strukturním vzorcům.\\
A) fluoren B) fenylalanin C) fulven D) farnesol E) fenanthren F) fenolftalein

\begin{center}
\includegraphics{/1/5-6}
\end{center}
\end{quotation} \dotfill \par 

Odpověď: 1A, 2E, 3B, 4D, 5F, 6C

Možná se zdá, že jsou některé ze struktur pojmenované úplně náhodně. V~některých názvech lze ale vystopovat vodítka k~identifikaci. Kupříkladu fenolftalein -- tato molekula nejspíše bude obsahovat OH skupinu na aromatickém jádře. Taková je pouze molekula 5. Farnesol pak lze vzhledem k~příponě -ol přiřadit k~jedinému zbývajícímu alkoholu --molekule 4. Poté, co ztotožníme poměrně známé molekuly fenylalaninu a fenantrenu, zbývá nám pouze fluoren a fulven. Jejich vzorce pak lze dohledat v~dostupné literatuře.

\hrulefill % \subsection*{Ročník 3, úloha č. 5.6 }
\begin{quotation}
\tri Svět chemiků je plný mnoha různých zkratek a symbolů, jimiž se brání
přesile složitých systematických názvů. Komu by se taky 
chtělo používat například celý název AIBN -- azobisizobutyronitrilu. Problémem
ovšem může být, že tyto zkratky mohou být identické s~běžně používanými
slovy, což může přinášet zmatky, a pouze ti znalejší ví, že skrývají
nosné informace dovolující jasné určení, ať už jde o~složení nebo
o strukturu (za všechny jeden příklad -- trinitrotoluen, TNT). Abyste
se případným potížím do budoucna vyhnuli, bude užitečné, když si tyto
zkratky procvičíte. Přiřaďte následující zkratky ke strukturám: 1)
DEAD, 2) TEA, 3) TEMPO, 4) DAST, 5) CAN a 6) PEE. 

\begin{center}
\includegraphics{/3/5-6}
\end{center}
\end{quotation} \dotfill \par 
Ubránili jste se přesile zkratek bez ztrát na životech? Doufáme, že
jste nezabili kolegu, který vás požádal o~\textbf{DEAD}! Pravděpodobně
se chystá provádět Mitsunobovu reakci a chce podat \textbf{D}i\textbf{E}thyl \textbf{A}zo\textbf{D}icarboxylate (\textbf{1f}). \textbf{TEA} není jen příjemně
vonící nápoj, který si Angličané dopřávají o~páté, ale i \textbf{T}ri\textbf{E}thyl\textbf{A}mine,
páchnoucí báze často používaná v~organické syntéze (\textbf{2d}). \textbf{TEMPO}
neoznačuje pouze rychlost hudby či chůze, ale i (2,2,6,6-\textbf{TE}tra\textbf{M}ethyl\textbf{P}iperidin-1-yl)\textbf{O}xyl,
stabilní radikál používaný například jako mediátor radikálových polymerací
nebo reagent v~organické syntéze (\textbf{3e}). Na \textbf{DAST} se rozhodně
nepráší v~žádné laboratoři věnující se organickým fluoroderivátům,
jde totiž o~\textbf{D}iethyl\textbf{A}mino\textbf{S}ulfur \textbf{T}rifluoride
-- často používané fluorační činidlo (\textbf{4a}). \textbf{CAN} kromě plechovky
od piva označuje i \textbf{C}eric \textbf{A}mmonim \textbf{N}itrate
oxidační činidlo, které se například využívá k~odchránění para-methoxybenzylových
chránicích skupin (\textbf{5b}). And last but not least: \textbf{PEE} není
jenom moč, ale i \textbf{P}hosphono\textbf{E}thoxy\textbf{E}thyl --
jeden z~postranních řetězců acyklických analogů nukleotidů, což je
třída antivirotik vyvinutá českým vědcem Antonínem Holým (\textbf{6c}).

Bylo též možné postupovat vylučovací metodou: jediná zkratka obsahující O~(TEMPO) jde hned přiřadit k~jediné molekule s~kyslíkem\footnote{Molekula DEAD sice také obsahuje kyslík, ale v~rámci karboxylové skupiny, tedy lze předpokládat, že tato molekula nebude mít ve zkratce písmeno O.}. Stejně tak DAST přiřadíme k~molekule obsahující síru, CAN pak k~jediné molekule obsahující cer a PEE k~jediné molekule obsahující fosfor. Zbývá nám přiřadit zkratky TEA a DEAD, což ovšem není nic složitého -- jednoduše lze vyvodit, že TEA bude zkratkou pro triethylamin.

\hrulefill % \subsection*{Ročník 1, úloha č. 7.5}
\begin{quotation}
\dva Syntézou zajímavých a/nebo symetrických molekul se lidstvo, zejména
jeho chemická část, bavilo odpradávna. Syntetizovat alkan ve tvaru
prasete, které si děti běžně kreslí na tabuli někdy v~první třídě
není za použití několika chemikálií z~lépe vybavené laboratoře velkým
problémem. Čeští chemici o~sobě dali vědět syntézou adamantanu, půvabné
tetraedrické molekuly (tricyk\-lo{[}3.3.1.1(3,7){]}de\-ka\-nu), američtí
vědci například syntetizovali kuban, dokonale krychlový osmiatomový
alkan.Dokázali byste určit na základě názvu vzorec takovéto molekuly s~hezkým tvarem? 

Co třeba pen\-ta\-cyk\-lo{[}$4.4.0.0^{2,5}.0^{3,8}.0^{4,7}${]}de\-kan, který se triviálně jmenuje basketan?
Postačí nám jeho sumární vzorec.
\end{quotation} \dotfill \par 
\begin{center}
\includegraphics{/1/7-5}
\end{center}


Pro určení sumárního vzorce není třeba znát přesnou konstituci sloučeniny. Z~názvu lze vyvodit, že v~molekule je pět cyklů. Každý cyklus sníží počet vodíků oproti nasycenému acyklickému uhlovodíku (v tomto případě se jedná o~dekan $\ce{C10H22}$, o~dva, výsledný sumární vzorec bude tedy po odečtení pěti cyklů $\mathrm{C_{10}H_{12}}$.

\hrulefill % \subsection*{Ročník 3, úloha č. 4.5 }
\begin{quotation}
\tri V~anglicky psané literatuře se v~souvislosti se strukturou a složením
organických molekul můžeme setkat s~prapodivnou zkratkou SODAR, označující
\textit{sum of double bonds and rings}. Častěji se ale uvádí jako
\textit{unsaturation number}, tedy stupeň nenasycenosti. Ten je roven
počtu molekul vodíku, se kterými by musela zkoumaná molekula zreagovat,
aby vznikla zcela nasycená acyklická sloučenina.
Stupeň nenasycenosti u~ethylenu, acetaldehydu či cyklohexanu je roven
1, u~cyklohexanonu 2,\\u~cyklopentadienu 3. Vaším úkolem je určit
stupeň nenasycenosti molekuly odvozené od \textit{N}-aminoazido\-tetrazolu,
jejíž sumární vzorec je C\textsubscript{2}N\textsubscript{14}.
\end{quotation} \dotfill \par 
Pro vyřešení úlohy je třeba si uvědomit, že nasycené alifatické uhlovodíky
jsou složeny z~jednotek --CH\textsubscript{2}--, na obou koncích řetězce
se pak nachází skupina --CH\textsubscript{3}. Výsledný homologický
vzorec je pak C\textsubscript{$4$}H\textsubscript{$2c+2$}. Aminy si
pak můžeme představit jako uhlovodíky, kde byl určitý počet skupin
--CH\textsubscript{2}-- nahrazen jednotkami --NH-- (atomy dusíku jsou
pouze trojvazné). Obecná nasycená acyklická sloučenina obsahující
atomy uhlíku a dusíku pak má sumární vzorec C\textsubscript{$c$}N\textsubscript{$n$}H\textsubscript{$2c+n+2$}.

Pro $c=2$ a $n=14$ je pak počet atomů vodíku roven 20, tomu odpovídá stupeň
nenasycenosti 10.

Nepravděpodobně vypadající sloučenina v~zadání
úlohy má systematický název (5-azido-\-1\textit{H}-tetrazol-1-yl)\-karboimidoyldiazid
a patří mezi energetické materiály, které mohou najít využití
v~pyrotechnice nebo raketovém pohonu. 
\noindent \begin{center}

\includegraphics{/3/4-5-1}

\par\end{center}

Skupina $\mathrm{-N_{3}}$ se nazývá azid.

Můžeme uvažovat i izomer $\mathrm{C_{2}N_{14}}$ s~totožným stupněm
nenasycenosti:
\noindent \begin{center}

\includegraphics{/3/4-5-2}

\par\end{center}

\hrulefill % \subsection*{Ročník 5, úloha č. 6.5 }
\begin{quotation}
\tri Jednou z~veselých částí chemie jsou chemické úsměvy, anglicky SMILES
(\textit{simplified molecular-input line-entry system}). Jedná se o~velmi
užitečný způsob reprezentace molekul pomocí textu, který lze přímo
přeložit do struktury. Pokud byste chtěli nakreslit složitou molekulu
v~chemickém editoru, můžete si najít její SMILES například na Wikipedii, zkopírovat
kód do vašeho editoru a, \textit{voilá}, molekula je nakreslená. Pojďme se podívat na pravidla
zápisu jednoduchých organických molekul.


\begin{enumerate}[label=\arabic*.]
\item Atomy jsou reprezentovány jejich značkami (C pro uhlík, O~pro kyslík
atd.).
\item Jednoduché vazby a atomy vodíku nejsou znázorněny (CCCCCC je hexan;
CCO znázorňuje ethanol).
\item Větvení je znázorněno závorkami (CCC(CC)CC(C)CC je 3-ethyl-5-methylheptan,
CC(O)C je propan-2-ol).
\item Dvojné a trojné vazby jsou znázorněny = respektive \#. (C\#C je ethyn,
CC(=O)C je aceton (propan-2-on); CC=CC je but-2-en.) 
\end{enumerate}
Mohli bychom pokračovat dále, ale toto nám jako základ postačí. Nakreslete
strukturní vzorce molekul \textbf{A--C}.


\begin{enumerate}[label=\textbf{\Alph*:}] 
\item OCC(CO)(CO)CO 
\item N\#CC(=O)C(=O)C\#N 
\item OOS(=O)(=O)O 
\end{enumerate}
\end{quotation} \dotfill \par 
Vždy je asi nejlepší začít nevětvenou částí molekuly. V~případě molekuly
A jde o~tučně vyznačenou část: \textbf{OCC}(CO)(CO)\textbf{CO}, které
odpovídá následující skelet:
\begin{center}
\includegraphics{/5/6-5-1} 
\par\end{center}

Poté musíme vzít v~potaz větvení, vyznačené závorkami OCC(CO)(CO)CO.
Skupina v~závorkách je navázána na první atom před závorkou. V~tomto
případě jsou obě skupiny stejné a jsou navázané na stejný atom. Výsledná
molekula vypadá takto: 
\begin{center}
\includegraphics{/5/6-5-2} 
\par\end{center}

V případě dalších molekul hrají roli i dvojné a trojné vazby, vyznačené
= resp. \#. Nevětvená část v~molekule B (\textbf{N\#CC}(=O)\textbf{C}(=O)\textbf{C\#N})
odpovídá: 
\begin{center}
\includegraphics{/5/6-5-3} 
\par\end{center}

Přidáme-li kyslíky navázané dvojnou vazbou, obdržíme: 
\begin{center}
\includegraphics{/5/6-5-4} 
\par\end{center}

Nevětvená část v~molekule C (\textbf{OOS}(=O)(=O)\textbf{O}) vypadá
takto: 
\begin{center}
\includegraphics{/5/6-5-5} 
\par\end{center}

Po přidání větvení obdržíme: 
\begin{center}
\includegraphics{/5/6-5-6} 
\par\end{center}

Jistě si stejně jako my nyní vzpomenete na tzv. Cimrmanův chemický
vtip\footnote{Poněkud vyšisovaná anekdota o~krátkém znění: $\ce{H2SO5}$}.

\section{Klasifikace a vlastnosti organických sloučenin}

 % \subsection*{Ročník 3, úloha č. 2.3 }
\begin{quotation}
\jeden \textit{Dobytí molekulárního dipólu}: Málokdo ví, že slavný
Jára Cimrman se po dobytí severního pólu nespokojil pouze s~pólem
jedním a rozhodl se, že dobude i dvojpól. Při tom se vynasnažil o
spojení vědy zeměpisné a chemické a čtenář musí seznat, že jde o~vskutku
bravurní dílo. 
\noindent \begin{center}

\includegraphics{/3/2-3}

\par\end{center}
Čtenář nechť se nenechá zmást jeho anekdotou, jež je zmíněna ve hře
\textit{Lijavec}, v~chemii byl Cimrman znalý! K~nastínění svého myšlenkového
experimentu využil Cimrman molekulu \textit{o}-dichlorbenzenu a tím o~roky
předběhl jiného génia, Bernarda Feringu, který té samé molekuly využil
pro tvorbu svých molekulárních stroječků a byl za ně posléze oceněn
Nobelovou cenou.

A tedy my se tážeme: Elektrický dipólový moment molekuly na obrázku
směřuje na jih. Jak byste změnili polohu jednoho atomu chloru, aby
dipólový moment nesměřoval nikam? Zakreslete!
\end{quotation} \dotfill \par 
Aby dipólový moment molekuly nesměřoval nikam (tzn. molekula by nebyla polární), je třeba, aby
se příspěvky vazeb C--Cl vzájemně odečetly. Správným řešením je tedy
1,4-dichlorbenzen (\textit{p}-dichlorbenzen).

\begin{center}
\includegraphics{/3/2-3-2}
\end{center}


\hrulefill % \subsection*{Ročník 3, úloha č. 2.6}
\begin{quotation}
\dva Svět chemiků je plný mnoha různých zkratek a symbolů, jimiž se brání
přesile složitých systematických názvů. Míru kyselosti u~kyselin chemici
neurčují ochutnáváním, protože to by u~některých kyselin (obzvláště
těch fluorovaných) nemuselo skončit dobře. K~vyjádření síly kyselin
se používá konstanta kyselosti, která se označuje jako $K_\mathrm{a}$ (a v~indexu je odvozeno z~anglického slova \textit{acidity} označujícího kyselost),
případně její záporný logaritmus označovaný jako \pKa.
Pokud chce chemik porovnat sílu dvou různých kyselin, může se spolehnout
na tabulky. Když tabulky nejsou k~dispozici, odhadne, která z~nich
je kyselejší, pomocí strukturních vzorců a elektronových posunů. Vyberte
v~následujících dvojicích organických kyselin vždy tu silnější: 
\end{quotation} \dotfill \par 
\begin{center}
\centering
\begin{tabular}[h!]{l|l}

k. propanová & k. 2-oxopropanová (pyrohroznová)\tabularnewline
\hline 
k. difluoroctová & k. chloroctová\tabularnewline
\hline 
k. octová & k. peroxoctová\tabularnewline
\hline 
k. benzoová & k. p-hydroxybenzoová\tabularnewline
\end{tabular}
\end{center}
\begin{quotation}
\noindent\textit{Nápověda: Stabilizace aniontu napomáhá odštěpení H}\textsuperscript{\textit{+}}\textit{.}
\end{quotation} \dotfill \par 

Silnější kyselina je vždy ta, u~které je tvorba aniontu podpořena
stabilizací záporného náboje. K~tomu dochází, když jsou v~blízkosti vznikajícího náboje elektronakceptorní skupiny. Silnější jsou proto tyto kyseliny:

\textbf{2-oxopropanová} (elektronakceptorní karbonylová skupina, \pKa\ 2,5 oproti 4,9),\\
\textbf{difluoroctová} (dva atomy F působí silněji než jeden
atom Cl, \pKa\ 1,3 oproti 2,9),\\ 
\textbf{octová} (u
aniontu peroxokyseliny není stabilizace rezonancí vůbec možná, \pKa\ 4,7 oproti 8,2),\\ 
\textbf{benzoová} (na aromatickém jádře se uplatní kladný
mezomerní efekt hydroxyskupiny, substituovaná kyselina je proto slabší, \pKa\ 4,20 oproti 4,54).

\hrulefill % \subsection*{Ročník 5, úloha č. 3.1 }
\begin{quotation}
\dva Některé molekuly, hlavně organické, jsou chirální. To znamená, že
nejsou totožné se svým zrcadlovým obrazem. Většinou pak obsahují jeden
nebo více atomů, které na sobě mají navázány čtyři různé substituenty.
Velice často to bývají uhlíky se čtyřmi různými substituenty.

Určete, kolik takových chirálních atomů uhlíku má vitamin C, kyselina
citronová, heroin a jedna konkrétní látka typu Novičok, výsledek zapište
číslem pro každou molekulu. 
\begin{center}
\includegraphics{/5/3-1} 
\par\end{center}

\end{quotation} \dotfill \par 
V molekulách se postupně nachází dva, nula, pět a jeden chirální atom
uhlíku. Doufáme, že jste se nenechali nachytat na symetrickou molekulu
kyseliny citronové.

\hrulefill % \subsection*{Ročník 4, úloha č. 3.2 }
\begin{quotation}
\dva Podle definice IUPAC jsou jako heterocyklické sloučeniny klasifikovány
ty, které v~kruhu obsahují atomy alespoň dvou různých prvků. Které
z následujících sloučenin nepatří mezi heterocyklické sloučeniny? 
\begin{itemize}
\item pyrrol 
\item imidazol 
\item triazol 
\item tetrazol 
\item pentazol
\end{itemize}
\end{quotation} \dotfill \par 

Heterocyklická sloučenina je definovaná jako taková sloučenina, která obsahuje cyklus tvořený alespoň ze dvou různých druhů atomů. Pentazol obsahuje samé atomy dusíku, představuje proto homocyklickou
sloučeninu\footnote{\textit{Pure and Applied  Chemistry}, 1995, 67, 1307. Glossary of class names of organic compounds
and reactivity intermediates based on structure (IUPAC Recommendations
1995), DOI: \href{https://doi.org/10.1351/pac199567081307}{\underline{10.1351/pac199567081307}}}.
Pro ilustraci uvádíme vzorce všech pětičlenných dusíkatých heterocyklů a pentazolu:

\noindent \begin{center}

\includegraphics{/4/3-2}

\par\end{center}

\hrulefill % \subsection*{Ročník 2, úloha č. 5.2 }
\begin{quotation}
\tri Výhodou psaní schémat v~organické chemii je především skutečnost,
že tímto způsobem zapsané reakce se většinou ponechávají nevyčíslené.
Pokud se totiž rozhodneme psát „klasickou“ vyčíslenou rovnici například
pro oxidaci komplikovanější molekuly, třeba derivátů alkaloidů fenanthrenového
typu, stechiometrické koeficienty mohou nepříjemně vzrůst. Bývá-li
nutné takovou reakci skutečně vyčíslit, mnohdy pomáhá zapsat průměrné
oxidační číslo atomu (typicky uhlíku). Při této příležitosti spočítejte
nebo kvalifikovaně odhadněte, jaké je průměrné oxidační číslo uhlíku
v běžném českém lese. Tolerujeme odchylku $\pm 0,5$. 
\end{quotation} \dotfill \par 
Převážná většina hmoty lesa je tvořena stromy. A převážná většina
hmoty stromů je tvořena celulosou. Obecný vzorec celulosy je $[\mathrm{C_{12}H_{22}O_{11}]_{n}}$.
Protože vodík má vždy oxidační číslo +1 a kyslík vždy -2, vychází
na uhlík nula.

\newpage %%
 % \subsection*{Ročník 1, úloha č. 5.4}
\begin{quotation}
\dva Opticky aktivní sloučeniny mnohdy napáchaly v~historii mnoho škody.
Za všechny zmiňme například Contergan, léčivo, které způsobilo vrozené
defekty tisíců dětí, vzniklé jeho podáním těhotným ženám jako
sedativum a hypnotikum. Jeden jeho enantiomer působil jako anestetikum,
druhý jako teratogen (škodil nenarozenému plodu). Snad se vám, drazí
řešitelé, neprotočí vaše optické aparáty při zadání následující úlohy:
kolik optických izomerů má 2,3,4,5,6-pentahydroxohexan-1-al?
\end{quotation} \dotfill \par 
Z~názvu jste mohli poznat, že se jedná o~nějaký šestiuhlíkatý cukr. Obecně mají $n$-uhlíkaté aldosy $2^{n-2}$ optických izomerů (majíce $n-2$ chirálních atomů
uhlíku). Pro uvedený cukr\footnote{Jedním z~možnách optických izomerů této sloučeniny je i všem známá glukosa.}, který má 6 uhlíků ($n=6$), a tedy čtyři stereogenní centra, je počet izomerů 16.

\section{Nasycené a nenasycené uhlovodíky}

% \subsection*{Ročník 5, úloha č. 0.21  }
\begin{quotation}
\jeden Uhlovodíky (angl. hydrocarbons) jsou typicky považovány za líné, nereaktivní
molekuly. I přes jejich zdánlivou ,,lenost`` však i tyto látky moderní společnost dnes
a denně  potřebuje. Kde přesně? Utvořte správné
dvojice uhlovodík-popis.\\
 \\
\end{quotation} \dotfill \par 
\begin{center}
\begin{tabular}{ r|l||r|l }

1 & toluen & A & významná složka zemního plynu \\\hline
2 & methan & B & nejjednodušší organický rostlinný hormon uvolňující se při zrání \\\hline
3 & oktan & C & významná složka ředidel, průmyslové rozpouštědlo \\\hline
4 & ethylen & D & významná složka benzínů \\
\end{tabular}
\end{center}
Správnými dvojicemi jsou 1C, 2A, 3D, 4B.

\hrulefill % \subsection*{Ročník 4, úloha č. 0.25 }
\begin{quotation}
\dva Nakreslete všech pět konstitučních izomerů látky se sumárním vzorcem
$\ce{C4H8}$.
\end{quotation} \dotfill \par 
\noindent \begin{center}

\includegraphics{/4/0-25}

\par\end{center}

Všech pět izomerů je na obrázku. V~případě \textit{E}/\textit{Z}-izomerie
u but-2-enu se nejedná o~konstituční izomerii, proto byly oba dva
izomery počítány jako jedna odpověď.

\hrulefill % \subsection*{Ročník 3, úloha č. 5.2 }
\begin{quotation}
\tri Napište strukturu alkenu C$_{7}$H$_{14}$, který poskytne stejné
produkty při oxidaci KMnO$_{4}$ i při ozonolýze za následujících
podmínek:

\[
\mathrm{C_{7}H_{14}\xrightarrow[2.\,Zn,\,H_{3}O^{+}]{1.\,O_{3}}}
\]
\[
\mathrm{C_{7}H_{14}\xrightarrow[H_{3}O^{+}]{KMnO_{4}}}
\]
\end{quotation} \dotfill \par 
Při ozonolýze za daných podmínek vzniknou dva ketony. Aby vznikly
dva ketony též reakcí s~manganistanem, musí být alken tetrasubstituovaný.
Požadovaný produkt tedy může mít pouze jeden vzorec:
\noindent \begin{center}

\includegraphics{/3/5-2}

\par\end{center}

\newpage %%
% \subsection*{Ročník 2, úloha č. 6.4}
\begin{quotation}
\tri Na poškozené etiketě lahve s~uhlovodíkem \textbf{A} bylo možné přečíst pouze
údaj o~molární hmotnosti 82 $\mathrm{g\cdot mol^{-1}}$ a ze vzorce se dalo rozeznat jen,
že obsahuje pětičlenný kruh. Experimentálně bylo zjištěno, že po přidání
bromu k~látce \textbf{A} se aduje jeden ekvivalent bromu za vzniku produktu \textbf{B}. Elektrofilní adice HCl na látku \textbf{A} vede ke vzniku dvou různých produktů
obsahujících chirální uhlík \textbf{C} a \textbf{D} v~poměru přibližně 1:1, tyto dva
produkty ovšem nejsou navzájem enantiomery (oba dva zmíněné produkty jsou racemickou směsí dvou enantiomerů). Napište strukturní vzorce
látek \textbf{A}, \textbf{B}, \textbf{C} a \textbf{D}.
\end{quotation} \dotfill \par 


Jediný sumární vzorec odpovídající molární hmotnosti 82 $\mathrm{g\cdot mol^{-1}}$ je $\mathrm{C_{6}H_{10}}$.

Stupeň nenasycenosti pro tento sumární vzorec vyjde 2. Z~toho plyne, že struktura obsahuje kromě
pětičlenného kruhu i druhý kruh nebo dvojnou vazbu. 

Adice bromu probíhá, takže struktura musí obsahovat dvojnou vazbu.
Možnost s~dvěma kruhy je tedy vyloučená. 

Adice HCl poskytuje dva produkty, poměr jejich množství je přibližně
1:1. Protože se nejedná o~enantiomery, dvojná vazba musí být mezi
terciárními atomy uhlíku.

Těmto kritériím odpovídá jediná struktura \textbf{A}.

Reakční schémata a vzorce látek \textbf{A}-\textbf{D}:

\begin{center}
\includegraphics{/2/6-4b}
\end{center}


\section{Areny a jejich reakce}

% \subsection*{Ročník 3, úloha č. 2.1 }
\begin{quotation}
\jeden Nitrace je jedna z~reakcí, bez nichž bychom dnes neměli tak širokou
paletu výbušnin. Nakreslete strukturní vzorce produktů nitrace toluenu
do prvního stupně. Vznikají dva izomery, napište oba.
\end{quotation} \dotfill \par 
Vzniká 2-nitrotoluen a 4-nitrotoluen. 
\noindent \begin{center}

\includegraphics{/3/2-1}

\par\end{center}

\hrulefill % \subsection*{Ročník 2, úloha č. 2.5}
\begin{quotation}
\jeden Elektrofilní aromatická substituce je jedním z~evergreenů středoškolské
organické chemie. Z~toho důvodu jsme ji neopomněli zařadit ani sem.
Substituce na samotném benzenu je triviální, podívejme se proto, jak
bude substituce probíhat na benzenu, kde už jedna funkční skupina
přítomna je. Seřaďte benzen a jeho monosubstituované deriváty podle
rostoucí reaktivity při elektrofilní aromatické substituci. Uvažujte
následující funkční skupiny: 
\[
\mathrm{-H,-COOH,-NO_{2},-NH_{2},-CH_{3}}
\]
\end{quotation} \dotfill \par 
Pořadí reaktivity je určeno mezomerním a indukčním efektem, přičemž
důležitější je zpravidla efekt mezomerní (výjimkou jsou zde například
halogeny). Pořadí reaktivity: 
\[
\mathrm{-NO_{2} < -COOH < -H < -CH_{3} < -NH_{2}}
\]


\newpage %%
 % \subsection*{Ročník 2, úloha č. 2.2 }
\begin{quotation}
\jeden První užití pojmu „aromaticita“ pro popis skupiny molekul je doloženo
v roce 1855, kdy tak August Wilhelm Hofmann označil sloučeniny obsahující
fenylovou skupinu. (Tento významný německý chemik se u~příležitosti
svých 70. narozenin dočkal povýšení do šlechtického stavu, od té doby
se jeho příjmení zapisuje jako „von Hofmann“.) Dodnes ale s\ jistotou
nevíme, proč se A. W. Hofmann rozhodl použít přídavné jméno vztahující
se k~smyslovému vjemu pro označení skupiny látek, z~nichž jen některé
mají charakteristickou vůni či zápach. Naproti tomu, mnoho přírodních
vonných látek patří do třídy terpenů, které z~chemického pohledu aromatické
nejsou. Označení tak pravděpodobně souvisí s~vyšším počtem nenasycených
vazeb, který je společným znakem „pravých“ aromátů i terpenů. Určete
u každé z~následujících organických molekul, zda její struktura nebo
její část představuje aromatický systém. 
\end{quotation} \dotfill \par 
\begin{center}
\includegraphics{/2/2-2}
\end{center}
 

Aby byla sloučenina aromatická, musí splňovat tato pravidla: musí
být planární (rovinná), musí existovat alespoň dvě odlišné resonanční
struktury vzniklé posunem $\pi$-elektronů a počet elektronů účastnících
se aromatického systému musí nabývat hodnot $4n+2$, kde $n$ je přirozené číslo (Hückelovo pravidlo).

\textbf{Cyklobutadien} na prvním obrázku nesplňuje Hückelovo pravidlo,
obsahuje pouze 4 $\pi$-elektrony. \textbf{3-nit\-roben\-zan\-thron}, který
je druhý zleva, obsahuje tři aromatická jádra, z~toho dvě kondenzovaná
a je tedy řazen mezi deriváty polycyklických aromatických uhlovodíků
(PAHs). Vyskytuje se ve zplodinách vznětových motorů, kde vzniká na
povrchu jemných částic reakcí ketonů (meziproduktů spalování) s~oxidy
dusíku. Třetí struktura zleva, \textbf{hexamethylentriperoxodiamin}
(HMTD) je vysokoenergetická pevná látka ze skupiny organických peroxidů.
Aromatický charakter je zcela vyloučen, jelikož molekula HMTD neobsahuje
žádnou násobnou vazbu (ani není planární). \textbf{Pyridin} zcela vpravo
naproti tomu aromatický je. Na atomu dusíku lokalizovaný volný elektronový pár, který
není zapojen do aromatického systému, může přijmout proton: pyridin
je proto zároveň příkladem bazické molekuly.

\hrulefill % \subsection*{Ročník 5, úloha č. 5.5}
\begin{quotation}
\jeden Biochemik Jardík byl donucen absolvovat předmět Laboratoř organické
chemie, aby získal potřebných 30 kreditů za semestr. Organická chemie
mu nebyla příliš blízká a ani při práci v~laboratoři nebyl příliš
zručný. Na konci laboratoří mu jedna z~baněk zůstala zanešena povlakem
špíny na dně, která nešla žádným způsobem seškrábat. Rozhodl se tedy
k jejímu odstranění chemickou cestou. Zkoušel všechna možná polární
i nepolární rozpouštědla, dokonce zkoušel i čištění chromsírovou a
peroxosírovou kyselinou, ale nic nepomáhalo. Nakonec se celý zoufalý
rozhodl ignorovat zásady bezpečnosti práce a začal náhodně zkoušet
různá rozpouštědla ve směsi s~kyselinou sírovou. Když do baňky nalil
kyselinu sírovou a toluen a zahřál ji horkovzdušnou pistolí, roztok
začal žloutnout. Jardík, celý šťastný, pokračoval v~proplachování
baňky touto směsí, ale nečistoty se na dně stále držely. Napište chemickou
reakci, která v~baňce při posledním ,,čištění`` probíhala.
\end{quotation} \dotfill \par 
Protože nevíme, jaké povahy byla nezničitelná špína na dně baňky,
usuzujme na chemickou reakci pouze mezi látkami v~baňce a látkami
dostupnými. Protože je v~baňce toluen s~kyselinou sírovou intenzivně
zahříván, dochází celkem nepřekvapivě k~sulfonaci toluenu. Žádanou
rovnicí je tedy\footnote{V rámci soutěže byly uznávány i isomery \textit{ortho}-, nicméně při sulfonaci vznikají převážně izomery \textit{para}-.} 

\begin{center}
\includegraphics{/5/5-5}   
\end{center}

\newpage %%
 % \subsection*{Ročník 5, úloha č. 7.4  }
\begin{quotation}
\tri Ze školních lavic jsou jména pánů Friedela a Craftse známá především
díky jednoduchým alkylacím, při kterých se nechává benzen reagovat
s alkylchloridem a chloridem hlinitým. Znalost mechanismu této reakce
však dovoluje mnohem hravější syntézy. Napište dominantní aromatický
produkt reakce, při které necháte benzen reagovat s~fosgenem a chloridem
hlinitým.
\end{quotation} \dotfill \par 
Není obtížné si domyslet, že prvním probíhajícím krokem bude reakce
ve smyslu Friedelovy--Craftsovy ,,acylace`` benzenu s~fosgenem,
která poskytne jako produkt benzoylchlorid. Ten sám je však v~naší
reakční směsi reaktivní částicí, která sama reaguje s~dalším ekvivalentem
benzenu nyní už přímo ve smyslu Friedelovy--Craftsovy acylace za
vzniku benzofenonu. Polyacylace je omezena ze dvou důvodů, jednak
záporný mezomerní efekt omezuje další acylace, druhak je ve směsi
(a zejména na začátku reakce) výrazně více benzenu. Dominantním produktem
je tedy \textbf{benzofenon}.

Schéma reakce můžete vidět na obrázku níže:

\begin{center}
\includegraphics{/5/7-4.eps}
\end{center}

\section{Halogenderiváty uhlovodíků, nukleofilní substituce}

 % \subsection*{Ročník 2, úloha č. 5.6}
\begin{quotation}
\tri Občas se organický chemik uchýlí i k~reakcím, které ve svém mechanismu obsahují radikály. Jedna taková ,,radikální`` -- chemickou řečí radikálová reakce je i na následujícím schématu. \textbf{Doplňte výchozí látku této reakce.}

\begin{center}
\includegraphics{/2/5-6}
\end{center}

Při čištění produktu této reakce jste si ale všimli, že oproti předpokladu
vznikl ještě jeden produkt. Je to oranžovohnědá kapalná látka, viskozitou
připomínající lehce připečený karamel a nevábně páchnoucí. \textbf{Co vám vzniklo a proč?}

\begin{enumerate}[label=\Alph*)]
\item Geminální dibromid. \textit{N}-bromsukcinimid
tvoří bromové radikály, které ochotně reagují v~benzylové poloze, některé molekuly brombenzenu můžou reagovat ještě jednou.
\item Benzonitril. Azobisisobutyronitril ochotně pouští při 120\,°C nitrilovou
skupinu, a tak vzniká nitril v~benzylové poloze. 
\item \textit{N}-benzylsukcinimid. Zbylá výchozí látka ochotně reaguje v~tetrachlormethanu se vzniknuvším sukcinimidem za vzniku \textit{N}-benzylsukcinimidu,
který nevábně páchne kvůli imidové skupině.
\end{enumerate}
\end{quotation} \dotfill \par 
\textit{N}-bromsukcinimid je bromační činidlo, které se využívá při radikálové bromaci. Výchozí látkou tedy byl toluen a onen „připečený, páchnoucí karamel“ byl geminální dibromid (A).

\hrulefill % \subsection*{Ročník 2, úloha č. 6.6 }
\begin{quotation}
\tri Ve vaší syntéze jste se posunuli opět o~kousek dál a po chvilce
čištění se vám podařilo oddělit smradlavý karamel, vzniklý v~předchozí úloze. První z~produktů, který jste získali destilací té oranžovohnědé
směsi za sníženého tlaku, necháte zreagovat stále pod atmosférou dusíku
s asi třemi ekvivalenty kyanidu draselného rozpuštěného ve skoro vroucím dimethylsulfoxidu (DMSO)\footnote{Zde je třeba upozornit na velkou prostupnost
DMSO přes běžné laboratorní rukavice. Proto postupujte při myšlenkovém
experimentu velice opatrně a snažte se rozpuštěný kyanid myšlenkově
nikde nerozlít.}. Po reakci, která trvala více než týden, jste nedočkavě
odlili malou část reakční směsi a tu nastříkli na již připravenou
kolonu pro plynovou chromatografii. Překvapilo vás ale, že místo jednoho
signálu produktu, který jste očekávali podle učebnice i na základě své
chemické intuice, vidíte na chromatogramu dva přibližně stejně velké
signály vypichující skutečnost, že vznikly dva různé produkty. Doplňte
chtěný produkt reakce a vyberte z~následujících možností, co bude druhým produktem.

\begin{enumerate}[label=\Alph*)]
\item Vznikla kyselina fenyloctová, protože došlo k~oxidaci vzdušným
kyslíkem. 
\item Vznikl benzyldikyanid, protože jsme přidali příliš mnoho kyanidu
draselného a reakce pokračovala do druhého stupně. 
\item Vznikl benzylalkohol, protože dimethylsulfoxid byl vlhký a
brom podstoupil nukleofilní substituci s~vodou jako nukleofilem. 
\item Vznikají dva stereoisomery, protože na methylové skupině vzniká
touto reakcí chirální centrum.
\end{enumerate}

\begin{center}
\includegraphics{/2/6-6}
\end{center}
\end{quotation} \dotfill \par 
Produktem této reakce je benzylkyanid, vedlejším produktem je pak
benzylalkohol, protože rozpouštědlem je velice hygroskopická látka,
dimethylsulfoxid. Správná je tedy odpověď C.

\hrulefill % \subsection*{Ročník 5, úloha č. 7.1 }
\begin{quotation}
\ctyri 2,4,5-trichlorfenoxyoctová kyselina byla používána jako neselektivní
herbicid. Tato látka se stala známou především jako jedna ze složek
defoliantu\footnote{Defoliant je druh herbicidu, který způsobí opadání listů.} Agent Orange, který byl použit ozbrojenými silami Spojených států
ve válce ve Vietnamu. 2,4,5-trichlorfenoxyoctová kyselina byla vyráběna
také v~Československu (Spolana Neratovice). Sama o~sobě tato látka
není pro člověka příliš toxická. Při její výrobě ale vzniká také
vedlejší produkt \textbf{Y}. Látka \textbf{Y} patří mezi dioxiny, je značně toxická,
karcinogenní a teratogenní. Doplňte vzorce \textbf{X}, \textbf{Y} a \textbf{Z}. Napovíme, že látka \textbf{Z}~obsahuje právě dva atomy uhlíku. 
\begin{center}
\includegraphics{/5/7-1} 
\par\end{center}

\end{quotation} \dotfill \par 
\begin{center}
\begin{tabular}{ccr}
\textbf{X}: & \textbf{Y}: & \textbf{Z}:\tabularnewline
\multirow{3}{*}{\includegraphics{/5/7-1-1}} & \multirow{3}{*}{\includegraphics{/5/7-1-2}} & $\ch{Cl-CH2-COOH}$\tabularnewline
 &  & nebo $\ch{Br-CH2-COOH}$\tabularnewline
 &  & nebo $\ch{I-CH2-COOH}$\tabularnewline
\end{tabular}
\par\end{center}

Látka \textbf{X} vzniká nukleofilní aromatickou substitucí tetrachlorbenzenu aniontem $\ch{OH-}$ za zvýšeného tlaku a teploty a následnou acidobazickou reakcí. Z~látky \textbf{X} potom nukleofilní substitucí některé z~halogenovaných kyselin octových (chlor-, brom-, či jodoctové) získáme látku \textbf{Z}. Látka \textbf{Y} vzniká dvojitou, vzájemnou nukleofilní aromatickou substitucí dvou molekul látky \textbf{X}, při níž dochází ke vzniku heterocyklu.

\newpage %nadpis%
\section{Dusíkaté deriváty}

 % \subsection*{Ročník 5, úloha č. 3.5  }
\begin{quotation}
\dva Diazotetrazol je značně nestálou látkou. Svědčí o~tom i fakt, že
Thiele, který tuto látku poprvé připravil diazotací 5-aminotetrazolu,
ji nebyl schopen identifikovat, jelikož v~roztoku při 0\,°C okamžitě
explodovala. 
\begin{center}
\includegraphics{/5/3-5} 
\par\end{center}
Napište rovnici rozkladu, který Thiele viděl a slyšel. Exploze je
natolik rychlým procesem, že při ní látka nestihne chemicky reagovat
s prostředím, ve kterém se nachází.
\end{quotation} \dotfill \par 
Diazotetrazol obsahuje šest atomů dusíku a jeden atom uhlíku. Dusík
s uhlíkem netvoří žádné jednoduché a zároveň velmi stálé binární sloučeniny,
a proto se diazotetrazol rozpadá až na prvky. Vyčíslená rovnice rozkladu
je potom 
\[
\ch{CN6 \ \rightarrow \mathrm{C} + 3 N2}
\]

\hrulefill % \subsection*{Ročník 4, úloha č. 7.1 }
\begin{quotation}
\ctyri Mannichova reakce je reakcí aldehydu, aminu a ketonu. 
\begin{center}

\includegraphics{/4/7-1-0}

\par\end{center}
Uplatnění v~organické syntéze nachází například při syntéze alkaloidů,
které tak lze pohodlně připravit z~běžných látek. Například při syntéze
atropinu, kterým se léčí otravy nervovými plyny a pesticidy, nebo
třeba pomalá tepová frekvence, narazíme na meziprodukt tropinon. Jaké
prekurzory potřebujeme pro syntézu tropinonu jednokrokovou Mannichovou
reakcí? 
\begin{center}

\noindent \begin{centering}
\includegraphics{/4/7-1-trop}
\par\end{centering}
Strukturní vzorec tropinonu

\par\end{center}
Nápověda: Jedním z~prekurzorů pro syntézu tropinonu je 1,4-dikarbonylová
sloučenina.
\end{quotation} \dotfill \par 
Při porovnání obecného schématu Mannichovy reakce a molekuly tropinonu
dospějeme k~závěru, že tropinon lze připravit jednokrokovou Mannichovou
reakcí ze sukcinaldehydu (butandialu), methylaminu a acetonu. 
\noindent \begin{center}

\includegraphics{/4/7-1}

\par\end{center}

Se sukcinaldehydem se můžeme setkat i ve formě hydrátu (2,5-dihydroxytetra\-hydrofuranu),
který se dá použít také. Methylamin je za standardních termodynamických
podmínek plynný, a tak raději než s~bombou plynu pracujeme s~jeho solemi,
které lze též uznat. Protože samotný aceton reaguje neochotně a poskytuje
tak nízké výtěžky, používá se namísto něj acetondikarboxylová kyselina,
která je reaktivnější díky karboxylovým skupinám. Kyselinu acetondikarboxylovou
lze uznat také, přestože musí po Mannichově reakci ještě podlehnout
dvojité dekarboxylaci a syntézu potom nelze nazvat jednokrokovou.

\section{Úlohy s~využitím NMR}

% \subsection*{Ročník 1, úloha č. 4.6}
\begin{quotation}
\jeden Svět chemiků je plný mnoha různých zkratek a symbolů, jimiž se brání
přesile složitých systematických názvů. Mnohdy už se objevují zkratky,
které nelze rozeznat od skutečných anglických slov -- představte
si třeba, že do baňky sypete sloučeninu, kterou běžný organik tituluje
výhradně jako DEAD. Oproti tomu existují mnohem ochočenější, probádanější
a hezčí zkratky. Tak například NMR je metodou analýzy struktury sloučenin,
jež využívá zajímavých vlastností jader atomů. Některá jádra (ta,
která mají nenulový spin) po vložení do magnetického pole absorbují
určité délky elektromagnetického záření, což se projeví jako signál
(pík) v~absorpčním spektru. Kdy a kolik signálů se může objevit? 

Představme si nějakou hezky symetrickou molekulu, třeba fluorid sírový.
Z~teorie VSEPR si i student nechemik zjistí, že fluorid sírový má
tvar oktaedru s~atomem síry v~jeho středu, všechny atomy jsou tedy
ve vzájemně ekvivalentních polohách a není mezi nimi jediného chemického
rozdílu. V~$\mathrm{^{19}F}$ NMR spektru se proto objeví jediný signál.
Podívejme se v~periodické tabulce jen o~políčko dál a zaměňme síru
za atom fosforu $\mathrm{^{31}P}$. Jádro fosforu je NMR-aktivní,
což se ve spektru projeví tak, že každých $n$ NMR-aktivních jader rozštěpí
signál ve spektru na $n+1$ menších píků. Pokud bychom tedy měli hypotetický
či reálný hexafluoridofosforečnanový anion, ve spektru $\mathrm{^{19}F}$
se v~důsledku štěpení atomem fosforu objeví 2 signály, ve spektru
$\mathrm{^{31}P}$ se na obrazovce počítače ukáže 7 signálů, což odpovídá
štěpení šesti atomy fluoru.\\Na základě nastíněného jednoduchého schématu
zvažte a následně autorům sdělte, kolik signálů se objeví v~$\mathrm{^{1}H}$
NMR spektru oxidu uhličitého. Nezapomeňte zvážit symetrii a hybridizaci
molekuly! 
\end{quotation} \dotfill \par 
Po důkladném prostudování textu úlohy si rádi povšimneme,
že zadavatelé úlohy se nás ptají na počet signálů molekuly $\mathrm{CO_{2}}$
ve „vodíkovém enemáru“. Molekula $\ce{CO2}$ nemá vodíky, tudíž krom šumu
se pražádný signál neobjeví. 

\hrulefill % \subsection*{Ročník 3, úloha č. 7.6 }
\begin{quotation}
\dva Svět chemiků je plný mnoha různých zkratek a symbolů, jimiž se brání
přesile složitých systematických názvů. Mnohdy už se objevují zkratky,
které nelze rozeznat od skutečných anglických slov -- ostatně o~tom
jste se přesvědčili v~jedné z~předchozích úloh. Teď ale následuje
jedna úloha opět o~nukleární magnetické resonanci, která se běžně
zkracuje jako NMR.

Představme si nějakou zajímavou molekulu, třeba bromaceton.
Ten kromě toho, že se po něm brečí, má v~$^{1}$H NMR spektru dva
signály, jeden o~chemickém posunu\footnote{Chemický posun je konstrukt, kteří si chemikové vymysleli, aby jejich signály měly malá čísla. Jedná se v~podstatě o~relativní změnu frekvence vůči nějaké základní frekvenci, při které daný atom absorbuje elektromagnetické záření. Pro vyřešení úlohy ale není znalost teorie NMR podstatná.} 3,86, druhý o~chemickém posunu 2,30. Toto plně odpovídá
teorii, jelikož atomy, které jsou chemicky ekvivalentní, produkují
pouze jeden signál. Na základě velikosti signálu se dá pak odvodit
počet vodíků příslušejících k~tomuto signálu. První signál příslušející
k CH$_{2}$Br skupině má skutečně oproti signálu CH$_{3}$ skupiny
dvoutřetinovou velikost.\\Látka \textbf{A} o~sumárním vzorci C$_{6}$H$_{12}$O
má NMR spektrum, která obsahuje signály s~následujícím posunem (v závorce uveden signálu příslušný počet atomů vodíku): 2,13 (3H); 1,20 (9H).\\Látka \textbf{B} o~sumárním vzorci C$_{5}$H$_{12}$O,
která je strukturně podobná látce \textbf{A}, má téměř stejné NMR
spektrum: 3,30 (3H); 1,19 (9H). Nakreslete strukturní
vzorce látek \textbf{A} a \textbf{B}.

\textit{Nápověda: Látka }\textbf{\textit{A}}\textit{ se triviálně
nazývá pinakolon. Veškeré signály jsou singlety, tedy nejsou nijak dále štěpené.}
\end{quotation} \dotfill \par 
Pinakolon (látka \textbf{A}) je, jak koncovka názvu napovídá, ketonem.
Jediný způsob, jak navázat zbylé atomy na karbonylovou skupinu, aby
odpovídaly spektru, je tento:

\begin{center}
\includegraphics{/3/7-6-1}
\end{center}
V této molekule jsou všechny tři methylové ($\ch{CH3 -}$) skupiny v~terc-butylu ekvivalentní, dají tedy jeden signál, který bude třikrát větší než signál methylové skupiny vpravo.
\newpage %text%

Analogickou strukturu má též látka \textbf{B}; je to ether následujícího vzorce:
\begin{center}
\includegraphics{/3/7-6-2}
\end{center}

\hrulefill % \subsection*{Ročník 3, úloha č. 6.6  }
\begin{quotation}
\tri Svět chemiků je plný mnoha různých zkratek a symbolů, jimiž se brání
přesile složitých systematických názvů\ldots\\ 
\ldots\-Ano, jsme si vědomi, že opakovaný vtip přestává být vtipem.
NMR spektra drtivé většiny látek obsahují spleť
složitě štěpených signálů, jejichž rozklíčování a souvislost mnohdy
umožní dešifrovat nebo výrazně poodhalit strukturu molekuly. Velmi
vzácnými okamžiky tak bývají ty, kdy se na obrazovce objeví jen málo
signálů nebo dokonce signál jediný, i ten však může být cennou informací;
běžně totiž jeho přítomnost znamená, že je molekula velmi pravidelně
či symetricky uspořádána a že jsou všechny vodíky chemicky naprosto
identické. Mezi takto mimořádně symetrické molekuly patří například
methan, 2,2-dimethylpropan nebo cyklopentan. 

Sami zkuste navrhnout a nakreslit pro následující tři sumární vzorce
struktury, pro které se v~jejich $^{1}$H NMR spektru objeví jediný
signál. 
\begin{description}
\item [{a)}] C$_{3}$H$_{6}$O 
\item [{b)}] C$_{8}$H$_{18}$ 
\item [{c)}] C$_{6}$H$_{6}$
\end{description}
\end{quotation} \dotfill \par 
V zadání jsme se pokusili napovědět trojicí
velmi symetrických struktur -- methanem (čtyřstěn s~atomy vodíku
ve vrcholech), 2,2-dimethylcyklopropanem (analogicky, jen s~methylovými
skupinami) a cyklopentanem jako obyčejnou cyklickou molekulou, naznačujíce
tak některé strukturní motivy, které by se mohly objevit i při řešení. Jejich vzorce Vám přikládáme zde:

\begin{center}
\includegraphics{/3/6-6-0}
\end{center}

Co tedy \textbf{a)} C$_{3}$H$_{6}$O? Ze vzorce vyplývá stupeň nenasycenosti\footnote{Stupněm nenasycenosti se myslí počet dvojných vazeb nebo kruhů, které molekula obsahuje. Stupeň nenasycenosti se dá i určit ze sumárního vzorce, ostatně o~tom též pojednávají některé další úlohy této sbírky.} 1, tedy
jedna dvojná vazba nebo jeden cyklus. Kdo by chtěl hledat stejné motivy,
mohl by vytvořit cyklus, avšak cyklus s~kyslíkem, oxetan, by neměl
šest vodíků v~ekvivalentní poloze. Ve vzorci však je možné nalézt
dvě methylové skupiny, po jejichž „odečtení“ zůstane CO, tedy karbonyl,
na který mohou být z~obou stran methylové skupiny nalepeny -- a vida,
první hledanou molekulou je aceton, který skutečně má rovinu symetrie
a šest ekvivalentních vodíků. 

\begin{center}
\includegraphics{/3/6-6-1}
\end{center}

Ze vzorce \textbf{b)} C$_{8}$H$_{18}$ můžeme vyčíst stupeň nenasycenosti 0,
tedy pouze větvení a ne cyklení. Je proto potřeba uspořádat menší
symetrické kousky symetrickým způsobem podobně jako u~2,2-dimethylpropanu.
Pokud bychom chtěli tvořit řetězec, okamžitě bychom narazili na problém
s rozdílností CH$_{3}$ a CH$_{2}$ skupin, můžeme tedy použít jen
methyly a žádné jiné umístění vodíků. Potřebujeme spojovat několik
kousků a ze tří methylů a jednoho atomu uhlíku slepíme s~volným koncem
nejvýše \textit{terc}-butyl, což odpovídá sumárnímu vzorci C$_{4}$H$_{9}$.\\
A hle, polovina problému vyřešena, protože na volný konec můžeme přilepi
jiný \textit{terc}-butyl a jsme u~cíle, tedy u~2,2,3,3-tetramethylbutanu, další
velmi symetrické molekuly! 

\begin{center}
\includegraphics{/3/6-6-2}
\end{center}

Na konec možná ten nejsnazší úkol \textbf{c)} -- mnohdy stačí chemikovi položit otázku
„Co je to C$_{6}$H$_{6}$?“, aby vám odpověděl: „Benzen“. A takto
přímočará odpověď stačí i tady -- molekula benzenu je planární a
opravdu vysoce symetrická\footnote{Dalším řešením by mohla být též molekula prismanu, o~němž je řeč v~jedné z~následujících úloh. Vzhledem k~tomu, že se jedná o~molekulu poněkud bizarní, tak zde neuvádíme její strukturu.}. 

\begin{center}
\includegraphics{/3/6-6-3}
\end{center}

\section{Kyslíkaté deriváty uhlovodíků}

 % \subsection*{Ročník 2, úloha č. 5.5}
\begin{quotation}
\dva V~nejmenovaných kruzích kolují legendy o~tom, že se pivo Ostravar
vyrábí z~černouhelného dehtu\footnote{Autorský kolektiv se od tohoto výroku distancuje.}. Takovému pivu by ale rozhodně chyběl
kýžený ethanol. Vaším úkolem je navrhnout syntézu ethanolu z~jiného
produktu koksárenského průmyslu -- acetylenu -- v~maximálně třech
reakčních krocích.
\end{quotation} \dotfill \par 
Možností je celkem  dost, například:\\Varianta A -- ekonomicky asi nejvýhodnější.
Katalyzátor se dá za určitých podmínek recyklovat. Syntéza obsahuje dva reakční kroky:

\begin{center}
\includegraphics{/2/5-5a}
\end{center}

Varianta B -- ,,one pot`` syntéza\footnote{Jako ,,one pot`` syntéza se označuje taková syntéza, která se dá provést v~jedné baňce postupným přidáváním činidel bez izolace meziproduktů.}, Reakce jsou z~didaktických důvodů rozepsané, suma sumárum se ale jedná o~dva kroky: hydroxymerkuraci a redukci borohydridem.

\begin{center}
\includegraphics{/2/5-5b}
\end{center}

Varianta C -- Nadbytek boranu redukuje vznikající aldehyd na aklohol.

\begin{center}
\includegraphics{/2/5-5c}
\end{center}

Varianta D -- Toto řešení je zbytečně komplikované, ale splňuje podmínku 3 reakčních kroků:

\begin{center}
\includegraphics{/2/5-5d}
\end{center}

Uznáváme každé správné řešení, které splňuje podmínku maximálně třech reakčních kroků. 

\newpage %%
% \subsection*{Ročník 5, úloha č. 0.12}
\begin{quotation}
\jeden Jakou molekulovou hmotnost má látka, která se odštěpuje při dehydrataci?
\end{quotation} \dotfill \par 
Odpověď je jednoduchá a pochází ze starořeckého \foreignlanguage{greek}{ὕδωρ} ({[}h{]}ýdôr{]}), tedy voda, jejíž molekulová hmotnost je rovna
18.


\hrulefill % \subsection*{Ročník 4, úloha č. 2.1 }
\begin{quotation}
\dva Biochemik Jardík si hrál v~organické laboratoři -- z~neznámé látky \textbf{A}
syntetizoval ethanol, z~nějž nakonc připravil látku \textbf{B}. Když to
zjistil jeho učitel, dal mu za úkol z~látky \textbf{B} zpátky připravit
látku \textbf{A}. Jardík to zvládl prostřednictvím přeměny nejprve na
látku \textbf{C}, z~níž pak dokázal připravit látku \textbf{A}. Vaším úkolem
je doplnit názvy a vzorce látek \textbf{A, B} a \textbf{C}. 

Nápověda: $\ce{LiAlH4}$ je silné redukční činidlo. Reakce ethanolu s~kyselinou
chromovou, stejně jako reakce látky \textbf{B} s~$\ce{LiAlH4}$ probíhá
až do posledního možného kroku.
\begin{center}

\includegraphics{/4/2-1}

\par\end{center}

\end{quotation} \dotfill \par 
Reakcí ethanolu se silným oxidovadlem, jako je kyselina chromová,
vznikla kyselina octová (\textbf{B}). Tu potom redukoval pomocí nadbytku
$\ce{LiAlH4}$ za opětovného vzniku ethanolu (\textbf{C}). Ethanol eliminací
vody v~přítomnosti kyseliny fosforečné poskytl ethen (\textbf{A}),
z nějž nešťastný Jardík vycházel.

\hrulefill % \subsection*{Ročník 5, úloha č. 6.1 }
\begin{quotation}
\ctyri Pinakolový přesmyk je reakce, které podléhají 1,2-dioly v~kyselém
prostředí za vzniku ketonů. Při reakci nejprve dochází k~protonaci
jedné OH skupiny, která se následně odštěpí ve formě vody za vzniku
karbokationtu, který podléhá samotnému přesmyku. Z~pinakolu takto
vzniká pinakolon, jak je zobrazeno níže. 
\begin{center}
\includegraphics{/5/6-1} 
\par\end{center}
Jaký bude produkt přesmyku následující látky? 
\begin{center}
\includegraphics{/5/6-1-1} 
\par\end{center}

\end{quotation} \dotfill \par 
Při prvotním porovnání struktury pinakolu a zadaného diolu nemusí
být produkt na první pohled zřejmý. Pokud ale budeme postupovat podle
nakresleného mechanismu, dospějeme k~hledanému produktu, spirocyklickému
ketonu. Mechanismus je nakreslen níže. Samotný přesmyk karbokationtu
probíhá tak, aby vznikl karbokation nejstabilnější, v~tomto případě
tedy vedle zbývající OH skupiny. 
\begin{center}
\includegraphics{/5/6-1-2} 
\par\end{center}

\newpage %%
 % \subsection*{Ročník 4, úloha č. 5.6  }
\begin{quotation}
\tri Pokaždé, když nestíháte nebo potřebujete dělat několik věcí zároveň,
se můžete dopustit zásadní chyby. Tím spíše to platí v~laboratoři,
pokud pracujete s~několika nepopsanými kádinkami. Takto jednou jeden
student bakalářského studia neopatrně rozpustil odvážený acetylchlorid
ve větším množství triethylaminu. Odměnou mu byl ještě nepříjemnější
smrad, než se kterým se dosud musel potýkat. Onen vznikající plyn
je jedovatý, výrazně páchne a je tak reaktivní, že je schopen dimerizovat
sám se sebou. Ochotně reaguje s~vodou, přičemž produkt této reakce
používáme běžně ve zředěné formě v\,kuchyni. Pokud dosud nepoznáváte,
napovím, že molární hmotnost tohoto plynu je po zaokrouhlení rovna\ldots
ano, 42 gramům na mol. Nakreslete strukturní vzorec tohoto plynu.
\end{quotation} \dotfill \par 
\noindent \begin{center}

\includegraphics{/4/5-6}

\par\end{center}

Reagující triethylamin a acetylchlorid oba obsahují nejvýše dvouuhlíkaté
zbytky. Sudá molární hmotnost 42~$\mathrm{g\cdot mol^{-1}}$ vylučuje přítomnost dusíku
v hledané molekule, bez dvou atomů uhlíku tedy zbývá 18 „volných“~$\mathrm{g\cdot mol^{-1}}$, které mohou odpovídat jedině jednomu atomu kyslíku a dvěma
atomům vodíku, sumární vzorec je tedy $\ce{C2H2O}$. Jelikož tento
fragment zjevně pochází z~molekuly acetylchloridu a podle rozdílu
sumárních vzorců vzešel „eliminací“ pomocí triethylaminu jako báze,
vznikla po eliminaci HCl další dvojná vazba. Kumulované dvojné vazby
také vysvětlují enormní reaktivitu vedoucí k~velmi rychlé dimerizaci.

\hrulefill % \subsection*{Ročník 4, úloha č. 4.1  }
\begin{quotation}
\dva Cannizzarova reakce je mimořádně zajímavá už jen z~toho důvodu, že
jde o~jeden z~mála vzácných případů disproporcionace organických sloučenin.
Této reakci podléhají v~bazickém prostředí aldehydy, které v~bezprostředním
sousedství aldehydové skupiny (v\,$\alpha$-poloze, tedy na uhlíku
hned vedle) nemají žádné potenciálně kyselé atomy vodíku; v~opačném
případě dochází k~aldolizaci a někdy i ke kondenzaci. Obecným schématem
této reakce je 
\[
\ce{2R\bond{-}CHO\rightarrow R\bond{-}COO}^{-}+\ce{R\bond{-}CH2OH}
\]
 Na základě výše uvedených rad rozhodněte, zdali níže uvedené reaktanty
mohou podléhat Cannizzarově reakci (zapsáním ANO/NE).
\begin{enumerate}[label=\alph*)]
\item hexan-1,6-dial,
\item 4-methoxybenzenkarbaldehyd,
\item 2,2-dimethylpropanal,
\item cyklopentankarbaldehyd,
\item 3-hydroxy-2,2-bis(hydroxymethyl)propanal. 
\end{enumerate}
\end{quotation} \dotfill \par 
\begin{enumerate}[label=\alph*)]
\item NE, kyselých $\alpha$-vodíků je mnoho u~obou aldehydových
skupin.
\item ANO, aromatické vodíky nejsou kyselé a jsou daleko.
\item ANO, v~$\alpha$-poloze jsou navěšené methylové skupiny, nikoli
vodíky.
\item NE, jeden kyselý vodík se skrývá na uhlíku kruhu.
\item ANO, ze stejného důvodu jako c), pouze methylové skupiny jsou
substituované, přítomnost hydroxylových skupin reakci neovlivňuje.
\end{enumerate}

\newpage %%
 % \subsection*{Ročník 1, úloha č. 4.2. }
\begin{quotation}
\dva Po esterifikaci byl úspěšně oddestilován ethylbutyrát a do výlevky
vylit roztok chloridu sodného. Navrhněte, z~jakých výchozích látek
byl ester syntetizován. 
\end{quotation} \dotfill \par 
\begin{center}
\includegraphics{/1/4-2}
\end{center}

Výchozími látkami mohou být chlorid kyseliny butanové (butanoylchlorid,
chlorid kyseliny máselné) a ethoxid sodný (ethanolát sodný), nebo
butanoát sodný (butyrát sodný) a chlorethan (ethylchlorid). 

\section{Širokozáběrové syntézní úlohy}

% \subsection*{Ročník 5, úloha č. 1.3 }
\begin{quotation}
\jeden Při syntéze hexafenylbenzenu můžeme pohodlně vycházet z~difenylacetylenu
a tetrafenylcyklopentadienonu. Reakce probíhá přes bicyklický intermediát.
Jaká látka \textbf{X} vzniká vedle chtěného produktu? 
\begin{center}
\includegraphics{/5/1-3} 
\par\end{center}

\end{quotation} \dotfill \par 
Nejprve dochází k~Dielsově--Alderově reakci za vzniku zmíněného
meziproduktu. Z~něho dospějeme k~hexafenylbenzenu extruzí oxidu uhelnatého
(CO). Jednodušší úvahou dojdeme ke vznikajícímu oxidu uhelnatému odečtením
sumárního vzorce hexafenylbenzenu od součtu sumárních vzorců výchozích
látek. 
\begin{center}
\includegraphics{/5/1-3-1} 
\par\end{center}

\newpage %%
 % \subsection*{Ročník 4, úloha č. 5.1 }
\begin{quotation}
\dva Otevřeme-li devátý díl Ottova slovníku naučného na straně 87, upoutají
naši pozornost hned dvě chemická hesla za sebou: Fenacetin a Fenanthren.
Podívejme se blíže na první z~nich: \\
\textit{Fenacetin, acetfenetidin, léčivý prostředek synthetické lučby doby
nejnovější. V~lékařství upotřebuje se ho hojně hlavně proti horečce (\ldots).
Příprava: natrium-paranitrofenol pomocí jodethylu promění se
v aethylaether fenolu $\ce{C6H4NO2OC2H5}$ (\textbf{I}), kterýž pak redukcí
se mění v~paraamidofenol $\ce{C6H4NH2OC2H5}$ (\textbf{II}). Vařením této sloučeniny
s ledovou kyselinou octovou obdržíme \textbf{f} (\textbf{III}). Výroba děje se po továrnicku.
\textbf{F}. přichází do obchodu v~bílých lesklých šlupičkách bez vůně a chuti;
rozpouští se snadno v~líhu, těžko ve studené vodě; v~zahřáté vodě
a v~glycerinu snadněji. Léčivo to nesmí se v~lékárnách bez předpisu
lékaře vydati. Předpisuje se v~dávkách 0,1 -- 1 g}\footnote{Ottův slovník naučný. Devátý díl. Praha : J. Otto, 1895.}.

Pomocí strukturních vzorců znázorněte produkty reakcí (\textbf{I}), (\textbf{II}) a (\textbf{III}) .

\end{quotation} \dotfill \par 
Jde po řadě o~(\textbf{I}) nukleofiní substituci (Williamsonovu syntézu etheru,
vzniká NaI), (\textbf{II}) redukci nitroskupiny na aminoskupinu a (\textbf{III}) tvorbu
amidu reakcí karboxylové kyseliny s~primárním aminem\footnote{Podotýkáme, že reakce \textbf{III} se laboratorně provádí spíše s~acetanhydridem z~důvodu jeho větší reaktivity.}. 
\noindent \begin{center}

\includegraphics{/4/5-1}

\par\end{center}

\hrulefill % \subsection*{Ročník 3, úloha č. 5.5 }
\begin{quotation}
\dva Karboxylová kyselina o~sumárním vzorci C$_{5}$H$_{8}$O$_{2}$ (\textbf{A})
tvoří dva geometrické izomery. Oba izomery po zredukování vodíkem
na platině tvoří další karboxylovou kyselinu (\textbf{B}), která se
v roztoku nachází ve formě dvou téměř stejně zastoupených stereoizomerů.
Nakreslete oba izomery od kyseliny \textbf{A} i \textbf{B}.
\end{quotation} \dotfill \par 
Ze sumárního vzorce kyseliny \textbf{A} a z~toho, že má kyselina tvořit
dva geometrické izomery, vyplývá, že tato kyselina bude mít jednu
dvojnou vazbu, která při zredukování na kyselinu \textbf{B} přejde
na jednoduchou. Vzhledem k~tomu, že kyselina \textbf{B} je dle zadání
chirální, musí redukcí dvojné vazby vzniknout chirální centrum (což
odpovídá zadání, při redukci vzniká racemát). Jeden z~uhlíků dvojné
vazby tedy na sobě musí mít dva různé substituenty a žádný vodík.
Oba izomery (\textit{E} i \textit{Z}) látky \textbf{A} jsou na obrázku níže:
\noindent \begin{center}

\includegraphics{/3/5-5-1}

\par\end{center}

Redukcí vznikne na $\upalpha$-uhlíku chirální centrum, oba dva enantiomery
(\textit{S}/\textit{R}) jsou na obrázku níže:
\noindent \begin{center}

\includegraphics{/3/5-5-2}

\par\end{center}

\newpage %%
 % \subsection*{Ročník 4, úloha č. 6.4 }
\begin{quotation}
\tri Sacharidy i přes svou sladkost umí pěkně zhořknout na jazyku a práce
s nimi je noční můrou pro nejednoho organického chemika. To je způsobeno
velkým množstvím hydroxylových skupin, které je potřeba chránit, a
tvorbou neočekávaných (a navzájem naprosto odlišných) produktů za
zdánlivě stejných reakčních podmínek. Jednou z~tradičních a velmi
často používaných chránících skupin je skupina acetalová/ketalová.
Po vás nyní chceme, abyste se vžili do role chemika, který chce studovat
mechanismus tvorby příslušných ace\-ta\-lů/ke\-ta\-lů, k~čemuž
se rozhodl použít izotopově značené karbonylové sloučeniny. Protože
je nám jasné, že chtít po vás celé mechanismy by bylo moc brutální
(a hlavně by se to blbě opravovalo), tak nepožadujeme nic víc než
napsat strukturní vzorce produktů reakcí \textbf{A} a \textbf{B} obsahujících kyslík
$\mathrm{^{18}O}$, pokud: 
\begin{description}
\item [{A}] $\upbeta$-D-glukopyranosa (2,3,4,5-tetrahydroxo-6-hydroxymethyloxan)
reaguje s~benzaldehydem--$\mathrm{^{18}O}$ za kyselé katalýzy (nápověda:
jeden z~produktů obsahuje 2 šestičlenné kruhy),
\item [{B}] $\upbeta$-D-glukopyranosa regauje s~acetonem--$\mathrm{^{18}O}$
za kyselé katalýzy (nápověda: jeden z~produktů obsahuje tři pětičlenné
kruhy).
\end{description}
\end{quotation} \dotfill \par 
\noindent \begin{center}

\includegraphics{/4/6-4}

\par\end{center}

Při tvorbě acetalu dojde k~protonaci karbonylového kyslíku benzaldehydu/aceto\-nu.
Poté je takto aktivovaná karbonylová skupina nukleofilně napadena
hydroxylem glukózy. Na vzniklém tetraedrickém intermediátu pak dojde
k intermolekulárnímu přenosu protonu a následnému odstoupení izotopově
značené vody (v mechanismu je kyslík $\ce{^{18}O}$ vyznačen červeně).
Následuje atak dalšího hydroxylu a po ztrátě katalytického protonu
vzniká příslušný monoacetal glukózy, který ovšem již neobsahuje žádný kyslík $\ce{^{18}O}$.

Přesto může v~průběhu reakce docházet k~„izotopovému zašpinění“ hemiacetalového
hydroxylu glukózy kyslíkem $\ce{^{18}O}$ sekvencí dějů: otevření pyranosového
cyk\-lu\ $\rightarrow$\ hyd\-ra\-tace karbonylové skupiny cukru izotopově značenou
vodou vzniklou v~re\-ak\-ci\ $\rightarrow$\ dehydratace hydratovaného karbonylu\ $\rightarrow$\ zpětné
uzavření cyklu (vyznačeno modře ve strukturách organických produktů
na obrázku). Míra tohoto „špinění“ je závislá na reakčních podmínkách.
Vzhledem k~tomu, že se reakce většinou provádí s~velkým nadbytkem
acetonu/benzaldehydu, je jeho míra minimální, protože aceton/benzaldehyd
mimoděk slouží i jako „scavenger“ vody. Nám ale stačila prostá odpověď\footnote{V případě benzaldehydu je preferován vznik stabilnějšího šestičlenného
acetalu, kde objemný fenyl zaujímá ekvatoriální polohu. Naopak v~případě
reakcí s~acetonem je termodynamicky preferován vznik o~něco málo méně
stálého kruhu pětičlenného, protože v~šestičlenném kruhu by byl jeden
z methylů nucen zaujmout axiální polohu, což je energeticky nevýhodné.
Tato znalost však k~řešení úlohy nebyla vůbec nutná.}, a to: 

\textbf{A} \ch{H2\-^{18}O}; \textbf{B} \ch{H2\-^{18}O}.

\newpage %%
 % \subsection*{Ročník 3, úloha č. 7.2 }
\begin{quotation}
\ctyri Chirální kyselina octová sehrála zásadní roli při určování mechanismů
některých enzymů. Její hlavní předností je značení radioaktivním tritiem,
jež lze velmi snadno detekovat. Vy si nyní můžete vyzkoušet tuto neobvyklou
chirální molekulu syntetizovat Arigoniho enovou kaskádou. Doplňte
meziprodukty \textbf{A--C} v~následující syntéze. Výchozí látka reaguje
s ethynyllithiem a po chirální resoluci poskytne (\textit{R})-enantiomer látky
\textbf{A} (pokud chcete, tak stereochemii ve svých odpovědích zanedbejte).
Látka \textbf{A} pak reaguje se deuterovaným methoxymethylchloridem
za přítomnosti báze a poskytuje sloučeninu \textbf{B}. Sloučenina
\textbf{B} potom reaguje s~butyllithiem a poté je do reakční směsi
přikapána supertěžká voda, což poskytne sloučeninu \textbf{C}. Syntéza
je dokončena zahřátím látky \textbf{C}, což spustí enovou a retro-enovou
reakci. Posledním krokem je brutální oxidace oxidem chromovým v~přítomnosti
kyseliny sírové, která poskytne chirální kyselinu octovou v~mizerném
výtěžku, ale slušném enantiomerním přebytku (93\% ee).

\begin{center}
\includegraphics[scale=0.95]{/3/7-2}
\end{center}
\end{quotation} \dotfill \par 
Syntéza této neobvyklé látky je zahájena adicí ethynyllithia, které
je dobrý nukleofil, na karbonylovou skupinu výchozí látky za vzniku
příslušného alkoholu jako směsi enantiomerů. Ta chirální resolucí
s kyselinou\\(\textit{R})-2-((1,3-dioxoisoindolin-2-yl)methyl)-4-methylpentanovou
poskytne alkohol \textbf{A}. Ten je poté za bazických podmínek naalkylován
deuterovaným methoxymethyl chloridem za vzniku derivátu \textbf{B}.
Butyllithium odtrhne nejkyselejší proton derivátu \textbf{B}, tím
je terminální alkynylový proton (\pKa přibližně 25),
za vzniku příslušného organokovu, který po rozložení supertěžkou vodou
poskytne příslušný tritiovaný derivát \textbf{C}. Poté v~klíčovém
kroku syntézy dojde k~efektní a efektivní enové-retroenové kaskádě.
Posledním krokem je Khunova-Rothova oxidace, která zoxiduje všechny atomy
uhlíku v~molekule kromě methylových na oxid uhličitý a poskytne tak
chirální kyselinu octovou.

\begin{center}
\includegraphics{/3/7-2-2}
\end{center}

\newpage %%
% \subsection*{Ročník 3, úloha č. 7.4 }
\begin{quotation}
\tri Z~ultrafialového záření o~vlnových délkách kolem 300 nm dopadajícího
na Zemi je velká část pohlcena stratosférickým ozonem. Významná část
ovšem projde a dopadá na povrch, kde způsobuje poškození dědičné informace
v organismech (mutageneze). 

Hlavní příčinou tohoto poškozování je přímá absorpce krátkovlnného
záření biomolekulami, která vede ke vzniku volných radikálů a dalších
reaktivních částic. Jedním z~možných způsobů poškození je tvorba aduktů
nukleových bází na vlákně DNA, které se projevuje chybami při expresi
genů. 

Podívejme se na případ, kdy vlákno obsahuje po sobě jdoucí pyrimidinové
báze T a C. UV záření může v~tomto případě vyvolat dimerizaci bazí, která probíhá jako {[}2+2{]}
cykloadice. Zakreslete vzorec výsledného produktu, který obsahuje tři cykly.
\begin{center}
\includegraphics{Base}
\par\end{center}
Pyrimidinové (nahoře) a purinové báze (dole). Zleva doprava: C, T, U; A, G.
\end{quotation} \dotfill \par 
Tohoto typu adice se účastní dvojné vazby pyrimidinových bází za vzniku
cyklobutadienového uspořádání. Výsledný produkt vypadá takto:

\begin{center}
\includegraphics{Adukt}
\end{center}

5' a 3' označují navázanou ribosa-fosfátovou páteř nukleové kyseliny.

Dimery, které nejsou některým z~mechanismů oprav DNA vráceny do původního
stavu nebo vystřiženy, způsobí narušení činnosti DNA-polymeras a zastavení
replikace. Významně tak přispívají ke vzniku melanomů.\footnote{\textit{Photochem Photobiol Sci.} 2013 Aug; 12(8): 1409--1415. \href{https://www.ncbi.nlm.nih.gov/pmc/articles/PMC3731422/}{PMCID: \underline{PMC3731422}}}

\newpage %%
% \subsection*{Ročník 4, úloha č. 8.1 }
\begin{quotation}
\ctyri V~organické chemii se můžete
setkat se širokou škálou bází. Přiřaďte každé reakci z~následujících
bází tu optimální. Každá báze smí být použita právě jednou. 
\begin{center}

\includegraphics[scale=0.85]{/4/8-1-bez}

\par\end{center}
\begin{itemize}
\item potaš 
\item hydroxid sodný 
\item pyridin 
\item hydrid lithný 
\item ethoxid sodný
\item LDA
\end{itemize}
\end{quotation} \dotfill \par 
Ač nám může na první pohled přijít, že jsou některé báze zaměnitelné,
brzy zjistíme, že je každá z~dobrých důvodů použitelná jen v~jedné
z nabízených reakcí.

Začneme-li methylací derivátu cyklohexanonu, vidíme, že probíhá ve
stericky méně bráněné poloze. Abychom potlačili vznik izomeru vznikajícího
methylací ve stericky bráněné poloze, musíme použít stéricky objemnou
bázi. Jediná taková v\,nabídce je LDA. 

Při syntéze 3-fenylpropynové kyseliny (respektive její soli) začínáme
deprotonací fenylacetylenu. Jeho koncová $\ce{#C-H}$ skupina má vysoké \pKa a k~deprotonaci tudíž potřebujeme silnou bázi. S~výhodou využijeme
hydridu lithného, který je dostatečně silný a tvoří solváty s~ethery,
což usnadňuje reakci. 

Claisenova kondenzace je běžně prováděna alkoxidem sodným, kde uhlovodíkový
řetězec odpovídá uhlovodíkovému řetězci daného esteru. Jinak totiž
probíhá současně do určité míry i transesterifikace. 

Methylace fenolů se běžně provádí reakcí s~methyljodidem. Báze se
do reakce přidává proto, aby zachytávala vznikající jodovodík a zabránila
tak zpětné reakci. Jelikož je jodovodík značně kyselý, stačí použít
slabou bázi, obvykle uhličitan draselný (potaš).

Bazická hydrolýza esterů se provádí výhradně roztoky hydroxidů, jelikož
jde o~nejsilnější rozumnou bázi, která je ve vodě stálá (silnější
báze primárně podlehnou hydrolýze). Potaš není použitelná, protože
není dostatečně bazická a reakce by neprobíhala rozumně rychle. 

Acetylace sacharidů se provádí především acetanhydridem v~přítomnosti
báze, která váže vznikající kyselinu octovou. Dříve se používal octan
sodný, dnes se používají pyridin a triethylamin.

\newpage %%
% \subsection*{Ročník 4, úloha č. 8.2 }
\begin{quotation}
\ctyri Češi jsou jedním z~nejvášnivějších kuřáckých národů na světě (za kuřáky
se považuje 22--25~\% Čechů). V~posledních letech se ale procento
kuřáků ve společnosti pomalu daří snižovat. Svůj podíl na tom má i
zavádění nových přípravků na odvykání kouření. Jedním z~těchto přípravků
je i Vareniklin. Toto první léčivo bez obsahu nikotinu, působící na
nikotin-acetylcholinové receptory $\upalpha4\upbeta2$, vyvolalo po zavedení
na trh v~roce 2006 kontroverze kvůli údajné zvýšené frekvenci sebevražd
uživatelů. Toto tvrzení bylo v~roce 2012 vyvráceno rozsáhlým přezkoumáním
klinických dat. Vareniklin je zajímavý nejen biologickými účinky nebo
kontroverzemi, ale i strukturně. Pojďme se tedy podívat na to, jak
je tato sloučenina syntetizována. Doplňte vzorce látek \textbf{A}--\textbf{E} a Vareniklinu.

Nápověda: Z~1,2-dibrombenzenu je vygenerován benzyn, který je okamžitě
zachycen pentadienem a vzniká meziprodukt \textbf{A}. Proces použitý na převedení
sloučeniny \textbf{A} na \textbf{B} je analogický ozonolýze. Látka \textbf{C} i Vareniklin obsahují
4 cykly. DCE je 1,2-dichlorethan, NMO je $N$-methylmorfolin-\textit{N}-oxid
(oxidační činidlo), TfOH je kyselina trifluormethansulfonová.
\end{quotation} \dotfill \par 
\noindent \begin{center}

\includegraphics[scale=0.85]{/4/8-2}

\par\end{center}

V prvním kroku dojde k~vygenerování benzynu halogen-lithiovou výměnou
následovanou $\upalpha$-eliminací (alternativními postupy, jak generovat
intermediáty benzynového typu, jsou například reakce 2-jodofenol\-triflátů
s~organolithnými činidly, reakce 2-halosilyl\-aromátů s~fluoridy
nebo rozklad 2-diazoniových solí benzoových kyselin). Vzniklý benzyn
pak reaguje s~cyklopentadienem v~Dielsově-Al\-de\-ro\-vě reakci za
vzniku tricyklické sloučeniny \textbf{A}. Touto reakcí je také ustanovena \textit{Z}-poloha substituentů v~benzylové poloze (methylenový můstek musí být
nutně na jedné straně molekuly), která je zachována po celou dobu
syntézy a promítne se i do finálního produktu.

Jak už bylo řečeno, druhý sled reakcí je analogií k~ozonolýze. Cenou
za vyhnutí se průmyslově obtížně aplikovatelné ozonolýze je ovšem
používání (byť jen katalytických množství) toxických sloučenin osmia
v~první z~tandemu reakcí, v~níž vzniká vicinální diol. Ten je poté
rozštěpen jodistanem na \textit{Z}-dialdehyd \textbf{B}, který je posléze reduktivně
aminován za vzniku terciálního aminu \textbf{C}. Hydrogenolýza meziproduktu
\textbf{C} poskytne sekundární amin \textbf{D}, který je následně nitrován. Vzniklý
dinitroderivát je poté hydrogenován za vzniku diaminu \textbf{E}. Diamin \textbf{E}
poté zkondenzuje s~glyoxalem za vniku Vareniklinu.
\noindent \begin{center}

\includegraphics{/4/8-2-1}

\par\end{center}

\newpage %%
 % \subsection*{Ročník 2, úloha č. 8.3}
\begin{quotation}
\tri Oxidací kumenu (isopropylbenzenu) a následným zpracováním vznikají
dvě látky (látka \textbf{A} a látka \textbf{B}), které mají široké využití v~průmyslu.
Sodná sůl látky s~vyšší molekulovou hmotností může reagovat s~oxidem
uhličitým, po následném okyselení směsi vznikne látka \textbf{C}, která je
významnou surovinou ve farmaceutické výrobě: produkt esterifikace
této látky (látku \textbf{D}) si můžete v~lékárně koupit jako acylpyrin.

 Látku s~nižší molekulovou hmotností si můžete koupit v~obchodě třeba jako odlakovač na nehty.

 Napište strukturní vzorce a názvy látek \textbf{A} až \textbf{D}.
\end{quotation} \dotfill \par 
\begin{center}
\includegraphics{/2/8-3}
\end{center}

Oxidací kumenu vzdušným kyslíkem vzniká radikálovým mechanismem fenol
a aceton. Fenolát sodný poté může v~Kolbeho-Schmittově reakci
reagovat s~oxidem uhličitým za vzniku 2-hydroxobenzoové kyseliny.
Ta může postoupit na hydroxylové skupině esterifikační reakci s~acetanhydridem
za vzniku kyseliny acetylsalicylové, která je známá a hojně používaná
pod názvem Acylpyrin.

\hrulefill % \subsection*{Ročník 3, úloha č. 8.3 }
\begin{quotation}
\ctyri V~létě roku 2017 uplynulo 5 let od smrti jednoho z~nejvýznamnějších česko(slovenských) vědců 20. století, Antonína Holého. Pojďme si ho
nyní připomenout a podívat se na zub syntéze (byť nepůvodní, upravené
pro průmyslové použití) jednoho z~jeho nejznámějších antivirotik,
Tenofoviru-disoproxilu (C$_{19}$H$_{30}$N$_{5}$O$_{10}$P), které
se používá na léčbu infekce virem HIV a hepatitidy typu B. Doplňte
vzorce sloučenin \textbf{A}, \textbf{B}, \textbf{C} i léčiva samotného, pokud víte,
že při vzniku produktu \textbf{B} dochází k~ataku pyrrolového (N-9)
dusíku na méně substituovaný atom uhlíku cyklického karbonátu \textbf{A},
a že TMSBr je činidlo pro hydrolýzu esterů fosfonových kyselin na
volné fosfonové kyseliny. Schéma syntézy je přiloženo níže:

\begin{center}
\includegraphics{/3/8-3-1}
\end{center}
\end{quotation} \dotfill \par 
První krok syntézy není nic jiného než prachobyčejná bazicky katalyzovaná
trans-esterifikace, kterou můžete znát například z~produkce bionafty.
Postupně dojde k~ataku obou hydroxylů výchozího alkoholu na sp$^{2}$
hybridizovaném atomu uhlíku diethylkarbonátu za vzniku cyklického
esteru kyseliny uhličité (karbonátu) \textbf{A}. V~dalším kroku hydroxid
rovnovážně deprotonuje N-9 dusík adeninu, který je v~celé molekule
nejkyselejší. Zatímco \pKa\ této N-H vazby je přibližně 17, hydroxid
sodný má \pKa\ 14,0, což znamená, že přibližně tisícina
molekul adeninu bude deprotonovaná. Tato rovnováha je však neustále
posouvána směrem k~produktům tím, že deprotonovaný adenin atakuje
méně substituovaný atom uhlíku karbonátu \textbf{A} a dochází ke vzniku
sloučeniny \textbf{B} za současného vyšumění oxidu uhličitého. V~následujícím
kroku dojde k~deprotonaci hydroxylové skupiny a nukleofilní substituci
tosylátu, který je výborná odstupující skupina a vznikne fosfonát
\textbf{C}. Při reakci s~trimethylsilyl\-bromidem dojde k~nasilylování
obou sp$^{3}$ hybridizovaných kyslíků fosfoesterové skupiny a následnému
ataku bromidového aniontu na C1 uhlících ethylových skupin, který
po zpracování (\textit{work-upu}) poskytne volnou fosfonovou kyselinu.
Ta poté v~přítomnosti Hünigovy báze reaguje v~nukleofilní substituční reakci s~chloromethyl\-isopropyl\-karbonátem, kdy se chlorid chová jako odstupující skupina
a vzniká Tenofovir-disoproxyl. 

\begin{center}
\includegraphics{/3/8-3-2}
\end{center}

Tenofovir-disoproxyl je proléčivem samotného tenofoviru (volné
fosfonové kyseliny), který je za fyziologického pH nabitý a špatně
by tak prostupoval buněčnými membránami. To je vyřešeno „zamaskováním“
fosfonové kyseliny jako methyl\-isopropyl\-karbonátu, který je uvnitř
buňky snadno hydrolyzován a poskytuje tak aktivní formu léčiva. Ta
je v~buňce okamžitě fosforylována na tenofovir difosfát, který vstupuje
do replikace, pro jejíž katalýzu retroviry používají enzym reverzní
transkriptázu. Jelikož tenofovir na rozdíl od normálního nukleotidu
postrádá 3'-hydroxylovou skupinu, nemůže po jeho zařazení do rostoucího
řetězce DNA dojít ke vzniku další fosfodiesterové vazby a enlongace
DNA je tak předčasně terminována. Toto léčivo tedy spadá do kategorie
léčiv označovaných jako NtRTIs -- nukleotidová analoga působící
jako inhibitory reverzní transkriptázy.

\chapter{Analytická chemie}

\section{Kvalitativní analýza}

% \subsection*{Ročník 3, úloha č. 3.4 }
\begin{quotation}
\dva Slovní spojení \textit{zlatý déšť} má v~chemii jeden zajímavý význam,
a tím je poměrně pěkná srážecí reakce. Představte si, že jednoho dne
stařičký profesor na přednášce oznámil, že právě zlatý déšť chce demonstrovat.
Bohužel už je lehce zapomnětlivý a pomíchal 0,05M roztoky Cd(NO$_{3}$)$_{2}$,
Pb(NO$_{3}$)$_{2}$, KI a Na$_{2}$S. Aby mohl určit, který roztok
je který, provedl vzájemné reakce všech roztoků. Jejich výsledky jsou
znázorněny v~následující tabulce (r. značí roztok, s. sraženinu). Pomozte profesorovi určit jednotlivé
roztoky v~kádinkách 1 až 4. Po okyselení roztoku 4 se uvolňuje páchnoucí
plyn.
\begin{center}
\begin{tabular}{c|c|c|c|c}

 & 1 & 2 & 3 & 4\tabularnewline
\hline 
1 & X & bezbarvý r. & bezbarvý r. & žlutá s.\tabularnewline
\hline 
2 & bezbarvý r. & X & žlutá s. & černá s.\tabularnewline
\hline 
3 & bezbarvý r. & žlutá s. & X & bezbarvý r.\tabularnewline
\hline 
4 & žlutá s. & černá s. & bezbarvý r. & X\tabularnewline

\end{tabular}
\end{center}

\end{quotation} \dotfill \par 
V chemickém slangu označuje \textit{zlatý déšť} srážecí reakci
olovnaté soli a jodidu, při které vznikají těžké zlaté krystaly jodidu
olovnatého, které padají ke dnu baňky a vytvářejí tak efektní podívanou.
Chemici ale nejsou jediná skupina obyvatelstva s~vlastním slangem.
Například zahrádkáři jako \textit{zlatý déšť} označují keře z~rodu zlatice
nebo štědřenec (povšimněte si, jak jsou zahradníci na rozdíl od chemiků
nejednotní a jedním termínem označuje dva naprosto odlišné typy rostlin),
pro punkera je zase nejvýznamnějším \textit{Zlatým deštěm} jedna pražská kapela.

Protože náš stařičký profesor není zahrádkář a číro na hlavě taky
nemá, měli bychom se radši vrátit k~chemii. Jak už bylo řečeno, olovnatá
sůl reaguje s~jodidem za vzniku žluté sraženiny ($\mathrm{Pb^{2+}+2\,I^{-}\rightarrow PbI_{2}}$),
dále poskytuje černou sraženinu se sulfidovým aniontem ($\mathrm{Pb^{2+}+S^{2-}\rightarrow PbS}$)
a s~kademnatou solí nijak nereaguje. Pak už došlo jen k~jedné další
reakci za vzniku žluté sraženiny -- kadmiové žluti. Tato sraženina
vzniká při reakci kademnaté soli se sulfidovým aniontem ($\mathrm{Cd^{2+}+S^{2-}\rightarrow CdS}$).
Další informace k~dispozici je, že z~roztoku číslo 4
se po okyselení uvolní plyn. Jediná z~našich sloučenin, která tak
může reagovat, je Na$_{2}$S ($\mathrm{Na_{2}S+2\,H^{+}\rightarrow H_{2}S+2\,Na^{+}}$).

Nyní už je chemická část za námi, zbývá pouze logicky přiřadit vzorce
rozpuštěných sloučenin k~číslům. Roztok číslo 4 je Na$_{2}$S. Protože
roztok 4 s~roztokem 1 poskytuje žlutou sraženinu, musí roztok číslo
1 obsahovat Cd(NO$_{3}$)$_{2}$. A protože roztok 4 poskytuje s~roztokem
2 černou sraženinu, musí být v~roztoku číslo 2 rozpuštěný Pb(NO$_{3}$)$_{2}$.
Na roztok číslo 3 tak zbývá KI, což je v~souladu s~tím, že reakcí
roztoků 2 a 3 vzniká zlatý déšť. 

\hrulefill % \subsection*{Ročník 5, úloha č. 5.3 }
\begin{quotation}
\tri Hydrid neznámého prvku \textbf{X} obsahuje více než 10 hmotnostních procent
vodíku. Při jeho reakci s~oxidem uhličitým za zvýšené teploty vzniká
sůl, která obsahuje 61,59~\% kyslíku. Hydrolýza soli dává vzniknout
organickému produktu. Napište značku prvku, který se skrývá pod označením
\textbf{X}.
\end{quotation} \dotfill \par 
10 hmotnostních procent vodíku omezuje molární hmotnost kationtu v
hydridu na nejvýše 9 pro stechiometrii XH, resp. 18 pro $\ch{XH2}$,
27 pro $\ch{XH3}$ a 36 pro $\ch{XH4}$. To nechává jako možnosti
LiH, $\ch{BeH2}$, $\ch{BH3}$, $\ch{AlH3}$, $\ch{NH3}$, $\ch{CH4}$,
$\ch{SiH4}$ a $\ch{H2O}$. Aby vznikala po zahřátí sůl, musí být
kation buď výrazněji elektropozitivní (což kvalifikuje LiH a $\ch{AlH3}$)
nebo tvořit jiný kation (kation amonný z~$\ch{NH3}$; poslední možnost
však nepřipadá snadno v~úvahu, protože při reakci s~oxidem uhličitým
není k~ruce žádný zdroj vodíku).

Hydrolýzou navíc vzniká organický produkt a nikoli zpětně oxid uhličitý,
takže sůl není uhličitanem. Máme k~dispozici pouze fragment s~jedním
atomem uhlíku, takže jako jediný rozumný anion připadá anion HCOO$^{-}$
vzniklý adicí hydridového aniontu. 

Ze zadání víme, že hmotnost kyslíku tvoří 61,59 \% hmotnosti soli. V~případě mravenčanového aniontu je relativní hmotnost všech atomů kyslíku\footnote{V jednom aniontu mravenčanu jsou dva atomy kyslíku, každý o~$M_r = 16$.} 32. Celková relativní hmotnost molekuly pak bude

\[
M_r = \frac{32}{0,6159} = 52
\]

Relativní atomová hmotnost, která náleží neznámému kovu, je pak rovna rozdílu takto spočtené hmotnosti a hmotnosti mravenčanového aniontu:
\[
52-32-12-1=7
\]

Hledaným prvkem \textbf{X} je lithium, Li.

\hrulefill % \subsection*{Ročník 1, úloha č. 5.3}
\begin{quotation}
\dva Stálým koloritem analytické chemie jsou důkazové reakce. Snad vám
následující úloha nevyvolá nepříjemné barevné změny ve tváři. Z~následujících
iontů vyberte všechny ty, které po přidání okyseleného roztoku $\mathrm{KMnO_{4}}$
do vodného roztoku těchto iontů budou vykazovat nějakou reakci. 
\[
\mathrm{SO_{4}^{2-},\;S^{2-},\;SO_{3}^{2-},\;Fe^{2+},\;Ba^{2+},\;K^{+},\;PO_{4}^{3-},\;I^{-},\;ClO_{4}^{-}}
\]
\end{quotation} \dotfill \par 
S manganistanem budou reagovat ty ionty, které se mohou oxidovat do vyššího oxidačního stavu. Z~výběru výše se jedná o~následující ionty: $\mathrm{S^{2-},\;SO_{3}^{2-},\;Fe^{2+}\;a\;I^{-}}$.

\hrulefill % \subsection*{Ročník 4, úloha č. 7.4  }
\begin{quotation}
\ctyri Každý z~vás se už jistě setkal s~důkazovými reakcemi anorganických
kationtů a aniontů. Stejně tak jistě mnozí z~vás tuší, že k~identifikaci
organických látek se využívají spektroskopické techniky (NMR, IČ ad.).
Před rozvojem těchto technik ve 20. století se však i organičtí chemici
museli spoléhat na důkazové reakce funkčních skupin. Vžijte se tedy
do role československého chemika počátku minulého století. Stejně
jako většina Čechoslováků s~oblibou popíjíte nejen pivo, ale i další
alkoholické nápoje.

Z~některých vašich oblíbených lahví se vám podařilo izolovat sloučeniny
s výraznou vůní či chutí: 

\textbf{A Eukalyptol}, který je obsažen v~pelyňku a je zodpovědný
za peprmintovou vůni a chladivý efekt absintu,

\textbf{B Prenyl-acetát}, jež se vyskytuje v~ginu a má ovocnou vůni,\textbf{ }

\textbf{D $\upbeta$-Damascenon} s~lehkým jablečným aromatem, který
se vyskytuje v~tmavých (ale ne bílých) rumech, 

\textbf{E „cis-dubový lakton“}, který dodává rumům a whiskey charakteristické
„dřevěné tóny“, 

\textbf{F Karyofylen} zodpovědný za chmelové aroma piva,

\textbf{G Humolon}, který dává pivu hořkost. 

Vialky s~vašimi látkami se vám bohužel pomíchaly a vy je teď musíte
rozlišit s~pomocí následujících testů:

\textbf{1. Bayerův test:} Smícháte 1 kapku neznámého vzorku s~1 kapkou
10\% manganistanu v~1 ml ethanolu,

\textbf{2. Reakce s~ceričitou solí: }Ke 2 kapkám roztoku ceričité
soli přidáte 2 kapky acetonitrilu a 2 kapky neznámého vzorku (vzorky
obsahující alkoholické skupiny poskytnou pestré koordinační sloučeniny),

\textbf{3. Reakce s~2,4-dinitrofenylhydrazinem (DNPH):} k~1 kapce
vzor\-ku v~1 ml ethanolu se přidá 1 ml roztoku 2,4-dinitrofenylhydrazinu, 

\textbf{4. Hydroxamátový test s~železitou solí:} Vzorek se povaří
v bazickém ethanolickém roztoku hydroxylaminu, roztok se okyselí a
následně se přidá roztok chloridu železitého (z esterů vzniknou hydroxamové
kyseliny, které s~železitou solí poskytnou sytě zbarvené komplexy).
Na základě následující tabulky přiřaďte sloučeniny A--G ke vzorkům
1--7. + značí pozitivní výsledek testu, -- značí výsledek negativní.
\begin{center}
\begin{tabular}{c|c|c|c|c|c|c}

Test/číslo vzorku & 1 & 2 & 3 & 4 & 6 & 7\tabularnewline
\hline 
\hline 
Bayerův test & $+$ & $+$ & $-$ & $+$ & $+$ & $-$\tabularnewline
\hline 
Reakce s~$\ce{Ce^{4+}}$ & $+$ & $-$ & $-$ & $-$ & $-$ & $-$\tabularnewline
\hline 
Reakce s~DNPH & $+$ & $-$ & $-$ & $+$ & $-$ & $-$\tabularnewline
\hline 
Hydroxamátový test s~$\ce{Fe^{3+}}$ solí & $-$ & $-$ & $+$ & $-$ & $+$ & $-$\tabularnewline

\end{tabular}
\end{center}


\begin{center}
\includegraphics{/4/7-4}
\end{center}
\end{quotation} \dotfill \par 

Se znalostí chemismu všech příslušných testů je nyní řešení úlohy
jednoduché: Vzorek 1 musí obsahovat dvojnou vazbu, alkoholovou funkční
skupinu, karbonylovou skupinu a neobsahuje ester, jediná sloučenina
splňující tyto požadavky je hořký humolon (\textbf{1G}). Vzorek 2
má ve své struktuře zabudovanou dvojnou vazbu a veškeré ostatní testy
po něj vyšly negativně, musí se tedy jednat o~karyofylen ze chmelu
(\textbf{2F}). Vzorek 3 reaguje v~hydroxamátovém testu, a tudíž obsahuje
ester, mohlo by tak jít o~látku \textbf{B} nebo \textbf{E}. Vzhledem
k negativnímu výsledku Bayerova testu ale musí jít o~cis-dubový lakton
(\textbf{3E}). Vzorek 4 je karbonylová sloučenina se dvojnou vazbou,
která neobsahuje alkoholovou funkční skupinu ani ester, může to být
jediná látka, a to $\upbeta$-Damascenon z~tmavých rumů (\textbf{4D}).
Vzorek 6 je nenasycený ester a jedná se tedy o~prenyl-acetát (\textbf{6B}).
Na vzorek 7 tak zbývá eukalyptol z~absinthu, který byl ke všem námi
prováděným testům inertní (\textbf{7A}). 

\hrulefill % \subsection*{Ročník 3, úloha č. 8.1 }
\begin{quotation}
\tri Po zesnulém profesorovi analytické chemie zbylo v~digestoři sedm roztoků.
Bohužel nebyly popsány, jediná další věc v~digestoři byl počmáraný
kus papíru, který obsahoval následující instrukce (profesor strašně
hrabal, a tak se vám podařilo vyluštit jen část): „\textit{0,05M roztoky
KSCN, HCl, H$_{2}$SO$_{4}$, BaCl$_{2}$, AgNO$_{3}$, Na$_{2}$CO$_{3}$,
KI, Na$_{2}$S, Pb(NO$_{3}$)$_{2}$, CdSO$_{4}$, Fe(NO$_{3}$)$_{3}$
a Na$_{2}$S$_{2}$O$_{3}$ připchrnmd}“. Je tedy jasné, že nebohý
profesor chystal nějaký experiment, ale před smrtí stihl připravit
jen oněch 7 roztoků. Chtěli jste zjistit, co každý roztok obsahuje,
a proto jste provedli vzájemné reakce vzorků roztoků. Vaše výsledky
jsou shrnuty v~následující tabulce. Určete jednotlivé roztoky 1-7,
víte-li, že obsahují vždy jedinou rozpuštěnou sloučeninu.

\textit{Legenda k~tabulce: r. -- roztok, sr. -- sraženina, rozp. -- rozpustná.}

\begin{center}
\begin{tabular}{c|c|c|c|c|c|c|c}
 
 & 1 & 2 & 3 & 4 & 5 & 6 & 7\tabularnewline

\hline 
\multirow{2}{*}{1} & \multirow{2}{*}{X} & \multirow{2}{*}{nažl. r.} & \multirow{2}{*}{bílá sr.} & \multirow{2}{*}{bílá sr.} & bílá sr. & \multirow{2}{*}{černá sr.} & \multirow{2}{*}{bílá sr.}\tabularnewline
 &  &  &  &  & rozp. v~NH$_{3}$ &  & \tabularnewline
\hline 
2 & nažl. r. & X & nažl. r. & črv. r. & nažl. r. & černá sr. & nažl. r.\tabularnewline
\hline 
3 & bílá sr. & nažl. r. & X & bb. r. & bb. r. & bb. r. & bílá sr.\tabularnewline
\hline 
4 & bílá sr. & črv. r. & bb. r. & X & bb. r. & bb. r. & bb. r.\tabularnewline
\hline 
\multirow{2}{*}{5} & bílá sr. & \multirow{2}{*}{nažl. r.} & \multirow{2}{*}{bb. r.} & \multirow{2}{*}{bb. r.} & \multirow{2}{*}{X} & \multirow{2}{*}{bb. r., plyn} & \multirow{2}{*}{bb. r.}\tabularnewline
 & rozp. v~NH$_{3}$ &  &  &  &  &  & \tabularnewline
\hline 
6 & černá sr. & černá sr. & bb. r. & bb. r. & bb. r., plyn & X & žl. s.\tabularnewline
\hline 
7 & bílá sr. & nažl. r. & bílá sr. & bb. r. & bb. r. & žl. s. & X\tabularnewline

\end{tabular}
\par\end{center}

\end{quotation} \dotfill \par 
První věc, které si můžete povšimnout, je, že roztok 2 je velmi často
popsán jako nažloutlý, je tedy rozumné předpokládat, že jde o~barvu
samotného roztoku. Z~uvedených sloučenin má v~roztoku nažloutlou barvu
pouze jediná z~nich -- Fe(NO$_{3}$)$_{3}$, tuto naši domněnku jen
potvrzuje to, že s~roztokem 4 poskytuje typické červené zabarvení,
které z~uvedených látek dává pouze kombinace Fe$^{3+}$ a SCN$^{-}$
(vzniká komplex {[}Fe(SCN)$_{6}${]}$^{3-}$). Určili jsme tedy první
dva roztoky: \textbf{roztok 2 obsahuje Fe(NO$_{3}$)$_{3}$ a roztok
4 KSCN}. 
\newpage
Dále si také můžeme všimnout, že roztok 2 reaguje s~roztokem 6 za
vzniku černé sraženiny, v~úvahu tedy připadá, že roztok 6 obsahuje
Na$_{2}$S nebo Na$_{2}$CO$_{3}$. Roztok 6 také reaguje s~roztokem
1 za vzniku černé sraženiny. Jelikož roztoky 1 a 4 spolu tvoří bílou
sraženinu, můžeme uvažovat, že v~roztoku 1 je rozpuštěna stříbrná
nebo olovnatá sůl. Aby reakcí tohoto roztoku s~roztokem 6 vznikla
černá sraženina, musí \textbf{roztok 6 obsahovat Na$_{2}$S}, protože
uhličitan nedává černou sraženinu ani se stříbrnými, ani s~olovnatými
kationty.

Reakce mezi roztoky 6 a 7 je typickou důkazovou reakcí kademnatých iontů za vzniku kadmiové
žluti, \textbf{roztok 7 tedy obsahuje CdSO}$_{\boldsymbol{4}}$. 

Při reakci roztoku sulfidu sodného (roztok 6) s~roztokem 5 dochází
k uvolnění plynu, roztok 5 tedy obsahuje kyselinu. Zároveň víme, že
roztok 3 reaguje s~roztokem 1 (který obsahuje stříbrnou nebo olovnatou
sůl) za vzniku bílé sraženiny, v~úvahu tedy přichází přítomnost chloridových,
síranových nebo thiosíranových aniontů v~roztoku 3. Zároveň ale víme,
že s~kyselým roztokem 5 roztok 3 nereaguje, což vylučuje přítomnost
thiosíranu, a sám roztok 3 také není kyselý, protože při reakci s
roztokem 6 nedochází k~uvolnění sulfanu. Z~těchto poznatků je jasné,
že \textbf{roztok 3 obsahuje BaCl}$\boldsymbol{_{2}}$\textbf{.} Pouhým
pohledem na reakci mezi roztoky 3 a 5 vidíme, že nedochází ke vzniku
sraženiny síranu barnatého, \textbf{roztok 5 tedy musí obsahovat HCl.}
Sraženina vzniklá reakcí roztoků 1 a 5 je rozpustná v~amoniaku, z
tohoto důvodu musí \textbf{roztok 1 obsahovat AgNO$\boldsymbol{_{3}}$}.

\hrulefill % \subsection*{Ročník 3, úloha č. 7.3  }
\begin{quotation}
\tri Těžko by se hledala sloučenina, která by nepodlehla tepelné zkáze
nebo alespoň chemickým změnám v~důsledku zvyšování teploty. Na této
skutečnosti je dokonce založená analytická metoda termogravimetrie,
která se používá ke stanovení přesného stechiometrického složení minerálů
a ke zjištění jejich obsahu v~rudách, její novější modifikace pak
třeba i při zkoumání degradace polymerů. Hezkým modelovým příkladem
je termogravimetrická analýza látky, pro niž bylo stanoveno prvkové
složení 27,43 \% Ca, 16,44 \% C, 1,38 \% H a 54,75 \% O. 

Při zahřívání přesně zjištěného množství této zcela čisté chemikálie
byly zjištěny následující změny:

$T=500\,\mathrm{K}$: pokles hmotnosti na 87,69 \% hmotnosti původní
navážky 

$T=775\,\mathrm{K}$: pokles hmotnosti na 68,51 \% hmotnosti původní
navážky 

$T=1140\,\mathrm{K}$: pokles hmotnosti na 38,39 \% hmotnosti původní
navážky 

Vašim úkolem je určit, které látky unikly při každém z~hmotnostních
úbytků. 
\end{quotation} \dotfill \par 
Z~elementární analýzy za použití relativních atomových hmotností (40,08
pro Ca, 12,0 pro C, 1,01 pro H a 16,00 pro O) snadno odvodíme, že
analyzovaný materiál má sumární vzorec CaC$_{2}$H$_{2}$O$_{5}$.
Nemáme údaje o~jeho hmotnosti, avšak při následných přepočtech by
po jejich vykrácení figurovala pouze molární hmotnost, která je rovna
146,09 $\mathrm{g\cdot mol^{-1}}$. Můžeme tedy pro jednoduchost uvažovat, že jsme nechali
rozkládat právě jeden mol této látky. Při prvním rozkladu došlo k
úbytku 12,31\% hmotnosti, což odpovídá poklesu molární hmotnosti o
18. Protože víme, jaké prvky se ve vzorku nacházejí, není těžké dovodit,
že první pokles odpovídá úniku jedné molekuly vody. Zůstává tedy látka
o sumárním vzorci CaC$_{2}$O$_{4}$, což lze jednoznačně identifikovat
jako šťavelan vápenatý, jehož monohydrát jsme právě důkladně „usušili“.
Druhý rozklad už signalizuje úbytek 31,49~\% hmotnosti, což odpovídá
poklesu M o~46 oproti původní hodnotě, a tedy o~28 od posledního skoku.
Toho lze docílit jediným způsobem, a to uvolněním molekuly CO, takže
v~kelímku se už žhaví obyčejný CaCO$_3$, uhličitan vápenatý. Proto lze
docela snadno uhodnout, že poslední rozklad má za následek únik oxidu
uhličitého, takže v~kelímku zůstane pouze oxid vápenatý. Postupně
tedy unikly voda (vodní pára), oxid uhelnatý a oxid uhličitý. 

\section{Titrace}

 % \subsection*{Ročník 4, úloha č. 4.6 }
\begin{quotation}
\dva Aminokyseliny vystupují v~prostředí bezvodé kyseliny octové jako bazické
látky. Mnohé z~nich tak proto lze přímo titrovat odměrným roztokem
kyseliny chloristé, přičemž dochází k~acidobazické reakci s~aminoskupinou
v~molekule kyseliny. 1,4382~g neznámé čisté aminokyseliny bylo rozpuštěno
v~bezvodé kyselině octové ve 100ml odměrné baňce. Do titrační baňky
bylo nedělenou pipetou odpipetováno 10 ml tohoto roztoku, který byl
naředěn dalšími 20 ml bezvodé kyseliny octové. Byly přidány dvě kapky
roztoku krystalové violeti a titrováno 0,1000 M $\ch{HClO4}$. Průměrná spotřeba
titračního činidla z~několika těchto stanovení byla 8,7 ml. O~jakou
z proteinogenních aminokyselin se jednalo?
\end{quotation} \dotfill \par 
Během titrace dochází k~reakci $\alpha$-aminoskupiny, která se kyselinou
chloristou protonuje. Rovnice reakce tedy vypadá zhruba takto:
\[
\ce{HClO4 +R\bond{-}NH2\rightarrow ClO4- +R\bond{-}NH3+}
\]

Jak je z~rovnice zřejmé, běžná aminokyselina reaguje s~kyselinou pouze
jednou\footnote{Vícekrát může reagovat např. lysin, ale na ten to nevychází.}.
Látkové množství použité kyseliny chloristé ponásobené zřeďovacím
faktorem tedy odpovídá látkovému množství aminokyseliny. Molární hmotnost
pak dopočítáme pomocí vzorce $M=\frac{m}{n}$.

\[
M=\frac{1,4382}{0,1\cdot0,0087\cdot10}=165,3
\]

Vypočtená molární hmotnost odpovídá aminokyselině fenylalaninu.
\noindent \begin{center}

\includegraphics{/4/4-6}

\par\end{center}

\hrulefill % \subsection*{Ročník 4, úloha č. 6.6 }
\begin{quotation}
\tri Magnetismus je v~přírodě vlastností jen malé skupiny minerálů.
Metalurgický pokrok dovolil vyvinout novou třídu magnetických materiálů,
které jsou výhodné díky svým lepším mechanickým vlastnostem jako
součástky v~elektromotorech. Takovouto slitinou byl v~půli minulého
století vyvinutý bismanol, jehož klíčovými složkami jsou bismut a
zinek. 

Vzorek slitiny o~hmotnosti 0,54281 g obsahující bismut a zinek byl
rozpuštěn ve vroucí koncentrované kyselině dusičné. Po vychladnutí
byl vzniklý roztok naředěný, zfiltrovaný a převedený do 500ml
odměrné baňky a doplněný po rysku destilovanou vodou. 

Do titrační baňky bylo odpipetováno 10,0 ml odměrného roztoku dusičnanu
olovnatého, který byl připravený rozpuštěním 0,3641 g $\ce{Pb(NO3)2}$
v 10 ml koncentrované kyseliny dusičné a doplněním na objem 100~ml. Následně byl odpipetovaný roztok zředěn destilovanou vodou na objem cca 50 ml. Následně bylo
přidáno asi 50 mg xylenolové oranži. Za stáleho míchání byl přidán
urotropin, dokud se roztok nezbarvil do červenofialova. Výsledný roztok
byl titrovaný odměrným roztokem Chelatonu III do čistě žlutého zbarvení.
Postup byl opakovaný 3krát. Výsledná průměrná spotřeba odměrného
roztoku byla 19,95~ml. 

Do titrační baňky bylo odpipetováno 25,0~ml roztoku vzorku směsi kationtů $\ce{Bi^{3+}}$ a $\ch{Zn^{2+}}$. Následně bylo přidáno
asi 50 mg xylenolové oranže. Vzniklý růžový roztok byl titrován
odměrným roztokom Chelatonu III do žlutého zbarvení, přičemž spotřeba
činila 9,10 ml. Následně byl za stáleho míchání do roztoku po částech
přidávaný pevný urotropin, dokud se barva roztoku nezměnila na růžovo\-fialovou.
V titraci odměrným roztokem Chelatonu III se poté pokračovalo až
do žlutého zbarvení. Celková spotřeba činila 27,7 ml. Tento postup byl
opakovaný ješte 3krát, byly zjištěny následující spotřeby v~ mililitrech: 9,00 a 27,80; 9,15
a 27,75; 9,10 a 27,80 ml. Vypočítejte hmotnostní zlomek bismutu a
zinku ve slitině. Předpokládejte, že jiné kovy než Bi a Zn při tomto
postupu nereagují. 

\textit{Při nižším pH reaguje s~chelatonem pouze bismut, při vyšším i zinek.}
\end{quotation} \dotfill \par 
Nejprve spočítáme koncentraci standardního roztoku dusičnanu olovnatého,
kte\-rý byl připraven rozpuštěním 0,3641 g v~100 ml vody:
\[
c_{\ch{Pb(NO3)2}}=\frac{m_{\ch{Pb(NO3)2}}}{M_{\ch{Pb(NO3)2}}\cdot V_{\mathrm{rozt\  \ch{Pb(NO3)2}}}}=\frac{0,3641}{331,2\cdot0,1}=0,010993\ \mathrm{mol\cdot dm^{-3}}
\]
Následně byl standardizován roztok Chelatonu III (Y), jeho koncentrace
je: 
\[
c(\mathrm{Y})=\frac{c_{\ch{Pb(NO3)2}}\cdot V_{\mathrm{st \ch{Pb(NO3)2}}}}{\bar{V}_{1}(\mathrm{Y})}=\frac{0,010993\cdot0,01}{0,01995}=5,5105\cdot10^{-3}\ \mathrm{mol\cdot dm^{-3}}
\]
Při stanovení vzorku je potřeba si uvědomit, co se v~titrační baňce
děje. Vzorek slitiny byl rozpuštěn v~koncentrované kyselině dusičné
a doplněn po značku. pH tohoto roztoku je velice nízké (přibližně
1-2) a při titraci takto kyselého roztoku směsi zinečnatých a bismutitých
kationtů vzniká stabilní komplex chelatonu pouze s~bismutem. Tudíž
z první spotřeby se dá spočítat koncentrace bismutu. Po úpravě pH
urotropinem už je možné vytvořit i komplex chelatonu se zinkem. Celková
spotřeba chelatonu je tedy spotřeba na oba kovy dohromady a spotřeba
na zinek je rozdíl celkové spotřeby a spotřeby na bismut. 

\newpage %text%
Průměrná spotřeba chelatonu na bismut: 

\[
\bar{V}_{\ch{Bi}}\left(\mathrm{Y}\right)=\frac{9,10+9,00+9,15+9,10}{4}=9,0875\ \mathrm{ml}
\]
Průměrná spotřeba chelatonu na zinek: 
\[
\bar{V}_{\ch{Zn}}\left(\mathrm{Y}\right)	=\frac{27,70-9,10+27,80-9,00+27,75-9,15+27,80-9,10}{4}=18,675\ \mathrm{ml}
\]

Spočítáme tedy koncentrace obou kovů v~roztoku vzorku: 

\[
c_{\ch{Bi^{3+}}}=\frac{c\left(\mathrm{Y}\right)\cdot\bar{V}_{\ch{Bi}}\left(\mathrm{Y}\right)}{V_{\mathrm{vzorek}}}=\frac{5,5105\cdot10^{-3}\cdot0,0090875}{0,025}=2,0031\cdot10^{-3}\ \mathrm{mol\cdot dm^{-3}}
\]
\[
c_{\ch{Zn^{2+}}}=\frac{c\left(\mathrm{Y}\right)\cdot\bar{V}_{\ch{Zn}}\left(\mathrm{Y}\right)}{V_{\mathrm{vzorek}}}=\frac{5,5105\cdot10^{-3}\cdot0,018675}{0,025}=4,1163\cdot10^{-3}\ \mathrm{mol\cdot dm^{-3}}
\]
Z~koncentrací je pak možné vypočítat celkovou hmotnost kovů ve vzorku:
\[
m_{\ch{Bi}}=c_{\ch{Bi^{3+}}}\cdot V_{\mathrm{vzorek,celk}}\cdot M_{\ch{Bi}}=2,0031\cdot10^{-3}\cdot0,5\cdot208.98=0,20930\ \mathrm{g}
\]
\[
m_{\ch{Zn}}=c_{\ch{Zn^{2+}}}\cdot V_{\mathrm{vzorek,celk}}\cdot M_{\ch{Zn}}=4,1163\cdot10^{-3}\cdot0,5\cdot65,38=0,13456\ \mathrm{g}
\]
Z~hmotností kovů umíme vypočítat hmotnostní zlomky v~slitině: 
\[
w_{\ch{Bi}}=\frac{m_{\ch{Bi}}}{m_{\mathrm{vz}}}=\frac{0,20930}{0,54281}=0,39
\]
\[
w_{\ch{Zn}}=\frac{m_{\ch{Zn}}}{m_{\mathrm{vz}}}=\frac{0,13456}{0,54281}=0,25
\]


\hrulefill % \subsection*{Ročník 3, úloha č. 7.1  }
\begin{quotation}
\tri Železo je pro civilizaci dávno známým a velmi důležitým prvkem, ať
už v~mikroskopickém (biologicky aktivním) nebo makroskopickém měřítku. V~přírodě se nachází v~mnoha horninách a minerálech, například
v hematitu (kreveli), limonitu (hnědeli) nebo pyritu (kočičím zlatě)
neboli disulfidu železnatém. Existuje však i sulfid železnatý, a to
dokonce ve dvou variantách: jako minerál troilit o~přesné stechiometrii
FeS, který se však nachází pouze v~dopadajících meteoritech, nebo jako
na Zemi běžně nalézaný minerál pyrhotin (kyz magnetový) o~nestechiometrickém
složení, jež bývá většinou zapisováno jako Fe$_{1-x}$S, kde $x$ se
pohybuje mezi hodnotami 0 až 0,2 (tj. např. Fe$_{0,85}$S pro $x =
0,15$). 

Odštěpek posledně zmíněného minerálu pyrhotinu o~hmotnosti 0,9789
g byl rozpuštěn v~koncentrované kyselině dusičné, přičemž došlo k
oxidaci veškerého železa do formy železitých kationtů. Vzniklý roztok
byl doplněn na objem 250 ml a pH upraveno na hodnotu 1. Ze zásobního
roztoku bylo odebráno 10 ml, k~nim bylo přidáno 20 ml destilované
vody, indikátor (xylenolová oranž) a roztok byl titrován roztokem
chelatonu z~fialového do žlutého zbarvení. Stanovení bylo několikrát
zopakováno, průměrná spotřeba roztoku chelaton o~koncentraci $c=0,01962\mathrm{\,mol\cdot dm^{-3}}$
byla 22,05~ml. Stanovte stechiometrii analyzovaného minerálu za předpokladu,
že obsahoval pouze železo a síru bez jakýchkoli nečistot, a zapište
ji ve výše ukázaném formátu Fe$_{1-x}$S. 
\end{quotation} \dotfill \par 
Pro týmy, které se rozhodnou v~nedostatku času „střílet od pasu“,
může být návodná první část zadání, ze~které plyne, že se bude stechiometrie
pohybovat mezi Fe$_{0,8}$S a FeS. Kam tato hodnota přesně spadne,
můžeme zjistit docela rychle. Látkové množství železa lze určit jediným
výpočtem, je však také klíčové také zjistit jeho hmotnost. 

\[
n_{\ch{Fe}}=c_{\mathrm{EDTA}}\cdot\-V_{\mathrm{titr}}\cdot\-\frac{V_{\mathrm{odm.b.}}}{V_{\mathrm{pip}}}=0,01962\cdot0,02205\cdot\frac{0,25}{0,01}=0,0108\,\mathrm{mol}
\]
\[
m_{\ch{Fe}}=n_{\ch{Fe}}\cdot\-M_{\ch{Fe}}=0,0108\cdot55,845=0,6040\,\mathrm{g}
\]

Protože je podle zadání v~minerálu přítomno pouze železo a síra, připadá
zbytek hmotnosti minerálu na ni, pro zjištění stechiometrie také spočítáme
její látkové množství. 

\[
m_{\ch{S}}=m_{\mathrm{vz}}-m_{\ch{Fe}}=0,9879-0,6040=0,3749\,\mathrm{g}
\]

\[
n_{\ch{S}}=\frac{m_{\ch{S}}}{M_{\ch{S}}}=\frac{0,3749}{32,065}=0,0117\,\mathrm{mol}
\]

Zbývá jediný krok, určení stechiometrického poměru.
\[
\frac{0,0108}{0,0117}=\frac{y}{1} \ \Rightarrow \ y=0,925
\]
Zkoumaný vzorek pyrhotinu má tedy stechiometrii Fe$_{0,925}$S.

\hrulefill % \subsection*{Ročník 1, úloha č. 7.4}
\begin{quotation}
\tri 1,745 g směsi dusičnanu hořečnatého a dusičnanu hlinitého bylo rozpuštěno
ve vodě a doplněno v~odměrné baňce na 100~ml roztoku. Bylo odpipetováno
10,0 ml roztoku, pH bylo upraveno roztokem amoniaku do kvantitativního
vytvoření sraženiny. Roztok byl přefiltrován, sraženina promyta a
zbylé ionty byly stanoveny titrací roztokem EDTA (c = 0,0500 M). Stanovení
bylo provedeno čtyřikrát se spotřebami 8,60 ml; 6,40 ml; 6,35 ml; 6,40 ml.
Určete hmotnostní zlomek hliníku ve směsi na 3~platné číslice.
\end{quotation} \dotfill \par 
Jedná se o~titrační výpočet, ve kterém je taktně zamlčeno
několik skutečností. Předně se tímto postupem stanoví obsah hořčíku
(hydroxid hlinitý se amoniakálním roztokem prakticky kvantitativně
sráží, navíc hliník se stanovuje v~prostředí značně kyselém), dále
je potřeba ignorovat první, zjevně přestřelenou hodnotu spotřeby chelatonu.

Průměrná spotřeba:

\[
V=\frac{V_{2}+V_{3}+V_{4}}{3}=6,367\, \mathrm{ml}
\]

Výpočet množství dusičnanu hořečnatého:

\[
m_{\ch{Mg(NO3)2}}=c_{\mathrm{EDTA}}\cdot V_{\mathrm{EDTA}}\cdot M_{\ch{Mg(NO3)2}}\cdot f_{\mathrm{zředění}}
\]
\[
m_{\ch{Mg(NO3)2}}=0,0500\cdot0,006367\cdot148,3\cdot10=0,47\ \mathrm{g}
\]

Výpočet hmotnosti dusičnanu hlinitého:

\[
m_{\ch{Al(NO3)3}}=m_{\mathrm{směs}}-m_{\ch{Mg(NO3)2}}=1,273\ \mathrm{g}
\]
\[
m_{\ch{Al}}=\frac{M_{\ch{Al}}}{M_{\ch{Al(NO3)3}}}\cdot\-m_{\ch{Al(NO3)3}}
\]
Výpočet hmotnostního zlomku hliníku:
\[
w_{\ch{Al}}=\frac{m_{\ch{Al}}}{m_{\mathrm{směs}}}=0,0913
\]


\hrulefill % \subsection*{Ročník 2, úloha č. 7.5}
\begin{quotation}
\tri Běžné nápojové kartony jsou vrstveným obalem, obsahujícím vrstvu hliníku
vloženou mezi polyethylenové folie. Z~kartonového obalu byl vystřižen
obdélník o~rozměrech $35\times45$~mm a z~něj odtržením a odmočením odstraněn
přilepený karton. Očištěná folie byla vložena do kádinky a k~ní přilito
asi 10~ml 20\% kyseliny chlorovodíkové. Směs byla po přikrytí hodinovým
sklem ponechána přes noc do rozpuštění veškerého hliníku. Obsah kádinky
byl přefiltrován, kádinka vypláchnuta destilovanou vodou a roztok
kvantitativně převeden do 100ml odměrné baňky. K~vlastnímu stanovení
bylo odpipetováno 10,00 ml tohoto roztoku do titrační baňky a přidáno
25,00 ml standardního roztoku Chelatonu III o~koncentraci 0,0204 $\mathrm{mol\cdot dm^{-3}}$.
Směs byla přivedena k~varu a na 10 minut umístěna na vodní lázeň.
Po ochlazení byl roztok zředěn destilovanou vodou a přidáno několik
kapek xylenolové oranže.\\
Za stálého míchání byl přidáván pevný urotropin,
dokud jasně žluté zbarvení neztratilo na zářivosti. Roztok byl pak
titrován roztokem dusičnanu olovnatého o~koncentraci 0,0501 $\mathrm{mol\cdot dm^{-3}}$
do červenofialového zbarvení. Byly zaznamenány tyto spotřeby titračního
činidla v~ml: 8,20; 8,25; 8,20. Vypočítejte průměrnou tloušťku
hliníkové vrstvy.

Potřebná data: $A_{\mathrm{r}}(\ch{Al})=26,98$; hustota hliníku
$\rho_{\ch{Al}}=2,7\ \mathrm{g\cdot cm^{-3}}$.
\end{quotation} \dotfill \par 
Popsaný postup stanovení je založen na nepřímé titraci, kdy je vzorek
hliníkové folie nejprve rozpuštěn v~silné kyselině za vzniku roztoku
$\mathrm{Al^{3+}}$ soli, která následně reaguje s~definovaným množstvím
Chelatonu III (komplexačního činidla), které je přidané v~nadbytku. Přebytek přidaného komplexačního činidla je následně
stanoven titrací standardním roztokem $\mathrm{Pb^{2+}}$ soli. 

Výhodná je zde skutečnost,
že Chelaton III (disodná sůl kyseliny ethylen\-diamin\-tetraoctové, EDTA) reaguje s~kationty kovů ve stechiometrickém poměru 1:1, nezávisle na náboji kationtu. Nejprve vypočteme látkové množství přidaného Chelatonu III: 

\[
n_{\mathrm{EDTA}}=c_{\mathrm{EDTA}}\cdot V_{\mathrm{EDTA}}=0,0204\cdot0,025=5,100\cdot10^{-4}\,\mathrm{mol}
\]

Dále je třeba na základě průměrné spotřeby při třech titracích určit
množství přidané $\mathrm{Pb^{2+}}$ soli: 

\[
n_{\ch{Pb}}=c_{\ch{Pb}}\cdot\overline{V_{\ch{Pb}}}=0,0501\cdot0,00822=4,1165\cdot10^{-4}\,\mathrm{mol}
\]
 

Množství $\mathrm{Al^{3+}}$, které zreagovalo s~komplexačním činidlem,
určíme jako rozdíl těchto látkových množství: 

\[
n_{\ch{Al}}=n_{\mathrm{EDTA}}-n_{\ch{Pb}}=9,8345\cdot10^{-5}\,\mathrm{mol}
\]

To je látkové množství hliníku obsažené v~10 ml vzorku. Jelikož tento
vzorek pochází z~odměrné baňky se 100~ml roztoku získaného rozpuštěním
hliníkové folie, je celkové množství hliníku desetinásobné, tj. $9,8345\cdot10^{-4}\,\mathrm{mol}$.
Vynásobením látkového množství molární hmotností (relativní atomovou
hmotností) hliníku získáme obsah 0,026533 g Al ve vzorku. Při zadané
hustotě hliníku má toto množství objem $9,8272\cdot10^{-9}\,\mathrm{m^{3}}$.
Vydělením tohoto objemu plochou vystřižené folie ($\mathrm{35\:mm\:\times\:45\:mm}=1,575\cdot10^{-3}\,\mathrm{m^{2}}$)
získáme tloušťku
\SI[exponent-product = \ensuremath{{}\cdot{}},output-decimal-marker = {,}]{6,239e-6}{\metre}
\\$\ =\ $\SI[output-decimal-marker = {,}]{6,239}{\micro\metre}. 

Při výrobě vícevrstvých obalů je nejčastěji opravdu používána hliníková
folie o~tloušťce mezi 6 a 9~\textmu m. 

\hrulefill % \subsection*{Ročník 2, úloha č. 8.4}
\begin{quotation}
\ctyri Fenol je možno stanovit pomoci bromatometrie. Jako standard byl navážen
bromičnan draselný, $m_{\ch{KBrO_{3}}}=2,0621\,\mathrm{g}$. Navážka bromičnanu
draselného byla kvantitativně převedena do 250ml odměrné baňky a doplněn
po značku. Z~roztoku bromičnanu draselného bylo odpipetováno 20 ml,
převedeno do zábrusové Erlenmeyerovy baňky, přidáno asi 1,2 g jodidu
draselného a 2,5 ml 5M roztoku kyseliny chlorovodíkové. Baňka byla
uzavřena špuntem, promíchána a nechána napospas osudu na tmavém místě
po dobu 5 minut. Následně byla zátka opláchnuta destilovanou vodou
a výsledný roztok titrován roztokem thiosíranu sodného. Tento postup
byl opakován třikrát. Průměrná spotřeba při titraci činila 28,1 ml.
Následně bylo odebráno 10 ml vzorku fenolu ze 100ml odměrné baňky
doplněné po značku. K~tomuto roztoku bylo přidáno 25 ml roztoku bromičnanu
draselného, asi 1 g bromidu draselného a 2,5 ml roztoku kyseliny chlorovodíkové.
Baňka byla uzavřena špuntem, promíchána a ponechána v~temnu po dobu
15 minut. Následně byla zátka opláchnuta destilovanou vodou a do roztoku
přidáno asi 1,2 g jodidu draselného. Roztok byl promíchán a ponechán
v temnu dalších 5 minut. Po opětovném opláchnutí zátky byl výsledný
roztok titrován roztokem thiosíranu sodného. Tento postup byl opakován
třikrát. Průměrná spotřeba při titraci činila 32,7 ml. Vypočítejte
hmotnost fenolu ve vzorku.\\$M_{\ch{Br}}=119\,\mathrm{g\cdot mol^{-1}}$;
$M_{\ch{KBrO3}}=167\,\mathrm{g\cdot mol^{-1}}$; $M_{\ch{KI}}=166\,\mathrm{g\cdot mol^{-1}}$;
$M_{\mathrm{fenol}}=94,1\,\mathrm{g\cdot mol^{-1}}$
\end{quotation} \dotfill \par 
Na vypočítání hmotnosti fenolu ve vzorku je potřeba nejprve zjistit
koncentraci thiosíranu sodného, který se v~první části úlohy standardizoval.
Rovnice probíhajícich reakcí: 

\[
\mathrm{BrO_{3}^{-}+6\,I^{-}+6\,H^{+}\rightarrow3\,I_{2}+Br^{-}+3\,H_{2}O}
\]
\[
\mathrm{I_{2}+2\,S_{2}O_{3}^{2-}\rightarrow2\,I^{-}+S_{4}O_{6}^{2-}}
\]

Pro koncentraci standardního roztoku bromičnanu draselného platí
vztah:

\[
c_{\ch{KBrO3}}=\frac{m_{\ch{KBrO3}}}{M_{\ch{KBrO3}}\cdot V_{\mathrm{baňka}}}
\]

který po dosazení poskytuje příslušnou koncentraci $c_{\ch{KBrO3}}=0,04939\,\mathrm{mol\cdot dm^{-3}}$.

Nyní můžeme odvodit vztah pro koncentraci thiosíranu s~využitím znalosti poměru látkových
množství reaktantů a produktů:

\[
c_{\ch{Na2S2O3}}=\frac{n_{\ch{Na2S2O3}}}{V_{\ch{Na2S2O3}}}=6\cdot\frac{n_{\ch{KBrO3}}}{V_{\ch{Na2S2O3}}}=6\cdot\frac{c_{\ch{KBrO3}}V_{\ch{KBrO3}}}{V_{\ch{Na2S2O3}}}
\]

Po dosazení čísel dostávame koncentraci thiosíranu sodného: $c_{\ch{Na2S2O3}}=\frac{6\cdot0,0493919\cdot0,02}{0,0281}=0,2109\,\mathrm{mol\cdot dm^{-3}}$

Pro výpočet koncentrace fenolu je třeba si uvědomit několik věcí.
V první řade je potřeba znát rovnice probíhajících reakcí: 

\[
\mathrm{BrO_{3}^{-}+5\,Br^{-}+6\,H^{+}\rightarrow3\,Br_{2}+3\,H_{2}O}
\]
\[
\mathrm{fenol+3\,Br_{2}\rightarrow2,4,6-tribr\acute{o}mfenol+3\,Br^{-}}
\]
\[
\mathrm{Br_{2}+2\,I^{-}\rightarrow I_{2}+2\,Br^{-}}
\]
\[
\mathrm{I_{2}+2\,S_{2}O_{3}^{2-}\rightarrow2I^{-}+S_{4}O_{6}^{2-}}
\]

V prvním kroku se \textit{in situ} (tedy přímo v~baňce) vygeneruje brom, který
ihned nabromuje fenol do poloh 2, 4 a 6. Z~důvodu prchavosti a toxicity
bromu se po reakci převede přebytečný brom na jód, který se následně
ztitruje thiosíranem sodným. 

Pro výpočet koncentrace fenolu je nutné si ujasnit poměry látkových
množství, které se dají odvodit z~reakcí. Platí vztah: $n_{\ch{KBrO3}}=n_{\mathrm{fenol}}$.
Z~tohoto vztahu vyplývá, že postačí bilancovat bromičnan, protože pouze z~něj se vytváří brom pro reakci. Koncentraci fenolu tedy spočítáme přímo z~koncecentrace bromičnanu:

\[
n_{\ch{KBrO3}}=c_{\ch{KBrO3}}\cdot V_{\ch{KBrO3}}
\]
\[
n_{\ch{KBrO3}}=0,0493919\cdot0,025=1,23480\cdot10^{-3}\,\mathrm{mol}
\]

Přebytečný bromičnan se spočítá ze spotřeby thiosíranu: 

\[
n_{\mathrm{preb,\ch{KBrO3}}}=\frac{1}{6}c_{\ch{Na2S2O3}}\cdot V_{\ch{Na2S2O3}}
\]

\[
n_{\mathrm{preb,\ch{KBrO3}}}=\frac{1}{6}\cdot0,210926\cdot0,0327=1,14955\cdot10^{-3}\,\mathrm{mol}
\]

Ze zbytku bromičnanu vznikal brom pro bromaci fenolu. Jeho množství
získáme jednoduchým odčítáním: 

\[
n_{\mathrm{zreag,\ch{KBrO3}}}=n_{\ch{KBrO3}}-n_{\mathrm{preb,\ch{KBrO3}}}
\]
\[
n_{zreag\,KBrO_{3}}=1,23480\cdot10^{-3}-1,14955\cdot10^{-3}=8,525\cdot10^{-5}\,\mathrm{mol}
\]

Nyní už konečně můžeme vypočítat koncentraci fenolu ve vzorku: 

\[
c_{\mathrm{fenol}}=\frac{n_{\mathrm{zreag,\ch{KBrO3}}}}{V_{\mathrm{fenol}}}
\]
\[
c_{\mathrm{fenol}}=\frac{8,525\cdot10^{-5}}{0,01}=8,525\cdot10^{-3}\mathrm{mol\cdot dm^{-3}}
\]

Pro hmotnost fenolu ve vzorku potom platí vztah:
\[
m_{\mathrm{fenol}}=M_{\mathrm{fenol}}\cdot n_{\mathrm{fenol}}=M_{\mathrm{fenol}}\cdot c_{\mathrm{fenol}}\cdot V_{\mathrm{fenol}}
\]
\[
m_{\mathrm{fenol}}=94,113\cdot8,525\cdot10^{-3}\cdot0,1=0,08023\ \mathrm{g}
\]

\hrulefill % \subsection*{Ročník 5, úloha č. 8.5 }
\begin{quotation}
\ctyri Určování množství vody ve vzorku je jednou ze základních úloh moderního
analytického chemika a nachází uplatnění v~široké škále odvětví od
potravinářského průmyslu, přes kosmetiku, až po organickou syntézu.
Dnes se pro tento účel používá coulometrická modifikace klasické titrace
podle Karla Fischera. V~analytické cele se nachází katoda a anoda
oddělené iontově-permeabilní membránou. V~anodovém prostoru se nachází
roztok jodidu, methanolu, imidazolu (Im), oxidu siřičitého a vzorku.
Z~jodidu je na anodě generován jód.

Princip titrace je následující: 
\begin{gather}
\ch{MeOH + SO2 + Im \rightarrow \mathrm{[ImH]}\mathrm{[SO3Me]}} \tag{1} \label{eq:KF1}\\
\ch{H2O + I2 + [ImH][SO3Me] + 2 Im \rightarrow \mathrm{[ImH]}\mathrm{[SO4Me]} + 2 [ImH]I}\tag{2} \label{eq:KF2}
\end{gather}
Představte si, že jste organickým chemikem, jehož úkolem je syntéza složitého
přírodního produktu -- beat\-agrad\-stu\-dent\-downotoxinu -- a už vám zbývá
provést jen poslední krok vaší syntézy: přípravu Grignardova činidla
(z~alkylbromidu a hořčíku) a jeho následnou adici na karbonyl. Bohužel
už vám ale zbývá jen 0,5 mg~vašeho pracně připraveného (45 reakčních
kroků a 5~let vaší práce) alkylbromidu ($M=5000\,\mathrm{g\cdot mol^{-1}}$).
Reakci provádíte v~THF (300~μl), u~nějž máte několik možností, jak
ho vysušit.

Kterou z~nich si vyberete, pokud znáte prošlý náboj při titraci sušeného
THF (5 ml) dle Karla Fischera, aby byl váš výtěžek co nejvyšší? Jaký je nejvyšší
možný výtěžek v~\%, pokud jako jedinou ztrátu počítáme kvantitativně
probíhající reakci vzniklého Grignardova činidla s~vodou?

Potřebná data: $M_{\ch{THF}}=72,107$, g $\cdot$ mol$^{-1}$;  $\rho_{\ch{THF}}=0,8892$ g $\cdot$ cm$^{-3}$; $M_{\ch{H2O}}=18,015$ g $\cdot$ mol$^{-1}$; \\ $F=96~485,333$~C$\cdot$mol$^{-1}$
\vspace{3mm}
\begin{center}
\begin{tabular}{l|l}
Metoda sušení & Prošlý náboj (5 ml THF) / C\tabularnewline\hline\hline
Na/benzofenon & 2,0455\tabularnewline\hline
3Å molekulová síta & 0,1907\tabularnewline\hline
Alumina & 0,2856
\end{tabular}
\par\end{center}

\end{quotation} \dotfill \par 
Při generaci jodu z~jodidu dochází k~dvouelektronové oxidaci: 
\[
\ch{2 I^{-} \rightarrow \mathrm{I_{2}} + 2 e^{-}}
\]
Jod pak reaguje s~vodu podle rovnice (2) v~poměru 1:1. Množství
vody ve vzorku je tedy přímo úměrné množství vygenerovaného jodu,
jehož množství je podle Faradayova zákona elektrolýzy přímo úměrné
prošlému náboji. Vygenerované Grignardovo činidlo reaguje s~vodou
podle rovnice: 
\[
\ch{RMgBr + H2O \rightarrow \mathrm{RH} + MgBrOH}
\]
Větší množství vody tudíž znamená nižší výtěžek. Nejvyšší výtěžek
tím pádem poskytne rozpouštědlo, při jehož titraci prošel nejmenší
náboj, tedy sušené 3Å molekulovými síty.

Z~Faradayova zákona elektrolýzy můžeme zjistit látkové množství vygenerovaného
jodu při titraci 5 ml THF:

\[
n(\ch{I2})=\frac{Q}{z\cdot F}=\frac{0,1907}{2\cdot96485}=9,8823\cdot10^{-7}\mathrm{~mol}
\]
Látkové množství vody v~5 ml THF je stejné jako látkové množství
jodu, ve 300 \textmu l THF tedy bude: 
\[
n(\ch{H2O})=\frac{0,3}{5\cdot988,23}\mathrm{~nmol}=59,29\mathrm{~nmol}
\]
Látkové množství pre-beatagradstudentdownotoxinu, jinými slovy našeho
bromidu, je 
\[
n(\ch{RBr})=\frac{m}{M}=\frac{0,0005}{5000}=10^{-7}\mathrm{~mol}=100\mathrm{~nmol}
\]
Jelikož Grignardovo činidlo reaguje s~vodou v~poměru 1:1, je výpočet
výtěžku triviální: 
\[
\eta=\frac{n_{\ch{RBr}}-n_{\ch{H2O}}}{n_{\ch{RBr}}}\cdot100\%=40,71\%
\]
Váš experiment tedy není příliš vhodně navržený. Vhodným řešením by
například bylo provést reakci v~menším objemu rozpouštědla (při větší
koncentraci), nebo, pokud to dostupnost vašeho karbonylu dovoluje,
vygenerovat Grignardovo činidlo transmetalací s~nadbytkem jiného,
komerčně dostupného Grignardova činidla, které nejdříve odreaguje
přítomnou vodu, a následné použití přebytku karbonylu.

\section{Gravimetrie, elektrochemie}

% \subsection*{Ročník 3, úloha č. 6.4 }
\begin{quotation}
\tri Vzorek rozpuštěné kontrabasové struny ve 100ml odměrné baňce byl doplněn
destilovanou vodou po značku a řádně promíchán. Ke stanovení bylo
odpipetováno 10,00 ml do 400ml kádinky, posléze byla směs zředěna
destilovanou vodou. Přidali jsme několik kapek alkoholického roztoku
bromthymolové modři a sráželi 1\% alkoholickým roztokem dimethylglyoximu
(15 až 20 ml). Ihned jsme neutralizovali přikapáváním zředěného NH$_{3}$
do změny barvy indikátoru na modrou.

Vzniklá sraženina byla izolována filtrací
přes skleněnou fritu předtím vysušenou v~sušárně do konstantní hmotnosti.
Sraženinu jsme pak promyli teplou destilovanou vodou a sušili v~sušárně
při 110 °C do konstantní hmotnosti. Prázdná frita vážila 12,3872 g,
frita se sraženinou pak 13,6179 g. Váženou látkou je dimethylglyoximát
nikelnatý, jehož $M_{\mathrm{r}}=288,92$. Kolik miligramů niklu obsahovala
struna? 
\end{quotation} \dotfill \par 
\noindent \begin{center}

\includegraphics{/3/6-4}

\par\end{center}

Dimethylglyoxim reaguje s~nikelnatými kationty ve stechiometrickém
poměru 2:1, tedy v~jedné molekule dimethylglyoximátu nikelnatého je
jeden atom niklu. Ze 100 ml zásobního roztoku jsme pipetovali 10 ml, výsledné množství tedy musíme adekvátně ponásobit zřeďovacím faktorem $f=\frac{100}{10}=10$. Hmotnost niklu
ve struně pak spočítáme jako

\[
m_{\mathrm{Ni}}=\frac{m_{\mathrm{DMG}}}{M_{\mathrm{DMG}}}\cdot M_{\mathrm{Ni}}\cdot f=\frac{m_{celkem}-m_{\mathrm{prázdná\ frita}}}{M_{\mathrm{DMG}}}\cdot M_{\mathrm{Ni}}\cdot f
\]
kde DMG v~indexu označuje dymethylglyoximát. Po dosazení vyjde přímo
hmotnost niklu ve vzorku.\\($M_{\mathrm{Ni}}=58,69\,\mathrm{g\cdot mol^{-1}}$)

\[
m_{\mathrm{Ni}}=\frac{13,6179-12,3872}{288,92}\cdot58,69\cdot10=2,5000\,\mathrm{g}
\]


\hrulefill % \subsection*{Ročník 5, úloha č. 6.4 }
\begin{quotation}
\dva Fluorid stříbrný se na rozdíl od všech ostatních halogenidů stříbra
nesráží. Toho může být s~úspěchem využito v~gravimetrické analýze.
Směs KF, KCl a KBr váží 5~g. K~úplnému vzniku směsi nerozpustných
sraženin postačí 145 ml 0,200 mol$\cdot$dm$^{-3}$ vodného roztoku
$\ch{AgNO3}$. Sraženina váží 4,5~g. Jaké je složení původní směsi
halogenidů? Výsledek uveďte v~hmotnostních procentech.
\end{quotation} \dotfill \par 
Začněme spočítáním látkového množství dusičnanu stříbrného potřebného
k~vysrážení všech nerozpustných halogenidů 
\[
n_{\ch{AgNO3}}=V_{\ch{AgNO3}}\cdot c_{\ch{AgNO3}}=0,029\mathrm{~mol}
\]

Celkové množství dusičnanu stříbrného se rovná součtu množství chloridů
a bromidů. 
\[
n_{\ch{AgNO3}}=n_{\ch{AgCl}}+n_{\ch{AgBr}}
\]
Z~toho vyplývá, že celková hmotnost sraženiny je dána hmotností sraženiny
chloridu a bromidu stříbrného: 
\[
m_{\text{sraženina}}=m_{\ch{AgCl}}+m_{\ch{AgBr}}
\]
Hmotnost původního vzorku je pak dána součtem všech jeho složek: 
\[
m_{\text{vzorek}}=m_{\ch{KCl}}+m_{\ch{KBr}}+m_{\ch{KF}}
\]
Drobnými úpravami a za použití vztahu $n=\frac{m}{M}$ dostaneme soustavu
tří rovnic o~třech neznámých, jejím řešením získáme 
\begin{align*}
m_{\mathrm{KCl}} & =1,5855\mathrm{~g}\\
m_{\mathrm{KBr}} & =0,9202\mathrm{~g}\\
m_{\mathrm{KF}} & =2,4943\mathrm{~g}
\end{align*}
To po přepočtu na hmotnostní procenta činí 
\begin{align*}
w_{\mathrm{KCl}} & =31,71\ \%\\
w_{\mathrm{KBr}} & =18,4\ \%\\
w_{\mathrm{KF}} & =49,89\ \%
\end{align*}


\newpage %%
 % \subsection*{Ročník 5, úloha č. 4.6 }
\begin{quotation}
\dva Jednoho dne si Standa nechal narůst dlouhé vlasy, to mu ovšem nestačilo
a potřeboval udělat něco dalšího, aby se stal pravým metalistou. Rozhodl
se tedy, že se nechá celý galvanicky pokovit. Abychom ho mohli pokovit,
nejprve ho pokryjeme tenkou vrstvou grafitu pro zajištění vodivosti
jeho těla, poté jej ponoříme do roztoku zinečnatých iontů a připojíme
do obvodu, kde bude sloužit jako katoda. Jako anoda poslouží dostatečně
velký kus zinku. Kolik dnů musíme nechat našeho metalového nadšence
pokovovat při proudu 5 A, aby se na celém jeho těle o~ploše povrchu 1,8 m$^{2}$
vytvořila 2 mm tlustá vrstva zinku?

\noindent\-Hustota kovového zinku $\rho=7140$ kg$\cdot$m$^{-3}$\\
Faradayova konstanta $F=96\,485$ C$\cdot$mol$^{-3}$\\
Molární hmotnost zinku $M=65,4$ g$\cdot$mol$^{-1}$
\end{quotation} \dotfill \par 
Nejdříve vypočítáme hmotnost zinku potřebnou na pokrytí Standy vrstvou
tlustou $2 \, \mathrm{mm}=0,002\ \mathrm{m}$.
\[
m=V\cdot\rho=S\cdot h\cdot\rho=1,8\cdot0,002\cdot7140=25,704\mathrm{~kg}
\]
Při galvanickém pokovování pracujeme s~prvním Faradayovým zákonem
elektrolýzy, který nám říká, že hmotnost vyloučeného kovu je násobkem
elektrochemického ekvivalentu $A$, proudu použitého na pokovování
$I$ a času pokovování $t$. 
\[
m=A\cdot I\cdot t
\]

Nyní vypočteme elektrochemický ekvivalent, který je zadefinován jako
molární hmotnost kovu dělená Faradayovou konstantou vynásobená počtem
elektronů potřebných na redukci iontů. Tato jednotka představuje hmotnost
kovu vyloučenou na jednotku náboje přivedenou do systému.

\[
A=\frac{M}{F\cdot Z}=3,39\cdot10^{-7}\mathrm{~kg\cdot C^{-1}}
\]
Z~rovnice prvního Faradayova zákona vyjádříme čas. Nyní již můžeme
dosadit do rovnice vypočtenou hmotnost, elektrochemický ekvivalent
a zadaný proud. 
\[
t=\frac{m}{A\cdot I}=\frac{25,704}{3,39\cdot10^{-7}\cdot5}=176\mathrm{~dní}
\]


\section{Instrumentální analýza}

 % \subsection*{Ročník 2, úloha č. 4.6}
\begin{quotation}
\jeden Těžiště kvantitativní analýzy v~moderní analytické chemii již několik
posledních let či desítek let neleží u~(na střední škole populárních)
titrací. Titrace jsou stále velmi často zařazovány do výuky na různém
stupni vzdělávání a jsou používány v~praxi jako velmi levná metoda,
avšak i naše soutěž by měla reflektovat posun směrem k~instrumentálním
metodám. Nevýhodou instrumentálních metod je fakt, že nezkušený chemik
často vůbec netuší, jak metoda funguje a celý přístroj se pro něj
stává jakousi „černou skříňkou plivající výsledky“. Přesto i s~takovým
handicapem se dá pracovat. Představte si, že máte před sebou jednu
takovou „černou skříňku“ a chcete zjistit koncentraci Cibalackrotu
(barevné látky z~předchozích úloh\footnote{Ročník 2, úloha č. 3.6, Ročník 2, úloha č. 8.6}) ve vzorku.\\
Ze 100ml zásobní lahve roztoku Cibalackrotu v~dichlormethanu o~koncentraci 1 $\mathrm{g\cdot dm^{-3}}$
jsme vždy do 25ml odměrných baněk pipetovali 1, 2, 3 a 4 ml tohoto
roztoku. Baňky jsme pak doplnili po rysku dichlormethanem a řádně
promíchali. Každý roztok jsme pak proměřili naší černou skříňkou,
která nám pro tyto roztoky ukázala následující hodnoty: 0,250, 0,487,
0,724 a 0,961. Hodnota téže veličiny pro vzorek o~neznámé koncentraci
činila za stejných podmínek měření 0,500. Kolik miligramů neznámé
látky je ve vzorku? Vzorek je podobně jako standardní roztoky předložen
v 25ml odměrné baňce a doplněn po rysku.
\end{quotation} \dotfill \par 
Je celkem rozumné předpokládat, že přístrojem vyplivnuté hodnoty lineárně
závisí na koncentraci. Abychom si to ověřili, můžeme použít data ze
čtyř standardních roztoků. Vynesením těchto hodnot v~závislosti
na koncentraci dostaneme takovýto graf:

\begin{center}
\includegraphics[scale=0.5]{/2/4-6}
\end{center}

Z~grafu je jasně patrné, že závislost skutečně lineární je. Její rovnice
je následující: $y=5,925x+0,013$ (spočítá se z~kterýchkoli dvou bodů
řešením soustavy dvou rovnic o~dvou neznámých). Pokud z~této rovnice vyjádříme
koncentraci ($x$), dostaneme vztah:

\[
x=\frac{y-0,013}{5,925}
\]

Koncentrace Cibalackrotu je tedy $x=\frac{0,5-0,013}{5,925}=0,0822\,\mathrm{g\cdot dm^{-3}}$.
Ve vzorku pak budou $0,0822\cdot25\doteq2$ mg Cibalackrotu.

\hrulefill % \subsection*{Ročník 3, úloha č. 5.1 }
\begin{quotation}
\dva Invertní cukr je ekvimolární směs glukosy a fruktosy, která vzniká
hydrolýzou, tj. inverzí sacharosy. Glukosa, fruktosa i sacharosa jsou
látky chirální, opticky aktivní; stáčejí rovinu polarizovaného světla.
Glukosa a sacharosa stáčí rovinu polarizovaného světla opačným směrem
než fruktosa, otáčivost je dále lineárně závislá na koncentraci opticky
aktivní látky. Otáčivost dále závisí na vlnové délce polarizovaného
světla. Vyjádřeno vzorcem: 

\[
\alpha=[\alpha]\cdot d\cdot c
\]
kde $\alpha$ je měřená otáčivost, $[\alpha]$ je specifická otáčivost
dané látky při určité vlnové délce (materiálová konstanta), $d$ je
délka kyvety v~decimetrech a $c$ je koncentrace chirální látky v
gramech na mililitr. Hydrolýzou sacharosy na glukosu a fruktosu se
změní celková otáčivost roztoku. Ta je aditivní vlastností, což znamená,
že se otáčivosti jednotlivých složek roztoku sčítají. Spočítejte,
jaký by musel být stupeň přeměny (v procentech) hydrolýzy čistého
1M roztoku sacharosy na invertní cukr, aby byla otáčivost roztoku
nulová. Potřebná data specifických otáčivostí: 

$[\alpha]_{\mathrm{sach}}=67,37\,\mathrm{ml\cdot g^{-1}\cdot dm^{-1}}$, 

$[\alpha]_{\mathrm{inv}}=-20,41\,\mathrm{ml\cdot g^{-1}\cdot dm^{-1}}$,

$[\alpha]_{\mathrm{glu}}=49,25\,\mathrm{ml\cdot g^{-1}\cdot dm^{-1}}$

Koncentrace invertního cukru se udává jako koncentrace odpovídajícího
množství sacharózy před inverzí.
\end{quotation} \dotfill \par 

\newpage %text%
Aby byla otáčivost nulová, musí platit

\[
\alpha=[\alpha]_{\mathrm{sach}}\cdot d\cdot c_{\mathrm{akt} }+[\alpha]_{\mathrm{inv}}\cdot d\cdot(c_{0}-c_{\mathrm{akt} })=0
\]
kde $c_{\mathrm{akt} }$ je aktuální koncentrace sacharózy a $c_{0}$ je původní
koncentrace sacharosy.

Stupeň přeměny (zde kvůli záměně s~otáčivostí označován jako $X$)
je definován jako

\[
X=\frac{c_{0}-c_{\mathrm{akt} }}{c_{0}}
\]

Po vyjádření je $c_{\mathrm{akt}}=c_{0}-c_{0}\cdot X$. Pokud dosadíme za $c_{\mathrm{akt}}$
do rovnice otáčivosti, vyjde

\[
[\alpha]_{\mathrm{sach}}\cdot d\cdot c_{0}-[\alpha]_{\mathrm{sach}}\cdot d\cdot c_{0}\cdot X+[\alpha]_{\mathrm{inv}}\cdot d\cdot c_{0}\cdot X=0
\]
Po podělení délkou kyvety $d$ a koncentrací $c_{0}$ se rovnice zjednoduší
na 
\[
[\alpha]_{\mathrm{sach}}-[\alpha]_{\mathrm{sach}}X+[\alpha]_{\mathrm{inv}}X=0
\]

Po vyjádření $X$ vyjde

\[
X=\frac{[\alpha]_{\mathrm{sach}}}{[\alpha]_{\mathrm{sach}}-[\alpha]_{\mathrm{inv}}}=0,767
\]


\hrulefill % \subsection*{Ročník 5, úloha č. 5.2 }
\begin{quotation}
\tri Tzv. rybičkometrie je hypotetickou metodou kvalitativní analýzy. Tato
metoda se dá použít k~určení neznámého kationtu v~čistém vzorku
kyanidu.

Princip metody je následující: v~jedné 1000ml odměrné baňce se v~destilované
vodě rozpustí 1,000 g čistého kyanidu sodného a v~druhé odměrné 1000ml
baňce se rozpustí 1,000 g neznámého kyanidu. Poté se vezmou dvě velká
identická akvária s~identickými rybičkami (odtud pochází název metody).
Do jednoho akvária se po malých přesně odměřených množstvích za
stálého intenzivního míchání přilévá roztok kyanidu sodného. Když
rybičky vyplavou bříškem nahoru na hladinu, zaznamená se spotřeba
roztoku kyanidu $V_{\mathrm{standard}}$. Stejný proces se opakuje
s~druhým akváriem a roztokem neznámého kyanidu, čímž obdržíme druhou
spotřebu $V_{\mathrm{nezn\acute{a}m\acute{y}}}$. Určete kation v~neznámém
kyanidu, je-li $V_{\mathrm{standard}}=365$ ml a $V_{\mathrm{nezn\acute{a}m\acute{y}}}=827$
ml.

Předpokládejte, že jedovatost kyanidu je dána pouze kyanidovým aniontem
a že kyanid perfektně disociuje.
\end{quotation} \dotfill \par 
Dle předpokladů v~zadání začnou rybičky umírat, pokud koncentrace
kyanidu v~akváriu dosáhne nějaké koncentrace $c_{letální}$, která
bude stejná v~obou případech. Protože se jedná o~identická akvária,
bude v~obou akváriích stejné látkové množství kyanidu. Pro látkové
množství $n$ platí $n=c\cdot V$, kde $c$ je koncentrace a $V$
je objem. Z~toho plyne $c_{\ch{NaCN}}\cdot V_{\mathrm{standard}}=c_{\mathrm{nezn\acute{a}m\acute{y}}}\cdot V_{\mathrm{nezn\acute{a}m\acute{y}}}$,
kde $c_{\ch{NaCN}}$ značí koncentraci daného kyanidu v~odměrném
roztoku. Protože koncentrace kyanidu v~odměrném roztoku je dána $c=\frac{n}{V}=\frac{m}{V\cdot M}$
platí následující rovnost: 
\[
\frac{V_{\mathrm{nezn\acute{a}m\acute{y}}}}{V_{\ch{NaCN}}}=\frac{M_{\mathrm{nezn\acute{a}m\acute{y}}}}{M_{\ch{NaCN}}}
\]
Z~toho dostaneme 
\[
M_{\mathrm{nezn\acute{a}m\acute{y}}}=\frac{V_{\mathrm{nezn\acute{a}m\acute{y}}}}{V_{\ch{NaCN}}}\cdot M_{\ch{NaCN}}=\frac{827}{365}\cdot49=111\,\mathrm{g\cdot mol^{-1}}
\]
tedy pro atomovou hmotnost neznámého kationtu za předpokladu, že má
náboj 1+, platí 
\[
M_{\mathrm{kation}}=M_{\mathrm{nezn\acute{a}m\acute{y}}}-M_{\ch{CN}}=111-26=85\,\mathrm{g\cdot mol^{-1}}
\]
Tato molární hmotnost odpovídá rubidiu.

\newpage %%
 % \subsection*{Ročník 3, úloha č. 6.1 }
\begin{quotation}
\dva Spektrofotometrie představuje jednoduchou a levnou možnost stanovení
obsahu barevných látek v~roztoku. V~případě, že se absorpční spektra
látek překrývají, je výpočet koncentrací poněkud obtížnější. V~kyvetě
o tloušťce 1,00 cm byly změřeny absorbance čistých látek X a Y o~koncentracích
\textit{$c_{\mathrm{X}}=0,001\,\mathrm{mol\cdot dm^{-3}}$} a $c_{\mathrm{Y}}=0,0001\,\mathrm{mol}\cdot\mathrm{dm^{-3}}$
při různé vlnové délce. 
\begin{center}
\begin{tabular}{c|c|c}

 & $\lambda_{1}=420$ nm & $\lambda_{2}=540$ nm\tabularnewline
\hline 
\textit{A}(X) & 0,825 & 0,187\tabularnewline
\hline 
\textit{A}(Y) & 0,030 & 0,580\tabularnewline

\end{tabular}
\par\end{center}
Vypočítejte koncentrace obou složek v~neznámé směsi, jestliže vzorek
směsi měl při nižší vlnové délce absorbanci 0,628 a při vyšší vlnové
délce absorbanci 0,395.

Uvažujte, že obě složky vykazují chování přesně dle Lambertova-Beerova
zákona. 
\end{quotation} \dotfill \par 
Podle Lambertova-Beerova zákona je absorbance (logaritmická míra pohlcení)
procházejícího záření úměrná součinu:
\begin{itemize}
\item dráhy záření v~absorbujícím prostředí (délce kyvety),
\item koncentrace absorbující látky,
\item její konstanty, tzv. molárního absorpčního koeficientu, který je
charakteristický pro danou látku a vlnovou délku:
\end{itemize}
\[
A=l\cdot c\cdot\varepsilon_{\lambda}
\]

Za předpokladu ideálního chování všech přítomných složek je výsledná
absorbance aditivní, tedy rovna součtu příspěvků přítomných látek:

\[
A=l\cdot\sum\left(c_{\mathrm{i}}\cdot\varepsilon_{\lambda \mathrm{i}}\right)=l\cdot(c_{1}\cdot\varepsilon_{\lambda,1}+c_{2}\cdot\varepsilon_{\lambda,2})
\]

Ze zadaných výsledků měření absorbance látek X a Y proto můžeme určit
jejich molární absorpční koeficienty:

\begin{align*}
0,825 & =1,00\cdot0,001\cdot\varepsilon_{420,\mathrm{X}}\\
\varepsilon_{420,\mathrm{X}} & =825\ \mathrm{l\cdot m^{-1}\cdot mol^{-1}}
\end{align*}
Podobně určíme 
\begin{align*}
\varepsilon_{540,\mathrm{X}} & =187\ \mathrm{l\cdot m^{-1}\cdot mol^{-1}}\\
\varepsilon_{420,\mathrm{Y}} & =300\ \mathrm{l\cdot m^{-1}\cdot mol^{-1}}\\
\varepsilon_{540,\mathrm{Y}} & =5800\ \mathrm{l\cdot m^{-1}\cdot mol^{-1}}
\end{align*}

Pro absorbance změřené ve směsi o~neznámém složení sestavíme dvě rovnice,
kde neznámými jsou koncentrace:
\begin{align*}
0,628 & =1,00\cdot(\mathrm{\mathit{c}_{X}}\cdot825+\mathrm{\mathit{c}_{Y}}\cdot300)\\
0,395 & =1,00\cdot(\mathrm{\mathit{c}_{X}}\cdot187+\mathrm{\mathit{c}_{Y}}\cdot5800)
\end{align*}

Řešením soustavy dvou rovnic o~dvou neznámých jakoukoliv validní metodou (její matematické řešení zde nebudeme provádět) dojdeme k~výsledkům:
\begin{align*}
\mathrm{\mathit{c}_{X}}=7,54\cdot10^{-4}\,\mathrm{mol\cdot dm^{-3}}\\
\mathrm{\mathit{c}_{Y}}=4,41\cdot10^{-5}\mathrm{\,mol\cdot dm^{-3}}
\end{align*}


\newpage %%
 % \subsection*{Ročník 4, úloha č. 7.3 }
\begin{quotation}
\dva Barbora kultivovala v~běžném živném médiu několik kolonií bakterie \textit{E. coli}, které se dělí každou půlhodinu. Do třech různých zkumavek
s médiem byly naočkovány bakterie v~7:00, 8:00 a 9:00. Během přípravy
čtvrtého vzorku si ovšem, nešika, smazala popisky ze vzorků připravených
v~7 a v~8 hodin. 

Vypočtěte z~následujících údajů, jaká by byla absorbance vzorku, který
Barbora připravila v~9:00, kdyby růst kolonií odpovídal ideálnímu průběhu
ve fázi exponenciálního růstu. Absorbance byla změřena v~10:00 v~kyvetě dlouhé 1~cm při 600~nm.

Absorbance vzorku A: 0,864

Absorbance vzorku B: 0,216 
\end{quotation} \dotfill \par 
Absorbance, kterou naměříme v~disperzi bakterií, bude úměrná množství
bakterií v~médiu. Proto vzorek A musel být připraven v~7:00 a vzorek
B v~8:00. Exponenciální růst bakterií znamená, že za daný časový úsek
(zde půl hodiny) se jejich počet zdvojnásobí. Tomuto odpovídá i
čtyřnásobné zvětšení absorbance mezi vzorky B a A. Jelikož byl třetí vzorek změřen o~hodinu později než vzorek B, bude jeho absorbance za uvedených předpokladů také čtyřikrát menší. Absorbance vzorku, který
Barbora připravila v~9:00, je proto 0,054. 

\hrulefill % \subsection*{Ročník 4, úloha č. 7.5 }

\textit{Tato úloha je věnována prof. Ing. Karlu Vytřasovi, DrSc. za
celoživotní přínos v~oblasti elektroanalytické chemie. Karel Vytřas
byl mj. dlouholetým vedoucím Katedry analytické chemie na Univerzitě
Pardubice, jako prorektor této školy se v~devadesátých letech významně
zasloužil o~její rozvoj. Kromě toho byl milujícím dědou prvního autora 
této publikace.}

\textit{Karel Vytřas odešel z~tohoto světa 25. ledna 2019 ve věku
74 let.}
\begin{quotation}
\tri Titrace je sice metodou středoškolsky oblíbenou, ale v~moderní analytice
se stále více používají instrumentální metody. Na stanovení fluoridů
se například téměř výhradně používají fluoridové iontově-selektivní
elektrody s~membránou z~fluoridu lanthanitého. Tímto způsobem se dá
stanovit i obsah fluoridů ve vodě či v~zubní pastě. Fluoridová elektroda
byla kalibrována šesti standardními roztoky NaF. Jejich koncentrace
a naměřené elektrodové potenciály (vůči referentní argentochloridové
elektrodě, solný můstek $\ce{KNO3}$) zachycuje následující tabulka: 
\begin{center}
\begin{tabular}{c|r||c|r} 
$c\;(\mathrm{mol \cdot dm^{3}})$ & $E\;(\mathrm{mV})$ & $c\;(\mathrm{mol \cdot dm^{3}})$ & $E\;(\mathrm{mV})$\tabularnewline
\hline 
\hline 
$1\cdot10^{-9}$ & 192,9 & $1\cdot10^{-6}$ & 188,6\tabularnewline
\hline 
$3\cdot10^{-9}$ & 192,8 & $3\cdot10^{-6}$ & 179,5\tabularnewline
\hline 
$1\cdot10^{-8}$ & 192,7 & $1\cdot10^{-5}$ & 160,8\tabularnewline
\hline 
$3\cdot10^{-8}$ & 192,5 & $1\cdot10^{-4}$ & 105,8\tabularnewline
\hline 
$1\cdot10^{-7}$ & 192,3 & $1\cdot10^{-3}$ & 45,7\tabularnewline
\hline 
$3\cdot10^{-7}$ & 191,3 & $1\cdot10^{-2}$ & --13,2\tabularnewline

\end{tabular}
\end{center}
0,3024 g zubní pasty bylo rozmícháno ve vodě s~přídavkem citrátového
pufru, následně byla suspenze kvantitativně převedena do 50ml odměrné
baňky a doplněna po rysku. Byl změřen elektrodový potenciál
fluoridové elektrody: 68,0~mV. Určete obsah fluoridu v~zubní pastě.
Výsledek uvádějte jako hmotnostní promile $\ce{F-}$.
\end{quotation} \dotfill \par 
Potenciál elektrody je lineárně závislý na logaritmu koncentrace fluoridových
aniontů\footnote{To lze odvodit také např. z~Nernstovy rovnice: $E=E^{\ominus}+\frac{RT}{zF}\ln\frac{c_{1}}{c_{2}}$}.
Pokud tedy budeme uvažovat, že potenciál je lineární funkcí záporně
vzatého dekadického logaritmu koncentrace $E=k\cdot\mathrm{p}c+q$,
můžeme ze dvou bodů určit rovnici této funkce a následně dopočítat
hodnotu koncentrace pro měřený potenciál.

Naměřený potenciál leží mezi potenciály pro koncentrace standardních
roztoků 0,001 a 0,0001 $\mathrm{mol \cdot dm^{3}}$. Pokud tedy chceme hledat
rovnici závislosti potenciálu na logaritmu koncentrace, je vhodné
použít tyto dvě hodnoty\footnote{Správnější by samozřejmě bylo určit rovnici této závislosti regresí všech dat v~lineární oblasti. Vzhledem k~tomu, že soutěžící mohli používat pouze kalkulačku a ne všechny kalkulačky disponují statistickými funkcemi, uvádíme zde i toto poněkud méně přesné řešení.}. Směrnice poté vychází

\[
k=\frac{\Delta E}{\Delta\mathrm{p}c}=\frac{105,8-45,7}{4-3}=60,1
\]

Úsek je pak

\[
q=45,7-60,1\cdot3=-134,6
\]

Pomocí získané závislosti $E=60,1\cdot\mathrm{p}c-134,6$ spočítáme
koncentraci fluoru ve vzorku:

\[
c_{\mathrm{F}}=10^{-\frac{68+134,6}{60,1}}=4,2555\cdot10^{-4}\,\mathrm{mol\cdot dm^{-3}}
\]

Pokud pak chceme znát hmotnost fluoru ve vzorku, pak už pouze stačí
vynásobit koncentraci objemem, v~němž jsme vzorek rozředili, a molární
hmotností fluoru.

\[
m_{\mathrm{F}}=c_{\mathrm{F}}\cdot M_{\mathrm{F}}\cdot V=4,2555\cdot10^{-4}\cdot0,050\cdot19=4,043\cdot10^{-4}\,\mathrm{g}
\]

Hmotnostní promile pak zjistíme podělením hmotnosti fluoru a celkové
hmotnosti vzorku\footnote{Obsah fluoru v~zubní pastě se častěji udává v~hmotnostních ppm (parts
per milion). To je bezrozměrná jednotka podobná procentům a promile, ovšem
ppm dělí celek na milion dílů. Výsledek tého úlohy v~ppm tedy je 1340~ppm.}:

\[
w=\frac{m_{\mathrm{F}}}{m_{\mathrm{vz}}}=\frac{4,043\cdot10^{-4}}{0,3024}=1,34\, \text{‰}
\]

\textit{Poznámka autora: Hodnoty potenciálu pro koncentrace menší
než $10^{-5}\:\mathrm{mol\cdot dm^{-3}}$ již nejsou lineárně závislé
na koncentraci, zde se elektroda dostává pod svou mez detekce.}

\chapter{Biochemie}

\section{Obecné}

 % \subsection*{Ročník 4, úloha č. 0.22 }
\begin{quotation}
\jeden \textit{,,Ferment (z lat. fermentum), kvasidlo. Na působení \textbf{f}-ů závisí velká
řada jevů biologických. Jsou činiteli snad ve všech buňkách zvířecích
i bylinných, podmiňujíce jich život i vzrůst, a jsou zplodem žlaz
organismu.} (\ldots) \textit{Fermentace jest v~podstatě reakcí chemickou. Změny,
které určitý \textbf{f}. v~látce chemicky přesně definované způsobuje, aniž
sám jakýmkoli změnám podléhá, jsou podobny pochodům, zvaným katalytické.}
(\ldots) \textit{Intensita fermentativná závisí na vnějších okolnostech, z~nichž
hlavní úkol hraje určitá teplota, koncentrace roztoku, alkaličnost
neb kyselost jeho, přítomnost solí a p.``}\footnote{Převzato z~Ottova slovníku naučného. Devátý díl. Praha : J. Otto,
1895.}

\textbf{Jak se fermenty nazývají dnes?}
\end{quotation} \dotfill \par 
Řešení: enzymy. Tento pojem odvozený od řeckého $\zeta\acute{\upsilon}\mu\eta$
(\textit{zýmé}, „kvašené těsto“, „droždí“) zavedl fyziolog Wilhelm Kühne, žák
Freidricha Wöhlera v~Gotinkách (Göttingen) a od roku 1871 nástupce
Hermanna von Helmholtze na univerzitě v~Heidelbergu. Již tehdy bylo
známo, že enzymy jsou schopny působit i mimo organismy, z~nichž byly
získány. Označení ferment přetrvává v~bulharštině a ruštině (\selectlanguage{russian}фермент\selectlanguage{czech}),
litevštině (fermentas) a lotyšštině (ferments)\footnote{Původní příspěvek je na straně 190 ve sborníku z~roku 1876: \href{https://archive.org/stream/verhandlungendes7477natu}{https://archive.org/stream/verhandlungendes7477natu}}.

\hrulefill % \subsection*{Ročník 2, úloha č. 1.3 }
\begin{quotation}
\jeden \textit{Pohybem ke zdraví} --- oblíbená věta všech zastánců zdravého životního
stylu. V~této úloze se zaměříme na biochemické děje probíhající v
lidském těle při zátěži. Jak známo, svaly jako vykonavatele pohybu
musíme sytit palivem. Základní molekula, která je zdrojem energie
nejen pro svaly, ale i většinu ostatních buněk v~těle, je přítomna
v krvi v~přísně regulované koncentraci. Jaký je název tohoto sacharidu?
\end{quotation} \dotfill \par 
Sacharid se jmenuje glukóza. Jeho koncentraci v~krvi řídí hormon inzulin.

\hrulefill % \subsection*{Ročník 1, úloha č. 3.3}
\begin{quotation}
\jeden Kterou látkou začíná citrátový cyklus? Nápovědou vám budiž skutečnost,
že tato látka vzniká reakcí acetyl-CoA, oxalaceátu a vody. Napište
název i strukturní vzorec.
\end{quotation} \dotfill \par 
Jedná se o~citrát, sůl kyseliny citronové\footnote{Pro účely soutěže jsme uznávali libovolně protonovanou formu.}. 

\begin{center}
\includegraphics{/1/3-3}
\end{center}


\newpage %%
% \subsection*{Ročník 3, úloha č. 3.6  }
\begin{quotation}
\dva Svět chemiků je plný mnoha různých zkratek a symbolů, jimiž se brání
přesile složitých systematických názvů. Například v~posledních několika
dekádách se stalo velmi módním a moderním důkladněji se zajímat o
potraviny, které denně požíváme, ať už jde o~jejich původ, vzhled,
konzistenci či prospěšnost. Tomuto účelu slouží také mnohé regulace
a zákonná opatření, za všechny můžeme zmínit například seznam přídavných látek
v jídle, které na štítku potraviny najdeme označené písmenem E a tří-
nebo čtyřciferným kódem. Do této kategorie spadají mnohé přísady zlepšující
kvalitu jídla, nehledě na jejich případnou prospěšnost či škodlivost;
proto se také mnozí ,,éčkům`` snaží co nejvíce vyhýbat. Je to ale možné
úplně? Sami posuďte platnost tohoto tvrzení a označte v~následujícím
výčtu ty látky, které běžně vznikají v~lidském těle. 
\begin{itemize}
\item E 121, citronová červeň 2 
\item E 270, laktát (kyselina mléčná) 
\item E 290, oxid uhličitý 
\item E 300, kyselina askorbová a askorbáty 
\item E 620, glutamát (kyselina glutamová)
\item E 926, oxid chloričitý
\item E 938, argon
\item E 954, sacharin. 
\end{itemize}
\end{quotation} \dotfill \par 
V lidském těle běžně vznikají pouze tři látky, a to laktát (E 270)
při anaerobní glykolýze a při svalovém zatížení, oxid uhličitý (E
290) jako konečný produkt kaskády oxidačních dějů a E 620 (glutamát)
jako proteinogenní aminokyselina a meziprodukt metabolismu ostatních
aminokyselin. Zbylé látky zadaný požadavek nesplňují (a to ani E 300,
kyselina askorbová, tu si lidské tělo nedokáže syntetizovat samo,
pouze ji metabolizuje!). 

\hrulefill % \subsection*{Ročník 4, úloha č. 4.4 }
\begin{quotation}
\dva Redoxní reakce organických sloučenin probíhají v~tělech všech živých organismů, pomocí nich získává organismus energii pro svůj život. Doufáme, že i vy máte ještě dost energie na řešení další úlohy.

Doplňte do tabulky vždy druhý biologicky významný člen redoxního páru,
tj. oxidovanou či redukovanou formu molekuly:
% \textit{Poznámka: v~zadání nebyly struktury pyruvátu, sukcinátu, malátu a acetoacetátu.}
\noindent \begin{center}
\begin{tabular}{c|c}

& \includegraphics{images_new/4/4-4/02.eps}\tabularnewline
pyruvát & laktát\tabularnewline
\hline
\includegraphics{images_new/4/4-4/03.eps} & \tabularnewline
fumarát & sukcinát\tabularnewline
\hline
\includegraphics{images_new/4/4-4/05.eps} & \tabularnewline
oxalacetát & malát\tabularnewline
\hline
& \includegraphics{images_new/4/4-4/08.eps}\tabularnewline
acetoacetát & $\upbeta$-hydroxobutyrát\tabularnewline

\end{tabular}
\end{center}
\end{quotation} \dotfill \par 

\newpage %text%
\-Řešení:
\noindent \begin{center}
\begin{tabular}{c|c}

\includegraphics{images_new/4/4-4/01.eps} & \includegraphics{images_new/4/4-4/02.eps}\tabularnewline
pyruvát & laktát\tabularnewline
\hline
\includegraphics{images_new/4/4-4/03.eps} & \includegraphics{images_new/4/4-4/04.eps}\tabularnewline
fumarát & sukcinát\tabularnewline
\hline
\includegraphics{images_new/4/4-4/05.eps} & \includegraphics{images_new/4/4-4/06.eps}\tabularnewline
oxalacetát & malát\tabularnewline
\hline
\includegraphics{images_new/4/4-4/07.eps} & \includegraphics{images_new/4/4-4/08.eps}\tabularnewline
acetoacetát & $\upbeta$-hydroxobutyrát\tabularnewline

\end{tabular}
\end{center}


\hrulefill % \subsection*{Ročník 3, úloha č. 6.3 }
\begin{quotation}
\tri Určete, kolik ATP lze získat z~jedné kyseliny stearové (C$_{17}$H$_{35}$COOH)
po odečtení nákladů na její aktivaci. Víme, že nejprve je nutné mastnou
kyselinu aktivovat v~cytosolu, a to za spotřeby jednoho ATP, ten však
přichází o~dva fosfáty, počítejme ho tedy energeticky za dva. Po přechodu
skrz první mitochondriální membránu dojde k~navázání na karnitin,
tato reakce je samovolná. V~momentě, kdy je mastná kyselina uvnitř
vnitřní mitochondriální membrány, dochází k~$\upbeta$-oxidaci. V~jednom
cyklu dojde ke vzniku jednoho NADH (+ H$^{+}$) a FADH$_{2}$. Produktem
$\upbeta$-oxidace je kyselina o~dva uhlíky kratší s~již navázaným CoA
a acetyl-CoA. Acetyl-CoA je metabolizován v~citrátovém cyklu za vzniku
3 molekul NADH, jedné molekuly ATP a jedné molekuly FADH$_{2}$. Z~jedné molekuly
FADH$_{2}$ je organismus schopný získat 1,5 ATP a jedno NADH odpovídá
2,5 ATP. $\upbeta$-oxidace se opakuje až do posledního rozštěpení MK
na dva acetyl-CoA.
\end{quotation} \dotfill \par 
Víme, že při jednom cyklu $\upbeta$-oxidace dochází ke vzniku 1 FADH$_{2}$,
1 NADH + H$^{+}$ a 1 acetyl-CoA. Pro kyselinu stearovou je třeba
8 cyklů, protože při osmém se čtyřuhlíkatá kostra mění na dva dvouuhlíkaté
zbytky, vznikají tedy rovnou dva acetyl-CoA. Celkem tedy jen $\upbeta$-oxidací
získáme 8 FADH$_{2}$ a 8 NADH + H$^{+}$. Zbylých 9 acetyl-CoA ($\nicefrac{18}{2}$)
je dále metabolizováno v~citrátovém cyklu, přičemž 1 acetyl-CoA odpovídá
3 NADH + H$^{+}$, 1 FADH$_{2}$ a jednomu ATP\footnote{Pro zjednodušení zde uvádíme, že při Krebsově cyklu vzniká jedna molekula
ATP. Ve skutečnosti se však nejedná o~ATP, ale o~jeho energetický
ekvivalent GTP. Tato molekula vypadá podobně jako ATP, jen místo
adeninu obsahuje jako nukleovou bázi guanin.}. Ze všech acetyl-CoA tedy získáváme 27 NADH + H$^{+}$, 9 FADH$_{2}$
a 9 ATP. Sečtením zisků z~$\beta$-oxidace a citrátového cyklu dojdeme
k mezivýsledku $8+27=35$ NADH + H$^{+}$ a $8+9=17$ FADH$_{2}$ a 9 ATP.
Zpracováním redukovaných kofaktorů organismus získá $35\cdot 2,5$
ATP (z~NADH + H$^{+}$) + $17\cdot 1,5$ ATP (z~FADH$_{2}$), zbylých
9 ATP pochází přímo z~citrátového cyklu. Nesmíme zapomenout odečíst
energetický ekvivalent dvou molekul ATP nutný pro aktivaci, výsledek
je tedy $35\cdot2,5+17\cdot1,5+9-2=120$ ATP. 

Výsledek lze zobecnit do vzorce $\mathrm{ATP_{total}}=(\frac{x}{2}-1)\cdot14+10-2$,
kde $x$ je počet atomů uhlíku v~molekule nasycené mastné kyseliny. 

\newpage %nadpis%
\section{Peptidy, proteiny}

% \subsection*{Ročník 5, úloha č. 0.25 }
\begin{quotation}
\jeden Kolagen je důležitou biomolekulou, která se vyskytuje ve velkém množství
v mimobuněčné hmotě, například v~kostech, vazivu či chrupavkách.
V této úloze se blíže podíváme na tuto zajímavou molekulu. Vaší úlohou
bude zodpovědět obě podotázky.


\begin{enumerate}[label=\Alph*.] 
\item Kolagen je\\
\begin{tabular}{ccccc}
 & 1. protein & 2. sacharid & 3. lipid & 4. nukleová kyselina\tabularnewline
\end{tabular}
\item Na obrázku níže je 3D struktura kolagenu\footnote{Zdroj: Wikimedia Commons, licence CC BY-SA 3.0. \href{https://en.wikipedia.org/wiki/File:Collagentriplehelix.png}{https://en.wikipedia.org/wiki/File:Collagentriplehelix.png}}. Jak můžete vidět, molekula
kolagenu je složená ze tří řetězců (na obrázku je každý jinou barvou).
Každý řetězec je sám o~sobě levotočivý, avšak celé uspořádání tří řetězců
je pravotočivé. Právě toto těsné uspořádání dodává kolagenu jeho pevnost.
Aby ale mohly být tři řetězce takto blízko sebe, je každá třetí aminokyselina
v kolagenu tou nejjednodušší kódovanou aminokyselinou, kterou známe.
Jak se tato aminokyselina jmenuje?
\end{enumerate}
\begin{center}
\includegraphics[scale=0.25, angle=10]{Collagentriplehelix.png}
\par\end{center}
\end{quotation} \dotfill \par 

\begin{enumerate}
\item Kolagen je samozřejmě proteinem, jak plyne z~textu otázky B. 
\item Nejjednodušší kódovanou aminokyselinou je glycin. 
\end{enumerate}

\hrulefill % \subsection*{Ročník 4, úloha č. 1.4 }
\begin{quotation}

\jeden Proteiny jsou mimořádně pestrou skupinou biomolekul, které v~organismech
zajišťují celou řadu funkcí. Přiřaďte k~uvedeným proteinům I až V
jejich úlohy A až E:
\noindent \begin{center}
\begin{tabular}{r|l||r|l}

I & fibrinogen & A & transportní\tabularnewline
\hline 
II & hemoglobin & B & ochranná, obranná\tabularnewline
\hline 
III & aktin a myosin & C & stavební\tabularnewline
\hline 
IV & kolagen & D & enzymatická, katalytická\tabularnewline
\hline 
V & chymotrypsin & E & zajištění pohybu\tabularnewline

\end{tabular}
\par\end{center}
\end{quotation}

I--B, II--A, III--E, IV--C, V--D

\textbf{Fibrinogen} figuruje jako důležitý protein při poranění: díky němu
se sráží krev.\\
\textbf{Hemoglobin} je protein, který po těle zajišťuje transport kyslíku.
Jednoznačně mu tedy přísluší transportní funkce.\\
\textbf{Aktin a myosin} jsou proteiny hrající roli při smršťování a uvolňování
svalů, přísluší jim tedy úloha zajištění pohybu.\\
\textbf{Kolagen} je základním stavebním proteinem těla, má tedy logicky stavební
funkci.\\
\textbf{Chymotrypsin} nám pomáhá rozkládat stravu v~našem trávicím traktu.
Je to enzym, tedy má jasně katalytickou úlohu.

\newpage %%
 % \subsection*{Ročník 1, úloha č. 1.5 }
\begin{quotation}
\jeden Syntéza peptidů zažila v~posledních desetiletích obrovský boom poté,
co byla vynalezena syntéza na pevné fázi. O~její rozvoj se velmi
významně zasloužili pánové Wang a Merrifield. Po vás však nic podobně
obtížného nechceme -- postačí nám, že nakreslíte strukturní vzorec
alanylglycinu.
\end{quotation} \dotfill \par 
\-Alanin bude navázaný na aminoskupinu glycinu peptidovou vazbou. Strukturní vzorec výsledného alanylglycinu je tedy:

\begin{center}
\includegraphics{/1/1-5}
\end{center}


\hrulefill % \subsection*{Ročník 5, úloha č. 2.5 }
\begin{quotation}
\dva Peptidy a proteiny jsou nepostradatelnou součástí lidského těla, bez nich bychom
zde doslova nebyli. Slouží mj. jako stavební kameny strukturních proteinů, jakož i
enzymů nezbytných pro správné fungování všech funkcí těla. Skládají se z~20
základních biogenních aminokyselin. Jejich variabilita může být opravdu
obrovská, to si dokážeme na následujícím příkladu. Napište pomocí
třípísmenných zkratek (od N-konce k~C-konci) \textbf{všechny možné
různé} tripeptidy, které lze složit ze tří nejlehčích aminokyselin
tak, abychom každou z~nich použili právě jednou.
\end{quotation} \dotfill \par 
Nejlehčí aminokyseliny jsou alanin, glycin a serin, sekvence tedy budou 
\begin{center}
\begin{tabular}{lcr}
Ala--Gly--Ser & Gly--Ala--Ser & Ser--Ala--Gly\tabularnewline
Ala--Ser--Gly & Gly--Ser--Ala & Ser--Gly--Ala\tabularnewline
\end{tabular}
\par\end{center}

U peptidů rozlišujeme N-konec a C-konec, bude tedy záležet
na pořadí aminokyselin a možných tripeptidů bude celkem šest.

\hrulefill % \subsection*{Ročník 4, úloha č. 3.5 }
\begin{quotation}
\dva Nejmenovaný autor básnil o~(ne)užitečnosti encyklopedických
znalostí pro řešení úloh Chemiklání. Spojujeme nyní encyklopedické
znalosti s~v autorském kolektivu široce neoblíbeným oborem biochemie.
Pojmenujte následující peptid s~využitím třípísmenných zkratek aminokyselin.
\begin{center}

\includegraphics{/4/3-5}

\par\end{center}

\end{quotation} \dotfill \par 
H-Tyr-Gly-Gly-Phe-Met-OH. Koncové skupiny lze rozepsat i explicitněji -- místo počátečního H-- lze uvést celou --NH$_2$ skupinu, místo koncového --OH pak celou --COOH skupinu (respektive jejich ionizované formy).

\newpage %%
% \subsection*{Ročník 2, úloha č. 4.1 }
\begin{quotation}
\tri Somatostatin je tetradekapeptid produkovaný v~hypotalamu, slinivce
břišní a dalších orgánech, tlumící uvolňování růstového hormonu. Zapište
jeho primární strukturu od N-konce k~C-konci třípísmennými zkratkami
a uveďte případné modifikace řetězce, víte-li, že: 
\begin{itemize}
\item Působením aminopeptidázy byl uvolněn alanin a následně menší množství
glycinu.
\item Enzymatickou hydrolýzou byly získány peptidové fragmenty těchto sekvencí:
\begin{itemize}
\item Phe-Trp 
\item Thr-Ser-Cys 
\item Lys-Thr-Phe
\item Thr-Phe-Thr-Ser-Cys 
\item Asn-Phe-Phe-Trp-Lys 
\item Ala-Gly-Cys-Lys-Asn-Phe 
\end{itemize}
\item V~řetězci somatostatinu je přítomna disulfidická vazba. 
\end{itemize}
\end{quotation} \dotfill \par 
\begin{center}
\includegraphics[scale=0.75]{/2/4-1}
\end{center}

Výsledek působení aminopeptidázy dává informaci o~aminokyselinách
na N-konci řetězce. Sekvence tedy začíná Ala-Gly. Z~fragmentů získaných
hydrolýzou tomu odpovídá pořadí prvních šesti aminokyselin Ala-Gly-Cys-Lys-Asn-Phe.
Následně vybereme z~dalších fragmentů ten, který navazuje na začátek
řetězce, tj. Asn-Phe-Phe-Trp-Lys. Takto sestavíme celou sekvenci 14
aminokyselin: Ala-Gly-Cys-Lys-Asn-Phe-Phe-Trp-Lys-Thr-Phe-Thr-Ser-Cys.
Správnost navržené sekvence ověříme tím, že v~ní jsou zastoupeny i
všechny kratší fragmenty. Disulfidická vazba nemůže být jinde, než
mezi zbytky cysteinu, které jsou 3. a 14. v~pořadí. 

\hrulefill % \subsection*{Ročník 4, úloha č. 4.5 }
\begin{quotation}
\dva Titin (podle titánů, obrů v~řecké mytologii), také nazývaný konektin,
je nejdelší v~dnešní době známý protein. Je zodpovědný za mechanickou
pružnost uvolněných svalů a v~lidském těle se jej běžně vyskytuje
půl kilogramu. Jeho systematický název začínající \textit{methionyl-} má délku
189 819 písmen a vyslovit jej trvá déle než hodinu. Relativní molekulová
hmotnost titinu je vyšší než $3\cdot10^{6}$. U~člověka obsahuje celkem
34350 zbytků aminokyselin s~procentuálním složením (početním,\\100
\% = 34350) uvedeným v~tabulce. Určete koeficient \textit{e} v~sumárním vzorci
lidského titinu $\ch{C_{\textit{a}}H_{\textit{b}}N_{\textit{c}}O_{\textit{d}}S_{\textit{e}}}$, tedy kolik
atomů síry je obsaženo v~jednom jeho vláknu.
\begin{center}
\begin{tabular}{c|c||c|c||c|c||c|c}

Ala & 6,067 & Ile & 6,003 & Gly & 6,015 & Pro & 7,328\tabularnewline
\hline 
Cys & 1,493 & Lys & 8,568 & His & 1,392 & Gln & 2,742\tabularnewline
\hline 
Asp & 5,007 & Leu & 6,163 & Arg & 4,774 & Val & 9,269\tabularnewline
\hline 
Glu & 9,295 & Met & 1,159 & Ser & 7,170 & Trp & 1,357\tabularnewline
\hline 
Phe & 2,643 & Asn & 3,234 & Thr & 7,412 & Tyr & 2,908\tabularnewline

\end{tabular}
\par\end{center}

\end{quotation} \dotfill \par 
Síra je obsažena v~aminokyselinách cysteinu (Cys) a methioninu (Met).
Tyto představují $1,493+1,159 =2,652\,\% $ aminokyselin v~titinu, což v~řetězci o~délce 34~350
zbytků aminokyselin znamená $910,962\approx911$ aminokyselin (obsahujících
po jednom atomu síry).

\section{Transkripce a translace}

 % \subsection*{Ročník 2, úloha č. 4.4}
\begin{quotation}
\dva Kolem DNA se dnes točí svět. Kary B. Mullis dostal v~roce 1993 Nobelovu
cenu za vynález metody PCR, která umožňuje zmnožit úseky DNA jednoduchým
procesem během několika málo hodin. V~této úloze si ale vyzkoušíme
opačný proces: analýzu. Analýzou úseku dvouvláknové DNA bylo zjištěno,
že 30,3 \% počtu jejích bází je tvořeno adeninem. Určete procentuální
obsah guaninu a uracilu ve zkoumaném úseku.
\end{quotation} \dotfill \par 
Ve dvoušroubovici DNA jsou k~sobě párovány vždy dvěma vodíkovými vazbami
adenin a thymin (A$=$T) a třemi vazbami cytosin a guanin (C$\equiv$G),
a proto platí rovnost obsahů $x_\mathrm{A} = x_\mathrm{T}$ a $x_\mathrm{C} = x_\mathrm{G}$. Z~toho také
plyne, že celkový obsah purinových bází (A+G) je roven obsahu pyrimidinových
bází (C+T). Tato pravidla poprvé uvedl rakousko-americký biochemik
Erwin Chargaff roku 1950. Využitím těchto pravidel a dopočtem do 100
\% zjistíme ze zadaného obsahu adeninu zastoupení zbývajících bází:
30,3 \% T a shodně 19,7 \% C a G. Uracil ve vzorku není přítomen,
jelikož se jedná o~DNA. (Uracil se vyskytuje v~RNA.) 

\hrulefill % \subsection*{Ročník 2, úloha č. 6.1}
\begin{quotation}
\textit{Přílohou k~této úloze byla kodonová tabulka, kterou si můžete stáhnout například zde:\\ \url{https://commons.wikimedia.org/wiki/File:Aminoacids_table.svg}}

\tri V~nedávné době byla objevena funkce cirkulárních RNA v~lidském genomu.
Tyto cirkulární RNA obsahují speciální místa, kam váží miRNA, čímž
brání sestřihu mRNA a post-transkripčním modifikacím. Takto funguje
například gen pro determinaci mužského pohlaví SRY, který po transkripci
do mRNA cirkularizuje a blokuje miR-138. U~bakterií však mohou tyto
RNA kódovat i určité proteiny, což je umožněno absencí intronů a jiným
než splicingovým vznikem. Určete, která z~následujících cirkulárních
RNA kóduje protein a zapište jeho sekvenci. 

Poznámka: Číslo 31 značí 31 bází A, resp. U v~kruhu. Obě RNA tedy sestávají
z~34 bází.
\end{quotation} \dotfill \par 
\begin{center}
\includegraphics[scale=0.35]{02_6_1_a.pdf}\hspace{2cm}
\includegraphics[scale=0.35]{02_6_1_b.pdf}
\end{center}

Translace mRNA začíná od AUG kodónu. Ten je v~obou mRNA právě
jeden, pokud tedy začneme translatovat mRNA s~adeniny, získáme: AUG
= methionin, 10$\times$AAA = 10$\times$~lysin, AAU = asparagin, GAA = glutamová kyselina,
9$\times$AAA = 9$\times$~lysin, AAA = lysin, UGA = stop kodón. Tento protein je
tedy možné vytvořit.\\
U druhé cirkulární mRNA s~uracily získáme : AUG = methionin, 10$\times$UUU = 10$\times$fenylalanin, UAU = tyrosin, GUU = valin, 9$\times$UUU = 9$\times$~fenylalanin, UUA = leucin, UGU = cystein, 10$\times$~UUU = 10$\times$~fenylalanin, AUG = methionin,~\ldots

Vzhledem k~tomu, že se v~druhé RNA nenachází žádný stop kodón a příslušný protein by se proto teoreticky translatoval donekonečna, není druhá RNA validní kódující sekvencí. Kódující RNA je tedy ta první, sekvence proteinu je Met--(Lys)$_{10}$--Asn--Glu--(Lys)$_{10}$. \\
Jedmopísmennými zkratkami pak MKKKKKKKKKKNEKKKKKKKKKK.

\section{Enzymy, hormony}

 % \subsection*{Ročník 2, úloha č. 3.1}
\begin{quotation}
\jeden Biochemik Jardík dostal v~laborce zoufalou chuť na sladké, nemůže
však najít žádný cukr. Které látky má zkombinovat, aby dostal něco
sladkého, když má v~laborce k~dispozici: chymosin, vzorek viru HIV,
latexové rukavice, alkoholdehydrogenázu, amylázu, polypropylen, pyruvátdekarboxylázu,
amoniak, škrob a kyselinu šťavelovou?
\end{quotation} \dotfill \par 
Smícháním amylasy a škrobu si může Jardík připravit jednodušší cukry, protože jejich vzájemnou reakcí dojde ke štěpení glykosidických vazeb v~makromolekule škrobu\footnote{Tímto rozhodně nikoho nenabádáme k~ochutnávání laboratorních vzorků.}.

\newpage %%
 % \subsection*{Ročník 4, úloha č. 4.2. }
\begin{quotation}
\dva Jedním z~nejvýkonnějších enzymů lidského těla je acetylcholinesteráza.
Tento enzym se vyskytuje na membránách neuronů a svalových buněk a
je nezbytně nutný k~odbourávání neurotransmiteru acetylcholinu. Napište vzorce produktů této reakce. Jako nápověda Vám může posloužit vzorec acetylcholinu:
\begin{center}
\includegraphics{/4/4-2-0}
\par\end{center}

\end{quotation} \dotfill \par 
Esteráza je enzym, který štěpí estery na karboxyláty a alkoholy. Acetylcholin se tedy působením esterázy rozštěpí na alkohol cholin a anion kyseliny octové (acetát). Celé reakční schéma vidíte níže:
\begin{center}
\includegraphics{/4/4-2}
\par\end{center}

\hrulefill % \subsection*{Ročník 3, úloha č. 4.6 }
\begin{quotation}
\dva Svět chemiků je plný mnoha různých zkratek a symbolů, jimiž se brání
přesile složitých systematických názvů. Mezinárodní biochemická unie
například klasifikuje enzymy podle čísel, kdy každému enzymu přiřazuje
čtyřmístný kód. První číslo určuje třídu enzymu, těch rozlišujeme
6\footnote{Od srpna 2018 je definována sedmá třída, tzv. translokasy.}:

EC 1-oxidoreduktasy: katalyzují redoxní děje 

EC 2-transferasy: přenášejí funkční skupiny 

EC 3-hydrolasy: katalyzují hydrolytické štěpení 

EC 4-lyasy: štěpí vazby bez vstupu vody

EC 5-izomerasy: katalyzují izomerizační reakce

EC 6-ligasy (synthetasy): spojují dvě molekuly kovalentní vazbou za
spotřeby ATP 

Určete u~následujících reakcí, k~jaké třídě patří enzym, který je
katalyzuje. K~jednotlivým enzymům napište číslo třídy\footnote{V zadání byly pouze názvy enzymů napsané nad reakční šipku, zde pro přehlednost uvádíme řešení rovnou pod příslušnou reakcí.}.
\end{quotation} \dotfill \par 
\begin{center}
\includegraphics{/3/4-6-1}
\end{center}

Enzym fosfatasa: Dochází zde k~odštěpení fosfátu za spotřeby jedné
molekuly vody. Fosfatasa proto patří mezi hydrolasy, skupina 3. Číselné
označení tohoto enzymu je EC 3.1.3.16.

\begin{center}
\includegraphics{/3/4-6-2}
\end{center}

Enzym fosfodiesterasa: Dochází zde k~rozdělení cyklu, pro toto
štěpení je třeba molekula vody, tedy se opět jedná o~hydrolasu, skupina
3. Celá identifikace enzymu: EC 3.1.4.1.

\begin{center}
\includegraphics[scale=0.8]{/3/4-6-3}
\end{center}

Enzym fosforylasa: katalyzuje reakci, kdy se z~polysacharidového
řetězce (škrobu) odštěpí jedna molekula glukosy s~navázaným fosfátem
(1-P-glukosa). Dalo by se říci, že to bude lyasa, vždyť žádná voda
nevstupuje. Ale je zde ten fosfát, který se na glukosu musel nějak
dostat, a to právě působením fosforylasy. Jedná se tedy o~transferasu,
došlo k~přenosu funkční skupiny (fosfátu). Skupina 2. Kompletní identifikace:
EC 2.4.1.1 

\hrulefill % \subsection*{Ročník 3, úloha č. 4.1 }
\begin{quotation}
\dva Dopamin, adrenalin a noradrenalin patří mezi sloučeniny účastnicí se přenosu vzruchu mezi neurony. Tyto tři příklady tzv. neurotransmiterů jsou postupně syntetizovány z~aminokyseliny tyrosinu. Hydroxylací
a následnou dekarboxylací vzniká dopamin, pomocí $\upbeta$-hydroxylázy
vzniká z~dopaminu noradrenalin, který je prekursorem
pro vznik adrenalinu. Molekula adrenalinu má stejné množství atomů
uhlíku jako tyrosin. Přiřaďte jednotlivé názvy ke vzorcům. 
\end{quotation} \dotfill \par 

Doplněné reakční schéma:
\noindent \begin{center}

\includegraphics{/3/4-1}

\par\end{center}

Přeměna tyrosinu na dopamin je dvoukroková reakce, nejprve dochází
k dekarboxylaci tyrosindekarboxylázou za vzniku \textsc{l}-DOPA, až poté vzniká
hydroxylací dopamin. V~praxi je využíváno toho, že \textsc{l}-DOPA (levodopa)
je prekurzorem dopaminu, a to při léčbě Parkinsonovy choroby. 

\section{Toxikologie}

 % \subsection*{Ročník 5, úloha č. 0.16  }
\begin{quotation}
\jeden Jedním z~nejsilnějších gymnaziálních zážitků autora této úlohy byla
hodina chemie, při které byl za vyrušování v~učebně přesazen do jiné
lavice. Minutu nato však byl znovu kárán, neboť jeho pozornost přitáhla
na lavici napsaná básnička: 
\begin{verse}
\textit{Nevydrží Vám myš v~klidu? }\\
 \textit{Použijte kyanidu! }\\
 \textit{Stačí jedna či dvě kapky, }\\
 \textit{už jí zpomalujou tlapky. }\\
 \textit{A po třetí kapce }\\
 \textit{spinká myška sladce.}
\end{verse}
Po několika letech autor tuto básničku náhodou objevil opět v~hlubinách
internetu, protože však chyběly vhodně jednoduché úlohy na Chemiklání,
zamyslel se místo vzpomínání nad tím, zdali by jeden gram, perorálně
požitý, některé z~následujících látek, dokázal trvale poškodit i
lidské zdraví. Vyberte tyto velmi nebezpečné látky z~následujícího
seznamu:\\sacharosa, chlorid sodný, kyanid draselný, ethanol, destilovaná
voda, nikotin.
\end{quotation} \dotfill \par 
Ve výčtu jsou pouze dva zjevné jedy, a to kyanid draselný a nikotin.
Všechny ostatní jsou více či méně součástí možného či běžného jídelníčku a jsou v~množství jednoho gramu průměrnému dospělému z~hlediska akutní toxicity neškodné.

\hrulefill % \subsection*{Ročník 3, úloha č. 1.4 }
\begin{quotation}
\jeden Na lékových interakcích v~těle se majoritně podílejí dva základní
mechanismy, a to změny ve vazbě na plazmatické bílkoviny (především
na albuminy), a změny v~aktivitě jaterních enzymů, metabolizujících většinu léčiv. Jaterními enzymy odbourávajícími především cizorodé látky v~organismu jsou cytochromy, tj. hemové bílkoviny oxidující/redukující různé substráty. Mají několik různých izoforem, které se různí svou
afinitou k~jednotlivým látkám. Nejznámější z~nich je cytochrom P450
v~isoformě 3A4. 

V následujícím textu vyberte z~nabídky vždy jedno slovo tak, aby byl text pravdivý.

Pokud budeme konzumovat velké množství citrusů (nejenom všeobecně
známý grep, ale třeba i pomeranč nebo pomelo), dojde k~inhibici, tj. 
\textbf{snížení / zvýšení} aktivity tohoto enzymu.
Pokud k~tomu zároveň užijeme zolpidem, který se účinkem cytochromu
P450 mění na biologicky inaktivní látku, dojde k~\textbf{zesílení
/ zeslabení}\ účinku léčiva. Při pravidelném
užívání čaje z~třezalky tečkované (Hypericum perforatum) dojde k~navýšení
produkce -- \textbf{indukci / depresi}\ cytochromu.
Při souběžném užívání hormonální antikoncepce, která se mění oxidací
na neúčinný metabolit, může dojít ke \textbf{snížení / zvýšení}\ účinnosti této metody ochrany proti početí. 
\end{quotation} \dotfill \par 
Součásti řešení jsou zvýrazněna tučně, nevhodná slova jsou přeškrtnuta:

\textbf{snížení }\textbf{\sout{zvýšení}} aktivity tohoto enzymu\\
\textbf{zesílení }\textbf{\sout{zeslabení}}\ účinku léčiva\\
navýšení produkce -- \textbf{indukci}\ \textbf{\sout{depresi}}\ cytochromu\\
\textbf{snížení}\ \textbf{\sout{zvýšení}}\ účinnosti

\textit{Poznámka autora: Po užití inhibitorů cytochromů dochází poměrně
často k~předávkování s~velmi závažnými účinky na pacienta.}

\hrulefill % \subsection*{Ročník 1, úloha č. 3.4}
\begin{quotation}
\textit{„Was ist das nicht Gift ist? Alle Ding sind Gift und nichts ohne Gift.
Allein die Dosis macht, dass ein Ding nicht Gift ist.“}\\
\dva Autorem známého
výroku o~tom, že pouze dávka rozhoduje, zda je látka jedem, je Phillipus
Aureolus Theophrastus Bombastus von Hohenheim (1493/4 -- 1541), známý jako Paracelsus,
který je považován za zakladatele toxikologie -- vědy zabývající
se zkoumáním interakcí cizorodých látek s~živými organismy, v~širším
povědomí známé jako vědy o~jedech.\\
Podle míry toxicity, vyjádřené
jako množství látky, které dokáže člověku způsobit závažné poškození
zdraví nebo smrt, lze látky rozdělit na prakticky netoxické, málo
toxické, mírně toxické, silně toxické, extrémně toxické a tzv. supertoxické
s letální dávkou menší než 5 mg na 1 kg živé váhy. Seřaďte uvedené
látky vzestupně podle jejich toxicity pro člověka při perorálním podání:\\
ethanol, kofein, $\mathrm{BaSO_{4}}$, $\mathrm{BaCl_{2}}$, NaCl.
\end{quotation} \dotfill \par 
Pětice látek seřazená vzestupně podle toxicity, s~doplněnými údaji\footnote{LD$_{50}$ je experimentálně stanovená dávka, při které zahyne 50 procent pokusných zvířat. Nižší LD$_{50}$ ukazuje na vyšší míru toxicity.} LD$_{50}$:\ 


\begin{center}
\begin{tabular}{ll}
$\mathrm{BaSO_{4}}$ & příliš vysoká pro změření (síran barnatý je téměř nerozpustný)\tabularnewline
ethanol & 7 $\mathrm{g\cdot kg^{-1}}$ (u dětí je to méně, ale přesto je méně toxický než NaCl)\tabularnewline
NaCl & 3 $\mathrm{g\cdot kg^{-1}}$\tabularnewline
kofein & 200 $\mathrm{mg\cdot kg^{-1}}$\tabularnewline
$\mathrm{BaCl_{2}}$ & asi 100 $\mathrm{mg\cdot kg^{-1}}$
\end{tabular}
\end{center}

\newpage %%
 % \subsection*{Ročník 2, úloha č. 4.5 }
\begin{quotation}
\tri Přiřaďte toxickým látkám jejich mechanismus účinku:
\begin{center}
% \begin{tabular}{|>{\centering}p{5cm}||>{\centering}p{5cm}|}
\begin{tabular}{c|p{4.8cm}||c|p{8cm}}

1 & rozpustné soli $\ch{Pb^{2+}}$ & A & reakce s~významnými ionty $\ch{Ca^{2+}}$ a $\ch{Mg^{2+}}$\newline\-za vzniku nerozpustných solí\tabularnewline
\hline 
% \multirow{1}{5cm}{2 & sarin, isopropylester kyseliny methylfluorofosfonové\includegraphics{/2/4-5}} & b & inhibitor enzymů s~SH skupinou v~aktivním místě a některých metaloproteinů\tabularnewline
2 & sarin, isopropylester kyseliny methylfluorofosfonové & B & inhibitor enzymů s~SH skupinou v~aktivním místě\newline\-a některých metaloproteinů\tabularnewline
\hline 
3 & rozpustné fluoridy, $\ch{F^{-}}$ & C & inhibitor cytochrom C oxidasy,\newline\-způsobující „buněčné dušení“\tabularnewline
\hline 
4 & fosgen, $\ch{COCl2}$ & D & inhibitor acetylcholinesterasy\newline\-\tabularnewline
\hline 
5 & kyanid draselný, KCN & E & hydrolýza v~organismu za vzniku žíravých produktů, poškození proteinů
reakcí s~aminoskupinami\tabularnewline

\end{tabular}
\end{center}
Vzorec sarinu, který se nevešel do tabulky:
\begin{center}
\includegraphics{/2/4-5}    
\end{center}
\end{quotation} \dotfill \par 

Správné přiřazení je: 1B, 2D, 3A, 4E, 5C.

Zmíněný sarin je velmi silný neurotoxin a je řazen mezi zakázané zbraně
hromadného ničení. Projevuje se nervově paralyticky a přímá expozice
je smrtelná již za velmi nízkých koncentrací. Působí jako nevratný
inhibitor acetylcholinesterasy, enzymu zajišťujícího reaktivaci nervových
synapsí po přenosu vzruchu. Na podobném principu fungují i v~Rusku populární nervově-paralytické látky typu Novičok.

\chapter{Ostatní}

\section{Slovní hříčky}

% \subsection*{Ročník 4, úloha č. 0.11 }
\begin{quotation}
\jeden Již při představení soutěže a vyhlášení pravidel jsme vám kladli na
srdce, abyste četli zadání s~nejvyšší pečlivostí a vyvarovali se tak
zbytečných chyb. Pro vás, mladší kategorii, toto zdůrazňujeme obzvlášť
a dovolujeme si prověřit vaši znalost lingvistiky a čtenářské gramotnosti
vskutku na elementární úrovni ‒ třeba vaši starší kolegové jsou již
nyní postaveni před skutečná umělecká díla! Ze značek chemických prvků
periodické soustavy sestavte české či slovenské názvy 4 členských
států Evropské unie k~31. prosinci 2018. Diakritická znaménka zanedbejte.
\end{quotation} \dotfill \par 
Například:

CeSKO 

FINSKO/FInSKO

IrSKO

KYPr 

RaKOUSKO 

ReCKO

HoLaNdSKO (pouze slovensky)

\hrulefill % \subsection*{Ročník 5, úloha č. 1.1 }
\begin{quotation}
\jeden V~následujících popisech odhalte chemické prvky a sestavte z~jejich značek
slovo, kterým lze nazvat Leonarda da Vinciho, W. A. Mozarta či Alberta
Einsteina. Možná jste to i vy! 
\begin{enumerate}
\item Polokov, který je ve sloučenině s~arsenem využíván v~elektronických
součástkách. 
\item Kov, který je jednou z~hlavních složek zemského jádra. 
\item Těžký kov, který lze využít jako jaderné palivo. 
\item Nekov, jehož kyslíkatá kyselina je průmyslově jednou z~nejdůležitějších
chemických sloučenin. 
\end{enumerate}
\end{quotation} \dotfill \par 
Hledaným slovem je \textit{GeNiUS}.

\newpage %%
% \subsection*{Ročník 3, úloha č. 0.11}
\jeden Vyluštěte křížovku a napište tajenku.
\begin{center}
\includegraphics[scale=0.47]{03_0_11prazdna orez.pdf}
\end{center}
\begin{quotation}
Legenda:
\begin{enumerate}[topsep=0mm,itemsep=0mm]
\item příjmení československého držitele Nobelovy ceny za chemii
\item odvětví chemie, kam lze zařadit elektrochemii
\item křestní jméno téhož nositele Nobelovy ceny (domácky)
\item prvek s~atomovým číslem 80 (slovensky)
\item první nositel Nobelovy ceny za chemii
\item příjmení československého držitele Nobelovy ceny za literaturu
\item značka prvku se $Z = 80$
\item objevitel elektromagnetické indukce, benzenu a zákonů elektrolýzy
\item název univerzity, kde nobelista z~otázek 1. a 3. získal titul doktora
filozofie
\item objevitel prvků Nd a Pr, přítel D.\,I. Mendělejeva, člen komise navrhovatelů
Nobelovy ceny za chemii
\item město, kde jsou na radnici slavnostně předávány Nobelovy ceny
\end{enumerate}
\end{quotation} \dotfill \par 
Vyplněná křížovka:
\begin{center}
\includegraphics[scale=0.47]{03_0_11 orez.pdf}
\end{center}

\textit{Hydrargyrum} je staré označení rtuti (přes latinu ze starořeckého \foreignlanguage{greek}{ὑδράργυρος}:
vodní, tj. tekuté stříbro).



\hrulefill % \subsection*{Ročník 4, úloha č. 1.1   }
\begin{quotation}
\textit{Pravděpodobně 8. červenec 2014, určitě hostinec Na Křížovce,
kolem 2:30 ráno}.

\textit{Honza Hrubeš: A jak tomu budeme říkat? }

\textit{Martin Balouch: Co já vím? Třeba} (tajenka)

\textit{Honza Hrubeš: To zní hrozně. Radši Chemiklání, obšlehneme název od Fyziklání. }

\textit{Martin Balouch: Tak jo,} (tajenka) \textit{zní fakt hrozně\footnote{Pozn. autora: Původní rozhovor byl pro potřeby soutěže převeden
do slušné a literárně publikovatelné podoby, hlavní aspekty však zůstaly
zachovány. Kolegové z~Fyziklání s~naší inspirací jejich názvem souhlasili.}.}\\

\dva Na adrese \href{https://bit.ly/KrizovkaChemiklani}{bit.ly/KrizovkaChemiklani} najdete křížovku.
Soutěžním úkolem bylo napsat znění tajenky.\\Řešení najdete na adrese \href{https://bit.ly/Tajenka}{bit.ly/Tajenka}.
\end{quotation}

\hrulefill % \subsection*{Ročník 1, úloha č. 1.3}
\begin{quotation}
\dva Člověk se periodickou tabulkou prokousával velmi pozvolna. V~průběhu
dějin tak byla klasifikace prvků několikrát upravena a (bohužel) bylo
nutné upustit od značení elementů pomocí alchymistických symbolů.
Do současné doby přežila po sérii úprav a přesunů tabulka Dmitrije
Mendělejeva. Samotný příliv prvků znepříjemnil život také autorům
-- v~dávných dobách mnohdy stačilo popsat latinským výrazem, z~čeho
byl prvek získán, v~současnosti však nemají chemici úlohu tak snadnou.\par
Mimochodem, které názvy prvků periodické tabulky mají geografickou
souvislost s~Německem? Postačí dva.
\end{quotation} \dotfill \par 

V tabule níže uvádíme pár prvků spojených s~Německem. Nelze si nevšimnout silného zastoupení Darmstadtu a jeho vyšších správních celků. V~Darmstadtu se totiž nachází výzkumné centrum pro vysokoenergetickou jadernou fyziku, kde byly připraveny prvky 107 až 112\footnote{\href{https://youtu.be/DtOsEPwtSkQ}{https://youtu.be/DtOsEPwtSkQ}}.

\begin{center}
\begin{tabular}{r|l|l}

Značka & Název & Původ\tabularnewline
\hline\hline
\ch{_{32}Ge} & Germanium & z~latinského \textit{Germania}, země Germánů, později Německo\tabularnewline
\hline
\ch{_{65}Eu} & Europium & podle Evropy, jíž je Německo součástí\tabularnewline
\hline
\ch{_{75}Re} & Rhenium & podle řeky Rýna\tabularnewline
\hline
\ch{_{108}Hs} & Hassium & podle spolkové země Hesenska (\textit{Hessen})\tabularnewline
\hline
\ch{_{110}Ds} & Darmstadtium & podle města Darmstadt v~Hesensku\tabularnewline

\end{tabular}
\end{center}


\hrulefill % \subsection*{Ročník 1, úloha č. 2.2 }
\begin{quotation}
\jeden Ani organizátoři se při sestavování úloh nenudili. Třeba při registraci
nějakého hlupáka napadlo, že bude u~každého slova zjišťovat, jestli
ho nelze složit z~názvů prvků periodické tabulky. Tak se k~blbnutí
přidejte a vymyslete aspoň tři známá, v~běžných (myšleno českých)
kalendářích dohledatelná jména, která se dají napsat výhradně s~použitím značek prvků.
\end{quotation} \dotfill \par 
Níže nabízíme několik jmen, která mohou být řešením. Musíme dodat, že i když jsme se snažili obsáhnout všechna jména, je možné, že nám nějaké uteklo.

ALiCe, NeLa, SVAtOPLuK, RuFIn, AgNeS, InEs, STeLa, FRaNCeSCa, IVONa,
YVONa, VInCeNC, IReNa, GeORg, OTa, VLaSTiSLaV, BLaHOSLaV, STaNiSLaV,
IVO, VIOLa, FiLiP, KAmILa, LAuRa, BrUNO, AlOIS, PaVLa, IVaN, PrOKOP,
NoRa, KArOLiNa, KArLa, LiBOSLaV, LuBOSLaV, KrISTiNa, OSKAr,
BOHUSLaV, OTaKAr, BrONiSLaV, OLiVEr, ReNaTa, NiNa, ErIK, EuGeN, OTmAr,
NiKOLa, IVa, SVAtOSLav, BArBORa, NiKOLaS, VRaTiSLaV, SiMoNa, AlBiNa,
SiMon, VLaSTa. 

\newpage %nadpis%
\section{Trolling čili Úlohy, kterými jsme si stříleli ze soutěžících}

 % \subsection*{Ročník 3, úloha č. 0.23}
\begin{quotation}

\jeden Před několika lety se objevila petice požadující zákaz látky DHMO.
Posuďte sami její argumenty:
\begin{itemize}
\item při vdechnutí většího množství způsobuje smrt,
\item byla nalezena v~rakovinných nádorech, 
\item umožňuje život škodlivým bakteriím, 
\item způsobuje korozi materiálů, 
\item je hlavní složkou kyselých dešťů, 
\item přispívá ke skleníkovému efektu,
\item je známá také jako oxidan, 
\item po kontaktu s~lidskou kůží na ní zůstává i po důkladném umytí.
\end{itemize}
Uveďte nejběžnější název této chemikálie. 
\end{quotation} \dotfill \par 
Jedná se o~obyčejnou vodu.

\hrulefill % \subsection*{Ročník 2, úloha č. 2.1}
\begin{quotation}
\jeden Teploty bodu varu čistých látek jsme zvyklí považovat za konstantní.
Je ale třeba si uvědomit, že tato teplota konstantou není, její hodnota
závisí na množství dalších parametrů. Voda v~rychlovarné konvici o~výkonu 1000 W vře při teplotě 373,02 K. Při jaké teplotě bude vřít
voda v~konvici o~výkonu 1500 W, je-li v~obou konvicích stejný tlak? 
\end{quotation} \dotfill \par 
Výkon konvice nemá na teplotu varu žádný vliv. Teplota v~druhé konvici
bude tedy stejná jako v~té první.

\hrulefill % \subsection*{Ročník 5, úloha č. 4.3 }
\begin{quotation}
\dva Určete tlak nasycených par nad vroucím toluenem v~otevřené baňce
za atmosférického tlaku, znáte-li tyto hodnoty:

Teplota varu toluenu $\theta=110,6$ °C.

Antoineova rovnice 
\[
p=\exp\left(13,9987-\frac{3096,52}{T-53,67}\right)
\]

kde $p$ je tlak v~kPa a $T$ teplota v~K.
\end{quotation} \dotfill \par 
Vroucí toluen očividně vře, tudíž je tlak jeho par roven atmosférickému tlaku. Platí tedy $p=p_{\mathrm{atm}}=101~325$~Pa.

\newpage %%
 % \subsection*{Ročník 3, úloha č. 4.4 }
\begin{quotation}
\textit{Součástí zadání jsou dva grafy fázové rovnováhy.}
\begin{center}
\includegraphics[scale=0.45]{images_new/3/4-4a.pdf}
\par
% \end{center}
% \begin{center}
\includegraphics[scale=0.45]{images_new/3/4-4b.pdf}
\par\end{center}

\dva Do vsádkové destilační aparatury byla vložena směs methanolu (15 \%
mol.) a vody. Jaký bude molární zlomek methanolu v~destilátu, pokud
směs předestilujeme beze zbytku?
\end{quotation} \dotfill \par 
Destilace je metoda umožňující separaci složek směsi na základě jejich
rozdílné těkavosti. Pokud ovšem do aparatury umístíme dvousložkovou
směs a zahříváme ji tak dlouho, až ve varné baňce nic nezbyde, bude mít destilát
složení totožné se vsádkou, tedy molární zlomek methanolu rovný 15
\%.

\newpage %%
 % \subsection*{Ročník 1, úloha č. 4.4}
\begin{quotation}
\dva Největší a nejtěžší zlatou cihlu na světě o~hmotnosti 250 kg nechala
odlít 11. června 2005 japonská společnost Mitsubishi Materials Corporation.
Základna cihly má rozměry $45,5 \mathrm{cm} \times 22,5 \mathrm{cm}$. Vrchní část má $38 \mathrm{cm} \times 15 \mathrm{cm}$. Úhel, který svírá základna se stranou cihličky, je všude stejný
a má hodnotu 70\,°. Zlato mělo v~době odlévání hodnotu 3,7 milionu
dolarů. Vypočtěte, jaký objem by měla za normálního tlaku při teplotě
varu, kdyby se vypařila. Předpokládejte ideální chování plynu, zlato
vře při 2970\,°C.
\end{quotation} \dotfill \par 
Úlohu lze vyřešit rychlým přepočtem ze stavové rovnice plynu.
Rozměry cihličky mají řešitele (a také nepozorné organizátory při
zkoušce výpočtu) pouze svést na scestí. 

\[
pV=nRT
\]
\[
V=\frac{mRT}{pM}=\frac{250000\cdot8,314\cdot(2970+273,15)}{101325\cdot196,97}=337,75\ \mathrm{m^{3}}
\]

\hrulefill % \subsection*{Ročník 3, úloha č. 5.4 }
\begin{quotation}
\dva Přiřaďte systematickým názvům polycyklických struktur jejich neformální,
ale častěji používané názvy:
\end{quotation} \dotfill \par 
\begin{center}
% \begin{table}
\begin{tabular}{c|l||c|l}
1 & bicyklo{[}2.1.0{]}pentan & A & prisman \\\hline
2 & 1,2,2,3-tetramethylbicyklo-{[}2.1.0{]}-pentan & B & hausan \\\hline
3 & tetracyklo{[}2.2.0.0\textsuperscript{2,6}.0\textsuperscript{3,5}{]}hexan & C & porkan \\\hline
4 & pentacyklo{[}4.2.0.0\textsuperscript{2,5}.0\textsuperscript{3,8}.0\textsuperscript{4,7}{]}oktan & D & kuban \\
\end{tabular}
% \end{table}
\end{center}
Klíčem k~vyřešení úlohy je určit z~názvů počet cyklů i atomů v~každé
molekule a zamyslet se nad anglickým či latinským významem neformálních
názvů. 

Správné přiřazení:

1B, 2C, 3A, 4D
\noindent \begin{center}

\noindent \begin{centering}
\includegraphics[scale=3.5]{Housane.pdf}\hspace{2cm}
\includegraphics[scale=1]{Porkan.pdf}
\par\end{centering}
Hausan a porkan

\par\end{center}

\noindent \begin{center}

\noindent \begin{centering}
\includegraphics[scale=0.7]{Prisman2.pdf}\hspace{1.5cm}
\includegraphics[scale=1]{Cuban.pdf}
\par\end{centering}
Prisman a kuban

\par\end{center}

\newpage %nadpis%
\section{Nositelé Nobelových cen}
% \subsection*{Ročník 1, úloha č. 1.1}
\begin{quotation}
\jeden Jmenujte českého držitele Nobelovy ceny za chemii a důvod jejího udělení. 
\end{quotation} \dotfill \par 
Jaroslav Heyrovský získal v~roce 1959 Nobelovu cenu za vynález polarografie
a její rozpracování pro analytické využití. První automatický polarograf
sestrojil s~japonským vědcem jménem Masuzó Šikata. Na Nobelovu cenu
byl nominován celkem osmnáctkrát.

\hrulefill % \subsection*{Ročník 2, úloha č. 2.4}
\begin{quotation}
\dva Nobelova cena za chemii je od roku 1901 téměř každoročně udělována
Královskou švédskou akademií věd a jejím prvním držitelem se stal
Jacobus Henricus van 't Hoff. Mezi vyznamenanými některé osobnosti
obzvlášť vystupují z~řady: v~průběhu 20. století byla celkem čtyřem
osobám udělena Nobelova cena dokonce dvakrát, ať samostatně nebo rozdělená
mezi více držitelů. Nám postačí, když uvedete jméno alespoň jednoho
z~takto vyznamenaných držitelů; pochopitelně se nemusíte omezovat
pouze na chemii.
\end{quotation} \dotfill \par 
V průběhu let byly více než jednou Nobelovou cenou vyznamenány tyto
osobnosti, v~chronologickém pořadí:
\begin{itemize}[topsep=0mm,itemsep=0mm]
\item Marie Skłodowska-Curie 
\begin{itemize}[topsep=0mm,itemsep=0mm]
\item Nobelova cenu za fyziku, 1903, \textit{za objev a výzkum radioaktivity} ($\nicefrac{1}{4}$
M. S.-C., dále $\nicefrac{1}{4}$ Pierre Curie, $\nicefrac{1}{2}$
Antoine Henri Becquerel)
\item Nobelova cenu za chemii, 1911, \textit{za objev prvků radia a polonia, izolaci
radia a výzkum vlastností a sloučenin tohoto pozoruhodného prvku} (neděleně)
\end{itemize}
\item Linus Carl Pauling 
\begin{itemize}
\item Nobelova cenu za chemii, 1954, \textit{za výzkum povahy chemické vazby a jeho
použití pro objasnění struktury komplexních sloučenin}
\item Nobelova cenu míru, 1962, \textit{za snahu o~omezení zbraní hromadného ničení}
(obě ceny neděleně)
\end{itemize}
\item John Bardeen 
\begin{itemize}
\item Nobelova cenu za fyziku, 1956, \textit{za výzkum polovodičů a objev tranzistorového
efektu} ($\nicefrac{1}{3}$ J. B., dále $\nicefrac{1}{3}$ William
Bradford Shockley, $\nicefrac{1}{3}$\,Walter Houser Brattain) 
\item Nobelova cenu za fyziku, 1972, \textit{za teorii supravodivosti} ($\nicefrac{1}{3}$
J. B., dále $\nicefrac{1}{3}$ Leon Neil Copper, $\nicefrac{1}{3}$
John Robert Schrieffer)
\end{itemize}
\item Frederick Sanger 
\begin{itemize}
\item Nobelova cenu za chemii, 1958, \textit{za práci na struktuře proteinů, zejména
inzulinu} (neděleně)
\item Nobelova cenu za chemii, 1980, \textit{za studium biochemie nukleových kyselin,
zejména rekombinantní DNA a za určení sekvencí bází v~nukleových kyselinách}
(¼ F. S., dále ¼ Walter Gilbert, ½\,Paul Berg)
\end{itemize}
\end{itemize}
Pokud kromě osobností vezmeme v~úvahu i vyznamenané instituce, je
třeba zmínit ještě Mezinárodní komisi Červeného kříže (Nobelova cena
míru 1917, 1944, 1963) a Úřad vysokého komisaře OSN pro uprchlíky
(Nobelova cena míru 1954, 1981).

\hrulefill % \subsection*{Ročník 3, úloha č. 1.1   }
\begin{quotation}
\dva Stejně jako v~předchozích ročnících je mezi úlohami Chemiklání jedna
zaměřená na držitele Nobelových cen. Vaším úkolem v~této úloze bude
spojit držitele Nobelových cen s~příspěvky, za niž byli touto cenou
oceněni. 
\begin{description}
\item [{A}] James D. Watson, Francis Crick, Maurice Wilkins
\item [{B}] Ernest Rutherford
\item [{C}] Jacobus Henricus van't Hoff
\item [{D}] Victor Grignard 
\item [{E}] Chandrasekhara Venkata Raman 
\newpage %text%
\item [{1}] Za objev zákonů chemické dynamiky a osmotického tlaku v~roztocích
\item [{2}] Za výzkum rozpadu prvků a chemii radioaktivních látek
\item [{3}] Za objev molekulární struktury nukleových kyselin a jejich
významu pro přenos dědičné informace v~živém materiálu 
\item [{4}] Za výzkum světelného rozptylu a objev po něm pojmenovaného
jevu
\item [{5}] Za objev Grignardova činidla 
\end{description}
\end{quotation} \dotfill \par 
Správné přiřazení: 1C, 2B, 3A, 4E, 5D.


\section{Makromolekulární chemie}

% \subsection*{Ročník 3, úloha č. 1.6 }
\begin{quotation}
\jeden Svět chemiků je plný mnoha různých zkratek a symbolů, jimiž se brání
přesile složitých systematických názvů. Přiřaďte k~názvům a zkratkám
jednotlivých polymerů jejich popis.

\noindent\textbf{1} polyethylentereftalát, PET\\
\textbf{2} polystyren, PS\\
\textbf{3} polytetrafluorethylen, PTFE\\
\textbf{4} polyethylen, PE\\
\textbf{5} polyvinylchlorid, PVC\par
\vspace{3mm}
\noindent\textbf{A} Tento strukturně nejjednodušší plast se používá například pro tvorbu
mikrotenových sáčků, jeho vysokohustotní, nerozvětvený typ se používá
na výrobu hraček, případně některých typů potrubí.\\
\textbf{B} Z~tohoto plastu se vyrábí například uzavíratelné obaly na nápoje.
Snaha o~environmentálně přívětivější obalovou politiku vede k~vysoké
recyklovatelnosti tohoto plastu, v~Německu jsou dokonce obaly z~tohoto
materiálu vratné podobně jako u~nás skleněné lahve od piva.\\
\textbf{C} Z~tohoto plastu se vyrábějí například potrubí (Novodur). Z~měkčené
varianty tohoto plastu se vyrábí též linoleum.\\
\textbf{D} Tento polymer je mimořádně chemicky odolný. Má jeden z~nejnižších
koeficientů tření, vyrábějí se z~něj díly a těsnění například pro
chemické sklo (kohouty v~dělicích nálevkách, těsnicí pásky ad.).\\
\textbf{E} Tento plast, obzvláště jeho pěnová forma, se používá jako isolační
materiál například na obklady domů. Dost často jsou jím vystlány krabice
s elektronikou, aby během převozu nedošlo k~jejímu poškození.
\end{quotation} \dotfill \par 
Správné přiřazení: 1B, 2E, 3D, 4A, 5C.

\hrulefill % \subsection*{Ročník 5, úloha č. 3.3  }
\begin{quotation}
\tri Pokud se lidstvu podaří kolonizovat vzdálenější části Sluneční soustavy,
nebo dokonce vzdálenějšího vesmíru, stane se novým zajímavým problémem
přenos informací mezi jednotlivými vesmírnými sídly a Zemí. Rychlost
přenosu je omezena rychlostí světla ve vakuu, ať přenášíme informace
jakkoliv. Je však stále oříškem bezdrátově přenášet velké objemy dat,
takže na Zemi stále většina dat teče kabely. To by ale bylo problematické
pro komunikaci se vzdálenými koloniemi.

Zkuste nyní řádově odhadnout, jak těžké by bylo vedení spletené ze
tří polyacetylenových vláken přenášející informace od Slunce k~nejbližší
hvězdě, Alpha Centauri A, vzdálené 4,37 světelného roku. (Světelný
rok je vzdálenost, kterou světlo ve vakuu urazí za 1 rok.)

Pokud si netroufáte na odhad, pomozte si výpočtem. Vzdálenost C--C
v polyacetylenu odhadněte jako 125 pm.
\end{quotation} \dotfill \par 
K výsledku přijdeme méně odvážnou cestou, tedy výpočtem. Vzdálenost 4,37 světelného
roku převedeme nejdřív do lépe představitelných jednotek, na metry. Jeden světelný rok je vzdálenost, kterou světlo urazí za rok. Rok má $365,25\cdot24\cdot60\cdot60$ sekund, to tedy při rychlosti světla $c=\num{299792458}\, \mathrm{m \cdot s^{-1}}$ znamená, že příslušná dráha činí
\[
60\cdot60\cdot24\cdot365,25\cdot\num{299792458} = 9,46 \cdot 10^{15} \si{\metre}
\]
Převod světelných let na metry je pak jednoduchý:
\[
4,37\mathrm{\ ly}=4,37\cdot60\cdot60\cdot24\cdot365,25\cdot299792458=4,134\cdot10^{16}\ \mathrm{m}
\]
Vzdálenost mezi dvěma atomy uhlíku je 125 pm. Na jednom metru tohoto vlákna tedy najdeme následující množství atomů uhlíku:
\[
\rho=\frac{1}{125\cdot10^{-12}}=8\cdot10^{9}\,\mathrm{m^{-1}}
\]

Nesmíme pak zapomenout, že vlákna jsou celkem tři ($N=3$). Celé vedení
by pak mělo hmotnost 
\[
m=l\cdot N\cdot\rho\cdot m_{\ch{C}}=19,78\mathrm{\ kg}
\]
Pro odhady byla dovolena tolerance v~rámci řádu, tedy 10 až 99 kg.

\hrulefill % \subsection*{Ročník 4, úloha č. 5.2   }
\begin{quotation}
\dva Nejmenovaná firma v~rámci hodnocení kvality nakupovaných surovin provedla
analýzu délky řetězců dodaného polystyrenu. Z~přístroje určeného k
této analýze vyjel následující graf četnosti výskytu molární hmotnosti
polystyrenu. Kolik strukturních jednotek bylo v~nejčastěji se vyskytujícím řetězci v~této várce polystyrenu?
\end{quotation} \dotfill \par 
\noindent \begin{center}

\includegraphics[scale=0.4]{/4/5-2}

\par\end{center}

Z~vrcholku grafu odečteme nejčastější molární hmotnost jako 210 $\mathrm{kg\cdot mol^{-1}}$.
Molární hmotnost monomeru polystyrenu (styrenu) je 104,15 $\mathrm{g\cdot mol^{-1}}$. Počet
monomerů v~tomto nejčastějším řetězci vypočítáme podělením obou hodnot:
\[
N=\frac{M_{\mathrm{graf}}}{M_{\mathrm{styren}}}=\frac{210\,000}{104,15}=2016
\]
 V~nejčastěji se vyskytujícím řetězci polystyrenu je 2016 monomerů
styrenu. Není bez zajímavosti, že je to zároveň rok, kdy se konal
první ročník Chemiklání. 

\section{Chemie ve světě}

% \subsection*{Ročník 3, úloha č. 0.12 }
\begin{quotation}
\jeden S~chemikáliemi se setkáváme nejen v~laboratoři, ale také téměř každý
den doma v~kuchyni. Pod jakým názvem bychom následující sloučeniny
koupili v~obchodě s~potravinami? 
\begin{itemize}[topsep=0mm,itemsep=0mm]
\item [{a)}] NaCl
\item [{b)}] NaHCO$_{3}$ 
\item [{c)}] CH$_{3}$COOH (8\% obj. vodný roztok)
\item [{d)}] C$_{12}$H$_{22}$O$_{11}$ 
\item [{e)}] CH$_{3}$CH$_{2}$OH (40\% obj. vodný roztok)
\end{itemize}
\end{quotation} \dotfill \par 
\newpage %text% 
Řešení:\\

 \begin{itemize}[topsep=0mm,itemsep=0mm]
 \item [{a)}] kuchyňská sůl
 \item [{b)}] jedlá soda
 \item [{c)}] ocet
 \item [{d)}] cukr
 \item [{e)}] vodka či jiná lihovina
 \end{itemize}

\hrulefill % \subsection*{Ročník 5, úloha č. 1.2  }
\begin{quotation}
\dva Svět chemiků je stále plný nejrůznějších názvů a zkratek\ldots

\ldots\-a ty vyvolávají především v~hlavách studentů velký zmatek. Především
tomu tak je proto, že mnohé známé molekuly kromě názvů systematických
označujeme také názvy triviálními, a co víc, ani v~názvech systematických
kýžený pořádek není. Mnohdy můžeme jednu molekulu dokonce pojmenovat
více způsoby, které vycházejí ze zcela odlišných principů -- posuďte
tuto skutečnost třeba na příkladech uvedených níže a uveďte, které
názvy si odpovídají:\\
acetylen, propanon, sirovodík, amoniak, aceton, čpavek, ethyn, sulfan.
\end{quotation} \dotfill \par 
Odpovídajícími dvojicemi jsou acetylen -- ethyn, sirovodík -- sulfan,
aceton -- propanon a čpavek -- amoniak.

\hrulefill % \subsection*{Ročník 2, úloha č. 1.2 }
\begin{quotation}
\jeden Chemik se bez správného skla v~laboratoři neobejde. Musí se však vyznat
v jeho názvech. Doplňte k~těmto kusům chemického nádobí jejich běžně
užívané názvy:
\end{quotation} \dotfill \par 
\begin{center}
\includegraphics[scale=0.35]{/2/1-2}
\end{center}

\hrulefill % \subsection*{Ročník 3, úloha č. 1.3 }
\begin{quotation}
\jeden Domácí potřeby obsahují celou řadu chemikálií. Přiřaďte, jaká chemikálie
je podstatou kterého drogistického zboží: 
\begin{center}
% \begin{tabular*}{9cm}{@{\extracolsep{\fill}}rlrl}
\begin{tabular}{c|p{5.5cm}||c|p{7cm}}
1 & tuhé mýdlo & A & hydroxid sodný\tabularnewline
\hline
2 & odlakovač na nehty & B & ethanol\tabularnewline
\hline
3 & Krtek na čištění odtoků a odpadů & C & sodné soli vyšších mastných kyselin\tabularnewline
\hline
4 & Okena & D & aceton, případně ethylacetát či 2-propanol\tabularnewline
\hline
5 & repelenty hmyzu & E & \textit{N},\textit{N}-diethyl-3-methylbenzamid\newline(diethyltoluamid, DEET)\tabularnewline

\end{tabular}
\end{center}
\end{quotation} \dotfill \par 
Správné přiřazení: 1C, 2D, 3A, 4B, 5E.

\newpage %%
 % \subsection*{Ročník 1, úloha č. 5.1 }
\begin{quotation}
\jeden Chemie je vědou výrazně méně nebezpečnou, než se široké veřejnosti
snaží namluvit některá média. Nehledě na tuto skutečnost je ale mnohdy třeba
dodržovat přísné předpisy, které se snaží zabránit nejrůznějším možným
náhodám, ne vždy se ale povede lidskou blbost překonat. Osvědčte své
znalosti a napište, co znamenají tyto bezpečnostní štítky, které zhusta
potkáváte při práci s~nejrůznějšími chemikáliemi.
\end{quotation} \dotfill \par 
\begin{center}
\includegraphics{/1/5-1}
\end{center}

Štítky v~této nové podobě odpovídají globálně harmonizovanému systému
klasifikace a označování chemikálií (GHS), který v~současnosti zavádí
OSN. Stanovuje pravidla pro identifikaci nebezpečných chemikálií a
informování o~nebezpečích prostřednictvím symbolů a vět na štítcích
obalů a bezpečnostních listů. 

\hrulefill % \subsection*{Ročník 1, úloha č. 6.4}
\begin{quotation}
\dva Autor této úlohy jednou chtěl spotřebovat zbytky od oběda. K~obědu
byla kachna s~červeným zelím a bramborovým knedlíkem, zbylo ale jen
zelí a knedlík. Autor se po zhodnocení svých kuchařských dovedností
rozhodl všechny zbytky hodit na jednu pánev. Ke knedlíkům a zelí přidal
ještě dvě vejce. Co se ale nestalo, po usmažení měl pokrm zvláštně
modrou barvu. Napište nejpravděpodobnější příčinu, proč pokrm zmodral\footnote{Tohle se opravdu stalo. Můžu říct, že na chuť pokrmu neměla barevná
změna žádný vliv.}. 
\end{quotation} \dotfill \par 
Nejpravděpodobnějším vysvětlením je zcela přirozeně chemická
příčina -- antokyany obsažené v~červeném zelí totiž patří mezi přírodní
indikátory, které mění svou barvu v~závislosti na pH. V~kyselém
prostředí jsou zbarvené červenofialově, při alkalizaci pozvolna
modrají a zelenají. Vzhledem k~tomu, že vejce jsou mírně zásaditá, změnilo zelí barvu. (Barevnou změnu můžete též občas pozorovat při umývání nádobí.)

\section{Co se jinam nevešlo}

 % \subsection*{Ročník 4, úloha č. 2.2   }
\begin{quotation}
\dva Entropie: pro někoho trochu záhadná fyzikální veličina, která nám
v nejjednodušší představě udává míru neuspořádanosti systému. Pokud
se podíváme do fyzikálně-chemických tabulek a budeme hledat například
entropii kapalné vody při 298 K, tak zjistíme hodnotu $S^{\ominus}=69,95\,\mathrm{J\cdot K^{-1}\cdot mol^{-1}}$.
Tato základní fyzikální veličina by si ale zasloužila uvést v~násobcích
opravdu základních jednotek SI, nemyslíte? Uveďte jednotku entropie
jako součin či podíl základních jednotek SI.
\end{quotation} \dotfill \par 
Kelvin i mol patří mezi 7 základních jednotek SI. Jediné, co je třeba
vyjádřit, je joule. Joule je jednotka energie, a vzpomeneme-li si na
nějaký vztah pro energii, můžeme tuto jednotku vyjádřit. Použijme
například vztah pro potenciální energii v\,gravitačním poli Země, kde vystupuje hmotnost, gravitační zrychlení a výška.
\[
E=mgh
\]
Dosazením jednotek veličin namísto jejich značek pak dostaneme jednotku energie.
\[
[E]=\SI[inter-unit-product = \ensuremath{{}\cdot{}}]{}{\kilo\gram\metre\per\second\squared\metre}=\SI[inter-unit-product = \ensuremath{{}\cdot{}}]{}{\kilo\gram\metre\squared\per\second\squared}
\]
Dosazením za joule do jednotky entropie získáme výsledný součin.
\[
[S]=\SI[inter-unit-product = \ensuremath{{}\cdot{}}]{}{\kilo\gram\metre\squared\per\second\squared\per\kelvin\per\mole}
\]

\newpage %%
 % \subsection*{Ročník 1, úloha č. 3.1}
\begin{quotation}
\dva Bez mučení přiznáváme, že velká část chemiků si příliš hlavu neláme
s tím, jaké používají jednotky. Běžně se setkáváme s~gramy, torry,
atmosférami a dalšími jednotkami neobsaženými v~SI. Někdy je ale znalost
jednotek nezbytná. Vezměme si například nám všem známou stavovou rovnici
ideálního plynu 
\[
pV=nRT
\]
kde $p$ je tlak, $V$ objem, $n$ látkové množství,
$T$ teplota v~Kelvinech a $R$ univerzální plynová konstanta. Možná si
pamatujete hodnotu $R$ jako 8,314. Ale čeho? Zapište jednotku univerzální
plynové konstanty v~násobcích základních jednotek SI.
\end{quotation} \dotfill \par 
Molární plynová konstanta je rovna přibližně $\SI[inter-unit-product = \ensuremath{{}\cdot{}}]{8,314}{\joule\per\kelvin\per\mole}$. Kelvin a mol jsou základní jednotky SI, joule je jednotkou energie,
kterou například podle vzorce $E=\frac{1}{2}m\cdot v^{2}$ vyjádříme
jako $[E]=\SI[inter-unit-product = \ensuremath{{}\cdot{}}]{}{\kilo\gram\metre\squared\per\second\squared}$.
Kombinací těchto vztahů dostaneme rozměr v~základních jednotkách SI: $R=\SI[inter-unit-product = \ensuremath{{}\cdot{}}]{8,314}{\kilo\gram\metre\squared\per\second\squared\per\kelvin\per\mole}$.

\hrulefill % \subsection*{Ročník 5, úloha č. 4.5 }
\begin{quotation}
\tri pH bývá na středních školách s~oblibou zjednodušeně (se zanedbáním
aktivit) definováno jako záporný dekadický logaritmus koncentrace
H$^{+}$. To obzvláště v~případě koncentrovaných kyselin může vést
k velice nepřesným výsledkům! S~použitím tohoto zjednodušeného vztahu
spočtěte pH kyseliny ze všech nejkoncentrovanější: samotného protonu.
Proton pro zjednodušení úlohy považujte za kouli o~poloměru 8,414$\cdot10^{-16}$
m.
\end{quotation} \dotfill \par 
pH jsme nadefinovali jako 
\[
\pH=-\log[\ch{H+}]
\]
Pro jeho výpočet tedy potřebujeme zjistit koncentraci protonu v~protonu.
Koncentrace je definována jako podíl látkového množství a objemu:
$c=n/V$. Objem protonu spočteme jako 
\[
V=\frac{4}{3}\pi r^{3}=2,491\cdot10^{-45}\mathrm{~m}^{3}
\]

Pokud si uvědomíme, že látkové množství je definováno jako podíl počtu
částic a Avogadrova čísla, můžeme dosadit do vztahu pro koncentraci
\[
c=\frac{N}{N_{\mathrm{A}}V}=6,6551\cdot10^{17}\mathrm{~mol\cdot dm^{-3}}
\]
Nyní už zbývá jen dosadit tuto koncentraci do definice pro pH: 
\[
\pH=-\log[\ch{H+}]=-17,82
\]
což je vskutku úctyhodná hodnota! Že tato hodnota nemá valný
fyzikální význam, jistě dokážete posoudit sami. Pro srovnání kyselostí
koncentrovaných kyselin (jíž jistě proton je) po dlouhou dobu byla
(a stále je) používána Hammettova kyselostní funkce\footnote{Hammett, L. P.; Deyrup, A. J., A SERIES OF SIMPLE BASIC INDICATORS. I. THE ACIDITY FUNCTIONS OF MIXTURES OF SULFURIC AND PERCHLORIC ACIDS WITH WATER. \textit{J. Am. Chem. Soc.} \textbf{1932}, 54 (7), 2721-2739. DOI:~\href{https://doi.org/10.1021/ja01346a015}{\underline{10.1021/ja01346a015}}}, ani ta však není pro kvantifikaci kyselosti protonu úplně vhodná.
V roce 2010 byla navržená nová unifikovaná škála pH, která umožnuje
kvantifikovat relativní kyselost všech species napříč všemi fázemi
a byla by tak vhodnější\footnote{Himmel, D.; Goll, S. K.; Leito, I.; Krossing, I.,
A Unified pH Scale for All Phases. \textit{Angew. Chem. Int. Ed.} \textbf{2010}, 49 (38), 6885-6888. DOI:~\href{https://onlinelibrary.wiley.com/doi/abs/10.1002/anie.201000252}{\underline{10.1002/anie.201000252}}}.

\hrulefill % \subsection*{Ročník 4, úloha č. 6.2  }
\begin{quotation}
\tri Mnohé chemické děje, jako třeba rozpouštění, koroze, adsorpce nebo
katalýza, se odehrávají na površích látek. V~makrosvětě je, co se
týče počtu zúčastněných molekul, povrch látky zanedbatelný vůči celkovému
objemu. S~klesající velikostí částic ale začíná přibývat počet atomů,
které jsou na povrchu. Dochází tak k~výrazným změnám chování, především
zesílení reaktivity či katalytických účinků.

Pokud se ale posuneme
ještě o~řád níže, přecházíme už do světa, kde začínají hrát prim kvantové
efekty. Nedávno objevené nanočástice -- kvantové tečky -- například
už mají energetické hladiny elektronů ne spojité, ale diskrétní (rozdělené)
podobně jako atomy. Jedním z~důsledků tohoto jevu je, že při osvícení světlem o~určité
vlnové délce samy začnou světlo o~jiné vlnové délce intenzivně vyzařovat. 
\newpage %text%
Zkuste sami vypočítat, jaká část atomů (v procentech) je na povrchu jasně zeleně svítící
kvantové tečky o~hraně 8,4 nm. Velikost atomu odhadněte jako 70 pm. 

\textit{Poznámka: Kvantovou tečku i atom můžeme v~rozumném přiblížení považovat
za krychli.}
\end{quotation} \dotfill \par 
Pokud předpokládáme „krychlovost“ atomu i kvantové tečky podle zadání,
je velice snadné si představit, že povrchem je svrchní vrstva
atomů na tečce. Vnitřek je pak krychlí s~hranou o~dva atomy kratší
(jedna vrstva z~každé strany tečky se odečítá jako povrch). Matematicky
vyjádřeno: 
\[
\alpha=\frac{V_{\mathrm{povrch}}}{V_{\mathrm{tečka}}}=\frac{V_{\mathrm{tečka}}-V_{\mathrm{vnitřek}}}{V_{\mathrm{tečka}}}=\frac{8400^{3}-(8400-2\cdot70)^{3}}{8400^{3}}=0,0491712\ldots\doteq4,917\,\%
\]
 Výsledek lze vyjádřit obecně a velmi přesně, v~soutěži byly proto uznávány
pouze hodnoty v~uzavřeném intervalu 4,91 až 4,93 procenta. 

\hrulefill % \subsection*{Ročník 5, úloha č. 7.5 }
\begin{quotation}
\tri \textit{„Gone, reduced to atoms.``} Toto prohlásil Thanos o~kamenech nekonečna,
když je zničil\footnote{\href{https://en.wikipedia.org/wiki/Infinity_Gems}{https://en.wikipedia.org/wiki/Infinity\_Gems}}. Z~toho můžeme usuzovat, že podobný osud potkal i
živé tvory na Zemi, když je Thanos vymazal z~existence. Průměrný
lidský jedinec váží 68 kg a je hmotnostně složen z: 66 \% kyslíku,
19,5 \% uhlíku, 9,7 \% vodíku, 3,3 \% dusíku a 1,5 \% vápníku. Ostatní
prvky zanedbáme. Jelikož množství možných typů vazeb překračuje obsah
této úlohy, počítejte s~průměrnou energií jedné vazby pro organické
sloučeniny 4~eV$\cdot$vazba$^{-1}$ a základní vazností prvků. Počet
lidí na Zemi je v~okamžiku psaní této úlohy 7\,737\,612\,322. Spočítejte,
kolik energie by Thanos spotřeboval na \textit{snapnutí}, tedy naprostou atomizaci
poloviny lidské populace a výsledek uveďte v~exajoulech. Jeden exajoule
odpovídá $10^{18}$~J.
\end{quotation} \dotfill \par 
Proces výpočtu je založen na určení počtu molů jednotlivých prvků
v lidském těle.
\[
n_{\mathrm{prvek}}=\frac{m_{\mathrm{člověk}}w_{\mathrm{prvek}}}{M_{\mathrm{prvek}}}
\]
Z~každého uhlíku jdou čtyři vazby, z~každého dusíku tři, z~každého vápníku dvě a z~každého vodíku jedna vazba. Celkový počet vazeb získáme vynásobením látkových množství jednotlivých prvků počtem vazeb, které z~nich vycházejí. Protože každé vazby se účastní dva atomy, je třeba ještě takto získané množství podělit dvěma.
\[
n_{\mathrm{vazeb}} = \frac12 ( 4\cdot n_{\mathrm{C}}+3\cdot n_{\mathrm{N}}+2\cdot n_{\mathrm{Ca}}+2\cdot n_{\mathrm{O}}+n_{\mathrm{H}})=8579\, \mathrm{mol}
\]
Vynásobením Avogadrovou konstantou a průměrnou energií jedné vazby (4~eV) získáme energii potřebnou na atomizaci jednoho člověka:
\[
E_{\mathrm{atomizace}}=n_{\mathrm{vazeb}}\cdot N_{A} \cdot E_{\mathrm{vazba}}=8579\cdot 6,022 \cdot 10^{23}\cdot 4=2,1\cdot 10^{28}\,\mathrm{eV} = 3,3 \, \mathrm{GJ}
\]
Protože dle zadání hledáme energii potřebnou pro atomizaci poloviny lidské populace, je třeba ještě energii řádně ponásobit počtem atomizovaných lidí.
\[
E_{final} = 3,3 \cdot 10^9 \cdot \frac{7\,737\,612\,322}{2} = 13 \, \mathrm{EJ}
\]

\newpage %%
 % \subsection*{Ročník 5, úloha č. 8.4  }
\begin{quotation}
\tri Svět chemiků je stále ještě plný nejrůznějších symbolů a zkratek…

Jednou z~chemických
zkratek, která je známá i širší veřejnosti, je LPG (angl. \textit{Liquefied
Petroleum Gas}), tedy zkapalněná směs propanu s~butanem, která se
používá jako alternativní palivo pro motorová vozidla. Malé množství
této směsi jsme zapálili v~přesně odpovídajícím množství kyslíku
a spaliny prohnali vymražovacím zařízením. Tlak v~trubici se spalinami
přitom poklesl o~57 procent.

Určete molární zlomek butanu v~LPG směsi, pokud předpokládáme, že
v ní nebyly žádné nečistoty a že došlo k~dokonalému spálení.
\end{quotation} \dotfill \par 

Klíčové pro výsledek je uvědomit si, co se při dokonalém spálení děje: propan i butan jsou dokonale spáleny na oxid uhličitý a vodu.

\[
\ch{C3H8 + 5 O2}\ \rightarrow\ \ch{3 CO2 + 4 H2O}
\]
\[
\ch{C4H10 +} \frac{13}{2}\ch{O2}\ \rightarrow\ \ch{4 CO2 + 5 H2O}
\]
Pokud byl LPG dokonale spálen ve stechiometrickém množství, odchází
jako spaliny pouze oxid uhličitý a voda. Ta je vymražena a dále běží
pouze oxid uhličitý. Jelikož je podle Daltonova a Avogadrova zákona
tlak závislý pouze na počtu částic, po spálení čistého propanu by
tlak spalin klesl o~čtyři sedminy a u~čistého butanu o~pět devítin. Problém
se tak redukuje na rovnici o~jedné neznámé:
\[
\frac{5}{9}x+\frac{4}{7}(1-x)=\frac{57}{100},
\]
jejímž řešením je hledaný molární zlomek butanu, přesně $x=0,09$.
\vfill\pagebreak
\chapter*{Na sbírce a na samotné soutěži spolupracovali}
\textbf{Editor, sazba, nápad:}\tabto{9cm}\-Jan Hrubeš

\textbf{Zdobení úloh omáčkou:}\tabto{9cm}\-Stanislav Chvíla

\textbf{Sazba grafů, korektury, mediace sporů:}\tabto{9cm}\-Adam Tywoniak

\textbf{Krocení megalomanie editora:}\tabto{9cm}\-Martin Balouch

\textbf{Holka pro všechno, hlavně pro biochemii:}\tabto{9cm}\-Pavlína Muchová

\textbf{Fundraising:}\tabto{9cm}\-Jan Hrabovský

\textbf{Grafický layout soutěže:} \tabto{9cm}\-Klára Scholleová

\section*{Autoři a recenzenti úloh:}
\textit{Pro přehlednost u~jmen neuvádíme akademické tituly.}
\setlength{\columnsep}{6mm}
\begin{multicols}{3}

Adam Jaroš 

Adam Přáda 

Adam Tywoniak

Agáta Holubová

Anna Carbolová

Anna Freislebenová

Anna Kovárnová 

Alena Budinská

Alexandr Zaykov

Clare Rees-Zimmerman

George Trenins

Heda Chaloupková

Jakub Režňák

Jan Bartáček

Jan Bartoň

Jan Hrabovský

Jan Hrubeš

Jan Němec

Jaroslav Cerman

Jiří Ledvinka

Juraj Malinčík

Jakub Petrús

Karel Berka

Klára Bělíčková

Ladislav Prener

Lukáš Marek

Martin Balouch

Martin Crhán

Marie Grunová

Matúš Drexler

Miroslava Novoveská

Nikola Vršková

Ondrej Kopilec

Ondřej Daněk

Pavlína Muchová

Pavel Měrka

Pavel Rysula

Petr Kalenda

Raz L. Benson

Richard Veselý

Roman Garassy

Simona Kožnarová

Stanislav Chvíla

Soňa Ondrušová

Tereza Dobrovolná

Tereza Gistrová

Vadim Kablukov

Veronika Holubová

Vladimír Finger

Vladimír Němec

Vojtěch Laitl

Vojtěch Pravda

Vít Procházka

Wojciech Jankowski

\vfill

\end{multicols}
\clearpage
\thispagestyle{empty}
\section*{Partneři}

Soutěž Chemiklání je pořádána Fakultou chemicko-technologickou Univerzity Pardubice a spolkem Alumni scientiae bohemicae.

\begin{center}
\begin{tabular}{ m{8cm} m{8cm} }
 \begin{center} \includegraphics[width=5cm]{fakulta_chemicko-technologicka.pdf} \end{center}  & \begin{center}  \includegraphics[width=3.5cm]{asblogo.png} \end{center}
\end{tabular}
\end{center}

Soutěž je spoluvyhlašována a spolufinancována z~Programu podpory soutěží a přehlídek v~zájmovém vzdělávání Ministerstva školství, mládeže a tělovýchovy ČR.

\begin{center}
\includegraphics[width=5cm]{MSMT_logotyp_text_CMYK_cz.pdf}    
\end{center}

K organizaci soutěže a tvorbě úloh se v~roce 2020 připojili též studenti Yusuf Hamied Department of Chemistry, University of Cambridge.

\begin{center}
\includegraphics[width=5cm]{department}    
\end{center}

Na recenzi finální podoby této sbírky se podíleli studenti Gymnázia Na Zatlance, Praha 5.
\begin{center}
\includegraphics[width=4.5cm]{zatlanka}    
\end{center}


\section*{Licence}
Úlohy jsou přístupné pod licencí Creative Commons BY-NC-SA 4.0 International\footnote{Plný text licence je dostupný na \url{https://creativecommons.org/licenses/by-nc-sa/4.0/}}.

\end{document}